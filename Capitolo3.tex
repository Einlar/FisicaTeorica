\documentclass[FisicaTeorica.tex]{subfiles}
\begin{document}
\chapter{Formulazione assiomatica Hilbertiana}

La meccanica quantistica è uno degli ambiti scientifici nei quali la fisica e la matematica più astratta manifestano reciproche influenze. Nella prima metà del novecento diversi matematici e fisici tentarono di costruire una solida base formale capace di assiomatizzare tutto ciò che fino ad allora era solo un fatto sperimentale. \\
La prima formulazione fu quella Hilbertiana (o Standard), elaborata da Hilbert, Dirac (1930) e von Neumann (1932).\\ Anche se la formulazione di Dirac e quella di von Neumann si possono considerare equivalenti, tuttavia la matematica utilizzata da Dirac (un \q{autentico} fisico) è diversa da quella utilizzata da von Neumann (un fisico-matematico).\\
\textbf{Il formalismo di \textbf{Dirac}} è il più seguito dai fisici per la sua maggiore flessibilità e semplicità\footnote{Ciò rispecchia il minimalismo di Dirac, il quale riteneva che le teorie potessero essere corrette solo se semplici.}, seppur a costo di molta difficoltà di formalizzazione rigorosa. Solo nel dopoguerra, con la teoria delle distribuzioni, fu possibile giustificare esaustivamente la matematica di Dirac.\\
\textbf{Il formalismo di von Neumann} è invece quello più seguito dai fisici-matematici, essendo caratterizzato da un maggior rigore matematico e una solida base formale, anche se formalmente è meno flessibile.

\section{Stati puri}\label{sec:stati_puri}
Gli stati puri sono codificati dalle funzioni d'onda $\psi \left(x\right)$ di Schrödinger, analogamente agli stati $\ket{\psi}$ stati di polarizzazione, e sono gli stati che comprendono la \textbf{massima informazione}.

\subsection{Spazio degli stati puri e sue proprietà}
Esaminiamo dunque le proprietà che vogliamo imporre sull'insieme (lo spazio) degli stati puri.
\begin{enumerate}
    \item Tutti i vettori di stato sono elementi di uno \textbf{spazio vettoriale}. \\
    Infatti gli stati possono essere sommati e moltiplicati per un numero complesso, e da questo segue immediatamente il \textbf{principio di sovrapposizione}: ogni vettore che\marginpar{Principio di sovrapposizione} rappresenta uno stato puro (detto \textbf{vettore di stato} o, con abuso di linguaggio, stato) può essere scritto come sovrapposizione, cioè \textbf{combinazione lineare} di altri vettori di stato dello spazio vettoriale a cui appartiene.
    \item Lo spazio vettoriale degli stati è uno spazio vettoriale \textbf{complesso}. \\
    Per la prima volta in fisica l'algebra naturale degli stati è in $\bb{C}$ e non in $\bb{R}$.
    \item In realtà\marginpar{Vettori definiti a meno di proporzionalità} abbiamo visto che $\langle \hat{x}\rangle_\psi$ e $\langle \hat{p}\rangle_\psi$ non cambiano se sostituiamo $\psi$ con $\alpha \psi$, con $\alpha \in \bb{C}\setminus \{0\}$. Inoltre nella definizione dei valori medi (o delle probabilità) abbiamo dovuto normalizzare le funzioni dividendo per $\norm{\psi} = (\int |\psi(x)|^2\, dx)^{1/2}$, e quindi necessariamente $0<\norm{\psi}<\infty$, in pratica escludiamo vettori di norma nulla o infinita.\\
    Per queste ragioni è più corretto definire lo spazio vettoriale a cui appartengono le $\psi$ come segue. \vspace{-0.3em}
    \begin{dfn}[Raggio vettore]\index{Raggio vettore}
    Dato $V$ spazio vettoriale normato, \marginpar{Raggio vettore}$\psi \in V$, l'insieme $\{\alpha\,\psi, \alpha \in \bb{C}\setminus \{0\}\}$ o equivalentemente $\{e^{i\alpha}\psi, e^{i\alpha} \in S^1, \norm{\psi} = 1 \}$, è detto \textbf{raggio vettore} di $V$. Quindi ad ogni \textbf{stato puro} corrisponde \textbf{uno e uno solo} raggio vettore\footnote{Questa in realtà non è una semplice sottigliezza matematica. Tale biiezione \q{a meno di un fattore moltiplicativo} è alla base dell'esistenza dei fermioni - e perciò della \q{materia}.}.
    \end{dfn}
    \item Lo spazio vettoriale $V$ dei vettori di stato ha bisogno di un \textbf{prodotto scalare} $\langle \cdot | \cdot \rangle$,\marginpar{Prodotto scalare} che si usa per esempio per calcolare il valor medio della posizione:
    \[
    \langle \hat{x}\rangle_\psi = \int \frac{\psi^*(x) \, x \, \psi(x)}{\norm{\psi}^2} dx
    \]
    Nel caso di un vettore 2-dimensionale il prodotto scalare è definito da $\braket{\psi|\phi} = \psi_x^* \phi_x + \psi_y^* \phi_y$.
    Vogliamo poi che per il momento valga:
    \[
    \langle \hat{p} \rangle_\psi = \int \frac{\tilde\psi^*(p) \, p \, \tilde\psi(p)}{\norm{\tilde{\psi}}^2} dp
    \]
    per cui per ogni funzione di stato \marginpar{Esistenza della trasformata} $\psi(x)$ deve esistere la sua trasformata di Fourier.
\end{enumerate}
Uno spazio su $\mathbb{R}^n$ che ha tutte queste proprietà\index{Spazio $L^2$} è $L^2\left(\mathbb{R}^n,d^nx\right)$. Per definire $L^2$ si  considerano le funzioni \q{modulo quadro integrabili} secondo Lebesgue, ovvero funzioni $f$ che soddisfano a:
\[
	\int \left|f\left(x\right)\right|^2d^nx<\infty
\]
(dove l'integrale è quello di Lebesgue). Due funzioni di questo tipo si considerano \textbf{equivalenti} se sono uguali quasi\marginpar{Classi di Equivalenza in $L^2$} ovunque (q.o.) rispetto alla misura di Lebesgue.
\begin{dfn}[Spazio vettoriale $L^2$]
Si denota con $L^2\left(\mathbb{R}^n,\ d^n x\right)$ lo spazio delle classi di equivalenza delle funzioni complesse $f$ su $\mathbb{R}^n$ modulo quadro integrabili secondo Lebesgue.
\end{dfn}
$L^2$ è dotato del \textbf{prodotto scalare} definito come:
\[
	\left\langle f,g\right\rangle=\int d^n x f^*\left(x\right)g\left(x\right) \qquad \forall f,g \in L^2\left(\mathbb{R},d^n x\right)
\]
Da qui si nota il perché abbiamo dovuto considerare le classi di equivalenza. Se non lo avessimo fatto $\left\langle f,f\right\rangle=0$ non implicherebbe che $f=0$, poiché anche funzioni quasi ovunque nulle avrebbero la stessa \q{norma} pari a $0$!
\begin{dfn}
Uno spazio vettoriale\marginpar{Spazi pre-Hilbertiani, e di Hilbert} dotato di un \textbf{prodotto scalare} è detto \textbf{pre-Hilbertiano}. Se è anche \textbf{completo} è allora detto di \textbf{Hilbert}.
\end{dfn}

\subsection{Isomorfismi tra spazi di Hilbert}
\begin{dfn}
Due spazi di Hilbert\marginpar{Spazi di Hilbert isomorfi, trasformazioni unitarie} $\mathcal{H}_1$ e $\mathcal{H}_2$ sono \textbf{isomorfi} se esiste una mappa biiettiva $U$, detta \textbf{trasformazione unitaria}, con $U:\mathcal{H}_1\rightarrow\mathcal{H}_2$ che preserva il prodotto scalare:
\[
	\forall \psi, \varphi \in \hs \qquad \left\langle U\psi,\ U\varphi\right\rangle_{\mathcal{H}_2}=\left\langle\psi,\ \varphi\right\rangle_{\mathcal{H}_1}
\]
\end{dfn}
In altre parole, per ogni coppia di elementi di $\mathcal{H}_1$, il loro prodotto scalare (seguendo la \q{regola} di $\mathcal{H}_1$) è lo stesso dei loro corrispettivi $U_\psi$, $U_\varphi$ in $\mathcal{H}_2$ (seguendo la \q{regola} di $\mathcal{H}_2$.

\textbf{Problema}: perché occorre la completezza per lo spazio delle funzioni d'onda, cioè perché ci serve prendere $L^2$? Potremmo considerare ad esempio per le $\psi\in\mathbb{R}^3$, lo spazio $V=L^2\left(\mathbb{R}^3\right)\cap C^2(\mathbb{R}^3)$ come suggerito dall'equazione di Schrödinger:
\[
	-\frac{\hbar^2}{2m} \nabla^2 \psi+V\psi=\mathcal{E}\psi
\]
Infatti siccome nell'equazione si deriva la $\psi$ due volte, è naturale richiedere che ciò sia possibile, mantenendo lo stesso prodotto scalare di $L^2$. La risposta a questa domanda è il fatto che lo spazio così definito risulta pre-Hilbertiano, ma non di Hilbert, e dunque manca la completezza. Ma tale proprietà è \textit{necessaria} per alcuni importanti teoremi, e quindi non possiamo rinunciarvi.\\
Per uno spazio di Hilbert (che è quindi uno spazio \textbf{completo}) valgono infatti i due seguenti teoremi:
\begin{thm}
Per ogni base ortonormale di $\mathcal{H}$\marginpar{Basi ON in spazio di Hilbert} è possibile scrivere ogni vettore di $\mathcal{H}$ in modo \textbf{univoco} come combinazione lineare (eventualmente infinita) di elementi della base (Teorema di Riesz-Fisher)
\end{thm}
Come vedremo  il teorema precedente è cruciale per l'interpretazione statistica della meccanica quantistica.
\begin{thm}
A meno di isomorfismi esiste un \textbf{unico spazio di Hilbert} \textbf{complesso} per ogni \textbf{cardinalità}\marginpar{Cardinalità della base $\leftrightarrow$ unico spazio di Hilbert} della base di vettori di $\mathcal{H}$.
\end{thm}
Quindi ci sono due casi:
\begin{itemize}
    \item La cardinalità (di una base) dello spazio di Hilbert $\mathcal{H}$ è di dimensione $N$ finita, per cui $\hs$ è isomorfo a $\mathbb{C}^N$  ($\mathcal{H}\cong\mathbb{C}^N$). In questo caso, fissata una base ON di $\hs$, ogni vettore in $\hs$ è identificato univocamente da $N$ coordinate complesse, ossia da un vettore in $\bb{C}^N$.
    \item Se la cardinalità (di una base) di $\hs$ è infinita ma numerabile, allora $\mathcal{H}$ è isomorfo a $l_2$, ossia lo spazio\marginpar{Spazio $l_2$} delle successioni (numerabili) in $\mathbb C$  che convergono in modulo-quadro:
    \[
	l_2 = \left \{ \{a_n\}_{n\in\bb{N}}, a_n \in \bb{C} \text{ t.c. } \sum_n |a_n|^2 < \infty \right \}
	\]
	In questo spazio vettoriale è definito il prodotto scalare come estensione di quello in $\mathbb{C}^N$:
	\[
	\left(\left\{a_n\right\},\ \left\{b_n\right\}\right)_{l_2}= \sum_{n}{a_n^* b_n}
	\]
    Rispetto a questo prodotto scalare $l_2$ è completo, e quindi è di Hilbert. 
    \end{itemize}
\begin{dfn}[Spazio separabile]
Uno spazio si dice \textbf{separabile} \marginpar{Spazio di Hilbert separabile} se in esso esiste un sottoinsieme \textbf{denso} e \textbf{numerabile}. Nel caso degli spazi di Hilbert, tale condizione è equivalente ad avere una \textbf{base ortonormale numerabile}.\\
\end{dfn}
\begin{expl}
Un esempio di spazio separabile è $\bb{R}$, in quanto contiene $\bb{Q}$, che è numerabile e denso in $\bb{R}$ (ogni intorno non vuoto di un numero reale contiene almeno un numero razionale\footnote{Ciò significa che si può approssimare con precisione arbitraria i reali (non numerabili) partendo solamente dai razionali, che sono numerabili. Analogamente, per uno spazio di Hilbert è possibile approssimare bene qualsiasi suo elemento usando una combinazione lineare a coefficienti razionali di vettori di una base \textit{numerabile}.}).
\end{expl}
Se $\mathcal{H}$ è separabile, per quanto detto prima sugli spazi con basi ON numerabili, è allora isomorfo a $\bb{C}^N$ o a $l_2$. \marginpar{$\mathcal{H}$ separabile isomorfo a $l_2$} Data una base ON numerabile $\left\{\phi_n\right\}$ di $\mathcal{H}$, tale isomorfismo si ottiene con:
\[
\psi = \sum_{n}{a_n\phi_n} \ \ \mapsto \ \ \left\{a_n\right\}_{\mathbb{C}^N,l_2} \qquad \forall \psi \in \mathcal{H}
\]
Analogamente al caso finito-dimensionale, identifichiamo un vettore $\psi \in \mathcal{H}$ con la successione - che sarà numerabile, essendo $\mathcal{H}$ separabile - dei suoi coefficienti rispetto ad una base ON.\\

\lesson{6}{11/10/2018} % Nuova lezione
A prima vista sembrerebbe che il numero di funzioni di $L^2$ sia molto più grande di quante siano le successioni in ${\mathbb C^N}$. Ma questo è falso perché quando si costruisce $L^2$ si creano le classi di equivalenza, e il risultato è \q{ridurre} di molto il numero di elementi. Intuitivamente si può pensare di \q{ruotare} la base delle funzioni di $L^2$ (si pensi al caso di ${\mathbb C}^N$) per passare da una rappresentazione all'altra. \\

\subsection{Equivalenza tra gli spazi della \MQ}
Tramite questi isomorfismi \marginpar{Equivalenza tra Heisenberg e Schrödinger} possiamo garantire l'\textbf{equivalenza} tra la \textbf{meccanica ondulatoria} di \textbf{Schrödinger}, dove le funzioni d'onda $\psi(x)$ vivono in $\mathcal{H}=L^2(\bb{R}, dx)$ e la \textbf{meccanica matriciale} di \textbf{Heisenberg} in cui $q_{mn}, p_{mn}$ agiscono come matrici $\infty$-dimensionali su $l_2$.\\
Ad ogni elemento $\psi(x)$ dello spazio delle funzioni di Schrödinger è associata una successione $\{a_n\}$ (un vettore $\infty$-dimensionale), cioè uno stato dello spazio di Heisenberg. Ad ogni operatore nello spazio di Schrödinger (il quale data una funzione restituisce un'altra funzione) è associata una matrice $\infty$-dimensionale che agisce sui vettori $\infty$-dimensionali di Heisenberg secondo le regole del prodotto matriciale. Il teorema non fornisce un modo pratico per passare da una all'altra rappresentazione, ma dimostra che questo modo esiste. \\
Più precisamente data una funzione $\psi$ di Schrödinger, siano $\phi_n$ le soluzioni dell'equazione stazionaria di Schrödinger per l'oscillatore armonico, che sono anche una base ortonormale di $L^2(\bb{R}, dx)$. Allora la mappa di isomorfismo è:
\[
L^2\left(\mathbb R,dx\right)\ni\psi \mapsto \left\{a_n=\left(\phi_n,\ \psi\right)\right\}_{i=1}^\infty\in l_2
\]
Lo stesso teorema garantisce inoltre\marginpar{Equivalenza tra rappresentazione in posizione e in momento} l'equivalenza tra lo spazio delle \textbf{funzioni d'onda} $\psi$ $\left(x\right)\in L^2(\bb{R}, dx)$ e delle funzioni d'onda del \textbf{momento} $\tilde{\psi}\left(p\right)\in L^2(\bb{R},dp)$ (sono entrambi spazi della stessa cardinalità, e di conseguenza sono isomorfi), con l'isomorfismo che è definito dalle \textbf{trasformate di Fourier} in $L^2$.\\

Perché tutta l'informazione fisica sullo stato è contenuta sia in $\psi(x)$ che in $\tilde{\psi}(p)$ che in $\left\{a_n\right\}$ evidentemente l'informazione sullo stato non è contenuta in uno spazio di Hilbert \q{concreto} come $L^2(\bb{R},dx)$ o $L^2(\bb{R},dp)$ o $l_2$, ma è contenuto in uno spazio di Hilbert \q{astratto}\footnote{La distinzione tra i due può essere ben compresa facendo riferimento alla nozione di \textit{vettore} in un normale spazio finito-dimensionale, come per esempio $\bb{R}^3$. Potremmo definire un vettore come una \textit{terna di numeri} (coordinate rispetto ad una base), oppure come le classi di equivalenza della relazione di equipollenza, che associa tra loro tutte le \q{coppie di punti} di tipo $AB$ (in uno spazio affine) che \q{hanno la stessa distanza} e puntano nella stessa direzione (che è quella che va da $A$ a $B$. Chiaramente in questo secondo modo stiamo definendo i vettori \textit{indipendentemente} da una base, come \q{entità matematica astratta} che semplicemente soddisfa determinate proprietà. Nell'introdurre la notazione dei \q{ket} Dirac compie un processo simile per la \MQ.}\marginpar{Spazio di Hilbert \q{astratto}}, cioè \textbf{definito a meno di isomorfismi}, in cui sono definite le rappresentazioni delle osservabili del sistema quantistico (questa era la visione di \textbf{Dirac}).\\
In altre parole, un qualsiasi spazio di Hilbert \q{concreto} ha delle informazioni ridondanti, che non ci servono per la rappresentazione fisica. Ma come facciamo a distinguerle?\\

Con una notazione dovuta a Dirac, un \textbf{vettore di stato} dello spazio di Hilbert \q{\textbf{astratto}} è denotato con il simbolo $|\psi\rangle$  e chiamato \q{\textbf{Ket}}\marginpar{Ket}.\\ 
\textbf{Warning}: Talora la distinzione tra \q{astratto} e \q{concreto} può non essere rispettata (per esempio per ragioni di brevità), e viene data per scontata.\\
\marginpar{Rappresentazione $\vec{x}$ e $\vec{p}$}
Quando consideriamo spazi concreti $L^2(\mathbb{R}^3,d^3x)$, o $L^2(\bb{R}^3,d^3p)$, diremo che i vettori (di stato) e le osservabili che agiscono su di essi sono considerati \q{in \textbf{rappresentazione}}, rispettivamente, $\vec{x}$ e $\vec{p}$. (Troveremo poi un metodo per \q{costruire} questi spazi di Hilbert concreti)\\
Nella notazione di Dirac, $|\psi\rangle$ in rappresentazione $\vec{x}$ sarà denotato come:
\[
\left\langle\vec{x}\middle|\psi\right\rangle=\psi \left(\vec{x}\right)\in L^2(\mathbb{R}^3,d^3x)
\]
In rappresentazione $\vec{p}$ invece si avrà:
\[
\left\langle\vec{p}\middle|\psi\right\rangle=\tilde{\psi}\left(\vec{p}\right)\in L^2(\mathbb{R}^3,d^3p)
\]

\subsection{Funzionali lineari e spazio duale}
I \textbf{funzionali lineari continui} (se e solo se limitati in quanto sono definiti su uno spazio di Hilbert\footnote{CFR pag. 18 \cite{spazi_hilbert}}) $F$ su $\hs$ che associano a un funzionale un numero complesso $\psi\in\mathcal{H}\mapsto F\left(\psi\right)\in \bb{C}$ e soddisfano (limitatezza):
\begin{equation}
    \norm{F} = \sup_{\psi \in \hs\setminus \{0\}} \frac{|F(\psi)|}{\norm{\psi}} < \infty
    \label{eqn:norma-funz}
\end{equation}
costituiscono uno spazio\marginpar{Spazio duale $\hs^*$} di Hilbert $\hs^*$, detto duale di $\hs$, e (anti-)isomorfo ad $\hs$ con mappa di isomorfismo data da:
\begin{equation}
F\in \mathcal{H}^*\rightarrow \psi_F\in \hs, \quad F(\phi) = (\psi_F, \phi)\> \forall \phi \in \hs, \quad \norm{F}_{\hs^*}=\norm{\psi_F}_\hs
\label{eqn:riesz-funz}
\end{equation}
(tale corrispondenza è garantita dal teorema di Riesz\footnote{CFR pag. 19 \cite{spazi_hilbert}}).\\
In altre parole, applicare un funzionale $F \in \hs^*$ a un vettore $\phi$ in $\hs$ equivale a moltiplicare scalarmente il vettore $\psi_F$ \q{associato a} $F$ tramite la relazione di dualità per $\phi$ stesso.\\
Combinando le definizioni (\ref{eqn:norma-funz}) e (\ref{eqn:riesz-funz}), otteniamo una nuova espressione per la norma del funzionale $F$: 
\begin{equation}
\norm{F}_{\hs^*} = \norm{\psi_F}_\hs = \sup_{\phi\in\hs \setminus \{0\}} \frac{|(\psi_F,\phi)|}{\norm{\phi}}
\label{eqn:normafunz}
\end{equation}

In notazione di Dirac i funzionali lineari continui su $\hs$ astratto sono detti \q{\textbf{Bra}} e denotati come $\langle\psi|$\marginpar{Bra}.
Applicando un \q{bra} $\langle \psi |\in \mathcal{H}^*$ su un \q{ket} $\left|\phi\right\rangle\in \hs$:
\[
\left\langle\psi\right|:\left|\phi\right\rangle\rightarrow \left\langle\psi\right|\phi \rangle \in \bb{C}
\]
otteniamo una \q{\textbf{braket}} $\braket{\psi|\phi}$ (parentesi).
Come vedremo, anche la notazione $\psi \left(\vec{x}\right)=\langle \vec{x}|\psi \rangle$  è \q{quasi} consistente.\\

\subsection{Osservazioni sullo spazio $\mathcal{S}$ degli stati puri in \MQ}
\begin{enumerate}
    \item Tutte le operazioni \textbf{lineari}\marginpar{Gli stati sono definiti a meno di una fase} hanno senso e vanno eseguite in $\hs$ e poi eventualmente proiettate nell'\textbf{insieme dei raggi vettori} in 
    \[ \mathcal{S}=\frac{\mathcal{H}\setminus\left\{0\right\}}{\mathbb{C}\setminus\left\{0\right\}} \]
    In quanto ogni stato puro è rappresentabile come un elemento di $\hs$ \textit{a meno di un complesso non nullo} (come avevamo visto alla sezione \ref{sec:stati_puri}).\\
    \textbf{Nota}: tale proiezione va fatta \textit{per ultima}. Ad esempio, se uno stato è descritto da un $|\phi \rangle$  normalizzato a meno di una fase e $|\psi \rangle$ similmente (con $\ket{\psi}$ e $\ket{\phi}$ non paralleli), $\frac{1}{\sqrt2}(|\psi \rangle +|\phi \rangle )$ non descrive uno stato \textit{a meno di} \textbf{due} \textit{fasi}, ma sempre a meno di una! Questo perché la fase relativa tra $|\psi \rangle$  e $|\phi \rangle$ è fisicamente osservabile. Matematicamente, infatti, i seguenti due stati \textit{non} sono equivalenti:
	\[
	\frac{1}{\sqrt2}\left(|\phi\right\rangle+|\psi \rangle )\not\equiv \frac{1}{\sqrt2}(\left|\phi\right\rangle+e^{i\gamma}|\psi \rangle )
	\]
	Anche se (intendendo sempre l'uguaglianza come equivalenza \textit{di stati}):
	\[ \frac{1}{\sqrt2}\left(|\phi\right\rangle+|\psi \rangle )\equiv e^{i\alpha}\frac{1}{\sqrt2}\left(|\phi\right\rangle+|\psi \rangle )
	\]
	dove è importante notare che le uguaglianze o disuguaglianze si intendono come uguaglianze o disuguaglianze \textit{tra stati}.\\
	Insomma, l'idea è che \textit{il vettore di stato finale} a cui si giunge al termine dei conti è definito a meno di una fase, esattamente come tutti i vettori di stato iniziali, se normalizzati.
\item In $\mathcal{S}$ è naturalmente definita una \textbf{distanza} (o metrica)\marginpar{Distanza su $\mathcal{S}$}:
\[
d(\ket{\psi},\ket{\phi}) = \left [ 1-\frac{|\braket{\psi|\phi}|^2}{\norm{\psi}^2 \norm{\phi}^2}\right]^{\frac{1}{2}}
\]
Si noti che vale $d(\alpha\ket{\psi}, \ket{\phi}) = d(\ket{\psi},\ket{\phi})$ come dev'essere.
\item $\hs$ e $\mathcal{S}$ sono comunque \textbf{spazi differenti}, e in quanto tali hanno differenti proprietà topologiche. Per esempio, benché $\hs$ sia contraibile a un punto, $\mathcal{S}=\frac{\mathcal{H}\setminus\left\{0\right\}}{\mathbb{C}\setminus\left\{0\right\}}$ non lo è (in gergo matematico: $\mathcal S$ è uno spazio topologicamente non banale).\\
	Ad esempio esistono curve chiuse in $\mathcal{S}$ non riducibili a un punto con continuità (essendo uno spazio proiettivo, intuitivamente può succedere che la \q{proiezione} di una curva aperta sia chiusa)\\
	Consideriamo una curva chiusa $C$ che descrive un moto ciclico (uno stato che ritorna dopo un po' su se stesso), e in questo ciclo si può dimostrare che lo stato acquista una fase che dipende solo dalla struttura topologica\footnote{Queste proprietà che sembrano puramente matematiche e astratte hanno in realtà applicazione in fisica - e spiegano alcune proprietà di materiali. In effetti il premio Nobel per la fisica del 2016 fu assegnato proprio per la ricerca sugli effetti fisici di proprietà topologiche.} di $\mathcal{S}$ e $C$ ed è la cosiddetta \textbf{fase di Berry} $e^{\oint \braket{\psi|d\psi}}$. 
\end{enumerate}
	
\section{Osservabili}
Occupiamoci ora di analizzare la descrizione matematica delle osservabili.\\
Abbiamo già visto che nella descrizione di Heisenberg le osservabili sono ottenute da matrici infinito-dimensionali ${\vec{q}}_{mn}$, ${\vec{p}}_{mn}$, mentre per Schrödinger le osservabili sono operatori $\vec{x}$, $-i\hbar \vec{\nabla}$ che agiscono sulle funzioni d'onda $\psi$ e i loro valor medi sono dati da:
\[
\langle X\rangle_\psi = \frac{(\psi, \vec{x}\psi)}{\norm{\psi}^2}, \quad \langle P\rangle_\psi = \frac{(\psi, -i\hbar \vec{\nabla}\psi)}{\norm{\psi}^2}
\]

\subsection{Prime proprietà: linearità e densità del dominio}
Per le osservabili dunque richiediamo le seguenti proprietà:
\begin{enumerate}
    \item \textbf{Linearità}: Le osservabili sono descritte da \emph{operatori lineari}\marginpar{Osservabili = operatori lineari} su $\hs$ (spazio dei vettori di stato).
    Oltre al fatto che gli operatori comuni (posizione e momento) sono evidentemente lineari, questa proprietà garantisce che i valori medi dipendano dallo stato e non dal vettore scelto per rappresentarlo. Infatti se l'osservabile $O$ è descritta dall'operatore $A$, con $A:\hs \to \hs$, dato $\alpha \in \bb{C}\setminus \{0\}$ si ha: \marginpar{Valor medio e osservabile}
    \[
    \avg{O}_{\alpha \psi} = \avg{A}_{\alpha \psi}  \ = \ \frac{(\alpha\psi,A(\alpha\psi))}{\norm{\alpha\psi}^2} \ \overset{(a)}{=} \ \frac{\alpha^2}{\alpha^2} \frac{(\psi,A\psi)}{\norm{\psi}^2} \ = \ \avg{A}_\psi = \avg{O}_\psi
    \]
    dove il passaggio $(a)$ è valido se $A$ è lineare, ovvero se $A(\alpha \psi) = \alpha A\psi$. 
    Pertanto il \textbf{valor medio} dell'\textbf{osservabile} nello \textbf{stato} $\psi$ non cambia se applichiamo $A$ ad una funzione d'onda $\psi$ moltiplicata per un $\alpha \in \bb{C}$, dato che $\alpha\psi$ descrive lo stesso \textbf{stato} di $\psi$, esattamente come vogliamo.
	\item \textbf{Densità del dominio}: \marginpar{Densità del dominio} se la dimensione di $\hs$ è finita ($\dim{\mathcal{H}<\infty}$) allora il \textbf{dominio} di un operatore \textit{lineare} $A$ è certamente $\hs$, ossia $A$ è applicabile senza problemi a tutti i vettori di $\hs$.\\
	Questo fatto non vale tuttavia nel caso infinito-dimensionale, cioè per $\dim{\mathcal{H}=+\infty}$, come si vede dal seguente esempio:
	\begin{es} Sia $X$ l'operatore \q{posizione} $X\equiv \hat{x}$, che agisce su $\psi$ come: $X\psi \left(x\right)=x\psi \left(x\right)$.
	In generale, se $\psi \left(x\right)\in L^2$ non è vero che $D\left(X\right)=L^2(\bb{R}, dx)$.
	Possiamo infatti trovare una funzione di $L^2$ a cui, applicando l'operatore, si ottiene un qualcosa di non definito. Per esempio consideriamo $\psi \left(x\right)=\frac{1-e^{-x^2}}{x}\in L^2$. Allora per $x\sim 0$, $\psi \left(x\right)\sim x$, e per $x\sim \infty$  $\psi \left(x\right)\sim \frac{1}{x}$, in modo che $\int_\bb{R} dx \frac{1}{x^2}<\infty$.
	Ma se applichiamo l'operatore otteniamo $x\psi \left(x\right)=1-e^{-x^2}$, $x\sim \infty$, e $\psi \left(x\right)\sim 1$, per cui $\int_\bb{R} dx 1=\infty$ (l'integrale diverge).
	\end{es}
	Perciò non si può sperare di definire gli operatori su tutti i vettori di $\hs$.\\
	Chiediamo\marginpar{Dominio denso} allora la cosa più vicina a $\hs$ che possiamo chiedere: che $D\left(A\right)$ sia \textbf{denso} in $\hs$, cioè:
	\[
	\forall \psi \in \hs,\> \exists \{\psi_n\} \subset D(A) \text{ t.c. } \norm{\psi_n -\psi} \xrightarrow[n\to\infty]{} 0
	\]
	Il significato fisico della densità è la possibilità di \q{approssimare qualsiasi $\psi$  bene quanto voglio restando in $D(A)$}.\\
\end{enumerate}
Partendo da un operatore $A$ lineare e con dominio $D(A)$ denso in $\hs$ possiamo estenderlo con unicità a un operatore definito su tutto $\hs$ solo se $A$ è \textit{continuo}, cosa che negli spazi di Hilbert è equivalente a essere \textbf{limitato}, ossia a verificare la proprietà\footnote{Pag. 39 di \cite{spazi_hilbert}}:\marginpar{Operatori lineari limitati}
	\[
	\norm{A} \equiv \sup_{\psi \in \hs \setminus \{0\}} \frac{\norm{A\psi}}{\norm{\psi}} < \infty
	\]
	Lo spazio degli operatori (lineari\footnote{In seguito, quando scriveremo \q{operatori} faremo riferimento implicitamente agli operatori lineari.}) limitati su $\hs$ è uno spazio vettoriale normato con la norma appena definita, ed è denotato con $\mathcal{B}\left(\mathcal{H}\right)$ (\textit{bounded})\marginpar{Spazio $\mathcal{B}(\hs)$ degli operatori limitati}.\\
	In particolare se $A$ descrive un'osservabile, $\norm{A}$ ha significato fisico: è il più grande valore in modulo che potete ottenere misurando $A$ (o meglio, l'osservabile $O$ descritto da $A$).\\
	Notiamo ora che il dominio degli operatori è strettamente legato al significato fisico delle informazioni che possiamo ricavare da essi.
	Infatti, se $\psi \in D(A)$, allora il valor medio $\langle A\rangle_\psi$ è ben definito, e in particolare $A\psi\in \hs$, $(\psi, A\psi)$ sono ben definiti, per cui: 
	\[
	\langle A \rangle_\psi = \frac{(\psi,A\psi)}{\norm{\psi}^2}
	\]
	è effettivamente il valor medio (che otterremmo da misure ripetute sperimentali) dell'osservabile descritto da $A$\marginpar{Operatori al di fuori del loro dominio}.\\
	Se invece $\psi \notin D(A)$, in generale $\langle A\rangle_\psi = \frac{(\psi, A\psi)}{\norm{\psi}^2}$ \textbf{non} è il valor medio.\\
	\begin{es}[Operatore energia nella buca $\infty$-profonda]
	Consideriamo l'operatore energia: 
	\[H=\frac{p^2}{2m}=-\frac{\hbar^2}{2m}\frac{d^2}{dx^2}\]
	e applichiamolo al caso della buca 1-dimensionale infinitamente profonda\footnote{Che può essere immaginata come una \q{scatola} con pareti impenetrabili, dove il potenziale è $+\infty$} tra $0$ e $1$.
	Allora $D\left(H\right)=\left\{\psi\text{\ regolare,\ e\ poi\ } \psi\left(0\right)=\psi\left(1\right)=0\right\}$.\footnote{L'ipotesi di regolarità è necessaria poiché applicare l'operatore $H$ significa calcolare una derivata seconda, e perciò le $\psi$ in $D(H)$ devono essere derivabili \textit{quasi ovunque} due volte. In generale rimarremo sul vago nella definizione precisa di regolarità, che richiederà essenzialmente che le operazioni che eseguiamo siano ben definite.}
	Prendiamo una $\psi=1$ in $[0,1]$, che chiaramente non appartiene a $D(H)$. 
	Allora:
	$\left(\psi H,\psi\right)=0$ (la derivata seconda di $1$ è nulla).
	Ma le energie della buca infinitamente profonda sono: $E_n\sim n^2, n>0$ e la media di numeri positivi non può essere un numero nullo, e quindi il risultato che abbiamo ottenuto applicando l'operatore ad una funzione non nel suo dominio porta ad un risultato assurdo!
	\end{es}

\subsection{Definizione mediante gli elementi di matrice}
\textbf{Se} $D(A)$ è \textbf{denso} in $\hs$ (cosa che abbiamo assunto per tutte le osservabili), per determinare $A$ possiamo dare\marginpar{Elementi di matrice di un operatore con dominio denso} gli \q{elementi di matrice di $A$}, ossia 
\[
\left\{\left(\phi,\ A\psi\right),\ \psi\in D\left(A\right),\ \phi\in D\text{\ denso\ in\ }\mathcal{H}\right\}
\]
Mostriamo che tale identificazione è univoca. Siano $A$ e $A'$ due operatori, con lo stesso dominio $D(A) = D(A')$, e stessi elementi di matrice $(\phi, A\psi) = (\phi, A'\psi)$ $\forall \psi \in D(A)$, $\forall \phi \in D$ (con $D$ denso in $\hs$). Vogliamo che da ciò derivi che $A = A'$.\\
Scrivendo $(\phi, A\psi)-(\phi, A'\psi) = 0$, prendendo la norma e dividendo per $\norm{\phi}$ all'interno di un $\sup$ possiamo costruire l'espressione della norma di un operatore $A-A'$ (come data da Riesz in (\ref{eqn:normafunz}): 
\[ 0 = \sup_{\phi \in D\setminus \{0\}}\frac{|(\phi, A'\psi)-(\phi, A\psi)|}{\norm{\phi}} = \sup_{\phi\in D\setminus\{0\}} \frac{|(\phi, (A'-A)\psi)|}{\norm{\phi}}
\]
che per Riesz corrisponde al modulo: %Sistemare [TO DO]
\[
\norm{(A'-A)\psi} = 0 \quad \forall \psi \in D(A)
\] 
	Ma $D(A)$ è denso e $A$ lineare, quindi $A^\prime=A$.\\
	Notiamo che possiamo esprimere gli elementi di matrice di $A$ per $D=D(A)$ in termini di valori medi utilizzando l'\textbf{identità di polarizzazione}:\marginpar{Identità di polarizzazione: da prodotto scalare a valor medi}
	\[
	\left(\phi, A\psi\right)= \sum_{n=0}^{3}{\frac{\left(-i\right)^n}{4}(\phi+i^n\psi,\ A\left(\phi+i^n\psi\right))}
	\]
	In questo modo, riscrivendo $\phi + i^n\psi = \varphi_n \in D(A)$, abbiamo riscritto il prodotto scalare $(\phi, A\psi)$ come somma di norme del tipo $(\phi, A\phi_n)$ che sono, a meno di un fattore di normalizzazione, interpretabili fisicamente come valori medi (vi è quindi un modo per \textit{misurare} il prodotto scalare tra due vettori).
	\begin{expl}
	Verifichiamo l'identità di polarizzazione. Per prima cosa svolgiamo la sommatoria, tralasciando il fattore $1/4$ per adesso (lo aggiungeremo di nuovo alla fine):
	\begin{align*}
	    (\phi, A\psi) &= (\phi+\psi, A(\phi +\psi))-i(\phi+i\psi, A(\phi+i\psi))+\\
	    & -(\phi-\psi, A(\phi-\psi)) +i(\phi-i\psi,A(\phi-i\psi))
	\end{align*}
	Svolgendo tutti i prodotti arriviamo a:
	\begin{align*}
	    &= \hlc{Yellow}{(\phi,A\phi)} + (\phi,A\psi) + (\psi,A\phi) + (\psi,A\psi)\\
	    &- \hlc{SkyBlue}{i(\phi,A\phi)} -i(\phi, iA\psi) -i(i\psi, -iA\phi) -i(i\psi, iA\psi)\\
	    &- \hlc{Yellow}{(\phi,A\phi)}  -(\phi, -A\psi) -(-\psi, A\phi) -(-\psi, -A\psi)\\
	    &+ \hlc{SkyBlue}{i(\phi, A\phi)} + i(\phi, -iA\psi) +i(-i\psi, A\phi) + i(-i\psi, -iA\psi)
	\end{align*}
	dove i termini evidenziati si elidono.\\
	Osserviamo ora, che per proprietà del prodotto scalare, se $a\in \bb{C}$, $a(A,B) = (A,aB)$, $(aA,B) = (A, a^* B)$ (dove l'asterisco indica il complesso coniugato).\\
	Così facendo possiamo effettuare alcune semplificazioni. Per esempio
	\[ i(-i\psi, iA\phi) = i(\psi,-A\phi) = -i(\psi,A\phi) \]
	e analogamente per gli altri termini.\\
	Giungiamo a:
	\begin{align*}%[TO DO] I conti son giusti?
	    &= (\phi,A\psi)+(\psi, A\phi)+(\psi, A\psi)\\
	    &+ (\phi, A\psi) -(\psi,A\phi) -i(\psi,A\psi)\\
	    &+ (\phi, A\psi) +(\psi,A\phi) -(\psi, A\psi)\\
	    &+ (\phi, A\psi) + (\psi, -A\phi) +i(\psi,A\psi)
	\end{align*}
	Notiamo che tutti i termini tranne quelli della prima colonna si elidono, e il risultato è $4(\phi, A\psi)$. Riportando il fattore $1/4$ prima trascurato si giunge al prodotto scalare $(\phi, A\psi)$, e pertanto l'identità è dimostrata.
	\end{expl}

\subsection{Valori medi reali: aggiunto e operatori simmetrici}
Un'altra proprietà degli operatori\marginpar{Valor medi reali} che rappresentano osservabili viene dalla richiesta che \textbf{i valor medi siano reali}. Matematicamente ciò significa che il complesso coniugato del valor medio sia il valor medio stesso\footnote{Essendo il prodotto scalare $(\psi, \phi) = \int \psi^* \phi$, si ha $(\psi, A\psi)^* = \left (\int \psi^* A\psi\right ) = \int \psi (A\psi)^* = (A\psi, \psi)$}:
	\[
	\left(\psi,A\psi\right)=\left(\psi,A\psi\right)^*=\left(A\psi,\psi\right)
	\]
	Specifichiamo questa scrittura con la definizione di aggiunto, la quale sarà necessaria per richiedere la proprietà appena vista.
	\begin{dfn}[Aggiunto di un operatore]
	Sia $A$ con $D(A)$ denso, allora l'\textbf{aggiunto} di $A$, denotato con $A^\dag$, è un operatore definito sul dominio:\marginpar{Aggiunto di un operatore}
	\[
	D(A^\dag) = \left \{\phi \in \hs \text{ t.c.} \sup_{\psi \in D(A)} \frac{|(\phi, A\psi)|}{\norm{\psi}}< \infty \right \}
	\]
	e tale che
	\[
	\left(A\psi,\phi\right)=\left(\psi,A^\dag\phi\right)\> \quad \forall \psi \in D\left(A\right), \phi \in D\left(A^\dag \right)
	\]
	Si noti che per $A^\dag$, l'insieme denso $D(A)$ svolge il ruolo di $D$ (denso in $\hs$) nella definizione degli elementi di matrice di $A^\dag$, che perciò sono ben definiti.
	\end{dfn}
	%[TO DO] Sistemare quest'ultima parte
	\begin{appr}
	Giustifichiamo la scelta del dominio di $A^\dag$. Si ha che $\forall \psi \in D(A^\dag)$ $A^\dag \phi \in \hs$. Ciò significa che:
	\[
	\infty > \norm{A^\dag \phi} \underset{(a)}{=} \sup_{\psi \in \hs\setminus \{0\}} \frac{|(A^\dag,\psi)|}{\norm{\psi}} \underset{(b)}{=} \sup_{\psi \in \bm{D(A)}} \frac{|(A^\dag\phi, \psi)|}{\norm{\psi}} \underset{(c)}{=} \sup_{\psi \in D(A)} \frac{|(\phi,A\psi)|}{\norm{\psi}}
	\]
	dove in (a) si è applicata la definizione della norma (per teorema di Riesz), mentre (b) è giustificato dal fatto che $D(A)$ è denso in $\hs$, e (c) deriva dalla definizione di operatore aggiunto $A^\dag$.
	\end{appr}
	Nota: se $\dim{\mathcal{H}<\infty}$ allora la matrice di $A^\dag$ (cioè la matrice composta dai suoi elementi di matrice) è la matrice trasposta e complesso-coniugata della matrice $A$:
	\[
	\left(A\psi,\phi\right)=\sum_{i,j=1}^{N}{\left(A_{ji}\psi_i\right)^*\phi_j=\left(\psi,A^\dag\phi\right)=\sum_{i,j=1}^{N}{\psi_i^* A_{ij}^*\phi_j} \quad \Rightarrow \quad A_{ij}^\dag=A_{ji}^*}
	\]
	Se allora consideriamo un'osservabile $O$, descritta da $A$ operatore lineare, con dominio $D(A)$ denso, affinché i valori medi siano reali $\left\langle A\right\rangle_\psi\in\bb{R}$, dobbiamo poter definire $A^\dag$ aggiunto in $D(A^\dag)$, ossia:
\begin{equation}
\left(A\psi,\phi\right) = \left(\psi,A^\dag\phi\right) \qquad \forall \psi \in D\left(A\right) \, \ \ \forall \phi \in D\left(A^\dag\right)
\label{eqn:aggiunto}
\end{equation}
Inoltre, è necessario imporre che tra operatore $A$ e il suo aggiunto $A^\dag$ esista una relazione precisa, che definiamo successivamente.
\begin{dfn}[Estensione di un operatore]
Dato un operatore $A$, un operatore $B$ si dice un'estensione di $A$, e si indica con $A \subseteq B$, se $D(A) \subseteq D(B)$ e se vale
\[
A \psi = B \psi \qquad \forall \psi \in D(A)
\]
\end{dfn}
\begin{dfn}[Operatore simmetrico]
$A$ si dice \textbf{simmetrico}\marginpar{Operatore simmetrico} se $A \subseteq A^\dag$, ovvero se il suo aggiunto è una sua estensione.
\end{dfn}
Per verificare che $A$ è simmetrico partendo da (\ref{eqn:aggiunto}) è sufficiente verificare:
\[ 
\left(\phi,A\psi\right)=\left(A\phi,\psi\right)\quad \bm{\forall \phi, \psi \in D\left(A\right)}
\]
Da cui segue immediatamente che se $\phi \in D\left(A\right)$ allora $\phi \in D\left(A^\dag\right)$, e quindi $D(A)\subseteq D(A^\dag)$, come richiesto.

\begin{thm}
Se A è simmetrico allora per ogni $\psi \in D(A)$ si ha \marginpar{Operatori simmetrici hanno valori medi reali} $\left(\psi,A\psi\right)\in \bb{R}$. In particolare ciò significa che un operatore simmetrico produce valori medi \textbf{reali}.
\end{thm}
\begin{proof}
Sia $A$ simmetrico, e quindi $D(A) \subseteq D(A^\dag)$ e $A^\dag \psi = A\psi$ $\forall \psi \in D(A)$. 
\[
(\psi, A\psi) \underset{(a)}{=} (\psi,A^\dag \psi) \underset{(b)}{=} (A\psi, \psi) = (\psi,A\psi)^*
\]
Nel passo $(a)$ si ha che $\psi \in D(A)$ è anche $\psi \in D(A^\dag)$, essendo $D(A)\subseteq D(A^\dag)$, e ciò rende ben definita l'uguaglianza $A^\dag \psi = A\psi$ della simmetria, che qui applichiamo.\\
In $(b)$, invece, applichiamo la definizione di aggiunto. Ma allora, per come è definito il prodotto scalare, otteniamo il coniugato di $(\psi,A\psi)$.\\
Perciò $(\psi,A\psi)$ è uguale al suo coniugato, e quindi è reale.
\end{proof}

\lesson{7}{12/10/2018}
\subfile{AppuntiGiornalieri/Ottobre/12_10.tex}

\lesson{8}{15/10/2018}
\subfile{AppuntiGiornalieri/Ottobre/15_10.tex}

\lesson{9}{16/10/2018}
\subfile{AppuntiGiornalieri/Ottobre/16_10.tex}

\lesson{10}{17/10/2018}
\subfile{AppuntiGiornalieri/Ottobre/17_10.tex}

\lesson{11}{18/10/2018}
\subfile{AppuntiGiornalieri/Ottobre/18_10.tex}

\lesson{12}{19/10/2018}
\subfile{AppuntiGiornalieri/Ottobre/19_10.tex}

\lesson{13}{22/10/2018}
\subfile{AppuntiGiornalieri/Ottobre/22_10gruppi.tex}

\lesson{14}{25/10/2018}
\subfile{AppuntiGiornalieri/Ottobre/25_10.tex}

\lesson{15}{26/10/2018}
\subfile{AppuntiGiornalieri/Ottobre/26_10.tex}

\lesson{16}{5/11/2018}
\subfile{AppuntiGiornalieri/Novembre/05_11.tex}

\lesson{17}{7/11/2018}
\subfile{AppuntiGiornalieri/Novembre/07_11.tex}

\lesson{18}{8/11/2018}
\subfile{AppuntiGiornalieri/Novembre/08_11.tex}

\lesson{19}{9/11/2018}
\subfile{AppuntiGiornalieri/Novembre/09_11.tex}
\end{document}