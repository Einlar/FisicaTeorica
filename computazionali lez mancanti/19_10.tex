\begin{table}
\begin{adjustbox}{center}
\begin{tabular}{llllr}
\toprule
Nome & Ordine locale & Ordine globale & Simpletticità & Self-starting \\
\midrule
Eulero & 2 & 1 & No & Sì \\
Eulero-Cromer & 2 & 1 & Sì & Sì \\
Störmer-Verlet & 4 & 2 & Sì & No \\
Verlet-velocità & 3 & 2 & Sì & Sì \\
Leap-frog & 3 & 2 & Sì & No \\
Eulero-Richardson & 3 & 2 & No & Sì \\
\bottomrule
\end{tabular}
\end{adjustbox}
\caption{Tabella riassuntiva degli algoritmi per ODE.}
\end{table}

\subsection{Mid-point implicito}
% Lezione 5
\lesson{5}{19/10/2018}
Dalla derivata centrata:
\[
\deriv{x}{t} \bigg|_{t_n + \frac{1}{2}} = \frac{x_{n+1} - x_n}{\Delta t} + \mathcal O(\Delta t^2)
\]
\[
x_{n+1} = x_n + \Delta t \, f(x_{n + \frac{1}{2}}, t_{n + \frac{1}{2}}) + \mathcal O(\Delta t^3)
\]
da
\[
x_{n + \frac{1}{2}} \simeq \frac{x_n + x_{n+1}}{2}
\]
si ottiene il metodo di mid-point implicito.
\[
x_{n+1 = x_n + \Delta t f\left(\frac{x_n + x_{n+1}}{2}, \, t_{n + \frac{1}{2}}\right)}
\]
\[
y_1 = \frac{x_{n+1} + x_n}{2} \quad , \quad x_{n+1} = 2 y_1 - x_n
\]
\[
y_1 = x_n + \frac{\Delta t}{2} f(y_1, \tau_1) \qquad \tau_1 = t_{n + \frac{1}{2}} = t_n + \frac{\Delta t}{2}
\]
Equazione implicita per $y_1$
\[
x_{n+1} = x_n + \Delta t \, f(y_1, \tau_1)
\]
Quindi:
\[
\begin{cases}
y_1 = x_n + \frac{\Delta t}{2} \, f(y_1, \tau_1) \\
x_{n+1} = x_n + \Delta t \, f(y_1, \tau_1)
\end{cases}
\]
Dunque $y_1$ è un passaggio intermedio. Si ottiene l'algoritmo di Eulero-Richardson:
\[
\begin{cases}
y_1 = x_n \\
\tau_1 = t_n \\
y_2 = x_n + \frac{\Delta t}{2} \, f(y_1, \tau_1) \\
\tau_2 = t_n + \frac{\Delta t}{2} \\
x_{n+1} = x_n + \frac{\Delta t}{2} \, f(y_2, \tau_2) \\
\end{cases}
\]



\section{Metodi di Runge-Kutta}
Gli algoritmi (o metodi) di Runge-Kutta (RK) è una classe di algoritmi il cui ordine può essere scelto in base al numero di passi intermedi. Tale classe generalizza molti algoritmi, tra cui alcuni di quelli già visti: rientrano negli algoritmi RK l'algoritmo di Eulero e quello di Eulero-Richiardson impliciti ed espliciti. Esistono anche algoritmi RK simplettici.

\begin{definition}
Il \emph{metodo di Runge-Kutta a $s$ passi} si compone di $s$ passi intermedi per posizioni intermedie $y_i$, e $s$ passi intermedi per tempi intermedi $\tau_i$, dove $i = 1,\dots,s$. Data una matrice $s \times s$ di parametri $a_{ij}$, un vettore di $s$ parametri $b_i$, e un vettore di $s$ parametri $c_i$, il metodo consiste in tre tappe:
\begin{enumerate}
\item si calcolano $\tau_i$ mediante l'equazione esplicita
\[
\tau_i = t_n + c_i \Delta t
\]
\item si determinano $y_i$ mediante l'equazione implicita
\[
y_i = x_n + \Delta t \sum_{j = 1}^s a_{ij} f(y_i, \tau_i)
\]
\item si calcola $x_{n+1}$ mediante l'equazione esplicita
\[
x_{n+1} = x_n + \Delta t \sum_{i = 1}^s b_i f(y_i, \tau_i)
\]
\end{enumerate}
\end{definition}

Una terna di elementi ($a_{ij}, b_i, c_i$) definisce un metodo di Runge-Kutta. Ogni terna di questo tipo, dunque ogni metodo di Runge-Kutta può essere visualizzato più facilmente mediante un \emph{Tableau di Butcher}:

\begin{center}
\begin{tabular}{c|cccc}
$c_1$ & $a_{11}$ & $a_{12}$ & $\dots$ & $a_{1s}$ \\
$c_2$ & $a_{21}$ & $a_{22}$ & $\dots$ & $a_{2s}$ \\
$\vdots$ & $\vdots$  & $\vdots$ & $\ddots$ & $\vdots$ \\
$c_s$ & $a_{s1}$ & $a_{s2}$ & $\dots$ & $a_{ss}$ \\
\midrule
      & $b_1$ & $b_2$ & $\dots$ & $b_s$ \\
\end{tabular}
\end{center}

\begin{remark}
Si noti che un metodo di Runge-Kutta è esplicito se e sole se $a_{ij} = 0$ per $i \le j$, ovvero se il rispettivo tableau è della forma
\begin{center}
\begin{tabular}{c|cccc}
$c_1$ & $0$ & $0$ & $\dots$ & $0$ \\
$c_2$ & $a_{21}$ & $0$ & $\dots$ & $0$ \\
$\vdots$ & $\vdots$  & $\vdots$ & $\ddots$ & $\vdots$ \\
$c_s$ & $a_{s1}$ & $a_{s2}$ & $\dots$ & $0$ \\
\midrule
      & $b_1$ & $b_2$ & $\dots$ & $b_s$ \\
\end{tabular}
\end{center}
In quel caso infatti $y_1 = x_n$ per $i = 1$ mentre per i passi successivi
\[
y_i = x_n + \Delta t \sum_{j = 1}^{i-1} a_{ij} f(y_j, \tau_j)
\]
dunque ogni $y_i$ si ottiene da quella precedente.
\end{remark}


Vogliamo ora trovare la condizione per la quale un metodo RK abbia ordine $m$ voluto.

Si riproduce lo sviluppo di Taylor di $x_{n+1} = x(t_n + \Delta t)$ fino all'ordine $m$.
\[
x_{n+1} = x_n + \Delta t \, f_n + \frac{(\Delta t)^2}{2} \deriv{f}{t} \bigg|_{t_n} + \dots
\]
\[
\deriv{f}{t} \bigg|_{t_n} = \pderiv{f}{x} \bigg|_{t_n} \, f_n + \pderiv{f}{t} \bigg|_{t_n}
\]
A questo punto
\[
f(y_i, \tau_i) = f\left(x_n + \sum{j = 1}^s a_{ij} f(y_i, \tau_j) \Delta t , \ t_n + c_i \Delta t \right)
\]
\[
\simeq f\left(x_n + \sum_{i = 1}^s a_{ij} f_n \Delta t , \ t_n + c_i \Delta t \right)
\]
\[
\simeq f_n + \pderiv{f}{x} \bigg|_{t_n} \sum_{j = 1}^s a_{ij} f_n \, \Delta t + \pderiv{f}{t} \bigg|_{t_n} \, \Delta t
\]
\[
x_{n+1} = x_n + \sum_{i = 1}^n b_i f_n \Delta t + (\Delta t)^2 \sum_i b_i \set{\pderiv{f}{x} \sum_{j = 1}^s a_{ij} + \pderiv{f}{t} c_i}
\]
Sviluppo in serie per la regola di Runge-Kutta

La condizione per avere un metodo di ordine 1  è
\[
\sum_{i = 1}^s b_i = 1
\]
Le condizioni per avere un metodo di ordine 2 sono
\[
b_i c_i = \frac{1}{2} \qquad c_i = \sum_{j = 1}^s a_{ij}
\]
Per l'ordine 3 si ha
\[
\sum_{i = 1}^s b_i c_i^2 = \frac{1}{3} \qquad \sum_{i,j} b_i a_{ij} c_j = \frac{1}{6}
\]
In generale per ottenere un'algoritmo di ordine $m$ il numero di condizioni aumenta molto velocemente all'aumentare di $m$. Fortunatamente alcune regole sono simli a tutto gli ordini successivi, ad esempio una regola è sempre
\[
\sum_{i = 1} b_i c^{m-1}_i = \frac{1}{m}
\]

\begin{proposition}
Un metodo $\mathrm{RK}$ implicito è simplettico se per ogni $i,j$ vale
\[
a_i a_{ij} + b_j a_{ij} = b_i b_j
\]
\end{proposition}
\begin{proof}
Omessa
\end{proof}