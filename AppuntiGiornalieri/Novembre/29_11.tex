\documentclass[../../FisicaTeorica.tex]{subfiles}

\begin{document}

\section{Lezione ?:\\ \large{Esercizio momento angolare}}
\vspace{-1em}
\begin{center}
    \small{(29/11/2018)}
\end{center}

Eravamo arrivati a:
\[
\underbrace{\hs_{j_1} \otimes \hs_{j_2}}_{\ket{j_1, j_2, m_1, m_2}} = \underbrace{\bigoplus_{j=|j_1-j_2|}^{j_1+j_2}\hs_j}_{\ket{J, M}}
\]
Dove:
\[
\ket{J, M}=\ket{j_1+j_2, j_1+j_2}=\ket{j_1, j_2, j_1, j_2}
\]
Applicando $J_- = J_-^{(1)}+J_-^{(2)}$:
\begin{align*}
\sqrt{(j_1+j_2)(j_1+j_2+1)-(j_1+j_2)(j_1+j_2-1)}\ket{j_1+j_2, j_1+j_2-1} = \\
\sqrt{j_1(j_1+1)-j_1(j_1-1)}\ket{j_1, j_2, j_1-1, j_2}+
\sqrt{j_2(j_2+1)-j_2(j_2-1)}\ket{j_1, j_2, j_1, j_2 -1}
\end{align*}
Se calcoliamo $\ket{j_1+j_2, m}$, 
\[
\braket{j_1+j_2, j_1+j_2-1|j_1+j_2-1, j_1+j_2-1}=0
\]
con la convenzione:
\[
\braket{j_1, j_2, j_1, j-j_1|j,j}\geq 0
\]

\subsection{Problema}
Una particella con spin $1/2$ ha funzione d'onda la cui parte spaziale in $\bb{R}^3$ è data da:
\[
\psi(\vec{x})=xe^{-a|\vec{x}|}\quad a>0 \text{ cost.}
\]
\begin{enumerate}
\item Si determinino i possibili risultati e le relative probabilità di una misura di $L_3$
\item Si assuma che la componente di spin lungo $\hat{z}$ sia $\hbar/2$. Si calcolino le probabilità che una misura di $J_3 = L_3 + S_3$ dia come risultato $0$ o $3\hbar/2$
\item Si calcoli la probabilità che una misura congiunta di $\vec{J^2}$ e $J_3$ dia come risultato $(3\hbar/4, -\hbar/2)$
\item Supponendo che la misura del punto $3$ sia stata eseguita con i risultati indicati, una successiva misura di $S_3$ a $t=0$ dà il valore $-\hbar/2$. Se l'Hamiltoniana è:
\[
H=c\vec{L}\cdot \vec{S} \qquad c \neq 0\text{ cost.}
\]
si determini l'evoluto temporale dello stato a $t>0$.
\end{enumerate}

\subsection{Soluzione}
\begin{enumerate}
\item Scriviamo la funzione d'onda in coordinate polari, sostituendo $x=r\sin\theta\cos\varphi$. Ricordando la forma delle armoniche sferiche:
\[
Y^m_l(\theta,\varphi)=\sqrt{\frac{2l+1}{4\pi}}\sqrt{\frac{(l+|m|)!}{(l-|m|)!}}\frac{1}{2^l l!} \frac{1}{\sin\theta^{|m|}}\frac{\partial^{l-|m|}}{\partial \cos\theta^{l-|m|}}(\sin\theta)^{2l} e^{im\varphi}(\op{sgn} m)^{m}(-1)^l
\]
Da:
\[
x=r\sin\theta\cos\varphi= r\sin\theta \frac{e^{i\varphi}-e^{-i\varphi}}{2}
\]
Basta srivere le armoniche con $|m|=1$ e $l=1$:
\begin{align*}
Y^1_1 &= \sqrt{\frac{3}{4\pi}}\sqrt{2!}\frac{1}{2}\sin\theta e^{i\varphi}(-1)=-\sqrt{\frac{3}{8\pi}}\sin\theta e^{i\varphi}\\
Y^{-1}_1 &= \sqrt{\frac{3}{8\pi}} \sin\theta e^{i\varphi}
\end{align*}
Perciò possiamo riscrivere la funzione d'onda in termini di armoniche sferiche:
\[
\psi(\vec{x})=x e^{-a|\vec{x}|}=-r e^{-ar}(Y_1^1 - Y^{-1}_1)
\]
Poiché $m=\pm 1$, i possibili valori ottenuti da una misura di $L_3$ sono $\pm\hbar$. Il fatto che abbiano coefficienti uguali significa che i due sono equiprobabili, e quindi:
\[
W_\psi^{L_3}(\pm \hbar) = \frac{1}{2}
\]
\item $\displaystyle \underbrace{\hs_{l}\otimes \hs_s}_{\ket{l,s,m,s_3}} = \underbrace{\bigoplus_{J=|l-s|}^{l+s}\hs_J}_{\ket{J, M}}$
Riscriviamo allora:
\[
\ket{\psi}=\frac{\ket{1,\frac{1}{2}, 1, \frac{1}{2}}-\ket{1,\frac{1}{2}, -1,\frac{1}{2}}}{\sqrt{2}}
\] 
Sappiamo che $M=J, J-1, \dots, -J$. $J$ può assumere valori da $l-s=\frac{1}{2}$ a $l+s=\frac{3}{2}$, ossia solo $\{\frac{1}{2}, \frac{3}{2}\}$. I valori di $M$ sono allora ottenuti da questi sottraendo un intero: notiamo che procedendo in tal modo non è mai possibile ottenere $0$. Perciò:
\[
W_\psi^{J_3}(0)=0
\]
D'altro canto, per la probabilità di ottenere $J_3=\frac{3}{2}$ dobbiamo calcolare:
\[
W_\psi^{J_3}(\frac{3}{2}\hbar) = |\braket{\frac{3}{2}, \frac{3}{2}|\psi}|^2
\]
Notiamo che $\frac{3}{2}$ è un massimo, e quindi l'unico modo di ottenerlo è scegliendo $l$ e $s$ massimi, ossia rispettivamente pari a $1$ e $1/2$. Scriviamo quindi:
\begin{align*}
|\braket{\frac{3}{2}, \frac{3}{2}|\psi}|^2 &=
|\braket{1,\frac{1}{2},1,\frac{1}{2}|\psi}|^2 =\\
&= \left|
\braket{1,\frac{1}{2},1,\frac{1}{2}|\frac{1}{\sqrt{2}}(\ket{1,\frac{1}{2},1,\frac{1}{2}}-\ket{1,\frac{1}{2},1,-\frac{1}{2}})}
 \right|^2 = \frac{1}{2}
\end{align*}
Dato che, scomponendo il braket in due prodotti, il primo è pari a $1/\sqrt{2}$ e il secondo si annulla (e prendendo il modulo quadro si ottiene il risultato).

\item Vogliamo ora calcolare:
\[
W_\psi^{\vec{J}^2, J_3}\left(\frac{3}{4}\hbar^2, -\frac{\hbar}{2}\right) =\left | \braket{\hlc{Yellow}{\underbrace{\frac{1}{2}}_{J}, \underbrace{-\frac{1}{2}}_{M}}|\psi}\right|^2
\]
\[
J(J+1)=\frac{3}{3}\Rightarrow J=\frac{1}{2}
\]
(Nei prossimi conti poniamo temporaneamente $\hbar=1$, e lo reintroduciamo alla fine)\\
Sappiamo che:
\[
\ket{\frac{3}{2}, \frac{3}{2}} =\ket{1, \frac{1}{2},1,\frac{1}{2}}
\]
\q{Abbassiamo} tale autovettore applicando $J_-$:
\[
\underbrace{J_- \ket{\frac{3}{2}, \frac{3}{2}}}_{=(J_-^{(1)}+J_-^{(2)})\ket{1,\frac{1}{2},1,\frac{1}{2}}} = \sqrt{\frac{3}{2}\left(\frac{3}{2}-1\right)-\frac{3}{2}\left(\frac{3}{2}-1 \right)}\ket{\frac{3}{2},\frac{1}{2}} = \hlc{SkyBlue}{\sqrt{3}\ket{\frac{3}{2},\frac{1}{2}}}
\]
Svolgendo allora i conti:
\begin{align*}
(J_-^{(1)} +J_-^{(2)})\ket{1,\frac{1}{2},1,\frac{1}{2}} &= \sqrt{1(1+1)-1(1-1)}\ket{1,\frac{1}{2},0,\frac{1}{2}} + \sqrt{\frac{1}{2}(\frac{1}{2}+1)-\frac{1}{2}(-\frac{1}{2})}\ket{1,\frac{1}{2},1,-\frac{1}{2}}=\\
&=\hlc{SkyBlue}{ \sqrt{2}\ket{1,\frac{1}{2},0,\frac{1}{2}}+\ket{1,\frac{1}{2},1, -\frac{1}{2}}}
\end{align*}
Tuttavia noi vogliamo \q{scendere} al termine evidenziato in giallo di sopra, ma ora siamo arrivati solo al termine azzurro.\\
La strategia è ora quella di utilizzare l'ortogonalità nel seguente modo:
\[
\ket{\frac{3}{2}, \frac{3}{2}} \xrightarrow{J_-} \ket{\frac{3}{2},\frac{1}{2}} \perp \ket{\frac{1}{2},\frac{1}{2}} \xrightarrow{J_-} \ket{\frac{1}{2},-\frac{1}{2}}
\]
Dall'uguaglianza dei due termini azzurri ricaviamo:
\[
\ket{\frac{3}{2},\frac{1}{2}}=\sqrt{\frac{2}{3}}\ket{1,\frac{1}{2},0,\frac{1}{2}} +\frac{1}{\sqrt{3}}\ket{1,\frac{1}{2},1,-\frac{1}{2}}
\]
Scriviamo allora il nostro obiettivo $\ket{\frac{1}{2},\frac{1}{2}}$ come combinazione lineare di questi autostati:
\[
\ket{\frac{1}{2},\frac{1}{2}}=c_1\ket{1,\frac{1}{2},0,\frac{1}{2}}+c_2\ket{1,\frac{1}{2},1,-\frac{1}{2}}
\]
E dall'ortogonalità:
\[
0=\braket{\frac{1}{2},\frac{1}{2}|\frac{3}{2},\frac{1}{2}} = c_1 \ket{1,\frac{1}{2},0,\frac{1}{2}} + c_2\ket{1,\frac{1}{2},1,-\frac{1}{2}}
\]

Applicando allora la condizione:
\[
\braket{j_1+j_2, j_1+j_2-1| j_1 + j_2-1, j_1+j_2-1} = 0
\]
Perciò (*):
\begin{align*}
0 = \braket{\frac{1}{2},\frac{1}{2}|\frac{3}{2},\frac{1}{2}} = c_1 \sqrt{\frac{2}{3}}+\frac{c_2}{\sqrt{3}}
\end{align*}
Per la convenzione:
\[
\braket{j_1, j_2, j_1, j-j_1 | j, j} \geq 0
\]
Si ha che:
\[
\braket{\frac{1}{2},\frac{1}{2}|1,\frac{1}{2},1,-\frac{1}{2}} > 0
\]
e quindi $c_2 > 0$. Ma allora due $c_1$, $c_2$ che soddisfano l'ortogonalità (*) e la convenzione di $c_2 > 0$ sono proprio:
\[
c_1 = -\sqrt{\frac{1}{3}}; \quad c_2 = \sqrt{\frac{2}{3}}
\]
Riepilogando, abbiamo ottenuto:
\begin{align*}
\ket{\frac{3}{2},\frac{3}{2}} &= \ket{1,\frac{1}{2},1,\frac{1}{2}}\\
\ket{\frac{3}{2},\frac{1}{2}} &= \sqrt{\frac{2}{3}}\ket{1,\frac{1}{2},0,\frac{1}{2}}+\frac{1}{\sqrt{3}}\ket{1,\frac{1}{2},1,-\frac{1}{2}}\\
\ket{\frac{1}{2},\frac{1}{2}} &= -\frac{1}{\sqrt{3}}\ket{1,\frac{1}{2},0,\frac{1}{2}}+\sqrt{\frac{2}{3}}\ket{1,\frac{1}{2},1,-\frac{1}{2}}
\end{align*}
Per ottenere $\ket{\frac{1}{2},-\frac{1}{2}}$ basta abbassare di $1$:
\[
J_- \ket{\frac{1}{2},\frac{1}{2}}=\sqrt{\frac{1}{2}(\frac{1}{2}+1)-\frac{1}{2}(\frac{1}{2}-1)} \ket{\frac{1}{2},-\frac{1}{2}}
\]
che è anche $=$, per le regole viste prima, all'applicazione di $(J_-^{(1)}+J_-^{(2)})$ al rispettivo autostato, che conduce ai conti:
\begin{align*}
&= (-\frac{1}{\sqrt{3}})\sqrt{1(1+1)-0}\ket{1,\frac{1}{2},-1,\frac{1}{2}} + (-\frac{1}{\sqrt{3}})\sqrt{\frac{1}{2}(\frac{1}{2}+1)-\frac{1}{2}(-\frac{1}{2})}\ket{1,\frac{1}{2},0,-\frac{1}{2}}+\\
&+ \sqrt{\frac{2}{3}}\sqrt{1(1+1)-1(1-1)}\ket{1,\frac{1}{2},0,-\frac{1}{2}} + 0
\end{align*}
(dato che l'ultimo autovettore è già il minimo, e non si può abbassare ulteriormente).\\
Svolgendo i calcoli:
\begin{align*}
&=
-\sqrt{\frac{2}{3}}\ket{1,\frac{1}{2},-1,\frac{1}{2}} -\frac{1}{\sqrt{3}}\ket{1,\frac{1}{2},0,-\frac{1}{2}} + \frac{2}{\sqrt{3}}\ket{1,\frac{1}{2},0,-\frac{1}{2}} =\\
&=\ket{\frac{1}{2},-\frac{1}{2}} = \frac{1}{\sqrt{3}}\ket{1,\frac{1}{2},0,-\frac{1}{2}}-\sqrt{\frac{2}{3}}\ket{1,\frac{1}{2},-1,\frac{1}{2}}
\end{align*}
Possiamo allora calcolare, finalmente, la probabilità cercata:
\begin{align*}
\left|\bra{\frac{1}{2},-\frac{1}{2}}\frac{1}{\sqrt{2}}(\ket{1,\frac{1}{2},-1,\frac{1}{2}}-\ket{1,\frac{1}{2},1,\frac{1}{2}})\right|^2 =
\left|\sqrt{\frac{2}{3}}\frac{1}{\sqrt{2}} \right|^2=\frac{1}{3}
\end{align*}
Da cui:
\[
W_\psi^{\vec{J}^2,J_3}(\hbar^2\frac{3}{4},-\frac{\hbar}{2})=\frac{1}{3}
\]
\item Dopo la misura di $\vec{J}^2$ e $J_3$ misuriamo $S_3$ e otteniamo $-\hbar/2$.\\ Ciò proietta lo stato $\ket{\frac{1}{2},-\frac{1}{2}}$
\[
\ket{\frac{1}{2},-\frac{1}{2}}=\underbrace{\frac{1}{\sqrt{3}}\ket{1,\frac{1}{2},0,-\frac{1}{2}}}_{s_3=-1/2}-\sqrt{\frac{2}{3}}\underbrace{\ket{1,\frac{1}{2},-1, \frac{1}{2}}}_{s_3 \neq -1/2}
\]
Dopo la misura di $S_3$ \textit{eliminiamo} la seconda componente (che non è compatibile con la misura effettuata), e quindi il nuovo stato sarà:
\[
\ket{1,\frac{1}{2},0,-\frac{1}{2}}=\ket{\psi(0^+)}
\]
Partiamo allora dall'espressione dell'Hamiltoniana:
\[
H=c\vec{L}\cdot \vec{S}
\]
Sappiamo che:
\[
\vec{J}=\vec{L}+\vec{S} = \vec{L}\otimes \bb{I} + \bb{I}\otimes \vec{S}
\]
e $[\vec{L},\vec{S}]=0$. Scriviamo:
\[
\vec{J}^2 = \vec{L}^2 + \vec{S}^2 + 2\vec{L}\cdot \vec{S} \Rightarrow  \vec{L}\cdot\vec{S} = \frac{1}{2}\left(\vec{J}^2 - \vec{L}^2 -\vec{S}^2 \right)
\]

Riscrivendo allora lo stato come:
\begin{align*}
\ket{1,\frac{1}{2},0,-\frac{1}{2}}=\sum_{J=\frac{1}{2}}^{\frac{3}{2}} \sum_{M=-J}^{J} \braket{J,M|1,\frac{1}{2},0,-\frac{1}{2}} \ket{J,M}
\end{align*}
Dato che $M=0+(-\frac{1}{2})=-\frac{1}{2}$ si ha che gli unici termini che contribuiscono alla sommatoria sono $\ket{\frac{3}{2},-\frac{1}{2}}$ e $\ket{\frac{1}{2},-\frac{1}{2}}$. Quest'ultimo l'abbiamo già ricavato:
\[
\ket{\frac{1}{2},-\frac{1}{2}}=-\sqrt{\frac{2}{3}}\ket{1,\frac{1}{2},-1,\frac{1}{2}} + \frac{1}{\sqrt{3}}\ket{1,\frac{1}{2},0,-\frac{1}{2}}
\]
E ci serve ricavare il primo. Ragionando come prima, possiamo per esempio abbassare $\ket{\frac{3}{2}, \frac{1}{2}}$ che già conosciamo:
\begin{align*}
J_- \ket{\frac{3}{2},\frac{1}{2}}&=\sqrt{\frac{3}{2}\left(\frac{3}{2}+1\right)-\frac{1}{2}\left(-\frac{1}{2}\right)}\ket{\frac{3}{2},-\frac{1}{2}} =\\
&=(L_- + S_-)\left(\sqrt{\frac{2}{3}}\ket{1,\frac{1}{2},0,\frac{1}{2}} +\frac{1}{\sqrt{3}}\ket{1,\frac{1}{2},1,-\frac{1}{2}}\right)=\\
&=\sqrt{\frac{2}{3}}\sqrt{1(1+1)+0}\ket{1,\frac{1}{2},-1,\frac{1}{2}}+\sqrt{\frac{2}{3}}\ket{\frac{1}{2}(\frac{1}{2}+1)-\frac{1}{2}(-\frac{1}{2})}\ket{1,\frac{1}{2},0,-\frac{1}{2}}+\\
&+\frac{1}{\sqrt{3}} \sqrt{1(1+1)-1(1-1)}\ket{1,\frac{1}{2},0,-\frac{1}{2}}=
\\
\ket{\frac{3}{2},-\frac{1}{2}} &= \frac{1}{\sqrt{3}}\ket{1,\frac{1}{2},-1,\frac{1}{2}}+\sqrt{\frac{2}{3}}\ket{1,\frac{1}{2},0,-\frac{1}{2}}
\end{align*}
E quindi:
\begin{align*}
\ket{1,\frac{1}{2},0,-\frac{1}{2}} &= \braket{\frac{3}{2},-\frac{1}{2}|1,\frac{1}{2},0,-\frac{1 }{2}}\ket{\frac{3}{2},-\frac{1}{2}}+ \braket{\frac{1}{2},-\frac{1}{2}|1,\frac{1}{2},0,-\frac{1}{2}}\ket{\frac{1}{2},-\frac{1}{2}}=\\
&=\sqrt{\frac{2}{3}}\ket{\frac{3}{2},-\frac{1}{2}} +\frac{1}{\sqrt{3}}\ket{\frac{1}{2},-\frac{1}{2}}
\end{align*}
Finalmente possiamo usare autostati di $H$:
\begin{align*}
H\ket{\frac{3}{2},-\frac{1}{2}}&=\frac{c}{2}(\vec{J}^2-\overbrace{\vec{L}^2}^{1}-\overbrace{\vec{S}^2}^{1/2})\ket{\frac{3}{2},-\frac{1}{2}}=\\
&= \frac{c}{2}\hbar^2 \left[\frac{3}{2}\left(\frac{3}{2}+1\right)-1(1+1)-\frac{1}{2}\left(\frac{1}{2}+1\right)\right]\ket{\frac{3}{2},-\frac{1}{2}}=\\
&=\frac{c}{2}\hbar^2\ket{\frac{3}{2},-\frac{1}{2}}
\end{align*}

\begin{align*}
H\ket{\frac{1}{2},-\frac{1}{2}}&=\frac{c}{2}\hbar^2\left[\frac{1}{2}\left(\frac{1}{2}+1\right)-1(1+1)-\frac{1}{2}\left(\frac{1}{2}+1\right)\right]\ket{1,-\frac{1}{2}}=\\
&=-c\hbar^2 \ket{\frac{1}{2},-\frac{1}{2}}
\end{align*}
E infine:
\begin{align*}
\ket{\psi(t)}=\exp\left(-i\frac{t}{\hbar}H\right) \ket{1,\frac{1}{2},0,-\frac{1}{2}} = \exp\left(-\frac{itH}{\hbar}\right) \left(\sqrt{\frac{2}{3}}\ket{\frac{3}{2},-\frac{1}{2}}+\frac{1}{\sqrt{3}}\ket{\frac{1}{2},-\frac{1}{2}}\right)
\end{align*}
Reintroducendo il termine radiale:
\begin{align*}
\ket{\psi(t)}=-r e^{-ar} \left[\exp\left(-i\frac{c}{2}\hbar t\right) \sqrt{\frac{2}{3}}\ket{\frac{3}{2},-\frac{1}{2}}+e^{ic\hbar t}\ket{\frac{1}{2},-\frac{1}{2}}\right]
\end{align*}
\end{enumerate}
\end{document}

