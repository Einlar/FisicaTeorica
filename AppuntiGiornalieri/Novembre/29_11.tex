\documentclass[../../FisicaTeorica.tex]{subfiles}

\begin{document}

\begin{comment}
Eravamo arrivati a:
\[
\underbrace{\hs_{j_1} \otimes \hs_{j_2}}_{\ket{j_1, j_2, m_1, m_2}} = \underbrace{\bigoplus_{j=|j_1-j_2|}^{j_1+j_2}\hs_j}_{\ket{J, M}}
\]
Dove:
\[
\ket{J, M}=\ket{j_1+j_2, j_1+j_2}=\ket{j_1, j_2, j_1, j_2}
\]
Applicando $J_- = J_-^{(1)}+J_-^{(2)}$:
\begin{align*}
\sqrt{(j_1+j_2)(j_1+j_2+1)-(j_1+j_2)(j_1+j_2-1)}\ket{j_1+j_2, j_1+j_2-1} = \\
\sqrt{j_1(j_1+1)-j_1(j_1-1)}\ket{j_1, j_2, j_1-1, j_2}+
\sqrt{j_2(j_2+1)-j_2(j_2-1)}\ket{j_1, j_2, j_1, j_2 -1}
\end{align*}
Se calcoliamo $\ket{j_1+j_2, m}$, 
\[
\braket{j_1+j_2, j_1+j_2-1|j_1+j_2-1, j_1+j_2-1}=0
\]
con la convenzione:
\[
\braket{j_1, j_2, j_1, j-j_1|j,j}\geq 0
\]
\end{comment}

\section{Esercizio \theEsercizio}\index{Esercizio!Composizione di momenti}
Una particella con spin $s=\frac{1}{2}$ è descritta, per la componente spaziale in $\bb{R}^3$, dalla funzione d'onda $\psi(\vec{r})$:
\[
\psi(\vec{r})=xe^{-a|\vec{r}|}\quad a>0 \text{ cost.}
\]
\begin{enumerate}
\item Si determinino i possibili risultati e le relative probabilità di una misura di $L_3$
\item Si assuma che la componente di spin lungo $\hat{z}$ sia $m_s=\hbar/2$. Si calcolino le probabilità che una misura di $J_3 = L_3 + S_3$ dia come risultato $0$ o $3\hbar/2$.
\item Si calcoli la probabilità che una misura congiunta di $\vec{J^2}$ e $J_3$ dia come risultato $(3\hbar/4, -\hbar/2)$.
\item Supponendo che la misura del punto $3$ sia stata eseguita con i risultati indicati, una successiva misura di $S_3$ a $t=0$ dà il valore $m_s=-\hbar/2$. Se l'Hamiltoniana è:
\[
H=c\vec{L}\cdot \vec{S} \qquad c \neq 0\text{ cost.}
\]
si determini l'evoluto temporale dello stato a $t>0$.
\end{enumerate}

\subsection{Soluzione}
\begin{enumerate}
\item Poiché abbiamo a che fare con gli operatori del momento angolare $\vec{L}^{\,2}$ e $L_3$, conviene riscrivere la $\psi(\vec{r})$ come combinazione lineare degli autoket comuni a $\vec{L}^{\,2}$ e $L_3$, ossia come combinazione delle \textit{armoniche sferiche}, che hanno la forma:
\[
Y^m_l(\theta,\varphi)=\sqrt{\frac{2l+1}{4\pi}}\sqrt{\frac{(l+|m|)!}{(l-|m|)!}}\frac{1}{2^l l!} \frac{1}{\sin\theta^{|m|}}\frac{\partial^{l-|m|}}{\partial \cos\theta^{l-|m|}}(\sin\theta)^{2l} e^{im\varphi}(\op{sgn} m)^{m}(-1)^l
\]
Partiamo passando in coordinate sferiche, notando che $e^{-a|\vec{r}|}$ ha già simmetria sferica, e infatti diviene $e^{-ar}$, mentre per il resto usiamo $x=r\sin\theta\cos\varphi$, giungendo a:
\begin{align*}
\psi(r, \theta,\varphi) = r\sin\theta \cos\varphi e^{-ar}
\end{align*}
Notiamo poi che possiamo riscrivere:
\[
x=r\sin\theta\cos\varphi= r\sin\theta \frac{e^{i\varphi}+e^{-i\varphi}}{2}
\]
come combinazione delle armoniche con $l=1$ e $m=\pm 1$:
\begin{align*}
Y^1_1 &= \sqrt{\frac{3}{4\pi}}\sqrt{2!}\frac{1}{2}\sin\theta e^{i\varphi}(-1)=-\sqrt{\frac{3}{8\pi}}\sin\theta e^{i\varphi}\\
Y^{-1}_1 &= \sqrt{\frac{3}{8\pi}} \sin\theta e^{-i\varphi}
\end{align*}
Perciò possiamo riscrivere la funzione d'onda in termini di armoniche sferiche:
\[
\psi(\vec{x})=x e^{-a|\vec{x}|}\propto-r e^{-ar}(Y_1^1 - Y^{-1}_1)
\]
Poiché $m=\pm 1$, i possibili valori ottenuti da una misura di $L_3$ sono $\pm\hbar$. Il fatto che abbiano coefficienti uguali significa che i due sono equiprobabili, e quindi:
\[
W_\psi^{L_3}(\pm \hbar) = \frac{1}{2}
\]
\item Per determinare le probabilità di una misura di $\vec{J}$ o $J_3$, è necessario spostarsi dalla base $\ket{l, s; m, m_s}$ a quella $\ket{J,M}$:
\begin{align*}
\displaystyle \underbrace{\hs_{l}\otimes \hs_s}_{\ket{l,s,m,m_s}} = \underbrace{\bigoplus_{J=|l-s|}^{l+s}\hs_J}_{\ket{J, M}}
\end{align*}
Partiamo riscrivendo lo stato $\ket{\psi}$ iniziale in notazione di Dirac, ricordando che dai dati del problema $l=1$, $s=\frac{1}{2}$, $m=\pm 1$, $m_s=\frac{1}{2}$. Trascuriamo, inoltre, la componente radiale, dato che non modifica in alcun modo le misure di $\vec{J}$ e $J_3$, e normalizziamo opportunamente lo stato così ottenuto\footnote{Matematicamente, stiamo utilizzando l'isomorfismo $L^2(\bb{R}^3, d^3x)\cong L^2(\bb{R}_+, r^2\,dr)\otimes L^2(S^2, d\Omega = \sin\theta d\theta d\varphi)$ per lavorare solo in $L^2(S^2)$, dove trascuriamo perciò la componente radiale.}:
\[
\ket{\psi}=\frac{\ket{1,\frac{1}{2}, 1, \frac{1}{2}}-\ket{1,\frac{1}{2}, -1,\frac{1}{2}}}{\sqrt{2}}
\]
Sappiamo che $M=J, J-1, \dots, -J$, e che $J$ può assumere valori da $l-s=\frac{1}{2}$ a $l+s=\frac{3}{2}$, ossia solo $\{\frac{1}{2}, \frac{3}{2}\}$. I valori di $M$ sono allora ottenuti da questi sottraendo un intero, e sono perciò $\left\{-\frac{3}{2},-\frac{1}{2},+\frac{1}{2},+\frac{3}{2} \right\}$. Notiamo che tra di essi non compare $0$, che perciò non potrà mai essere il risultato di una misura Perciò:
\[
W_\psi^{J_3}(0)=0
\]
D'altro canto, per la probabilità di ottenere $J_3=\frac{3}{2}$ dobbiamo calcolare:
\[
W_\psi^{J_3}\left(\frac{3}{2}\hbar\right) = \left|\braket{\frac{3}{2}, \frac{3}{2}|\psi}\right|^2
\]
Ricordando che $J_3=M=m+m_s$, se scriviamo le prime combinazioni:
\begin{align*}
\bm{M} && \bm{m} && \bm{m_s}\\
\frac{3}{2} && \hlc{Yellow}{1} && \hlc{Yellow}{\frac{1}{2}}\\
\frac{1}{2} && \hlc{ForestGreen}{0} && \hlc{ForestGreen}{\frac{1}{2}}\\
&& \hlc{ForestGreen}{\frac{1}{2}} && \hlc{ForestGreen}{0}\\
\vdots && \vdots && \vdots
\end{align*}
notiamo che $\frac{3}{2}$ è il \textit{massimo} valore di $J_3$, e quindi l'unico modo di ottenerlo è scegliendo $l$ e $s$ massimi, ossia rispettivamente pari a $1$ e $1/2$ (evidenziati in giallo). Scriviamo quindi:
\begin{align*}
W_\psi^{J_3}\left(\frac{3}{2}\hbar\right) &= \left|\braket{\frac{3}{2}, \frac{3}{2}|\psi}\right|^2 =
\left|\braket{1,\frac{1}{2};1,\frac{1}{2}|\psi}\right|^2 =\\
&= \left|
\braket{1,\frac{1}{2};1,\frac{1}{2}|\frac{1}{\sqrt{2}}\left(\ket{1,\frac{1}{2};1,\frac{1}{2}}-\ket{1,\frac{1}{2};1,-\frac{1}{2}}\right)}
 \right|^2 = \frac{1}{2}
\end{align*}
Nel svolgere i conti del braket abbiamo utilizzato l'ortonormalità della base $\ket{l,s;m,m_s}$.

\item Vogliamo ora calcolare:
\begin{align}
W_\psi^{\vec{J}^2, J_3}\left(\frac{3}{4}\hbar^2, -\frac{\hbar}{2}\right) =\left | \langle{\hlc{Yellow}{\underbrace{\frac{1}{2}}_{J}, \underbrace{-\frac{1}{2}}_{M}}|\psi\rangle}\right|^2
\label{eqn:probability-angular-3}
\end{align}
Dato che $\vec{J}^2$ ha autovalori della forma $\hbar^2 J(J+1)$, da cui abbiamo ricavato:
\[
\hbar^2J(J+1)=\frac{3}{4}\hbar^2\Rightarrow J=\frac{1}{2}
\]
Poiché $\ket{\psi}$ è scritta nella base $\ket{l,s;m,m_s}$, per svolgere il braket conviene riscrivere anche $\bra{\frac{1}{2},-\frac{1}{2}}$ in questa base. Possiamo farlo per \q{abbassamenti} successivi, partendo da:
\[
\ket{\frac{3}{2}, \frac{3}{2}} =\ket{1, \frac{1}{2};1,\frac{1}{2}}
\]
Applicando $J_-$ e utilizzando l'identità (\ref{eqn:abbasso-alto}), che è la forma già semplificata per il caso comune di abbassare un autoket \textit{massimo}:
\[
\underbrace{J_- \ket{\frac{3}{2}, \frac{3}{2}}}_{=(J_-^{(1)}+J_-^{(2)})\ket{1,\frac{1}{2},1,\frac{1}{2}}} = \sqrt{2\left(1+\frac{1}{2}\right)}\ket{\frac{3}{2},\frac{1}{2}} = \hlc{SkyBlue}{\sqrt{3}\ket{\frac{3}{2},\frac{1}{2}}}
\]
\begin{align*}
(J_-^{(1)} +J_-^{(2)})\ket{1,\frac{1}{2};1,\frac{1}{2}} &= \sqrt{2\cdot 1}\ket{1,\frac{1}{2};0,\frac{1}{2}} + \sqrt{2\cdot \frac{1}{2}}\ket{1,\frac{1}{2};1,-\frac{1}{2}}=\\
&=\hlc{SkyBlue}{ \sqrt{2}\ket{1,\frac{1}{2};0,\frac{1}{2}}+\ket{1,\frac{1}{2};1, -\frac{1}{2}}}
\end{align*}
Dall'uguaglianza dei due termini azzurri ricaviamo:
\begin{align}
\ket{\frac{3}{2},\frac{1}{2}}=\sqrt{\frac{2}{3}}\ket{1,\frac{1}{2};0,\frac{1}{2}} +\frac{1}{\sqrt{3}}\ket{1,\frac{1}{2};1,-\frac{1}{2}}
\label{eqn:tremezzi}
\end{align}
Vogliamo ora scendere ulteriormente, per arrivare a $\ket{\frac{1}{2},-\frac{1}{2}}$. Per farlo, dovremo anche abbassare l'autovalore di $J$, per cui usiamo l'ortogonalità:
\begin{align*}
\braket{j_1+j_2, j_1+j_2-1| j_1 + j_2-1, j_1+j_2-1} = 0 \Rightarrow \ket{\frac{3}{2},\frac{1}{2}} \perp \ket{\frac{1}{2},\frac{1}{2}}
\end{align*}
Notiamo che quest'ultimo autoket può essere combinazione solo di due $\ket{l,s; m, m_s}$, come si può notare dalle prime combinazioni di $M=m+m_s$ scritte sopra, di cui stavolta ci interessano solo i termini evidenziati in azzurro.

Scriviamo allora il nostro obiettivo $\ket{\frac{1}{2},\frac{1}{2}}$ come combinazione lineare di questi autostati:
\[
\ket{\frac{1}{2},\frac{1}{2}}=c_1\ket{1,\frac{1}{2},0,\frac{1}{2}}+c_2\ket{1,\frac{1}{2},1,-\frac{1}{2}}
\]
Calcolando allora il prodotto scalare con (\ref{eqn:tremezzi}) (sfruttando l'ortonormalità dei $\ket{l,s;m,m_s}$) e imponendo che il risultato sia nullo per l'ortogonalità:
\begin{align*}
0 \overset{!}{=} \braket{\frac{1}{2},\frac{1}{2}|\frac{3}{2},\frac{1}{2}}
= c_1 \sqrt{\frac{2}{3}}+\frac{c_2}{\sqrt{3}}
\end{align*}
E per la normalizzazione si deve avere $c_1^2 + c_2^2 =1$, con $c_2>0$ per la convenzione:
\[
\braket{j_1, j_2, j_1, j-j_1 | j, j} \geq 0 \Rightarrow c_2 = \braket{\frac{1}{2},\frac{1}{2}|1,\frac{1}{2},1,-\frac{1}{2}} > 0
\]
In questo caso il modo più rapido per trovare $c_1$ e $c_2$ è utilizzare il fatto di algebra lineare per cui dato un vettore $(a, b)$, i vettori perpendicolari \textit{con la stessa norma} sono $(-b, a)$ o $(b,-a)$. Possiamo allora partire da (\ref{eqn:tremezzi}) e da $(\sqrt{\frac{2}{3}},\frac{1}{\sqrt{3}})$ e trovare $(c_1, c_2)$ seguendo lo schema $(-b, a)$, che rispetta la convenzione di $c_2>0$:
\[
c_1 = -\sqrt{\frac{1}{3}}; \quad c_2 = \sqrt{\frac{2}{3}}
\]
Riepilogando, abbiamo ottenuto:
\begin{align}\nonumber
\ket{\frac{3}{2},\frac{3}{2}} &= \ket{1,\frac{1}{2};1,\frac{1}{2}}\\ \nonumber
\ket{\frac{3}{2},\frac{1}{2}} &= \sqrt{\frac{2}{3}}\ket{1,\frac{1}{2};0,\frac{1}{2}}+\frac{1}{\sqrt{3}}\ket{1,\frac{1}{2};1,-\frac{1}{2}}\\
\ket{\frac{1}{2},\frac{1}{2}} &= \hlc{Yellow}{-\frac{1}{\sqrt{3}}}\ket{1,\frac{1}{2};0,\frac{1}{2}}+\hlc{Yellow}{\sqrt{\frac{2}{3}}}\ket{1,\frac{1}{2};1,-\frac{1}{2}}
\label{eqn:unmezzo}
\end{align}
Per ottenere $\ket{\frac{1}{2},-\frac{1}{2}}$ basta abbassare $\ket{\frac{1}{2},\frac{1}{2}}$ mediante $J_-$. Utilizziamo allora la formula (\ref{eqn:ricorsione_generale}), con $j=\frac{1}{2}$ e $m=\frac{1}{2}$. Il membro di sinistra diviene: 
\begin{align*}
J_- \ket{\frac{1}{2},\frac{1}{2}}&= \sqrt{j(j+1)-m(m-1)}\ket{j,m-1}=\\
&=\sqrt{\frac{1}{2}\left(\frac{1}{2}+1\right)-\frac{1}{2}\left(\frac{1}{2}-1\right)} \ket{\frac{1}{2},-\frac{1}{2}}=\hlc{ForestGreen}{\ket{\frac{1}{2},-\frac{1}{2}}}
\end{align*}
Mentre quello di destra è dato da:
\begin{align*}
\sum_{m_1' = -j_1}^{+j_1} \sum_{m_2' = -j_2}^{+j_2} \braket{j_1, j_2; m_1', m_2'|j,m} \Big ( \sqrt{j_1(j_1+1)-m_1'(m_1'-1)} \ket{j_1, j_2; m_1'-1, m_2'} +\\
+\sqrt{j_2(j_2+1)-m_2'(m_2'-1)}\ket{j_1, j_2; m_1', m_2'-1} \Big)
\end{align*}
dove $j_1 = l = 1$, $j_2 = s = \frac{1}{2}$, $j=m=\frac{1}{2}$, e $\braket{j_1, j_2; m_1', m_2'|j,m}$ sono i coefficienti evidenziati in giallo in (\ref{eqn:unmezzo}), e sono non nulli solo per $(m_1', m_2') = (0,\frac{1}{2}), (1,-\frac{1}{2})$. Svolgendo allora i conti otteniamo:
\begin{align} \nonumber
&\left(-\frac{1}{\sqrt{3}}\right)\left[\sqrt{1(1+1)-0\left(0-\frac{1}{2}\right)}\ket{1,\frac{1}{2};-1,\frac{1}{2}}+ \sqrt{\frac{1}{2}\left(\frac{1}{2}+1\right)-\frac{1}{2}\left(\frac{1}{2}-1\right)}\ket{1,\frac{1}{2};0,-\frac{1}{2}} \right] +\\ \nonumber
+&\left(\sqrt{\frac{2}{3}}\right)\left[\sqrt{1(1+1)-1(1-1)} \ket{1,\frac{1}{2};0,-\frac{1}{2}}+\sqrt{\frac{1}{2}\left(\frac{1}{2}+1\right)-\left(-\frac{1}{2}\right)\left( -\frac{1}{2}-1\right)}\ket{1,\frac{1}{2};1,-\frac{3}{2}} \right]=\\ \nonumber
=&
-\sqrt{\frac{2}{3}}\ket{1,\frac{1}{2};-1,\frac{1}{2}} -\frac{1}{\sqrt{3}}\ket{1,\frac{1}{2};0,-\frac{1}{2}} + \frac{2}{\sqrt{3}}\ket{1,\frac{1}{2};0,-\frac{1}{2}}=\\ \label{eqn:ket-finale}
=& -\sqrt{\frac{2}{3}}\ket{1,\frac{1}{2};-1,\frac{1}{2}}+\frac{1}{\sqrt{3}}\ket{1,\frac{1}{2};0,-\frac{1}{2}} = \hlc{ForestGreen}{\ket{\frac{1}{2},-\frac{1}{2}}}
\end{align}
\textbf{Nota}: dato che $\ket{1,\frac{1}{2};1,-\frac{1}{2}}$ ha già lo spin minimo, non è possibile abbassarlo ulteriormente, e infatti il coefficiente di $\ket{1,\frac{1}{2};1,-\frac{3}{2}}$ risulta nullo.\\

\textbf{Riepilogando}, per ottenere $\ket{\frac{1}{2},-\frac{1}{2}}$ abbiamo usato questa strategia:
\[
\ket{\frac{3}{2}, \frac{3}{2}} \xrightarrow{J_-} \ket{\frac{3}{2},\frac{1}{2}} \perp \ket{\frac{1}{2},\frac{1}{2}} \xrightarrow{J_-} \ket{\frac{1}{2},-\frac{1}{2}}
\]

Usando il risultato ottenuto in (\ref{eqn:ket-finale}), possiamo allora finalmente calcolare il braket in (\ref{eqn:probability-angular-3}),  e quindi la probabilità cercata:
\begin{align*}
W_\psi^{\vec{J}^2, J_3}\left(\frac{3}{4}\hbar^2, -\frac{\hbar}{2}\right)  =\left|\bra{\frac{1}{2},-\frac{1}{2}}\frac{1}{\sqrt{2}}\left(\ket{1,\frac{1}{2},-1,\frac{1}{2}}-\ket{1,\frac{1}{2},1,\frac{1}{2}}\right)\right|^2 =
\left|\sqrt{\frac{2}{3}}\frac{1}{\sqrt{2}} \right|^2=\frac{1}{3}
\end{align*}
\item La misura di $\vec{J}^{\,2}$ e $J_3$, di risultato $\left(\frac{3\hbar^2}{4},-\frac{\hbar}{2}\right)$, proietta lo stato del sistema sull'autoket $\ket{\frac{1}{2},-\frac{1}{2}}$, che abbiamo già determinato al punto precedente:
\begin{align}
\ket{\frac{1}{2},-\frac{1}{2}}=\underbrace{\frac{1}{\sqrt{3}}\ket{1,\frac{1}{2},0,-\frac{1}{2}}}_{m_s=-1/2}-\sqrt{\frac{2}{3}}\underbrace{\ket{1,\frac{1}{2},-1, \frac{1}{2}}}_{m_s \neq -1/2}
\label{eqn:autoket-angolare_misura}
\end{align}
In questo stato misuriamo $S_3$, ottenendo $-\frac{\hbar}{2}$, ossia $m_s = -\frac{1}{2}$. Ciò fa sì che tutti i termini di (\ref{eqn:autoket-angolare_misura}) che non sono compatibili con tale risultato svaniscano, e perciò il nuovo stato $\ket{\psi(0^+)}$ conterrà solo i rimanenti:
\[
\ket{\psi(0^+)}=\ket{1,\frac{1}{2};0,-\frac{1}{2}}
\]
Vogliamo ora calcolare l'evoluto temporale $\ket{\psi(t)}$:
\begin{align}
\ket{\psi(t)}=\exp\left(-\frac{i}{\hbar}tH\right)\ket{\psi(0^+)}
\label{eqn:evoluto-temporale-angolare}
\end{align}
Partiamo allora dall'espressione dell'Hamiltoniana:
\[
H=c\vec{L}\cdot \vec{S}
\]
In questa forma, non sappiamo ricavare direttamente lo spettro di $\sigma(\vec{L}\cdot \vec{S})$, per cui risulterebbe difficile riscrivere $\ket{\psi(0^+)}$ come combinazione di autostati di $H$. Notiamo però che possiamo riscrivere tale $H$ in funzione di operatori \q{che sappiamo gestire}.\\
Sappiamo infatti che:
\[
\vec{J}=\vec{L}+\vec{S} = \vec{L}\otimes \bb{I} + \bb{I}\otimes \vec{S}
\]
e $[\vec{L},\vec{S}]=0$. Allora possiamo elevare al quadrato \q{senza problemi}:
\[
\vec{J}^{\,2} = \vec{L}^{2} + \vec{S}^2 + 2\vec{L}\cdot \vec{S} \Rightarrow  \vec{L}\cdot\vec{S} = \frac{1}{2}\left(\vec{J}^2 - \vec{L}^2 -\vec{S}^2 \right)
\]
E quindi otteniamo:
\begin{align*}
H=\frac{c}{2}(\vec{J}^{\,2}-\vec{L}^2 - \vec{S}^2)
\end{align*}
Vogliamo inoltre che $\ket{\psi(0^+)}$ sia scritta come combinazione di autoket di $H$, ossia di autoket comuni a $\vec{J}^{\,2}$, $\vec{L}^2$, $\vec{S}^2$. Attualmente $\ket{\psi(0^+)}$ è però scritta nella base $\ket{l,s;m, m_s}$, che è fatta di autoket di $\vec{L}^2$ e $\vec{S}^2$, ma non di $\vec{J}^2$. Dobbiamo allora passare in base $\ket{l,s; J, M}$ riscrivendo $\ket{\psi(0^+)}$ come:
\begin{align}
\ket{1,\frac{1}{2},0,-\frac{1}{2}}=\sum_{J=\frac{1}{2}}^{\frac{3}{2}} \sum_{M=-J}^{J} \braket{J,M|1,\frac{1}{2},0,-\frac{1}{2}} \ket{J,M}
\label{eqn:cambio-base-JM}
\end{align}
I coefficienti di Clebsh-Gordan sono non nulli solo per $M=m+m_s=0+(-\frac{1}{2})=-\frac{1}{2}$ e perciò gli unici termini che contribuiscono alla sommatoria sono $\ket{\frac{3}{2},-\frac{1}{2}}$ e $\ket{\frac{1}{2},-\frac{1}{2}}$. Quest'ultimo l'abbiamo già ricavato:
\[
\ket{\frac{1}{2},-\frac{1}{2}}=-\sqrt{\frac{2}{3}}\ket{1,\frac{1}{2},-1,\frac{1}{2}} + \frac{1}{\sqrt{3}}\ket{1,\frac{1}{2},0,-\frac{1}{2}}
\]
E ci manca solo ricavare il primo. Ragionando come prima, possiamo per esempio abbassare $\ket{\frac{3}{2}, \frac{1}{2}}$ che già conosciamo:
\begin{align*}
J_- \ket{\frac{3}{2},\frac{1}{2}}&=\sqrt{\frac{3}{2}\left(\frac{3}{2}+1\right)-\frac{1}{2}\left(-\frac{1}{2}\right)}\ket{\frac{3}{2},-\frac{1}{2}} =\\
&=(L_- + S_-)\left(\sqrt{\frac{2}{3}}\ket{1,\frac{1}{2},0,\frac{1}{2}} +\frac{1}{\sqrt{3}}\ket{1,\frac{1}{2},1,-\frac{1}{2}}\right)=\\
&=\sqrt{\frac{2}{3}}\left[\sqrt{1(1+1)+0}\ket{1,\frac{1}{2},-1,\frac{1}{2}}+\sqrt{\frac{1}{2}\left(\frac{1}{2}+1\right)-\frac{1}{2}\left(-\frac{1}{2}\right)}\ket{1,\frac{1}{2},0,-\frac{1}{2}}\right]+\\
&+\frac{1}{\sqrt{3}}\left[ \sqrt{1(1+1)-1(1-1)}\ket{1,\frac{1}{2},0,-\frac{1}{2}} + 0 \right]\\
\Rightarrow \ket{\frac{3}{2},-\frac{1}{2}} &= \frac{1}{\sqrt{3}}\ket{1,\frac{1}{2},-1,\frac{1}{2}}+\sqrt{\frac{2}{3}}\ket{1,\frac{1}{2},0,-\frac{1}{2}}
\end{align*}
Sostituendo in (\ref{eqn:cambio-base-JM}) e calcolando, otteniamo:
\begin{align*}
\ket{1,\frac{1}{2},0,-\frac{1}{2}} &= \braket{\frac{3}{2},-\frac{1}{2}|1,\frac{1}{2},0,-\frac{1 }{2}}\ket{\frac{3}{2},-\frac{1}{2}}+ \braket{\frac{1}{2},-\frac{1}{2}|1,\frac{1}{2},0,-\frac{1}{2}}\ket{\frac{1}{2},-\frac{1}{2}}=\\
&=\sqrt{\frac{2}{3}}\ket{\frac{3}{2},-\frac{1}{2}} +\frac{1}{\sqrt{3}}\ket{\frac{1}{2},-\frac{1}{2}}
\end{align*}
Detti $\ket{\varphi_n} = \ket{J,M}$ gli autostati di $H$, possiamo espandere la (\ref{eqn:evoluto-temporale-angolare}) come:
\begin{align*}
\ket{\psi(t)} = \sum_n \exp\left(-\frac{i}{\hbar}\mathcal{E}_n(t-t_0) \right) \ket{\varphi_n} \braket{\varphi_n| \psi(0^+)}
\end{align*} 
Calcoliamo gli $\mathcal{E}_n$ dalle equazioni agli autovalori:
\begin{align*}
H\ket{\frac{3}{2},-\frac{1}{2}}&=\frac{c}{2}(\vec{J}^2-\overbrace{\vec{L}^2}^{1}-\overbrace{\vec{S}^2}^{1/2})\ket{\frac{3}{2},-\frac{1}{2}}=\\
&= \frac{c}{2}\hbar^2 \left[\frac{3}{2}\left(\frac{3}{2}+1\right)-1(1+1)-\frac{1}{2}\left(\frac{1}{2}+1\right)\right]\ket{\frac{3}{2},-\frac{1}{2}}=\\
&=\frac{c}{2}\hbar^2\ket{\frac{3}{2},-\frac{1}{2}} \Rightarrow  \mathcal{E}_1 =\frac{c}{2}\hbar^2
\end{align*}

\begin{align*}
H\ket{\frac{1}{2},-\frac{1}{2}}&=\frac{c}{2}\hbar^2\left[\frac{1}{2}\left(\frac{1}{2}+1\right)-1(1+1)-\frac{1}{2}\left(\frac{1}{2}+1\right)\right]\ket{1,-\frac{1}{2}}=\\
&=-c\hbar^2 \ket{\frac{1}{2},-\frac{1}{2}} \Rightarrow \mathcal{E}_2=-c\hbar^2
\end{align*}
E infine:
\begin{align*}
\ket{\psi(t)}&=\exp\left(-i\frac{t}{\hbar}H\right) \ket{1,\frac{1}{2},0,-\frac{1}{2}} =\\
&= \exp\left(-\frac{i}{\hbar}\mathcal{E}_1 t\right)\sqrt{\frac{2}{3}}\ket{\frac{3}{2},-\frac{1}{2}}+
\exp\left(-\frac{i}{\hbar}\mathcal{E}_2 t \right)\frac{1}{\sqrt{3}}\ket{\frac{1}{2},-\frac{1}{2}}\\
\end{align*}
Inserendo allora i valori di $\mathcal{E}_1$ e $\mathcal{E}_2$ e reintroducendo il termine radiale, otteniamo infine:
\begin{align*}
\ket{\psi(t)}=-r e^{-ar} \left[\exp\left(-i\frac{c}{2}\hbar t\right) \sqrt{\frac{2}{3}}\ket{\frac{3}{2},-\frac{1}{2}}+e^{ic\hbar t}\ket{\frac{1}{2},-\frac{1}{2}}\right]
\end{align*}
\end{enumerate}
\end{document}

