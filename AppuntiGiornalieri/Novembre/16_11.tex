\documentclass[../../FisicaTeorica.tex]{subfiles}

\begin{document}
\section{Sistemi quantistici composti}
Consideriamo due sistemi quantistici $S^1$ e $S^2$ con spazio di Hilbert $\hs_1$ e $\hs_2$. Come si descrive il sistema $S^1 \cup S^2$, e in particolare i suoi stati puri?\\

In \MC, se $\Omega_1$ e $\Omega_2$ sono gli spazi delle fasi di $S^1$ e $S^2$, lo spazio delle fasi del sistema composto $S^1 \cup S^2$ è il \textbf{prodotto cartesiano} $\Omega_1\times \Omega_2$ e quindi gli stati puri sono dati da un prodotto di delta:
\[
\delta^{(1)} (q^{(1)}-q_0^{(1)})\delta(p^{(1)}-p_0^{(1)}) \delta^{(2)}(q^{(2)}-q_0^{(2)})\delta(p^{(2)}-p_0^{(2)}) \qquad (q^{(1)}, p^{(1)})\in \Omega_1,\> (q^{(2)}, p^{(2)}) \in \Omega_2
\] 
Quindi gli stati puri di $S^1 \cup S^2$ sono il prodotto degli stati puri dei sottosistemi.\\

Proviamo ora a costruire l'analogo in \MQ, in cui considereremo all'inizio il caso di due particelle \textit{distinguibili}.  Partiamo dall'equazione di Schr\"odinger della particella singola:
\begin{equation}
i\hbar \frac{\partial \psi(\vec{x},t)}{\partial t}=H(\vec{x}, -i\hbar\vec{\nabla})\psi(\vec{x},t)
\label{eqn:schrody_1}
\end{equation}
in cui l'$H$ quantistica è ottenuta da quella classica con la sostituzione $\vec{p} \to -i\hbar\vec{\nabla}^2$.\\
Poiché nel caso di due particelle l'Hamiltoniana classica è della forma:
\[
H(\vec{x}^{\,(1)}, \vec{p}^{\,(1)}, \vec{x}^{\,(2)}, \vec{p}^{\,(2)})
\]
Potremmo pensare di estendere (\ref{eqn:schrody_1}) nel seguente modo:
\[
i\hbar \frac{\partial \psi(\vec{x}^{\,(1)}, \vec{x}^{\,(2)},t)}{\partial t} = H(\vec{x}^{\,(1)},-i\hbar \vec{\nabla}^{\,(1)}, \vec{x}^{\,(2)}, -i\hbar\vec{\nabla}^{\,(2)})\psi(\vec{x}^{\,(1)},\vec{x}^{\,(2)},t)
\]
Quindi la funzione d'onda a tempo fissato diviene:
\[
\psi(\vec{x}^{\,(1)}, \vec{x}^{\,(2)})\in L^2(\bb{R}^6, d^3 x^{\,(1)} d^3 x^{\,(2)})
\]
Per capire come si descrivono i sistemi composti dobbiamo capire la relazione tra lo spazio $L^2(\bb{R}^6, d^3 x^{\,(1)} d^3 x^{\,(2)})$ (suggerito dall'estensione del caso classico) e gli spazi di Hilbert \q{iniziali} delle due particelle \textit{considerate separamente}:
\[
\hs^{(1)}=L^2(\bb{R}^3, d^3 x^{(1)}); \quad \hs^{(2)}=L^2(\bb{R}^3, d^3 x^{(2)})
\]
\end{document}



