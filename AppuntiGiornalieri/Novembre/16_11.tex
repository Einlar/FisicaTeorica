\documentclass[../../FisicaTeorica.tex]{subfiles}

\begin{document}

\section{Lezione ?:\\ \large{Titolo}}
\vspace{-1em}
\begin{center}
    \small{(16/11/2018)}
\end{center}
Riprendiamo l'esercizio iniziato la volta scorsa.
\begin{enumerate}
\item $\mathcal{E}_n = \frac{\hbar^2}{2m} \frac{\pi^2 n^2}{a^2}$ sono gli autovalori, associati alle autofunzioni:
\[
\varphi_n(x)=\begin{cases}
\sqrt{\frac{2}{a}}\cos\left(\frac{n\pi x}{a}\right) & n\in \bb{N} \text{ dispari}\\
\sqrt{\frac{2}{a}}\sin\left(\frac{n\pi x}{a}\right) & n\in \bb{N} \text{ pari}
\end{cases}
\]
Lo stato è dato da:
\[
\psi(x,t)=\sum_{n=1}^\infty c_n \exp\left(-\frac{i}{\hbar}\mathcal{E}_n t\right) \varphi_n(x)
\]
con $c_n=(\varphi_n, \psi(x,0^+))$, che vale:
\begin{align*}
\text{$n$ pari} && \begin{cases}
c_n = 0 \quad n \neq 2\\
c_2 = \frac{1}{\sqrt{2}}
\end{cases}\\
\text{$n$ dispari} && c_n=\int_0^\frac{a}{2} \sqrt{\frac{2}{a}}\cos\left(\frac{n\pi x}{a}\right)\frac{2}{\sqrt{a}}\sin\left(\frac{2\pi x}{a}\right)\,dx=\\
\end{align*}
Usando le formule di Werner:
\[
\sin\alpha \cos \beta = \frac{1}{2}(\sin(\alpha+\beta)+\frac{1}{2}\sin(\alpha-\beta))
\]
Da cui:
\begin{align*}
c_n &= \int_0^{\frac{a}{2}}\frac{2\sqrt{2}}{a}\frac{1}{2}\left[\sin\frac{(2+n)\pi x}{a}+\sin\frac{(2-n)n x}{a}\right]dx =\\
&= \frac{\sqrt{2}}{a}\left[ \frac{a}{(2+n)\pi}\left(-\cos \frac{(n+2)\pi x}{a}\right)\Big|_{0}^{\frac{a}{2}}+\frac{a}{(2-n)\pi}\left(-\cos\frac{(2-n)\pi x}{a}\right)\Big|_{0}^{\frac{a}{2}} \right] =\\
&=\sqrt{2}\left[\frac{1}{(2+n)\pi} + \frac{1}{(2-n)\pi}\right]=\sqrt{2}\frac{4}{4-n^2}\frac{1}{\pi}
\end{align*}
\item \begin{align*}
P^H_{\psi(t)}(\mathcal{E}>\mathcal{E}_2)&=1-W^H_{\psi(t)}(\mathcal{E}_1)-W^H_{\psi(t)}(\mathcal{E}_2)\\
W^H_{\psi(t)}(\mathcal{E}_n) &= |(\psi(t),\varphi_n)|^2 = |(\psi(0^+),\varphi_n)|^2 = |c_n|^2
\end{align*}
dato che $\psi(t)=U(t)\psi(0^+)$, e $U(t)$ è unitario, cioè aggiunge solo una fase $\exp\left(-\frac{i}{\hbar}\mathcal{E}_n t\right)$ senza modificare il modulo.
\[
P_{\psi(t)}^H(\mathcal{E}>\mathcal{E}_2)=1-|c_1^2-|c_2|^2=1-\left|\ \sqrt{2}\left(\frac{1}{\pi}+\frac{1}{3\pi}\right)
\right|^2 -\frac{1}{2} \sim 0.14
\]
\item Calcoliamo ora la probabilità che la parità applicata a $\psi(t_0)$ dia $+1$, ossia $W_{\psi(t_0)}^\mathcal{P}(+1)$. Sappiamo che:
\[
\mathcal{P}\psi(x,t_0)=\psi(-x,t_0)
\]
Per calcolare $W_{\psi_{t_0}}(1)$ dobbiamo costruire il proiettore $P^\mathcal{P}(\{+1\})$ che proietta nell'autospazio di $\mathcal{P}$ di autovalore $+1$. Una base di questo autospazio è per esempio data dalle $\varphi_n$ con $n$ dispari (i $\cos$, che sono pari), e quindi $+1$ è autovalore a degenerazione $\infty$.\\
Dato che $[H,\mathcal{P}]=0$, $H$ e $\mathcal{P}$ ammettono una base di autovettori comuni:
\begin{align*}
H, \mathcal{P}=+1 & \cos \frac{n\pi x}{a} & \ket{n}, \sum_{n \text{ dispari}}\ket{n}\bra{n}\\
H, \mathcal{P}=-1 & \sin\frac{n\pi x}{a} & \sum_{n \text{ pari}} \ket{n}\bra{n}
\end{align*}
Tuttavia posso evitare di calcolare una somma infinita per trovare la base, notando che $\mathcal{P}^2=\bb{I}$, da cui $(1+\mathcal{P})/2$ è un proiettore. Infatti:
\[
\left(\frac{1+\mathcal{P}}{2}\right) \left(\frac{1+\mathcal{P}}{2}\right)=\frac{1}{4}+\frac{1}{2}\mathcal{P}+\frac{1}{4}\mathcal{P}^2=\frac{1+\mathcal{P}}{2}
\]
E tale operatore agisce come:
\[
\frac{1+\mathcal{P}}{2}\psi(x)=\frac{\psi(x)+\psi(-x)}{2} = \begin{cases}
\psi(x) & \text{se $\psi(x)$ è pari}\\
0 & \text{se $\psi(x)$ è dispari}
\end{cases}
\]
Quindi $(1+\mathcal{P})/2$ è il proiettore che proietta nell'autospazio $+1$ di $\mathcal{P}$, e quindi:
\begin{align*}
W_{\psi(t)}^\mathcal{P}(+1)&=(\psi(t),\frac{\bb{I}+\mathcal{P}}{2}\psi(t))=\left(
\sum_n c_n \exp\left(-\frac{i}{\hbar}\mathcal{E}_n t\right) \varphi_n, \left(\frac{\bb{I}+\mathcal{P}}{2}\right)\underbrace{
\sum_m c_m \exp\left(-\frac{i}{\hbar}\mathcal{E}_m t\right) \varphi_m}_{m \text{ dispari}} \right) = \\
&=\sum_{n, m \text{ dispari}} c_n^* c_m \exp\left(-\frac{i}{\hbar}(\mathcal{E}_m-\mathcal{E}_n)t\right)\underbrace{(\varphi_n, \varphi_m)}_{\delta_{nm}}=\sum_{n \text{ dispari}}|c_n|^2
\end{align*}
Di nuovo, non serve calcolare la somma infinita, dato che sappiamo:
\[
W_{\psi(t)}^\mathcal{P}(+1)=1-W_{\psi(t)}^\mathcal{P}(-1)=1-|c_2|^2=1-\frac{1}{2}=\frac{1}{2}
\]
\item Calcoliamo $(\Delta X)_{\varphi_n}(\Delta P)_{\varphi_n}$, dove:
\begin{align*}
(\Delta X)_{\varphi_n}&=(\Delta X)_{n} = \sqrt{\langle X^2\rangle_n - \langle X\rangle^2_n}\\
\underbrace{\langle X^2\rangle_n}_{n \text{ dispari}} &= \int_{-\frac{a}{2}}^{+\frac{a}{2}}\sqrt{\frac{2}{a}}\left(\cos\frac{n\pi x}{a}\right)x^2 \sqrt{\frac{2}{a}}\left(\cos\frac{n\pi x}{a}\right)dx =\\
&=\int_{-\frac{a}{2}}^{\frac{a}{2}} x^2 \cos^2 \frac{n\pi x}{a} dx = \frac{2}{a} \int_{-\frac{a}{2}}^{\frac{a}{2}} \frac{x^2}{2}\left(1+ \cos\frac{2n\pi x}{a}\right)dx =\\
&=\frac{1}{a}\frac{x^3}{3}\Big|_{-\frac{a}{2}}^{+\frac{a}{2}}+\frac{1}{a}\frac{a}{2\pi n}\int_{-\frac{a}{2}}^{+\frac{a}{2}} x^2 \frac{d}{dx}\sin\frac{2n\pi x}{a}dx \underset{\text{per parti}}{=}\\
&=\frac{a^2}{12}+\frac{1}{2\pi n} x^2 \bcancel{\sin\frac{(2n\pi x)}{a}\Big|_{-\frac{a}{2}}^{+\frac{a}{2}}} - \frac{1}{2\pi n}\int_{-\frac{a}{2}}^{+\frac{a}{2}} 2x \sin \frac{2n\pi x}{a} dx=\\
&=\frac{a^2}{12} - \frac{a}{2(\pi n)^2} x \cos\frac{2\pi n x}{a}\Big|_{-\frac{a}{2}}^{\frac{a}{2}} + \frac{a}{2(n\pi)^2}\int_{-\frac{a}{2}}^{\frac{a}{2}} \cos \frac{2n\pi x}{a}=\\
&=\frac{a^2}{12}+\frac{a}{2(n\pi)^2}\frac{a}{2}2(-1)^n + \frac{a}{2(\pi n)^2}\frac{a}{2\pi n}\bcancel{\sin\frac{2\pi n x}{a}\Big|_{-\frac{a}{2}}^{\frac{a}{2}}} =\\
&=a^2 \left(\frac{1}{12}-\frac{1}{2(n\pi)^2}\right)
\end{align*}
Perciò ($n$ dispari):
\begin{align*}
(\Delta X)_n &= a\sqrt{\frac{1}{12}-\frac{1}{2(n\pi)^2}}\\
\langle X\rangle_n &= \frac{2}{a} \int_{-\frac{a}{2}}^{\frac{a}{2}} x \cos^2 \frac{n\pi x}{a} dx =0
\end{align*}
Per quanto riguarda il momento:
\[
\langle P^2\rangle_n = \langle 2mH\rangle_n = 2m \frac{\hbar^2}{2m}\frac{\pi^2 n^2}{a^2}=\frac{\hbar^2 \pi^2 n^2}{a^2}
\]
$P$ è definito con condizioni periodiche, e la $\psi$ anche
\[
\psi\left(-\frac{a}{2}\right)=\psi\left(\frac{a}{2}\right)
\]
da cui $\psi \in D(P)$.\\
Le $\varphi_n \in D(P)$? Sì, perché:
\[
\varphi_n\left(-\frac{a}{2}\right)=0=\varphi_n\left(\frac{a}{2}\right)
\]
E allora:
\[
\langle P \rangle_n = \int_{-\frac{a}{2}}^{\frac{a}{2}} dx \sqrt{\frac{2}{a}}\sin\frac{n\pi x}{a}\sqrt{\frac{2}{a}}\cos \frac{n\pi x}{a} dx=0
\]
E infine:
\[
(\Delta X)_n(\Delta P)_n = \frac{\hbar \pi n}{a} a \sqrt{\frac{1}{12}-\frac{1}{2(n\pi)^2}}=\hbar \pi n \sqrt{\frac{1}{12}-\frac{1}{2(\pi n)^2}} \neq 0
\]
(Il conto nel caso di $n$ pari è lasciato per esercizio)
\item $H$ e $P$ sono compatibili?\\
Se $H$ e $P$ sono compatibili, ammettono una base di autovettori comuni. $\varphi_n$ autostati di $H$ sono non degeneri, e per questo i $\varphi_n$ dovrebbero essere anche autostati di $P$. Ma ciò non è possibile perché $(\Delta P)_n \neq 0$ come abbiamo appena visto, e perciò $H$ e $P$ \textbf{non} sono compatibili.\\
Nota che $P$ e $P^2$ sono compatibili, $P^2$ e $H$ sono compatibili, ma $P$ e $H$ non lo sono! In generale:
\[
[P^2, H]=0, \> [P,P^2]=0 \not\Rightarrow [P,H]=0
\]
La compatibilità è una questione delicata da valutare attentamente sempre \textit{mediante} la definizione, dato che potrebbero sorgere problemi di dominio (per esempio $P$ e $H$ sono compatibili nel caso libero, ma non in questo caso \textit{compatto}).
\item Calcoliamo $P_{\psi(t)}^X \left(\left[-\frac{a}{2},0\right]\right)$, $0<t<t_0$.
\begin{align*}
P_{\psi(t)}^X \left(\left[-\frac{a}{2},0\right]\right) &= \left(\psi(t), P^X\left(\left[-\frac{a}{2},0\right]\right)\psi(t)\right)=\int_{-\frac{a}{2}}^{0} |-\psi(x,t)|^2 dx=\\
&= \int_{-\frac{a}{2}}^0 \sum_n c_n^* \sum_m c_m \exp\left(\frac{i}{\hbar}(\mathcal{E}_n-\mathcal{E}_m)t\right)\varphi_n^*(x)\varphi_m(x)dx=\\
&=\sum_{n,m}c_n^* c_m \gamma_{nm}\exp\left(\frac{i}{\hbar}(\mathcal{E}_n-\mathcal{E}_m)t\right) \quad \gamma_{nm}\equiv \int_{-\frac{a}{2}}^0 \varphi_n^*(x) \varphi_n(x) dx 
\end{align*}
Per $m,n$ dispari:
\begin{align*}
\gamma_{mn}&=\frac{2}{a}\int_{-\frac{a}{2}}^0 \cos \frac{n\pi x}{a}\cos \frac{m\pi x}{a}dx=\frac{1}{a}\int_{-\frac{a}{2}}^{0} \left[
\cos\frac{(n+m)\pi x}{a}+ \cos\frac{(n-m)\pi x}{a}
\right]=\\
&=\begin{cases}
0 & n\neq m\\
\displaystyle
\frac{1}{a}\int_{-\frac{a}{2}}^0 1\, dx=\frac{1}{2} & n=m
\end{cases}
\end{align*}
Negli altri casi, dobbiamo esaminare solo per $n=2,m=2$, oppure uno dei due dispari e l'altro $2$ (tutti gli altri sono nulli). Immediatamente si ha $\gamma_{22}=\frac{1}{2}$, mentre, per $n$ dispari:
\begin{align*}
\gamma_{2n}&=\gamma_{n2}=\frac{2}{a}\int_{-\frac{a}{2}}^0 \sin\frac{2\pi x}{a}\cos \frac{n\pi x}{a} dx=\frac{1}{a}\int_{-\frac{a}{2}}^0 \left[
\sin\frac{(n+2)\pi x}{a}+\sin\frac{(2-n)\pi x}{a}
\right]dx=\\
&= -\left[\frac{1}{\pi(n+2)}+\frac{1}{n(2-n)}\right]=\frac{1}{\pi} \frac{4}{n^2-4}
\end{align*}
Allora:
\begin{align*}
P^X_{\psi(t)}\left(\left[-\frac{a}{2},0\right]\right)&=
\sum_{n,m} c_n^2 c_m \exp\left(\frac{i}{\hbar}(\mathcal{E}_n-\mathcal{E}_m)t\right)\gamma_{nm}
=\\
&=\frac{|c_2|^2}{2}+\sum_{n \text{ dispari}}\frac{|c_n|^2}{2}+\\
&+\sum_{n \text{ dispari}}\frac{1}{\pi}\frac{4}{(n^2-4)}c_n c_2\left(\exp\left(-\frac{i}{\hbar}(\mathcal{E}_n-\mathcal{E_2})t\right) + \exp\left(\frac{i}{\hbar}(\mathcal{E}_n-\mathcal{E}_2)t\right)\right) =\\
&=\frac{1}{2}+\sum_{n \text{ dispari}}\underbrace{\frac{4}{\pi(n^2-4)}}_{\gamma_{2n}}\sqrt{2} \underbrace{\frac{4}{(4-n^2)}\frac{1}{n}}_{c_n}
\end{align*}
Dato che:
\[
\sum_{n} |c_n|^2 = 1 = |c_2|^2 + \sum_{n \text{ dispari}}|c_n|^2
\]
Arriviamo perciò a:
\[
P_{\psi(t)}^X\left(\left[-\frac{a}{2},0\right]\right) = \frac{1}{2}-\sum_{n \text{ dispari}}\left(\frac{1}{\pi}\frac{4}{n^2-4}\right)^2 2\cos\left(\frac{t}{\hbar}(\mathcal{E}_n-\mathcal{E}_2)\right)
\]
\end{enumerate}

\section{Sistemi quantistici composti}
(Dopo questo argomento discuteremo le simmetrie, poi considereremo 3 sistemi, oscillatore armonico, atomo di idrogeno e scattering a potenziale centrale, ed eventuali argomenti finali se resta tempo).\\
Consideriamo due sistemi quantistici $S^1$ e $S^2$ con spazio di Hilbert $\hs_1$ e $\hs_2$. Come si descrive il sistema $S^1 \cup S^2$, in particolare i suoi stati puri?\\
In \MC, se $\Omega_1$ e $\Omega_2$ sono gli spazi delle fasi di $S^1$ e $S^2$, lo spazio delle fasi di $S^1 \cup S^2$ è il \textbf{prodotto cartesiano} $\Omega_1\times \Omega_2$ e quindi gli stati puri sono dati da ($q^1 \in \Omega_1$, $q_2 \in \Omega_2$):
\[
\delta^1 (q^1-q_0^1)\delta(p^1-p_0^1) \delta^2(q^2-q_0^2)\delta(p^2-p_0^2)
\] 
(Non confondere gli apici con esponenti!)\\
Con $q_0^{(1)}$, $p_0^{(1)} \in \Omega^1$ e $q_0^{(2)}$, $p_0^{(2)}\in \Omega^2$.\\
Quindi gli stati puri di $S^1 \cup S^2$ sono il prodotto degli stati puri dei sottosistemi.\\
In \MQ abbiamo input da Schrodinger. 
Per due particelle (\textit{distinguibili}), se partiamo dall'equazione della particella singola:
\[
i\hbar \frac{\partial \psi(\vec{x},t)}{\partial t}=H(\vec{x}, -i\hbar\vec{\nabla})\psi(\vec{x},t)
\]
in cui l'$H$ quantistica è ottenuta da quella classica con la sostituzione $\vec{p} \to -i\hbar\vec{\nabla}^2$.\\
Poiché nel caso di due particelle l'Hamiltoniana classica è della forma:
\[
H(\vec{x}^{(1)}, \vec{p}^{(1)}, \vec{x}^{(2)}, \vec{p}^{(2)})
\]
L'equazione di Schrodinger suggerisce: %Potrebbe esserci un - davanti (controlla)
\[
i\hbar \frac{\partial \psi(\vec{x}^{(1)}, \vec{x}^{(2)},t)}{\partial t} = H(\vec{x}^{(1)},-i\hbar \vec{\nabla}^{(1)}, \vec{x}^{(2)}, -i\hbar\vec{\nabla}^{(2)})\psi(\vec{x}^{(1)},\vec{x}^{(2)},t)
\]
Quindi la funzione d'onda a tempo fissato:
\[
\psi(\vec{x}^{(1)}, \vec{x}^{(2)})\in L^2(\bb{R}^6, d^3 x^{(1)} d^3 x^{(2)})
\]
Per capire come si descrivono i sistemi composti dobbiamo capire la relazione tra $L^2(\bb{R}^6, d^3 x^{(1)} d^3 x^{(2)})$ (suggerito dall'estensione del caso classico) con gli spazi di Hilbert delle due particelle:
\[
\hs^{(1)}=L^2(\bb{R}^3, d^3 x^{(1)}); \quad \hs^{(2)}=L^2(\bb{R}^3, d^3 x^{(2)})
\]
\end{document}

