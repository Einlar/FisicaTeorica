\documentclass[../../FisicaTeorica.tex]{subfiles}

\begin{document}

%\section{Lezione ?:\\ \large{Titolo}}
\vspace{-1em}
\begin{center}
    \small{(12/11/2018)}
\end{center}
\section{Conseguenze degli assiomi} %Nuovo capitolo
\textbf{Principio di indeterminazione}. Possiamo ora enunciare il principio di indeterminazione\index{Principio di indeterminazione} per una qualsiasi coppia di variabili (non siamo più confinati a posizione-momento).\\
Supponiamo di voler misurare simultaneamente due osservabili $A$ e $B$: che vincolo abbiamo sulla precisione della misura simultanea?\\
Una delle conseguenze degli assiomi è il \textit{principio di indeterminazione} che generalizza a coppie di osservabili arbitrarie quello trovato (euristicamente) da Heisenberg nel 1927 per $X$ e $P$.\\
Ricordiamo che la fluttuazione di una osservabile $A$ in uno stato $\psi$ è definita come:
\[
(\Delta A)_\psi = \sqrt{\langle(A-\langle A\rangle_\psi)^2\rangle_\psi}
\]
\begin{thm}
Siano $A$ e $B$ \marginpar{Principio di indeterminazione} autoaggiunti (ossia corrispondenti a osservabili) in $D(A)$ e $D(B)$. Supponiamo che esista un dominio $D \subset \{\psi \in \hs | \psi, A\psi, B\psi \in D(A)\cap D(B)\}$ denso in $\hs$ 
(ossia tale che applicare $A$ o $B$ ad un elemento dell'intersezione dei domini $D(A)\cap D(B)$ produce un risultato che resta ancora nell'intersezione). Questa scelta del dominio è tale che tutte le operazioni usate in seguito siano ben definite.\\
Vale allora, $\forall \psi \in D$:
\[
(\Delta A)_\psi (\Delta B)_\psi \geq \left |\frac{\langle [A,B]\rangle_\psi}{2i} \right |
\]
\end{thm}
\textbf{Dimostrazione}. Definiamo per semplicità di notazione:
\[
\bar{A}=A-\langle A \rangle_\psi \bb{I}; \quad \bar{B}=B-\langle B \rangle_\psi \bb{I}
\]
Allora:
\[
[A,B]=[\bar{A},\bar{B}]; \quad (\Delta A)^2_\psi = \langle \bar{A}^2 \rangle_\psi
\]
Per proprietà del prodotto scalare:
\[
\langle \left(\frac{\bar{A}}{(\Delta A)_\psi}\pm i \frac{\bar{B}}{(\Delta B)_\psi} \right)^\dag \left ( \frac{\bar{A}}{(\Delta A)_\psi}\pm i \frac{\bar{B}}{(\Delta B)_\psi }\right ) \rangle \geq 0
\]
dato che:
\[
\langle c^\dag c\rangle_= \langle c\psi | c\psi \rangle =\norm{c\psi}^2 \geq 0
\]
Sviluppando i prodotti:
\[
\underbrace{\frac{\langle \bar{A}^2 \rangle_\psi}{(\Delta A)^2_\psi}}_{=1} +
\underbrace{\frac{\langle \bar{B}^2\rangle_\psi}{(\Delta B)^2_\psi}}_{=1} \mp i \frac{\langle \bar{B}\bar{A}-\bar{A}\bar{B}\rangle_\psi}{(\Delta A)_\psi (\Delta B)_\psi}\geq 0
\]
(per come abbiamo definito $(\Delta A)^2_\psi$, $(\Delta B)^2_\psi$.\\
Ma allora:
\[
\mp \frac{\langle [\bar{A},\bar{B}]\rangle_\psi}{2i} = \mp \frac{\langle [A,B]\rangle_\psi}{2i}\leq (\Delta A)_\psi (\Delta B)_\psi \Rightarrow (\Delta A)_\psi (\Delta B)_\psi \geq \left |\frac{\langle [A,B]\rangle_\psi}{2i} \right |
\]
dato che:
\[
|x|\leq a \leftrightarrow \begin{cases}
x \leq a\\
-x \leq a
\end{cases}
\]

In particolare, se $A=X$ e $B=P$, $D=\mathcal{S}(\bb{R})$, $\forall \psi \in D$:
\[
(\Delta X)_\psi (\Delta P)_\psi \geq \left | 
\frac{\langle [X,P]\rangle_\psi}{2i}
\right | = \frac{\hbar}{2}
\]
Il principio di indeterminazione di Heisenberg è quindi \textbf{solo} una conseguenza dell'algebra $[X,P]=i\hbar$.\\

\textbf{Osservazioni}
\begin{itemize}
\item Il principio di indeterminazione vale anche per \textbf{stati misti}.\marginpar{Indeterminazione degli stati misti} Se $A$, $B \in \mathcal{B}(\hs)$, ($D=\hs$), definiamo:
\[
\langle A\rangle_\rho = \op{Tr} \rho A; \quad (\Delta A)^2_\rho = \op{Tr}(\rho(A-\op{Tr}(\rho A))^2
\]
Vale allora il teorema di prima:
\[
(\Delta A)_\rho (\Delta B)_\rho \geq \left |
\frac{\langle [A,B]\rangle_\rho}{2i}
\right|
\]
Infatti:
\[
\rho = \sum_i p_i \ket{\psi_i}\bra{\psi_i}; \quad \langle c^\dag c\rangle_\psi \geq 0 \Rightarrow \op{Tr}\rho(c^\dag c)=\sum_i p_i\bra{\psi_i}c^\dag c\ket{\psi_i}\geq 0
\]
\item Talvolta\marginpar{Importanza delle ipotesi del teorema} il principio di indeterminazione \textbf{sembra violato}, ma ciò è dovuto a \textit{problemi di dominio}, per cui cadono le ipotesi fatte nel teorema.\\
Per esempio, sia $\hs=L^2[-\frac{1}{2}, \frac{1}{2}], dx)$, $X$ è limitato in $\hs$, e quindi $D(X)=\hs$.\\
Per $P=-i\hbar\frac{d}{dx}$ con condizioni periodiche per $D(P)$ (in modo che sia autoaggiunto).\\
Poiché $X$ è limitato, $(\Delta X)_\psi \leq 1$ (non posso avere un'indeterminazione più grande del dominio!), se scegliamo come $\psi$ un autostato $\psi_n$ di $P$, dove $\sigma(P)$ è unicamente discreto, sappiamo che:
\[
(\Delta P)_{\psi_n} = 0
\]
Ma allora:
\[
(\Delta X)_{\psi_n}(\Delta P)_{\psi_n}=0
\]
in apparente violazione del principio di indeterminazione.\\
Tuttavia, notiamo che applicando $X$ a $\psi_n(x)$ non otteniamo un elemento nel dominio di $P$ (e quindi non può stare nell'intersezione tra $D(X)\cap D(P)$ come richiesto dalle ipotesi del teorema). Infatti:
\[
-i\hbar \frac{d}{dx}\psi_n(x)=\lambda_n \psi_n(x)\Rightarrow \psi_n(x)=\exp\left( -i\frac{\lambda_n}{\hbar}x\right)
\]
Imponendo le condizioni di contorno:
\[
\psi\left(\frac{1}{2}\right)\overset{!}{=}\psi\left(-\frac{1}{2}\right) \Rightarrow \lambda_n = 2\pi n\hbar
\]
E sostituendo nella soluzione:
\[
\psi_n(x)=e^{-i2\pi x n}
\]
Tuttavia:
\[
X\psi_n(x)=x\psi_n(x) \notin D(P)
\]
perché non è periodica.\\
Quindi:
\[
\psi_n \notin D=\{\psi\in \hs|\psi, X\psi, P\psi \in D(X)\cap D(P) = D(P)\}
\]

\end{itemize}

\subsection{Osservabili compatibili}
Il principio di indeterminazione implica che in generale non è possibile effettuare misure simultanee arbitrariamente precise se il commutatore $[A,B]\neq 0$, ma suggerisce che se $[A,B]=0$ ciò potrebbe essere possibile. Dimostriamo che, in effetti, \textbf{simultanea misurabilità} e \textbf{commutatività} sono in corrispondenza biunivoca.\\
Iniziamo col caso di spettro discreto e premettiamo una nozione matematica.\\
\begin{dfn}
Se $\hs_1$ \marginpar{Somma diretta} e $\hs_2$ sono spazi di Hilbert, l'insieme delle \textbf{coppie} $\{\psi_1 \in \hs_1, \psi_2 \in \hs_2\}$ denotate $\psi_1 \oplus \psi_2$ con il prodotto scalare:
\[
(\psi_1 \oplus \psi_2, \varphi_1 \oplus \varphi_2)\equiv (\psi_1, \varphi_1)_{\hs_1} + (\psi_2, \varphi_2)_{\hs_2}
\] 
è uno spazio di Hilbert detto \textbf{somma diretta} di $\hs_1$ e $\hs_2$ e denotato con $\hs_1 \oplus \hs_2$.\\
Nel caso finito dimensionale, $\hs_1 \cong \bb{C}^N$, $\hs_2 \cong \bb{C}^M$, si ha:
\[
\hs_1 \oplus \hs_2 \cong \bb{C}^{N+M}
\]
e i suoi vettori sono vettori colonna del tipo:
\[
\begin{pmatrix}
\psi_1 \in \bb{C}^N\\
\psi_2 \in \bb{C}^M
\end{pmatrix}
\]
Si noti che tra spazi di Hilbert in somma diretta non si ha \q{interferenza}, cioè tutti gli stati di $\hs_1$ pensati in $\hs_1 \oplus \hs_2$, ossia come $\psi_1 \oplus 0$, sono ortogonali a quelli di $\hs_2$ pensati in $\hs_1\oplus \hs_2$ come $0\oplus \psi_2$.
\end{dfn}

\begin{dfn}
Due osservabili\marginpar{Osservabili compatibili} \textbf{limitate} a \textbf{spettro discreto} si dicono \textbf{compatibili}\index{Compatibilità} se una misura di prima specie di una osservabile eseguita su un autostato dell'altra la lascia nello stesso autovalore (ma non necessariamente nello stesso autostato), cosicché l'ordine delle misure di $A$ e $B$ è irrilevante, e sia $A$ e $B$ sono simultaneamente misurabili. 
\end{dfn}

\begin{thm}
Se $A$ e $B$ sono autoaggiunti e limitati e a spettro discreto sono \textbf{compatibili} se e solo se \textbf{commutano}.\ Allora ammettono una \textbf{base ortonormale di autovettori comuni}.
\end{thm}
\textbf{Dimostrazione} Compatibilità $\Rightarrow$ commutatività.\\
Sia $\sigma(A)=\{\lambda_n\}_{n\in N}$, $\sigma(B)=\{\mu_m\}_{m\in M}$. Definiamo lo spazio degli autovettori comuni di $A$ e $B$ di autovalori $\lambda_n, \mu_m$:
\[
\hs_{(\lambda_n, \mu_m)}=\{\ket{\phi}\in \hs| A\ket{\phi}=\lambda_n \ket{\phi}, 
B\ket{\phi}=\mu_m \ket{\phi}
\}
\]
che è ben definito poiché se $\ket{\phi}\in \hs_{\lambda_n}= \{\ket{\psi}\in \hs| A\ket{\psi}=\lambda_n\ket{\psi}\}$ per la compatibilità $B\ket{\phi} \in \hs_{\lambda_n}$ e viceversa.\\
Se $\hs_{(\lambda_n, \mu_m)}\neq 0$ diremo che $(\lambda_n, \mu_m)$ autovalori comuni di $A$ e $B$ e lo spettro:
\[
\sigma(A,B)\equiv \{
(\lambda_n, \mu_m) \in \sigma(A) \times \sigma(B) \text{ autovalori comuni }
\} \subseteq \sigma(A)\times \sigma(B)
\]
(è solo un sottoinsieme, dato che per una certa scelta di $\lambda_n$ potrebbe essere che non tutti i $\mu_m$\ sono ammessi!)\\
Essendo $\hs_{(\lambda_n, \mu_m)}$ di Hilbert ammette una base ortonormale:
\[
\{\ket{(\lambda_n, \mu_m), j}, j = 1, \dots, \op{dim}\hs_{(\lambda_n, \mu_m)}\equiv d(\lambda_n, \mu_m)\}
\]
(dove $j$ indica la degenerazione dell'autovettore, nel caso alla stessa coppia di autovalori corrispondano più autovettori).\\
Inoltre, se $(\lambda_n, \mu_m)\neq (\lambda_{n'}, \mu_{m'})$ gli stati in $\hs_{(\lambda_n, \mu_m)}$ sono ortogonali agli stati in $\hs_{(\lambda_{n'}, \mu_{m'})}$ in quanto autostati di autovalori diversi di operatori autoaggiunti a spettro discreto.\\ %Come avevamo già visto, mettere un ref
Quindi:
\[
\hs = \bigoplus_{(\lambda_n, \mu_m)\in \sigma(A,B)} \hs_{(\lambda_n, \mu_m)}
\]
e $\{\ket{(\lambda_n, \mu_m), j}\}, j=1, \dots, d(\lambda_n, \mu_m), (\lambda_m, \mu_m)\in \sigma(A,B)\}$ è una base ortonormale in $\hs$ di autovettori comuni.\\
Pertanto posso esprimere ogni vettore di $\hs$ come combinazione lineare di elementi di questa base. $\forall \ket{\psi}\in \hs$ vale:
(*)\[
\ket{\psi}=\sum_{\substack{n\in N\\m\in M}} \sum_{j=1}^{d(\lambda_n, \mu_m)}c_{nmj}\ket{(\lambda_n, \mu_m),j}
\]
Essendo $\ket{(\lambda_n, \mu_m), j}$ autostati di $A$ e di $B$:
\[
(AB-BA)\ket{(\lambda_n, \mu_m), j} = (A\mu_m -B\lambda_n)\ket{(\lambda_n, \mu_m),j}= (\lambda_n \mu_m - \mu_m \lambda_n)\ket{(\lambda_n, \mu_m),j} =\ 0
\]
e per (*) si ha:
\[
[A,B]\ket{\psi} = 0; \quad \forall \ket{\psi}\in \hs
\]

Dimostriamo ora il viceversa, ossia che commutatività $\Rightarrow$ compatibilità.\\
Sia $[A,B]=0$, $\ket{\phi_n}$ autostato di $A$ di autovalore $\lambda_n$, allora:
\[
AB\ket{\phi_n}\underset{(a)}{=}BA\ket{\phi_n} \underset{(b)}{=} \lambda_n B\ket{\phi_n}
\]
In (a) si è usata la commutatività, e in $(b)$ il fatto che sia un autostato di $A$. Otteniamo quindi che $B\ket{\phi_n}$ è autostato di $A$ di autovalore $\lambda_n$.\\
Poiché $\hs_{\lambda_n}$ è ortogonale a $\hs_{\lambda_n'}$ perché $\lambda_n\neq \lambda_{n'}$ (autospazi di autovalori distinti), si ha che:
\[
\bra{\underbrace{\phi_{n'}}_{\in \hs_{\lambda_{n'}}}}\underbrace{B \ket{\phi_n}}_{\in \hs_{\lambda_n}}=0 \quad \lambda_n \neq \lambda_{n'}
\]
Quindi $B$ mappa $\hs_{\lambda_n}$ in se stesso e $\hs=\oplus_{n\in N}\hs_{\lambda_n}$ e $B$ ristretto a $\hs_{\lambda_n}$ (denotato $B\Big|_{\hs_{\lambda_n}}$) è autoaggiunto in $\hs_{\lambda_n}$, cioè $B$ è \q{diagonale a blocchi}:
\[
B=\begin{pmatrix}
\boxed{B\Big|_{\hs_{\lambda_1}}} & 0 & 0\\
0 & \boxed{B\Big|_{\hs_{\lambda_2}}} & 0\\
0 & 0 & \ddots
\end{pmatrix}
\]
Essendo $B\Big|_{\hs_{\lambda_n}}$ autoaggiunto in\ $\hs_{\lambda_n}$ ammette in $\hs_{\lambda_n}$ una base di autovettori, ciascuno corrispondente a una coppia $\lambda_n \in \sigma(A)$, $\mu_m \in \sigma(B)$ che denotiamo $\ket{(\lambda_n, \mu_m), j}$. Essi sono autovettori comuni di $A$ e $B$ e pertanto $A$ e $B$ sono compatibili.
\begin{flushright}
$\square$
\end{flushright}

Poiché gli operatori autoaggiunti sono in corrispondenza biunivoca con le loro famiglie spettrali e queste famiglie spettrali sono composte di operatori limitati a spettro discreto (sono proiettori, quindi con solo autovalori $0$ e $1$) è naturale la definizione:

\begin{dfn}
$A$ e $B$ si dice che \textbf{commutano}\marginpar{Commutatività di operatori} se commutano le corrispondenti famiglie spettrali. 
\end{dfn}
Nel caso di operatori limitati a spettro discreto le due cose sono equivalenti.\\
In effetti, se $A$ e $B$ sono limitati e a spettro discreto possiamo scrivere le famiglie spettrali:
\begin{align*}
P^A(\lambda) &= \sum_{\lambda_n \in \sigma(A)}\sum_{r=1}^{d(\lambda_n)}H(\lambda-\lambda_n)\ket{\lambda_n,r}\bra{\lambda_n, r}\\
P^B(\mu) &= \sum_{\mu_m \in \sigma(B)}\sum_{s=1}^{d(\mu_m)}H(\mu-\mu_m)\ket{\mu_m, s}\bra{\mu_m, s}
\end{align*}
sono le loro famiglie spettrali e quindi 
\[
[P^A(\lambda), P^B(\mu)]=0 \Leftrightarrow [\ket{\lambda_n,r}\bra{\lambda_n, r}, \ket{\mu_m, s}\bra{\mu_m, s}] = 0 \quad \forall \lambda_n, \mu_m
\]
Ma poiché:
\begin{align*}
A &= \sum_{\lambda_n \in \sigma(A)}\lambda_n \sum_{r=1}^{d(\lambda_n)}\ket{\lambda_n,r}\bra{\lambda_n,r}\\
B &= \sum_{\mu_m \in \sigma(B)} \mu_m \sum_{s=1}^{d(\mu_m)}\ket{\mu_m, s}\bra{\mu_m, s}
\end{align*}
e allora $\leftrightarrow [A,B]=0$ e la definizione coincide con la precedente.\\
Ma per $A$, $B$ autoaggiunti in generale entrano problemi di dominio nella definizione di commutatore e la definizione data (di \q{commutano}) risulta la definizione corretta  perché valga:
 \begin{thm}
 $A$ e $B$ autoaggiunti sono compatibili (proprietà fisica) se e solo se commutano (proprietà algebrica).
 \end{thm}
 (Dimostrazione omessa)\\
 
 Nella prossima lezione ci concentreremo sull'analizzare le caratteristiche di un insieme massimale su cui sono definite coppie di operatori.

\end{document}

