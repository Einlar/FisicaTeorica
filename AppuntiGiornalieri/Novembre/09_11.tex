\documentclass[../../FisicaTeorica.tex]{subfiles}

\begin{document}

\subsection{Buca finitamente profonda}
\begin{figure}[H]
\centering
\definecolor{qqqqff}{rgb}{0.,0.,1.}
\begin{tikzpicture}[line cap=round,line join=round,>=triangle 45,x=1.0cm,y=1.0cm,scale=0.55]
\draw[->,color=black] (-8.494387027773634,0.) -- (9.293322703354024,0.);
\foreach \x in {-8.,-7.,-6.,-5.,-4.,-3.,-2.,-1.,1.,2.,3.,4.,5.,6.,7.,8.,9.}
\draw[shift={(\x,0)},color=black] (0pt,-2pt);
\draw[->,color=black] (0.,-5.795646939212975) -- (0.,4.380206694230023);
\foreach \y in {-5.,-4.,-3.,-2.,-1.,1.,2.,3.,4.}
\draw[shift={(0,\y)},color=black] (2pt,0pt) -- (-2pt,0pt);
\clip(-8.494387027773634,-5.795646939212975) rectangle (9.293322703354024,4.380206694230023);
\draw [line width=1.2pt,color=qqqqff] (-8.494387027773634,0.)-- (-4.501858550554776,0.);
\draw [dash pattern=on 2pt off 2pt,color=qqqqff] (-4.501858550554776,0.)-- (-4.501858550554776,-4.720199120914791);
\draw [line width=1.2pt,color=qqqqff] (-4.501858550554776,-4.720199120914791)-- (4.501858550554776,-4.720199120914791);
\draw [dash pattern=on 2pt off 2pt,color=qqqqff] (4.501858550554776,-4.720199120914791)-- (4.501858550554776,0.);
\draw [line width=1.2pt,color=qqqqff] (4.501858550554776,0.)-- (9.,0.);
\draw (-5.241319392261099,1.5277626835956077) node[anchor=north west] {$-\frac{a}{2}$};
\draw (4.053159566346146,1.5918625490031226) node[anchor=north west] {$\frac{a}{2}$};
\draw (-7.084190220260811,-3.968800775098799) node[anchor=north west] {$\overline V < 0$};
\draw (-7.949538261234589,1.1431634911505182) node[anchor=north west] {$V = 0$};
\draw (8.31579991632809,-0.186908716055417) node[anchor=north west] {$x$};
\draw (-2.276701103739822,4.2359819970631145) node[anchor=north west] {$V(x)$};
\draw (-8.485438406347642,-1.7573554185395333) node[anchor=north west] {regione 1};
\draw (-3.818323745652703,-3.215627356560498) node[anchor=north west] {regione 2};
\draw (5.543481192467652,-1.8054303175951694) node[anchor=north west] {regione 3};
\begin{scriptsize}
\draw [fill=black] (0.,0.) circle (1.0pt);
\end{scriptsize}
\end{tikzpicture}

\caption{Potenziale della buca finitamente profonda}
\end{figure}

%\lesson{?}{9/11/2018}
Consideriamo ora il caso della \textit{buca di potenziale finita}, ovvero il sistema quantistico 1-dimensionale con potenziale
\[
V(x)=\bar{V}\chi_{\left[-\frac{a}{2},\frac{a}{2}\right]}(x)
\]
Se $\mathcal{E}>0$ la situazione è analoga a quella della barriera di potenziale per $\mathcal{E}>\bar{V}$, solo che in questo caso $\bar{V}<0$ mentre per la barriera $\bar{V}>0$. In entrambi i casi comunque $\sigma(H)$ è continuo con degenerazione $2$.\\
Anche qui c'è il fenomeno di \emph{risonanza di trasmissione}; una particella \emph{con abbastanza energia} per oltrepassare la barriera ha una certa probabilità di essere riflessa. In particolare per determinate energie la particella \emph{trascorre} più tempo nella regione della buca, e per altri di meno\footnote{rispettivamente al tempo calcolato classicamente}. Questo fenomeno si può intuire  se \textit{estendiamo} gli stati discreti presenti nella buca \q{al di sopra} di essa. Quando una particella ha un'energia tale per cui \textit{corrisponderebbe} a uno stato discreto \textit{esteso}, allora tale particella \q{è in risonanza} e trascorre più tempo nella regione della buca. In maniera analoga ai calcoli dei paragrafi precedenti, per $\bar{V}< \mathcal{E}<0$ se si pone:
\[
\chi=\frac{\sqrt{-2m\mathcal{E}}}{\hbar}; \quad k_2=\frac{\sqrt{2m(\mathcal{E}-\bar{V})}}{\hbar}
\]
allora le autofunzioni dell'hamiltoniana $H$ sono:
\[
\varphi_\mathcal{E}(x)=\begin{cases}
c^1_+ e^{\chi x} + c^1_- e^{-\chi x} & x<-\frac{a}{2} \quad \qquad \, (1)\\
c^2_+ e^{ik_2 x} + c^2_- e^{-ik_2 x} & -\frac{a}{2}<x<\frac{a}{2}\quad \, (2)\\
c^3_+ e^{ik_3 x}+c^3_- e^{-ik_3 x} & x > \frac{a}{2} \ \quad \ \qquad \, \, (3)
\end{cases}
\]
Poiché qualsiasi autofunzione (sia nel caso appartenga ad $\hs$ che ad $\mathcal{S}'$) non può divergere esponenzialmente, abbiamo immediatamente che $c^1_- = c^3_+ = 0$.\\

A questo punto non serve svolgere tutti i conti. Notiamo infatti che, dal caso già studiato della barriera con\ $\mathcal{E}<\bar{V}$ otteniamo il caso della buca finitamente profonda mediante la sostituzione $k_1 \to -i\chi$ e imponendo che $c^1_- = c^3_+ = 0$. Sempre in analogia con quanto fatto per la barriera si ha
\[
\begin{pmatrix}
c^3_+\\
c^3_-
\end{pmatrix} = (M'M)
\begin{pmatrix}
c^1_+\\
c^1_-
\end{pmatrix}
= (M'M)
\begin{pmatrix}
c^1_+\\
0
\end{pmatrix}
\]
Il primo termine diagonale di $M'M$ è nullo, infatti
\[
0 = c^3_+ = (M'M)_{11} \underbrace{c^1_+}_{\neq 0} \Rightarrow (M'M)_{11}=0 
\]
da cui, sostituendo $k_1$ con $-i \chi$ si ottiene
\begin{align}
0 = (M'M)_{11} & \underset{k_1\to-i\chi}{=}e^{-\chi a}
\left [
\frac{\chi^2-k_2^2-\overbrace{\sin k_2 a}^{(e^{ik_2 a}-e^{-ik_2 a})/(2i)} + \overbrace{\cos(k_2 a)}^{(e^{ik_2 a}+e^{-ik_2 a})/2} 2\chi k_2}{2\chi k_2}
\right ] \notag \\
& = e^{-\chi a}
\left [
\frac{\chi^2-k_2^2- \displaystyle\frac{e^{ik_2 a}-e^{-ik_2 a}}{2i} + \displaystyle\frac{e^{ik_2 a}+e^{-ik_2 a}}{2} 2\chi k_2}{2\chi k_2}
\right ] \notag
\end{align}
Semplificando si arriva alla seguente equazione:
\begin{equation}
\label{eqn:expandsquared}
(ik_2 + \chi)^2 e^{ik_2 a}=(-ik_2 + \chi)^2 e^{-ik_2 a}
\end{equation}
A questo punto estraendo la radice si ha che
\[
\frac{\chi + ik_2}{\chi-ik_2} = \pm e^{ik_2 a} \quad \Rightarrow \quad c^2_+ = \pm c^2_-
\]
Infatti valgono le uguaglianze:
\[
c^2_+ = M_{11}c^1_+; \quad c^2_- = M_{21} c^1_+
\]
dove i coefficienti $M_{11}$ e $M_{21}$ valgono:
\begin{align*}
M_{11} &= \frac{1}{2i}(\chi + ik_2)e^{ik_2 \frac{a}{2}}e^{-\chi\frac{a}{2}}\\
M_{21} &= -\frac{1}{2i}(\chi-ik_2)e^{ik_2 \frac{a}{2}}e^{-\chi\frac{a}{2}}
\end{align*}
Inoltre si noti che:
\[
\varphi_\mathcal{E}(-x)=\pm \varphi_\mathcal{E}(x)
\]
dunque prendendo il segno $+$ si ottiene una funzione pari, mentre con il $-$ una dispari. Dalla \eqref{eqn:expandsquared}, prendendo la parte reale o immaginaria si ottiene:
\begin{align*}
\op{Re}&\Rightarrow \frac{\chi^2-k_2^2}{\chi^2+k_2^2} = \pm \cos k_2 a\\
\op{Im} &\Rightarrow \frac{2k \chi}{\chi^2 + k_2^2}=\pm \sin k_2 a
\end{align*}
Pertanto si giunge all'equazione:
\[
-1+2\left(\frac{\mathcal{E}}{-\bar{V}}\right)=\frac{\chi^2-k_2^2}{\chi^2+k_2^2}=\pm \cos k_2 a = -1+2\begin{cases}
\cos^2\left(k_2\frac{a}{2}\right)\\
\sin^2\left(k_2 \frac{a}{2}\right)
\end{cases}
\]
Ed esplicitando il seno e coseno si ottengono le soluzioni finali, date da:
\begin{align*}
(1):\>
\begin{cases}
|\cos \frac{k_2 a}{2}| = \sqrt{\frac{\mathcal{E}}{-\bar{V}}} = \frac{k_2}{\bar{k}}\\
\sin k_2 a >0
\end{cases} \quad (2):\
\begin{cases}
|\sin \frac{k_2 a}{2}|= \sqrt{\frac{\mathcal{E}}{-\bar{V}}} = \frac{k_2}{\bar{k}}\\
\sin k_2 a < 0
\end{cases}
\end{align*}
\begin{figure}
\centering
\definecolor{zzttqq}{rgb}{0.6,0.2,0.}
\definecolor{wqwqwq}{rgb}{0.3764705882352941,0.3764705882352941,0.3764705882352941}
\definecolor{ccqqqq}{rgb}{0.8,0.,0.}
\definecolor{ffwwqq}{rgb}{1.,0.4,0.}
\begin{tikzpicture}[line cap=round,line join=round,>=triangle 45,x=1.0cm,y=1.0cm,scale=1.8]
\draw[->,color=black] (-0.3543430138962053,0.) -- (8.074306562308825,0.);
\foreach \x in {,0.5,1.,1.5,2.,2.5,3.,3.5,4.,4.5,5.,5.5,6.,6.5,7.,7.5,8.}
\draw[shift={(\x,0)},color=black] (0pt,-2pt);
\draw[->,color=black] (0.,-0.6826625623217666) -- (0.,2.57489659821151);
\foreach \y in {-0.5,0.5,1.,1.5,2.,2.5}
\draw[shift={(0,\y)},color=black] (2pt,0pt) -- (-2pt,0pt);
\clip(-0.3543430138962053,-0.6826625623217666) rectangle (8.074306562308825,2.57489659821151);
\fill[line width=0.pt,color=zzttqq,fill=zzttqq,fill opacity=0.1] (1.5707963267948966,2.590083354204672) -- (1.5707963267948966,0.) -- (3.1462042425276855,0.) -- (3.1462042425276855,2.590083354204672) -- cycle;
\fill[line width=0.pt,color=zzttqq,fill=zzttqq,fill opacity=0.1] (4.71238898038469,0.) -- (6.2898627331122645,0.) -- (6.2898627331122645,2.590083354204672) -- (4.71238898038469,2.590083354204672) -- cycle;
\draw[line width=1.2pt,color=ffwwqq,smooth,samples=100,domain=-0.3543430138962053:8.074306562308825] plot(\x,{2.0*abs(cos(((\x))*180/pi))});
\draw [color=ccqqqq,domain=-0.3543430138962053:8.074306562308825] plot(\x,{(-0.--1.5118236786901609*\x)/7.633890638507108});
\draw (1.4756610832798418,-0.07519232259528148) node[anchor=north west] {$\frac{\pi}{a}$};
\draw (3.0398969505755504,-0.04481881060895723) node[anchor=north west] {$\frac{2\pi}{a}$};
\draw (4.55097917189519,-0.037225432612376166) node[anchor=north west] {$\frac{3\pi}{a}$};
\draw (6.206335575149872,-0.014445298622632975) node[anchor=north west] {$\frac{4\pi}{a}$};
\draw [dash pattern=on 1pt off 1pt,color=wqwqwq] (1.5707963267948966,0.) -- (1.5707963267948966,2.57489659821151);
\draw [dash pattern=on 1pt off 1pt,color=wqwqwq] (3.1462042425276855,0.) -- (3.1462042425276855,2.57489659821151);
\draw [dash pattern=on 1pt off 1pt,color=wqwqwq] (4.71238898038469,0.) -- (4.71238898038469,2.57489659821151);
\draw [dash pattern=on 1pt off 1pt,color=wqwqwq] (6.2898627331122645,0.) -- (6.2898627331122645,2.57489659821151);
\draw [dash pattern=on 1pt off 1pt,color=wqwqwq,domain=0.0:8.074306562308825] plot(\x,{(--6.292408485055371-2.126671484248277E-5*\x)/3.1462042425276855});
\draw (7.656670772496864,-0.029632054615795104) node[anchor=north west] {$k_2$};
\draw (7.633890638507121,1.5421971906764853) node[anchor=north west] {$\displaystyle\frac{k_2}{\overline k}$};
\draw (0.23794046983712117,2.5976767322012533) node[anchor=north west] {$\left| \cos{\left(\frac{k_2 a}{2}\right)}\right|$};
\draw (1.4932968730918027,2.3774687703004025) node[anchor=north west] {regione proibita};
\draw (4.616581851690057,2.3622820143072403) node[anchor=north west] {regione proibita};
\draw (-0.22525558795432643,2.2939416123380107) node[anchor=north west] {$1$};
\begin{scriptsize}
\draw [fill=black] (0.,0.) circle (1.5pt);
\draw [fill=black] (1.5707963267948966,0.) circle (1.5pt);
\draw [fill=black] (4.71238898038469,0.) circle (1.5pt);
\draw [fill=black] (1.4288359127729424,0.28296815715393125) circle (1.5pt);
\draw [fill=black] (4.27526166628322,0.8466772876040689) circle (1.5pt);
\draw [fill=black] (7.077445924389052,1.4016247860779527) circle (1.5pt);
\draw [fill=black] (3.1462042425276855,0.) circle (1.5pt);
\draw [fill=black] (6.2898627331122645,0.) circle (1.5pt);
\end{scriptsize}
\end{tikzpicture}

\vspace{1cm}

\definecolor{zzttqq}{rgb}{0.6,0.2,0.}
\definecolor{wqwqwq}{rgb}{0.3764705882352941,0.3764705882352941,0.3764705882352941}
\definecolor{ccqqqq}{rgb}{0.8,0.,0.}
\definecolor{ffwwqq}{rgb}{1.,0.4,0.}
\begin{tikzpicture}[line cap=round,line join=round,>=triangle 45,x=1.0cm,y=1.0cm,scale=1.8]
\draw[->,color=black] (-0.3948520356962474,0.) -- (8.379484009221937,0.);
\foreach \x in {,0.5,1.,1.5,2.,2.5,3.,3.5,4.,4.5,5.,5.5,6.,6.5,7.,7.5,8.}
\draw[shift={(\x,0)},color=black] (0pt,-2pt);
\draw[->,color=black] (0.,-0.8150177690400646) -- (0.,2.57614454021208);
\foreach \y in {-0.5,0.5,1.,1.5,2.,2.5}
\draw[shift={(0,\y)},color=black] (2pt,0pt) -- (-2pt,0pt);
\clip(-0.3948520356962474,-0.8150177690400646) rectangle (8.379484009221937,2.57614454021208);
\fill[line width=0.pt,color=zzttqq,fill=zzttqq,fill opacity=0.1] (0.,2.57489659821151) -- (0.,0.) -- (1.566781619238815,0.) -- (1.566781619238815,2.57489659821151) -- cycle;
\fill[line width=0.pt,color=zzttqq,fill=zzttqq,fill opacity=0.1] (3.1462042425276855,2.57489659821151) -- (3.1462042425276855,0.) -- (4.718033487819975,0.) -- (4.718033487819976,2.57489659821151) -- cycle;
\draw[line width=1.2pt,color=ffwwqq,smooth,samples=100,domain=-0.3948520356962474:8.379484009221937] plot(\x,{2.0*abs(sin(((\x))*180/pi))});
\draw [color=ccqqqq,domain=-0.3948520356962474:8.379484009221937] plot(\x,{(-0.--1.5118236786901609*\x)/7.633890638507108});
\draw (1.4785872820025,-0.07196588775870891) node[anchor=north west] {$\frac{\pi}{a}$};
\draw (3.0437391170419597,-0.04034665876801291) node[anchor=north west] {$\frac{2\pi}{a}$};
\draw (4.553557301347701,-0.03244185152033892) node[anchor=north west] {$\frac{3\pi}{a}$};
\draw (6.205662016111575,-0.008727429777316923) node[anchor=north west] {$\frac{4\pi}{a}$};
\draw [dash pattern=on 1pt off 1pt,color=wqwqwq] (3.1462042425276855,0.) -- (3.1462042425276855,2.57614454021208);
\draw [dash pattern=on 1pt off 1pt,color=wqwqwq] (6.2898627331122645,0.) -- (6.2898627331122645,2.57614454021208);
\draw (7.660146549683599,-0.02453704427266492) node[anchor=north west] {$k_2$};
\draw (7.6364321279405765,1.5485195980144606) node[anchor=north west] {$\displaystyle\frac{k_2}{\overline k}$};
\draw (1.7394459211757431,2.599858961955102) node[anchor=north west] {$\left| \sin{\left(\frac{k_2 a}{2}\right)}\right|$};
\draw (-0.06705658035604597,2.3943339735155784) node[anchor=north west] {regione proibita};
\draw (3.053645828691943,2.354809937277208) node[anchor=north west] {regione proibita};
\draw (-0.22094627624741847,2.291571479295816) node[anchor=north west] {$1$};
\draw [dash pattern=on 1pt off 1pt,color=wqwqwq] (1.566781619238815,0.) -- (1.566781619238815,2.57614454021208);
\draw [dash pattern=on 1pt off 1pt,color=wqwqwq] (4.718033487819975,0.) -- (4.718033487819975,2.57614454021208);
\draw [dash pattern=on 1pt off 1pt,color=wqwqwq,domain=0.0:8.379484009221937] plot(\x,{(--3.133537985318499-0.*\x)/1.566781619238815});
\draw (6.204425462532493,2.3785243590202305) node[anchor=north west] {(regione proibita...)};
\begin{scriptsize}
\draw [fill=black] (0.,0.) circle (1.5pt);
\draw [fill=black] (8.436027293701917,1.670679660035962) circle (1.5pt);
\draw [fill=black] (3.1462042425276855,0.) circle (1.5pt);
\draw [fill=black] (6.2898627331122645,0.) circle (1.5pt);
\draw [fill=black] (2.8549829065558936,0.5654037976949899) circle (1.5pt);
\draw [fill=black] (5.685229235738035,1.1259087361975673) circle (1.5pt);
\draw [fill=black] (1.566781619238815,0.) circle (1.5pt);
\draw [fill=black] (4.718033487819975,0.) circle (1.5pt);
\end{scriptsize}
\end{tikzpicture}
\caption{(1) e (2) in funzione di $k_2$, con il plot del membro di sinistra, e il membro di destra con le intersezioni (cioè le soluzioni).}
\label{fig:solution1and2}
\end{figure}
dove:
\[
\bar{k}=\frac{\sqrt{-2m\bar{V}}}{\hbar}
\]
Le soluzioni (1) e (2) possono sono esplicitate graficamente in figura \ref{fig:solution1and2}, e sono l'intersezione tra la retta e le funzioni seno e coseno in valore assoluto (dove sono accettate solo quelle nelle regioni non proibite). Esse sono le soluzioni dell'equazione agli autovalori per $H$ con $\mathcal{E}<0$. Notiamo che lo spettro dell'hamiltoniana è unicamente discreto, e quindi le autofunzioni $\varphi_\mathcal{E}(x)$ appartengono a $L^2$. Inoltre $\sigma(H)|_{\mathcal{E}<0}=\sigma_P(H)$ è non degenere (degenerazione $1$). Se $\bar{V}\to\infty$ la retta $k_2/\bar{k}$ tende all'asse $x$ ed esaminandone le intersezioni con il coseno e il seno ritroviamo i risultati della buca infinitamente profonda.\\

\subsection{Regole generali per $\sigma(H)$ per potenziali in $1D$ (con al più salti)}
\begin{figure}[H]
\centering
\definecolor{qqqqff}{rgb}{0.,0.,1.}
\begin{tikzpicture}[line cap=round,line join=round,>=triangle 45,x=1.0cm,y=1.0cm]
\draw[->,color=black] (-6.676479077874541,0.) -- (8.002026159205059,0.);
\foreach \x in {-6.,-5.,-4.,-3.,-2.,-1.,1.,2.,3.,4.,5.,6.,7.,8.}
\draw[shift={(\x,0)},color=black] (0pt,-2pt);
\draw[->,color=black] (0.,-1.67819159971909) -- (0.,3.569434852014872);
\foreach \y in {-1.,1.,2.,3.}
\draw[shift={(0,\y)},color=black] (2pt,0pt) -- (-2pt,0pt);
\clip(-6.676479077874541,-1.67819159971909) rectangle (8.002026159205059,3.569434852014872);
\draw [shift={(-5.990479550422295,-0.503060335720243)},line width=1.2pt,color=qqqqff]  plot[domain=0.921463135200873:1.5745998322606323,variable=\t]({1.*2.503078441283073*cos(\t r)+0.*2.503078441283073*sin(\t r)},{0.*2.503078441283073*cos(\t r)+1.*2.503078441283073*sin(\t r)});
\draw [shift={(-4.087433525065524,1.9823996714100185)},line width=1.2pt,color=qqqqff]  plot[domain=4.042484887064738:5.68560406525681,variable=\t]({1.*0.6273784139167422*cos(\t r)+0.*0.6273784139167422*sin(\t r)},{0.*0.6273784139167422*cos(\t r)+1.*0.6273784139167422*sin(\t r)});
\draw [shift={(-3.4056119847292083,1.5137154376894562)},line width=1.2pt,color=qqqqff]  plot[domain=0.5137956939737662:2.524824404105156,variable=\t]({1.*0.20002216638462805*cos(\t r)+0.*0.20002216638462805*sin(\t r)},{0.*0.20002216638462805*cos(\t r)+1.*0.20002216638462805*sin(\t r)});
\draw [shift={(-2.9526513154048626,1.782574813850848)},line width=1.2pt,color=qqqqff]  plot[domain=3.6906518165667066:5.6252594582745346,variable=\t]({1.*0.32679849406407757*cos(\t r)+0.*0.32679849406407757*sin(\t r)},{0.*0.32679849406407757*cos(\t r)+1.*0.32679849406407757*sin(\t r)});
\draw [shift={(-0.9591334354802966,0.12882822270892555)},line width=1.2pt,color=qqqqff]  plot[domain=1.6195474014402678:2.4440924849192416,variable=\t]({1.*2.263597170879376*cos(\t r)+0.*2.263597170879376*sin(\t r)},{0.*2.263597170879376*cos(\t r)+1.*2.263597170879376*sin(\t r)});
\draw [shift={(-1.0778080450689216,1.4151124162679072)},line width=1.2pt,color=qqqqff]  plot[domain=0.3193308056715693:1.5622132016919401,variable=\t]({1.*0.9746595027453426*cos(\t r)+0.*0.9746595027453426*sin(\t r)},{0.*0.9746595027453426*cos(\t r)+1.*0.9746595027453426*sin(\t r)});
\draw [shift={(-39.042773991560445,-14.365568665496033)},line width=1.2pt,color=qqqqff]  plot[domain=0.3360384106053099:0.3922105553849662,variable=\t]({1.*42.08610264399796*cos(\t r)+0.*42.08610264399796*sin(\t r)},{0.*42.08610264399796*cos(\t r)+1.*42.08610264399796*sin(\t r)});
\draw [shift={(1.165296596585405,-0.3180909616447532)},line width=1.2pt,color=qqqqff]  plot[domain=3.4839282860822745:5.834909745929153,variable=\t]({1.*0.5052276175177605*cos(\t r)+0.*0.5052276175177605*sin(\t r)},{0.*0.5052276175177605*cos(\t r)+1.*0.5052276175177605*sin(\t r)});
\draw [shift={(3.2013198737488575,1.237545130990266)},line width=1.2pt,color=qqqqff]  plot[domain=0.5404927744988528:2.7834973068748012,variable=\t]({1.*0.7488204882226647*cos(\t r)+0.*0.7488204882226647*sin(\t r)},{0.*0.7488204882226647*cos(\t r)+1.*0.7488204882226647*sin(\t r)});
\draw [shift={(5.964690534938655,2.98677397119538)},line width=1.2pt,color=qqqqff]  plot[domain=3.713006848920577:4.202967265019543,variable=\t]({1.*2.5219323972745222*cos(\t r)+0.*2.5219323972745222*sin(\t r)},{0.*2.5219323972745222*cos(\t r)+1.*2.5219323972745222*sin(\t r)});
\draw [shift={(7.273630176071871,5.41887749611166)},line width=1.2pt,color=qqqqff]  plot[domain=4.211168131805285:4.690363138666036,variable=\t]({1.*5.283735136003091*cos(\t r)+0.*5.283735136003091*sin(\t r)},{0.*5.283735136003091*cos(\t r)+1.*5.283735136003091*sin(\t r)});
\draw [line width=1.2pt,color=qqqqff] (7.157260872005046,0.13642397785855528)-- (8.109235623814909,0.13642397785855528);
\draw [line width=1.2pt,color=qqqqff] (-6.,2.)-- (-6.7208191960118535,2.);
\draw (-1.2679573336892247,3.408540644768312) node[anchor=north west] {$V(x)$};
\draw (5.1554357675285765,1.3911748154460577) node[anchor=north west] {$V(x \to \infty) = 0 $};
\draw (7.234684401448789,-0.18063782457803) node[anchor=north west] {$x$};
\draw (0.8731737000500424,-0.9851088608108308) node[anchor=north west] {$V_\text{min}$};
\draw (-6.552714278236433,2.7649638157820715) node[anchor=north west] {$V(x \to -\infty) > 0$};
\draw [shift={(-32.53547830222711,15.417631168735921)},line width=1.2pt,color=qqqqff]  plot[domain=5.8460770521613865:5.9050571062125305,variable=\t]({1.*37.69861001701251*cos(\t r)+0.*37.69861001701251*sin(\t r)},{0.*37.69861001701251*cos(\t r)+1.*37.69861001701251*sin(\t r)});
\begin{scriptsize}
\draw [fill=qqqqff] (1.1610008339262132,-0.8233003161957635) circle (1.5pt);
\end{scriptsize}
\end{tikzpicture}
\caption{Potenziale $V(x)$ generico.}
\end{figure}

Consideriamo ora il potenziale generico $V(x)$ in figura; richiediamo che $V$ sia regolare con al più discontinuità di salto. Inoltre richiediamo che $V(x\to + \infty)=0$, e che $V(x\to-\infty) > 0$. Supponiamo poi che il minimo assoluto di $V$ sia negativo. Si hanno tre possibili situazioni qualitative per l'energia $\mathcal E$:




\subsubsection{Caso $\bm{V_{\text{min}} < \mathcal{E} < 0}$}
Le soluzioni sono del tipo $\varphi_\mathcal{E}\sim e^{\mp c_\pm x}$. La funzione non può divergere, e in particolare $\varphi_\mathcal{E}\in \hs$. Lo spettro è inoltre non degenere, cioè ha degenerazione 1. Per dimostrarlo supponiamo che $\psi_1$ e $\psi_2$ siano due soluzioni corrispondenti alla stessa energia $\mathcal{E}$, e dimostriamo che rappresentano lo stesso stato. Si ha che
\begin{align*}
\psi_1'' + \frac{2m}{\hbar^2}(\mathcal{E}-V(x))\psi_1 &= 0\\[0.6em]
\psi_2'' + \frac{2m}{\hbar^2}(\mathcal{E}-V(x))\psi_2 &= 0
\end{align*}
Dividendo la prima per $\psi_1$ e la seconda per $\psi_2$ si ottiene:
\begin{align*}
\frac{\psi_1''}{\psi_1}=\frac{2m}{\hbar^2}((V(x)-\mathcal{E})=\frac{\psi_2''}{\psi_2} & \quad \Rightarrow \quad \psi_1''\psi_2 - \psi_1\psi_2'' = 0\\
& \Rightarrow \quad \frac{d}{dx}(\psi_1' \psi_2 - \psi_1 \psi_2') = 0
\end{align*}
e quindi la funzione $\psi_1'\psi_2 - \psi_1 \psi_2'$ deve essere una costante. Tale funzione è regolare perché $\psi_{1,2}$ è derivabile due volte. Inoltre tale funzione è in $\hs$ (perché appartiene a $\mathcal{E}\in \sigma_P(H)$), ma quindi è quadrato sommabile e necessariamente tende a 0 all'infinito. L'unica funzione costante in $\hs$ è quella identicamente nulla. Quindi
\[
(\psi_1' \psi_2 - \psi_1 \psi_2')(x) = 0 \quad \Rightarrow \quad \frac{\psi_1'(x)}{\psi_1(x)} = \frac{\psi_2'(x)}{\psi_2(x)}
\]
e integrando entrambi i membri in $dx$ si ottiene
\[
\ln \psi_1(x) = \ln \psi_2(x) + c \quad
\Rightarrow \quad \psi_1(x) = e^{c} \, \psi_2(x)
\]
dove $c$ è una costante di integrazione. Quindi $\psi_1$ e $\psi_2$ differiscono per un fattore moltplicativo, ma quindi rappresentano lo \textbf{stesso stato}.
Segue che lo spettro  $\sigma(H)|_{V_\text{min} < \mathcal{E} < 0}$ è non degenere.

\subsubsection{Caso $\bm{0 < \mathcal{E} < V(-\infty)}$}
La soluzione ha un andamento del tipo:
\[
\varphi_\mathcal{E}(x) = \begin{cases}
c_+ e^{ikx} + c_-e^{-ikx} & x\to\infty\\
e^{\chi x} & x\to-\infty
\end{cases} \qquad \chi =\ \frac{\sqrt{2m(V(+\infty)-\mathcal{E})}}{\hbar}
\]
Poiché la soluzione $\varphi_{\mathcal E}$ oscilla infinitamente non può appartenere ad $\hs$, e quindi lo spettro $\sigma(H)|_{0<\mathcal{E}<V(-\infty)}$ dell'hamiltoniana è necessariamente continuo.
Inoltre siccome $\varphi_\mathcal{E}(-\infty) = 0$ possiamo applicare gli stessi ragionamenti del teorema visto al punto precedente e quindi affermare che $\sigma(H)$ è non degenere.

\subsubsection{Caso $\bm{\mathcal{E} > V(-\infty)}$}
Le soluzioni risultano
\[
\varphi_\mathcal{E}(x) = \begin{cases}
c^1_+ e^{ikx} + c^1_- e^{-ikx} & x\to +\infty\\
c^2_+ e^{ik_1 x} + c^2_- e^{-ik_1 x} & x\to -\infty
\end{cases}
\]
dove
\[
k_1 =\ \frac{\sqrt{2m(\mathcal{E}-V(-\infty))}}{\hbar}
\]
Anche in questo caso le autofunzioni non possono essere elementi di $\hs$, perchè presentano oscillazioni infinite. Quindi lo spettro $\sigma(H)|_{\mathcal{E}>V(-\infty)}$ è continuo. Inoltre lo spettro ha degenerazione 2, perché sono $4$ costanti da determinare a cui vanno sottratte le $2$ condizioni di raccordo ($4 - 2 = 2$).

\subsubsection{Caso della parità del potenziale}
Qualcosa in più sul sistema si può dedurre se si suppone che il potenziale sia una funzione pari, ovvero $V(x) = V(-x)$. In questo caso specifico se $\varphi_\mathcal{E}(x)$ è soluzione dell'equazione di Schrödinger stazionaria, allora anche $\varphi_\mathcal{E}(-x)$ lo è. Se poi supponiamo anche che lo spettro $\sigma(H)$ sia non degenere allora
\[\varphi_\mathcal{E}(x)=c \varphi_\mathcal{E}(-x) = c^2\varphi_\mathcal{E}(x) \quad \Rightarrow \quad c^2 = 1 \quad \Rightarrow \quad \varphi_{\mathcal E}(x) = \pm \varphi_{\mathcal E}(-x)
\]
Quindi ci sono solo due possibilità: la funzione d'onda può essere pari, dunque $\varphi_\mathcal{E}(x) =\varphi_\mathcal{E}(-x)$, oppure può essere dispari, ovvero $\varphi_\mathcal{E}(x)=-\varphi_\mathcal{E}(-x)$.\\
Se poi si definisce l'operatore di parità $\mathcal{P}$ come:
\[
(\mathcal{P}\psi)(x)=\psi(-x)
\]
allora si ha che gli autostati di $H$ sono anche autostati di $\mathcal{P}$, con autovalore $+1$ oppure $-1$.


\subsection{Esercizio \theEsercizio}\index{Esercizio!Buca infinita (1)}
Una particella di massa $m$ si trova nello stato fondamentale (cioè di energia minima) in una buca di potenziale infinita in $[0,a]$. Al tempo $t = 0$ la parete impenetrabile posta in $a$ viene improvvisamente spostata in $x = 2a$. \begin{enumerate}
\item Calcolare la probabilità che al tempo $\bar{t}>0$ l'energia della particella sia uguale a quella a $t=0^-$
\item Che sia invece minore.
\end{enumerate}
%[PLOT] della funzione d'onda a t=0^- e t=0^+
\begin{figure}[H]
\centering


\tikzset{every picture/.style={line width=0.75pt}} %set default line width to 0.75pt        

\begin{tikzpicture}[x=0.75pt,y=0.75pt,yscale=-1,xscale=1]
%uncomment if require: \path (0,382); %set diagram left start at 0, and has height of 382

%Straight Lines [id:da7445328699952485] 
\draw    (120.67,276.67) -- (235.17,276.67) ;

%Straight Lines [id:da5602696309049731] 
\draw    (120.67,276.67) -- (120.33,200) ;

%Straight Lines [id:da6575428372633421] 
\draw    (235.17,276.67) -- (234.83,200) ;

%Curve Lines [id:da7063477548895307] 
\draw [color={rgb, 255:red, 255; green, 0; blue, 0 }  ,draw opacity=1 ]   (120.67,276.67) .. controls (149.67,220) and (206.33,220) .. (235.17,276.67) ;

%Straight Lines [id:da41224257304284606] 
\draw    (303.67,276.67) -- (418.17,276.67) ;

%Straight Lines [id:da6792934740519954] 
\draw    (303.67,276.67) -- (303.33,200) ;

%Straight Lines [id:da7039651188453786] 
\draw    (532.67,276.67) -- (532.33,200) ;

%Curve Lines [id:da5751890548206524] 
\draw [color={rgb, 255:red, 255; green, 0; blue, 0 }  ,draw opacity=1 ]   (303.67,276.67) .. controls (332.67,220) and (389.33,220) .. (418.17,276.67) ;

%Straight Lines [id:da04877587672718464] 
\draw [color={rgb, 255:red, 255; green, 0; blue, 0 }  ,draw opacity=1 ][fill={rgb, 255:red, 255; green, 0; blue, 0 }  ,fill opacity=1 ]   (418.17,276.67) -- (532.67,276.67) ;

% Text Node
\draw (155,162) node   {$t=0^{-}$};
% Text Node
\draw (121,292) node   {$0$};
% Text Node
\draw (236,292) node   {$a$};
% Text Node
\draw (332,163) node   {$t=0^{+}$};
% Text Node
\draw (304,292) node   {$0$};
% Text Node
\draw (419,292) node   {$a$};
% Text Node
\draw (531,291) node   {$2a$};

\end{tikzpicture}

\caption{Funzione d'onda $\psi$ per $t=0^-$ e $t=0^+$}
\end{figure}
\subsubsection{Soluzione}
\begin{enumerate}
\item Per $t<0$, $\sigma(H_1)$ è quello della buca in $[0,a]$, e gli autovalori sono:
\[
\mathcal{E}^1_n = \frac{\hbar^2}{2m}\frac{\pi^2 n^2}{a^2}
\]
Per $t>0$, la buca è in $[0,2a]$ e gli autovalori saranno:
\[
\mathcal{E}^2_n = \frac{\hbar^2}{2m}\frac{\pi^2 n^2}{4a^2}
\]
Dopo il cambiamento del sistema, la $\psi$ non è più autofunzione di $H$. Confrontando le due espressioni per gli autovalori si ha che $\mathcal{E}^1_1 = \mathcal{E}^2_2$, quindi per rispondere alla prima richiesta occorre calcolare $W_{\psi(t>0)}^{H_2} (\mathcal{E}_2^2)$, ossia la probabilità che una misura di energia a $t>0$ dia come risultato $\mathcal{E}_2^2$. In notazione di Dirac:
\begin{align*}
W_{\psi(t)}^{H_2}(\mathcal{E}_2^2)&=|\braket{\psi(t)|\mathcal{E}_2^2}|^2 = |\bra{\psi(0^+)}e^{i\frac{t}{\hbar}H_2}\ket{\mathcal{E}_2^2}|^2 = \\
&=|\braket{\psi(0^+)|\mathcal{E}^2_2} e^{i\frac{t}{\hbar}\mathcal{E}_2^2}|^2 = |\braket{\psi(0^+)|\mathcal{E}_2^2}|^2
\end{align*}
dove abbiamo semplificato l'evoluzione temporale riducendola al calcolo della $\psi$ in $t=0^+$.
Abbiamo quindi:
\[
\psi(x,t=0^+) = \psi(x,t=0^-) = \begin{cases}
\sqrt{\frac{2}{a}} \sin \frac{\pi}{a}x & x \in [0,a]\\
0 & \text{altrove}
\end{cases}
\]
ed è già normalizzata in $L^2([0,2a],dx)$ (era normalizzata prima, e abbiamo aggiunto solo degli zeri). Calcoliamo quindi l'autofunzione di $H_2$:
\[
\braket{x|\mathcal{E}_n^2}=\varphi_n(x)=\sqrt{\frac{1}{a}} \sin \frac{\pi x n}{2a}
\]
Vogliamo quindi calcolare il prodotto scalare per esprimere la $\psi$ sulla base delle autofunzioni di $H_2$:
\begin{align*}
\braket{\psi(0^+)|\mathcal{E}_2^2} &= \int_0^a \sqrt{\frac{2}{a}}\sin \frac{\pi x}{a} \sqrt{\frac{2}{a}}\sin \frac{2\pi x}{2a} dx +\int_a^{2a} 0\cdot \sqrt{\frac{1}{a}} \sin \frac{2\pi x}{2a}dx =\\
&= \frac{\sqrt{2}}{a}\int_0^a \sin^2 \frac{\pi x}{a}= \frac{\sqrt{2}}{a}\frac{a}{2}=\frac{1}{\sqrt{2}}
\end{align*}
Ma quindi:
\[
W^{H_2}_{\psi(t>0)}(\mathcal{E}_2) = \frac{1}{2}
\]
\item Dato che l'unico autovalore di $H_2$ con energia minore di $\mathcal{E}_2^2=\mathcal{E}_1^1$ è quello per $n=1$, basta calcolare $W^{H_2}_{\psi(t)}(\mathcal{E}_1^2)$. Utilizzando la notazione di Dirac si ha:
\[
W^{H_2}_{\psi(t)}(\mathcal{E}_1^2) = |\braket{\psi(0^+)|\mathcal{E}_1^2}|^2
\]
Successivamente si calcola il prodotto scalare:
\begin{align*}
\braket{\psi(0^+)|\mathcal{E}_1^2} &= \int_0^a \sqrt{\frac{2}{a}} \sin \frac{\pi x}{a} \sqrt{\frac{1}{a}} \sin \frac{\pi x}{2a} dx\underset{(a)}{=} \frac{\sqrt{2}}{a} \int_0^a \frac{1}{2} \left [
\cos \frac{\pi a}{2a}-\cos \frac{3\pi x}{2a}
\right ]dx =\\
&\underset{(b)}{=}\frac{\sqrt{2}}{a}\frac{1}{2}\left [
\frac{2a}{\pi} \sin y \Big |_0^{\frac{\pi}{2}} - \frac{2a}{3\pi}\sin y \Big|_{0}^{\frac{3}{2}\pi}\right ] = \sqrt{2}\left [
\frac{1}{\pi}+\frac{1}{3\pi}
\right] = \sqrt{2}\frac{4}{3\pi}
\end{align*}
dove in (a) si è usata una delle formule di Werner:
\[
\sin \alpha \sin \beta= \frac{1}{2}[\cos(\alpha-\beta)-\cos(\alpha+\beta)]
\]
mentre in (b) si è effettuato il cambio di variabili $y=\pi x/(2a)$.
Arriviamo quindi al risultato finale:
\[
W_{\psi(t)}^{H_2}(\mathcal{E}_1^2)=\frac{32}{9\pi^2}
\]
\end{enumerate}
\end{document}