\documentclass[../../FisicaTeorica.tex]{subfiles}

\begin{document}

\section{Oscillatore armonico}

Occupiamoci ora di studiare il comportamento di un \textbf{oscillatore armonico}, che ha applicazioni per lo studio di svariati fenomeni, come le vibrazioni di cristalli, o per l'espansione dei campi in modi normali: si ha infatti che ogni moto periodico nei pressi di un \textit{minimo} dell'energia potenziale (ossia un punto di equilibrio) è ben approssimato da un oscillatore armonico\footnote{CFR \cite{griffiths} pag. 47.}.\\

In questa sezione ci limiteremo al caso \textbf{unidimensionale}. In \MC, dette $q$ e $p$ rispettivamente la posizione e il momento di un punto di massa $m$, legato ad una molla di costante elastica $k$ centrata all'origine, si ottiene la seguente Hamiltoniana:
\begin{align*}
H=\frac{p^2}{2m}+\frac{1}{2}m\omega^2 q^2\qquad \omega = \sqrt{\frac{k}{m}}
\end{align*}
dove $\omega$ è la \textit{pulsazione} del moto armonico, che è esplicitamente dato da:
\begin{align*}
q(t) = A\sin(\omega t) + B\cos(\omega t)
\end{align*}
Poiché $p$ e $q$ possono potenzialmente assumere qualsiasi valore ($\sigma(p)=\sigma(q)=\bb{R}$), e in \MQ lo spettro di una somma è la somma degli spettri, si ha che $\sigma(H) = \bb{R}$, ed è quindi \textit{puramente continuo}.\\

Scopriremo che, invece, in \MQ $\sigma(H)$ è \textit{discreto}, e un ICOC per il sistema è dato dal solo $H$.\\
Lavoriamo in rappresentazione $\{x\}$, fissando $\hs=L^2(\bb{R},dx)$. L'Hamiltoniana quantistica $H_0$ si ottiene sostituendo le \q{versioni quantistiche} di posizione e momento, ossia $P$ e $X$, che in rappresentazione $\{x\}$ sono dati da:
\begin{align*}
X = x \qquad P=-i\hbar\frac{d}{dx}
\end{align*}
Giungiamo allora a:
\[
H_0 = -\frac{\hbar^2}{2m}\frac{d^2}{dx^2}+m\frac{\omega^2}{2}x^2; \quad D(H_0)=\mathcal{S}(\bb{R})
\]
Si dimostra che $H_0^\dag \equiv H$ è autoaggiunto (cioè $H_0^\dag = (H_0^\dag)^\dag$) e questa è l'hamiltoniana dell'oscillatore armonico, se vogliamo essere matematicamente precisi. Ma per semplicità consideriamo in seguito le operazioni definite in $D(H_0)$.\\ 
L'equazione di Schr\"odinger stazionaria è allora:
\begin{align*}
\left[-\frac{\hbar^2}{2m} \frac{d^2}{dx^2} + \frac{1}{2}m\omega^2 x^2\right]\psi(x) = \mathcal{E}\psi(x)
\end{align*}
Per risolverla si può procedere in modo \textit{diretto} e \textit{analitico} tramite le serie di potenze, oppure - come faremo - sfruttando delle ingegnose proprietà algebriche, adottando tecniche simili a quelle usate nel caso del momento angolare.\\
Partiamo riscrivendo $H_0$ in una forma più \q{suggestiva}:
\begin{align*}
H_0 = \frac{1}{2m}\left[\left(\frac{\hbar}{i}\frac{d}{dx}\right)^2 + (m\omega x)^2 \right]=\frac{1}{2m}\left[P^2+(m\omega X)^2\right]
\end{align*}
L'idea è di \textit{fattorizzare} $H_0$, usando l'identità dell'algebra:
\begin{align*}
u^2+v^2 = (u-iv)(u+iv)
\end{align*}
Dato che stiamo lavorando con operatori che in genere \textit{non commutano}, la fattorizzazione non sarà perfetta - ma ci consentirà comunque di ottenere una soluzione.\\
Prima di fattorizzare, raccogliamo un fattore $m\omega$ per \q{simmetrizzare} i due quadrati, e anche un $\hbar$ per \q{adimensionalizzarli}. Tale passaggio serve unicamente per velocizzare i conti successivi, togliendo di torno delle fastidiose costanti:
\begin{align*}
H_0 = \frac{\hbar\omega}{2}\left[\underbrace{\frac{P^2}{\hbar m\omega}}_{B^2} + \underbrace{\frac{m\omega X^2}{\hbar}}_{A^2}\right] \to \hbar \omega \left(\frac{A-iB}{\sqrt{2}}\right)\left(\frac{A+iB}{\sqrt{2}}\right)
\end{align*}
$A$ e $B$ non sono altro che \q{riscalamenti} di momento $P$ e posizione $X$, e quindi li chiamiamo in accordo:
\begin{align*}
A\equiv X' \equiv \sqrt{\frac{m\omega}{\hbar}}X \qquad
B\equiv P' \equiv \frac{1}{\sqrt{m\hbar \omega}}P 
\end{align*}
I due termini ottenuti dal raccoglimento sono invece denotati con $a$ e $a^\dag$:
\begin{align*}
a=\frac{X'+iP'}{\sqrt{2}} \qquad a^\dag = \frac{X'-iP'}{\sqrt{2}}
\end{align*}
Notiamo che possiamo riscrivere $X$ e $P$ in termini di $a$ e $a^\dag$:
\begin{align*}
X = \sqrt{\frac{\hbar}{2m\omega}}(a+a^\dag) \qquad P = i\sqrt{\frac{\hbar m\omega}{2}}(a^\dag-a)
\end{align*}

Come ci si può aspettare, $a$ e $a^\dag$ \textbf{non} commutano, dato che sono combinazione di $X'$ e $P'$, che a loro volta non commutano:
\begin{align*}
[X,P]&=i\hbar \Rightarrow  [X',P']=
\left[\sqrt{\frac{m\omega}{\hlc{SkyBlue}{\hbar}}}X,\frac{1}{\sqrt{m\omega\hlc{SkyBlue}{\hbar}}}P\right]= \frac{[X,P]}{\hbar}=
i\\
[a,a^\dag]&=\frac{1}{2}[X'+iP', X'-iP'] = \frac{i}{2}[P',X']-\frac{i}{2}[X',P']=-\frac{i^2}{2}-\frac{i^2}{2}=1
\end{align*}

Ci aspettiamo allora che la \q{fattorizzazione} fatta sopra non sia \textit{perfetta}, ma introduca una differenza (evidenziata in giallo) data proprio dal commutatore.\\
Svolgendo infatti i conti:
\begin{align*}
a^\dag a=\left(\frac{X'-iP'}{\sqrt{2}}\right)\left(\frac{X'+iP'}{\sqrt{2}}\right) &= \frac{X'^2}{2}+\frac{P'^2}{2}+\hlc{Yellow}{\frac{i[X',P']}{2}}=\\
&=\frac{m\omega}{2\hbar} + \frac{p^2}{2m\omega\hbar}-\frac{1}{2} = H'-\frac{1}{2}\\
&\Rightarrow H =h\omega H' =h\omega\left(a^\dag a + \frac{1}{2}\right)
\end{align*}
Dove con $H'$ denotiamo l'Hamiltoniana da cui abbiamo rimosso le costanti $h\omega$ precedentemente raccolte - ossia tale che $H=h\omega H'$.\\

Definiamo allora l'operatore $N = a^\dag a$. Poiché $H$ è autoaggiunto in $D(H)$, anche $N$ lo è, e in effetti uno è funzione dell'altro:
\begin{align}
H=\hbar \omega \left(N+\frac{1}{2}\right) \quad \sigma(H) = \hbar \omega\left(\sigma(N)+\frac{1}{2}\right)
\label{eqn:relazione_spettrale}
\end{align}
Perciò, se troviamo lo spettro $\sigma(N)$, automaticamente abbiamo lo spettro $\sigma(H)$ che ci interessa.\\

Mimando quanto fatto nel caso del momento angolare, esaminiamo allora l'algebra di $N$, $a$ e $a^\dag$, studiandone i commutatori:
\begin{align*}
[N,a] &= [a^\dag a, a] = [a^\dag, a]a = -a\\
[N,a^\dag] &= [a^\dag a, a^\dag] = a^\dag[a, a^\dag] = a^\dag
\end{align*}
Denotando $a^\dag \equiv a_+$, $a\equiv a_-$, ritroviamo allora una relazione del tipo $[N,a_\pm]=\pm a_\pm$, che è del tutto simile alla relazione $[J_3, J_\pm]=\pm\hbar J_\pm$ precedentemente ottenuta. In un certo senso, perciò, comparando i commutatori con quelli ottenuti studiando il momento angolare, avremmo $a\sim J_-$, $a^\dag \sim J_+$, $N\sim J_3$.\\
Ciò ci suggerisce come continuare. Supponiamo infatti che esista un certo $\psi_\lambda$ autovettore di $N$ di autovalore $\lambda$, per cui $N\psi_\lambda =\ \lambda \psi_\lambda$.\\
Tale autovalore deve essere positivo\footnote{Ciò deriva da un fatto generale, per cui l'energia \q{associata} ad una qualsiasi soluzione normalizzabile dell'equazione di Schr\"odinger stazionaria deve essere sempre $\geq$ $V_{min}$. In questo caso, $N$ determina l'energia associata ad un autoket, dato che si può calcolare $H$ da esso, e $V_{min}=0$ per l'Hamiltoniana $H_0$ di partenza. L'idea, esaminata nel problema 2.2 a pag. 39 di \cite{griffiths}, è che da $\frac{d^2\psi}{dx^2}=\frac{2m}{\hbar^2}[V(x)-\mathcal{E}]\psi$, se $V(x)>\mathcal{E}$ la derivata seconda di $\psi$ e $\psi$ stessa hanno sempre lo stesso segno. Ma se $\psi$ è normalizzabile, deve \q{partire da $0$} per $x\to\pm\infty$, e non appena vi si allontana non può tornare indietro, data l'assenza di punti di flesso - non può perciò essere $\neq 0$ e normalizzabile allo stesso tempo.
}, ossia $\lambda \geq 0$.\\
Dimostriamolo. Applicando l'equazione agli autovalori al calcolo del valor medio, si ha:
\begin{align*}
(\psi_\lambda, N\psi_\lambda) = (\psi_\lambda, \lambda \psi_\lambda)=\lambda \norm{\psi_\lambda}^2
\end{align*}
Ma per la definizione di $N = a^\dag a$, vale anche:
\[
(\psi_\lambda, N\psi_\lambda) = (\psi_\lambda, a^\dag a\psi_\lambda) = (a\psi_\lambda, a\psi_\lambda)= \norm{a\psi_\lambda}^2 \geq 0
\]
E perciò $\lambda \geq 0$, e anche $\lambda = 0 \Leftrightarrow a\psi_\lambda = 0$.\\

Partendo da un qualsiasi autovettore $\psi_\lambda$ di $N$, notiamo che $a\psi_\lambda$ è ancora autovettore di $N$, ma di autovalore $\lambda-1$ (ammesso che $a\psi_\lambda \neq 0$), e analogamente lo è $a^\dag \psi_\lambda$, con autovalore $\lambda+1$. In altre parole, $a$ e $a^\dag$ rispettivamente \textit{abbassano} e \textit{alzano} gli autovalori di $N$.\\
Dimostriamolo:
\begin{align*}
N a\psi_\lambda&= ([N,a]+aN)\psi_\lambda = (-a +a\lambda) \psi_\lambda = (\lambda-1)a \psi_\lambda\\
N a^\dag \psi_\lambda &= ([N, a^\dag]+a^\dag N) \psi_\lambda = (a^\dag + a^\dag \lambda) \psi_\lambda = (\lambda+1) a^\dag
\end{align*} 
Per questa ragione, $a$ è chiamato \textbf{operatore di distruzione} (o annichilazione), dato che \textit{abbassa} un autovalore, e $a^\dag$ \textbf{operatore di creazione}\footnote{Una motivazione migliore di tali nomi deriva dal fatto che i campi sono sviluppati come modi normali di oscillatori armonici, e in teoria dei campi le particelle sono \textit{eccitazioni} di questi campi, si ha che tali operatori effettivamente \textit{creano} e \textit{distruggono} particelle.}.
Ma allora iterando $n$-volte l'applicazione di $a$ si ha che anche $\lambda-n$ è un autovalore corrispondente all'autovettore $a^n\psi_\lambda$ (se è diverso da $0$).\\
Tuttavia, abbiamo dimostrato che $\lambda \geq 0$, ossia che tutti gli autovalori di $N$ sono positivi. Ma scegliendo un $n$ sufficientemente grande, si ha che $\lambda-n < 0$ - e ciò non può essere. L'unico modo è allora che $\lambda$ sia un intero positivo, che chiamiamo $m$, cosicché dopo $m$ iterazioni $a^m\psi_\lambda = (\lambda-m)\psi_\lambda = 0$ e la catena si interrompe. Avremo quindi una successione di autovalori di $N$ dati da $m$, $m-1, \dots, 0$.\\
Ma allora applicando a $\psi_0$ (che esiste se esiste $\psi_\lambda$) potenze di $a^\dag$, possiamo alzare di quanto vogliamo gli autovalori, dato che non esiste in generale un \q{tetto} per l'energia. Partendo da $0$ e salendo \textit{a passi interi}, otteniamo che tutti i numeri naturali fanno parte dello spettro di $N$.\\

Naturalmente, tutto ciò funziona ammesso che esista un $\psi_\lambda$ iniziale da cui partire. Poiché partendo da un \textit{qualsiasi} $\psi_\lambda$ consente di generare tutti gli autovalori, conviene, per semplicità, cercare di dimostrare l'esistenza per $\lambda=0$, ossia risolvere l'equazione agli autovalori:
\begin{align*}
a\ket{\psi_0} = 0 \Rightarrow \frac{X'+iP'}{\bcancel{\sqrt{2}}} \ket{\psi_0} = 0
\end{align*}
Passando in rappresentazione in $\{x\}$ otteniamo un'equazione differenziale a variabili separabili:
\begin{align*}
\left(\sqrt{\frac{m\omega}{\hbar}}
x+i\left(-\frac{i\hbar}{\sqrt{m\hbar \omega}}\frac{d}{dx} \right)
\right)\psi_0(x) =0
\end{align*}
Con la sostituzione:
\begin{align*}
x'=\sqrt{\frac{m\omega}{\hbar}}x
\end{align*}
Il problema si semplifica notevolmente. Separando le variabili e integrando in $x'$:
\begin{align*}
\left(x'+\frac{d}{dx'}\right)\psi_0 (x') = 0 \Rightarrow \frac{\psi_0'(x')}{\psi_0(x')}=-x' \Rightarrow \ln|\psi_0(x')|=-\frac{x^2}{2}+c
\end{align*}
Dopodiché basta esponenziare per trovare la soluzione cercata, che ha la forma di una \textit{gaussiana}:
\[
\psi_0(x') = Ae^{-\frac{x^2}{2}} =\ A \exp\left(-\frac{m\omega}{2\hbar}x^2\right) \in L^2
\]
Dove $A$ è fissata dalla normalizzazione:
\[
A = \left(\frac{m\omega}{\pi \hbar}\right)^{1/4}
\]
Riconosciamo in $\psi_0(x')$ la funzione di Hermite $h_0(x')$. Possiamo ora iterare  $a^\dag$ per ottenere tutti gli altri autostati, che corrispondono ancora a funzioni di Hermite $h_n(x')$ di ordine $n$-esimo:
\[
(a^\dag)^n \psi_0(x) = \left(x'-\frac{d}{dx'}\right)^n h_0(x') \sim h_n(x')
\]
Più precisamente le funzioni di Hermite $h_n(x)$ sono definite da:
\begin{align*}
    h_n(x) = \left(2^n n! \sqrt{\pi}\right)^{-\frac{1}{2}}\left( x - \frac{d}{dx}\right)^n \exp\left(-\frac{x^2}{2}\right)
\end{align*}
Si dimostra che le $\{h_n(x)\}_{n\in \bb{N}}$ costituiscono una base ON\footnote{CFR pag. 17 di \cite{spazi_hilbert}} in $L^2(\bb{R},dx)$. Ne segue allora che gli autovalori $n\in \bb{N}$ esauriscono $\sigma(N)$, quindi scrivendo $h_n(x)=\braket{x|n}$ vale la completezza di Dirac $\sum_{n\in \bb{N}}\ket{n}\bra{n}=\bb{I}$. Perciò non c'è $\sigma_C(N)$ e lo spettro è non degenere.\\
Notiamo ora che:
\begin{enumerate}
\item Dalla relazione tra gli spettri in (\ref{eqn:relazione_spettrale}) ricaviamo $\sigma(H)$:
\begin{align*}
\sigma(H)=\sigma_P(H)=\hbar \omega\left(\bb{N}+\frac{1}{2}\right)
\end{align*}
Perciò, come conseguenza diretta dell'\textbf{algebra} di $X$ e $P$ ($\Leftrightarrow a, a^\dag$), si ha che lo spettro dell'energia dell'oscillatore armonico è \textbf{quantizzato}\footnote{In realtà abbiamo utilizzato anche l'esistenza della soluzione di un'equazione differenziale agli autovalori. In realtà si può dimostrare - in maniera più lunga e \textit{diccifile} senza questa ipotesi}.
\item Perché lo spettro è puramente \textbf{discreto} e \textbf{non degenere} dal teorema sulle osservabili compatibili segue che $H$ è un ICOC per l'oscillatore armonico quantistico.\\
\end{enumerate}

Facciamo, infine, qualche osservazione sui risultati ottenuti:
\begin{enumerate}
\item A differenza di quanto accade in \MC, lo spettro della somma di due operatori \textit{non} è la somma degli spettri: 
\begin{align*}
\sigma(H)\sim\sigma(X^2+ P^2) \neq \sigma(X^2) + \sigma(P^2)
\end{align*}
dato che $\sigma(X^2)$ e $\sigma(P^2)$ sono continui, mentre $\sigma(H)$ è discreto.
\item La famiglia spettrale di $H$ si ottiene, in analogia a quanto visto per $X$, sommando funzioni a gradino di Heaviside $H(\lambda)$ centrate sui singoli autovalori:
\[
P^H(\lambda) = \sum_{n\in \bb{N}}H\left(\lambda-\hbar\omega\left(n+\frac{1}{2}\right) \right)\ket{n}\bra{n}
\] 
Tramite questa definizione, possiamo dare esplicitamente il dominio di autoaggiuntezza di $H$ come:
\begin{align*}
D(H) &= \{\psi \in \hs \>|\> \int \lambda^2 d(\psi, P^H(\lambda)\psi) < \infty\} \\
&=\{\psi \in \hs \>|\> \ket{\psi}=\sum_{n\in \bb{N}}c_n\ket{n}, \quad \sum_{n\in \bb{N}} (\hbar \omega)^2 \left(n+\frac{1}{2}\right)^2 |c_n|^2 < \infty\}
\end{align*}

\item Alzare o abbassare autostati tramite $a^\dag$ e $a$ \textit{non} produce autostati normalizzati. In generale, la procedura introduce un fattore $c(n)$, per cui:
\begin{align*}
a^\dag \ket{n} = c_+(n) \ket{n+1} \qquad a\ket{n}=c_-(n)\ket{n-1}
\end{align*}
Fissiamo allora:
\begin{align*}
\braket{n|n} \overset{!}{=}1 \qquad \forall n
\end{align*}
e cerchiamo di determinare le costanti $c_+$ e $c_-$.\\
Partiamo considerando $a^+$, e applicandolo a sinistra e a destra di $\braket{n|n}$. Otteniamo due espansioni equivalenti:
\begin{align*}
\braket{a^\dag n | a^\dag n} &= |c_+(n)|^2 \braket{n+1|n+1}\\
\braket{a^\dag n | a^\dag n} &= \bra{n}a a^\dag \ket{n} = \bra{n}(\hlc{Yellow}{[a,a^\dag]}+\hlc{SkyBlue}{a^\dag a}) \ket{n} = \bra{n}(\hlc{Yellow}{1}+\hlc{SkyBlue}{N})\ket{n} \underset{(a)}{=} (n+1)\braket{n|n}
\end{align*} 
dove in (a) si è usato il fatto che $\ket{n}$ è autovalore di autovalore $n$ per $N$.\\
Uguagliando le due espressioni si ottiene:
\begin{align*}
|c_+(n)|^2 = (n+1) \Rightarrow c_+(n) = \sqrt{n+1}
\end{align*}
D'altro canto, ripetendo gli stessi ragionamenti nel caso di $a$ si ottiene:
\begin{align*}
\braket{an|an} &= |c_-(n)|^2 \braket{n-1|n-1}\\
\braket{an|an} &= \bra{n}a^\dag a \ket{n} = \bra{n}N\ket{n} \underset{(a)}{=} n\braket{n|n}
\end{align*}
Da cui:
\begin{align*}
|c_-(n)|^2 = n \Rightarrow c_-(n) = \sqrt{n}
\end{align*}
\textbf{Riepilogando}:
\begin{align}
a^\dag \ket{n} = \sqrt{n+1}\ket{n+1}\qquad a\ket{n}=\sqrt{n}\ket{n-1}
\label{eqn:normalizzazione-costruzione-distruzione}
\end{align}
\item Nella \textit{old quantum theory} Planck era giunto a $\mathcal{E}_n \cong \hbar \omega n$, da cui però manca il termine $\hbar \omega/2$, che dà l'\textit{energia di punto zero} $\hbar \omega (n+1/2)$, necessaria per il principio di indeterminazione.\\
Dimostriamo, infatti, che l'energia dell'oscillatore armonico ha un minimo $\neq 0$. Data una $\psi \in D(H)$ generica, il valor medio di $H$ è dato da:
\begin{align*}
(\psi, H\psi)&=\left(\psi, \frac{P^2}{2m}\psi\right) + m\frac{\omega^2}{2}(\psi, X^2\psi) =\\
&=\frac{1}{2m}\avg{P^2}_\psi + \frac{m\omega^2}{2}\avg{X^2}_\psi =\\
&\underset{(a)}{=}
\frac{1}{2m}\left[
(\Delta P)^2_\psi + \avg{P}_\psi^2
\right ] +\frac{m\omega^2}{2}\left[
(\Delta X)^2_\psi + \avg{X}^2_\psi
\right]\\
&\underset{(b)}{\geq} \frac{(\Delta P)^2_\psi}{2m} + \frac{m\omega^2}{2}\hlc{Yellow}{(\Delta X)^2_\psi} \underset{(c)}{\geq} \frac{(\Delta P)^2_\psi}{2m} + \frac{m\omega^2}{2}\hlc{Yellow}{\frac{\hbar^2}{4(\Delta P)^2_\psi}} \\
&\underset{(d)}{\geq} \frac{m\omega \hbar }{2 \cdot 2m} + \frac{m\omega^2}{2} \frac{\hbar^2 2}{4m\omega \hbar} = \frac{\hbar\omega}{2}
\end{align*}
Dove in $(a)$ abbiamo usato la formula per la fluttuazione:
\begin{align*}
(\Delta A)^2_\psi = \avg{A^2}_\psi - \avg{A}^2_\psi \Rightarrow \avg{A^2}_\psi = (\Delta A)^2_\psi + \avg{A}^2_\psi
\end{align*}
In $(b)$ si è maggiorata la somma rimuovendo i termini (positivi) $\avg{P}^2$ e $\avg{X}^2$, mentre in $(c)$ si è applicato il principio di indeterminazione di Heisenberg, in modo da ottenere un'espressione \textit{in una sola variabile} facile da minimizzare:
\begin{align*}
(\Delta X)^2_\psi (\Delta P)^2_\psi \geq \left(\frac{\hbar}{2}\right)^2 \Rightarrow  (\Delta X)^2_\psi \geq \frac{\hbar^2}{4(\Delta P)^2_\psi}
\end{align*}
In $(d)$, infine, si è maggiorata la somma con il suo \textit{minimo}, per cui $(\Delta P)^2 = m\omega \hbar/2$. Infatti, posto $y\equiv(\Delta P)^2_\psi$ per semplicità di notazione, il minimo è dato da:
\begin{align*}
(\Delta P)^2_\psi \equiv y &\Rightarrow \frac{d}{dy}\left[
\frac{y}{2m}+\frac{m\omega^2}{2}\frac{\hbar^2}{4y}
\right] \overset{!}{=} 0\\
&\Rightarrow \frac{1}{2m} + \frac{m\omega^2 \hbar^2}{8}\left(-\frac{1}{y^2}\right)=0\\
 &\Rightarrow y^2 = \frac{m^2 \omega^2 \hbar^2}{4}\Rightarrow y =\frac{m\omega \hbar}{2}
\end{align*}
Notiamo che lo stato fondamentale dell'oscillatore armonico, per cui $H=\hbar\omega/2$ soddisfa il principio di indeterminazione con \textit{l'uguaglianza}. In altre parole, in un sistema costituito da un oscillatore armonico nello stato fondamentale, si possono determinare posizione e momento con la \textit{minima incertezza}.
\end{enumerate}


\subsection{Esercizio \theEsercizio}\index{Esercizio!Oscillatore armonico} \stepcounter{Esercizio}
L'Hamiltoniana di una particella di massa $m$ in 1D è data da:
\[
H=\frac{P^2}{2m} + AX^2 + BX + C; \quad A,B,C \in \bb{R}, \> A > 0
\]
Fisicamente ciò corrisponde ad un oscillatore armonico in un campo elettrico costante o in presenza di forza gravitazionale.
\begin{enumerate}
\item Mostrare che il sistema con una opportuna traslazione di parametro $\alpha$ diventa equivalente a un oscillatore armonico, con hamiltoniana $H'$, determinando $\alpha$.
\item Si calcolino spettro e autofunzioni di $H$, denotando con $\psi_n(x)$ gli autostati dell'Hamiltoniana dell'oscillatore armonico $H'$ ottenuta in $1$, in rappresentazione $\{x\}$.
\item All'istante $t=0$ lo stato della particella sia descritto da:
\[
\ket{\psi}=\frac{1}{\sqrt{2}}(\ket{\phi_0}+\ket{\phi_1})
\]
ove le $\ket{\phi_n}$ sono i ket corrispondenti al livello energetico $n$-esimo (ordinati in modo crescente) di $H$.\\
Si determini il valor medio di $X$ per $t>0$.
\item Si dimostri che l'operatore $\mathcal{P}_1$ così definito:
\[
\mathcal{P}_1 \equiv \exp\left(i\frac{B}{2A\hbar}P\right) \mathcal{P} \exp\left(-i\frac{B}{2A\hbar}P\right)
\]
con $\mathcal{P}$ parità e $P$ momento, commuta con $H$.
\item Si esprima $\mathcal{P}_1$ in funzione di $H$ (dato che $H$ costituisce un ICOC, e i due operatori sono compatibili per il punto precedente, i due devono essere funzionalmente dipendenti).
\end{enumerate}

\subsubsection{Soluzione}
\begin{enumerate}
\item Una generica \textit{traslazione spaziale} $U(\alpha)$  ha la forma:
\begin{align*}
U(\alpha) =\exp\left(-i\frac{\alpha}{\hbar}P\right)\\
\end{align*}
In visuale di Heisenberg, gli operatori $X$ e $P$ sono \textit{trasformati} da $U(\alpha)$ secondo le relazioni:
\begin{align*}
U^\dag (\alpha)XU(\alpha) = X+\alpha \qquad
 U^\dag (\alpha)PU(\alpha) = P
\end{align*}
Come ci si aspetta, una traslazione spaziale altera solo le posizioni e non il momento.\\

Scriviamo la relativa trasformazione dell'Hamiltoniana:
\begin{align*}
H' = U^\dag(\alpha)HU(\alpha) &= \frac{P^2}{2m}+ A(X+\alpha)^2 + B(x+\alpha) + C=\\
&=\frac{P^2}{2m} + AX^2 + (B+2\alpha A)X + \alpha^2 A + B\alpha +C
\end{align*}
Vogliamo che la traslazione conduca all'Hamiltoniana di un oscillatore armonico, che nel caso generale ha la forma:
\begin{align*}
H=\frac{P^2}{2m}+\hlc{Yellow}{\frac{m^2 \omega^2}{2}} X^2 + V_0
\end{align*}

Dobbiamo allora imporre che il termine lineare in $X$  di $H'$ scompaia, ossia che $B+2\alpha A = 0$. Ciò significa che abbiamo fissato $\alpha$:
\[
\alpha = -\frac{B}{2A}
\]
E allora $H'$ è data da:
\[
H' = \frac{P^2}{2m} + \hlc{Yellow}{A}X^2 + \underbrace{\left(C-\frac{B^2}{4A}\right)}_{\equiv V_0}
\]
\item Cerchiamo ora spettro e autofunzioni di $H$. $U(\alpha)$ è una trasformazione unitaria, che perciò non intacca gli spettri, e quindi $\sigma(H)=\sigma(H')$. Ma $\sigma(H')$ è lo spettro di un oscilltore armonico, che è unicamente determinato una volta che sappiamo la sua $\omega$. Basta allora invertire:
\[
A=\frac{m\omega^2}{2}\Rightarrow \omega = \sqrt{\frac{2A}{m}}
\]
E quindi:
\[
\sigma(H) = \sigma(H') = \hbar \omega\left(\bb{N}+\frac{1}{2}\right) + V_0
\]

Occupiamoci ora delle autofunzioni. Siano $\ket{\psi_n}$ gli autostati di $H'$, che soddisfano l'equazione agli autovalori per $H'$:
\begin{align*}
H'\ket{\psi_n} = \left[\hbar \omega \left(n+\frac{1}{2}\right) + V_0\right] \ket{\psi_n}
\end{align*}

Ma $H'=U^\dag(\alpha) H U(\alpha)$. Moltiplicando a sinistra per $U(\alpha)$ entrambi i membri, otteniamo:
\begin{align*}
\cancel{\textcolor{Blue}{U(\alpha)}U^\dag(\alpha)}HU(\alpha)\ket{\psi_n} = \left[\hbar \omega \left(n+\frac{1}{2}\right)+V_0\right] \textcolor{Blue}{U(\alpha)}\ket{\psi_n}
\end{align*}

Perciò $\ket{\phi_n}\equiv U(\alpha)\ket{\psi_n}$ è soluzione dell'equazione agli autovalori per $H$. Passando allora in rappresentazione $\{x\}$, in notazione di Dirac si ha:
\begin{align*}
\psi_n(x) = \braket{x|\psi_n} \qquad \phi_n(x)=\braket{x|\phi_n}
\end{align*}
E gli autostati di $H$, che denotiamo con $\phi_n(x)$, si ottengono traslando le $\psi_n(x)$ tramite $U(\alpha)$:
\begin{align*}
\phi_n(x) = \bra{x}U(\alpha)\ket{\psi_n} = \psi_n(x-\alpha)
\end{align*}
ossia gli autostati di $H$ sono le funzioni di Hermite traslate di $\alpha = -B/(2A)$.
\item Vogliamo ora calcolare l'evoluto temporale del valor medio di $X$, ossia:
\begin{align}
\avg{X}_{\psi(t)} =\bra{\psi} \exp\left(\frac{itH}{\hbar}\right) X \exp\left(-\frac{itH}{\hbar}\right)\ket{\psi} \qquad \ket{\psi}=\frac{1}{\sqrt{2}}\left(\ket{\phi_0} + \ket{\phi}_1\right)
\label{eqn:valor-medio-punto3}
\end{align}
Per fare i conti, conviene passare ad $H'$, che è collegata ad $H$ da una traslazione. Partiamo notando che traslazione temporale $U(\alpha)$ ed evoluzione temporale $T(t)$ sono \textit{indipendenti}, e commutano tra loro: $[U(\alpha),T(t)]=0$. Perciò possiamo applicare le due trasformazioni nell'ordine che vogliamo. Per esempio, in visuale di Schr\"odinger, lo stato \textit{trasformato} $\ket{\psi'}$ si ottiene:
\begin{align*}
\ket{\psi'}=T(t)U(\alpha)\ket{\psi} = U(\alpha) T(t)\ket{\psi}
\end{align*}
Ma allora, passando in visuale di Heisenberg, possiamo traslare l'operatore di evoluzione temporale, oppure evolvere l'operatore di traslazione. Scegliamo la prima via (più semplice). Avremo allora:
\begin{align*}
\exp\left(\pm \frac{i}{\hbar}tH'\right)=U(\alpha)^\dag \exp\left(\pm \frac{i}{\hbar}tH\right)U(\alpha)
\end{align*}
Invertendo, in modo da ottenere la sostituzione da fare in (\ref{eqn:valor-medio-punto3}), giungiamo a:
\begin{align*}
\exp\left(\pm \frac{itH}{\hbar}\right) = U(\alpha) \exp\left(\pm \frac{itH'}{\hbar}\right) U(\alpha)^\dag
\end{align*}
Sostituendo allora in (\ref{eqn:valor-medio-punto3}):
\begin{align}\nonumber
(\ref{eqn:valor-medio-punto3}) &= \bra{\psi}U(\alpha) \exp\left(\frac{itH'}{\hbar}\right) \underbrace{U(\alpha)^\dag X U(\alpha) }_{X+\alpha}\exp\left(-\frac{itH'}{\hbar}\right) U^\dag(\alpha)\ket{\psi} =\\
&\underset{(a)}{=} \alpha + \bra{\psi}U(\alpha) \exp\left(\frac{itH'}{\hbar}\right) X \exp\left(-\frac{itH'}{\hbar}\right) U^\dag(\alpha)\ket{\psi} = \label{eqn:valor-medio-punto3_2}
\end{align}
dove in (a) abbiamo spostato fuori il parametro $\alpha$, che è lasciato invariato dalle esponenziali (dato che quelle a sinistra annullano quelle a destra).
Cominciamo a calcolare l'espressione \q{a ritroso}, partendo da destra, ossia da $U^\dag(\alpha)\ket{\psi}$.\\
Sappiamo che:
\begin{align*}
\ket{\phi_n} = U(\alpha)\ket{\psi_n}\Rightarrow  U(\alpha)^\dag \ket{\phi_n} =\ket{\psi_n}
\end{align*}
E allora:
\begin{align*}
U(\alpha)^\dag \ket{\psi} = U(\alpha)^\dag \left(\frac{\ket{\phi_0}+\ket{\phi_1}}{2}\right) = \frac{\ket{\psi_0}+\ket{\psi_1}}{2}
\end{align*}
Per $\bra{\psi}U(\alpha)$ basta prendere il coniugato di quanto appena trovato:
\begin{align*}
\bra{\psi}U(\alpha)=\frac{\bra{\psi_0}+\bra{\psi_1}}{2}
\end{align*}
Riportando il tutto in (\ref{eqn:valor-medio-punto3_2}):
\begin{align}
(\ref{eqn:valor-medio-punto3_2}) = \alpha + \left(\frac{\bra{\psi_0}+ \bra{\psi_1}}{\sqrt{2}}\right) \exp\left(\frac{itH'}{\hbar}\right) X \hlc{Yellow}{\exp\left(- \frac{itH'}{\hbar}\right)
\left( \frac{\ket{\psi_0}+\ket{\psi_1}}{\sqrt{2}}\right)}
\label{eqn:valor-medio-punto3_3}
\end{align}
Occupiamoci ora del termine evidenziato. Le $\ket{\psi_0}$ e $\ket{\psi_1}$ sono già autostati di $H'$, e perciò possiamo subito espandere l'esponenziale:
\begin{align*}
\exp\left(-\frac{itH'}{\hbar}\right) \left(
\frac{\ket{\psi_0}+\ket{\psi_1}}{\sqrt{2}}\right) &=
\frac{1}{\sqrt{2}}\left[\exp\left(-\frac{i}{\hbar}t\mathcal{E}_0\right)\ket{\psi_0} + \exp\left(-\frac{i}{\hbar}t \mathcal{E}_1\right)\ket{\psi_1} \right] =
\\
\span= \frac{1}{\sqrt{2}} \left[
\exp\left({-\frac{i}{\hbar}\left(\hbar\frac{\omega}{2} + V_0\right)t}\right)\ket{\psi_0} +
\exp\left({-\frac{i}{\hbar}\left(\frac{3}{2}\hbar \omega + V_0\right)t}\right)\ket{\psi_1}
\right]
\end{align*}
Dato che:
\begin{align*}
\mathcal{E}_n = \hbar \omega \left(n+\frac{1}{2}\right) + V_0
\end{align*}
Tutto ciò si può scrivere in maniera più semplice estraendo dagli esponenziali un termine $\exp\left(-i\left(\frac{1}{2}\omega t+\frac{V_0}{\hbar}\right)\right)$:
\begin{align*}
\frac{1}{\sqrt{2}}\exp\left(-i\left(\frac{1}{2}\omega t+\frac{V_0}{\hbar}\right)\right)\left[ \ket{\psi_0} +e^{-i\omega t} \ket{\psi_1}\right]
\end{align*}
Prendendo il complesso coniugato, abbiamo calcolato anche la rispettiva parte a sinistra. Sostituendo nell'espressione generale, e semplificando gli esponenziali comuni raccolti nel bra e nel ket (che sono solamente fattori moltiplicativi, e quindi li possiamo spostare):
\begin{align}
(\ref{eqn:valor-medio-punto3_3}) =\alpha + \frac{1}{2}\left[ \bra{\psi_0} + e^{i\omega t}\bra{\psi_1}
\right] X \left[ \ket{\psi_0} +e^{-i\omega t}\ket{\psi_1}\right]
\label{eqn:valor-medio-punto3_4}
\end{align}
\begin{comment}
\begin{align}\nonumber
(\ref{eqn:valor-medio-punto3_3}) = \alpha +\frac{1}{2}\left[\exp\left(\frac{i}{\hbar}\left(\hbar \frac{\omega}{2}+ V_0\right)t \right)\bra{\psi_0}+\exp\left(\frac{i}{\hbar}\left(\frac{3}{2}\hbar \omega + V_0\right)t\right)\bra{\psi_1}\right] X\\
\left[\exp\left(-\frac{i}{\hbar}\left(\hbar \frac{\omega}{2}+V_0\right)t\right) \ket{\psi_0} + \exp\left(-\frac{i}{\hbar}\left(\frac{3}{2}\hbar \omega + V_0\right) t \right)\ket{\psi_1}\right]
\end{align}
\end{comment}
Le $\ket{\psi_0}$ e $\ket{\psi_1}$ non sono però autostati di $X$. Tuttavia, possiamo riscrivere $X$ come combinazione di $a$ e $a^\dag$, di cui conosciamo l'azione sulle $\ket{\psi_n}$. Usiamo allora l'identità:
\begin{align*}
X=\left(\frac{\hbar}{2m\omega}\right)^{\frac{1}{2}} (a+a^\dag)
\end{align*}
Gli operatori $a$ e $a^\dag$ rispettivamente \textit{abbassano} e \textit{alzano} gli autostati. Perciò, dei $4$ braket che compaiono espandendo la (\ref{eqn:valor-medio-punto3_4}), ne restano solo $2$, come si dimostra usando le relazioni di normalizzazione ricavate dalla formula in (\ref{eqn:normalizzazione-costruzione-distruzione}) e l'ortonormalità delle $\ket{\psi_n}$:
\begin{align*}
\bra{\psi_0}a\ket{\psi_1} &= \braket{\psi_0|\sqrt{1}\psi_0}=1 && \bra{\psi_0} a^\dag \ket{\psi_1} =\braket{\psi_0|\sqrt{2}\psi_2}=0\\
\bra{\psi_1}a\ket{\psi_0} &= \braket{\psi_1|0}= 0 && \bra{\psi_1}a^\dag \ket{\psi_0} = \braket{\psi_1|\sqrt{1}\psi_1}=1
\end{align*}
E perciò arriviamo a:
\begin{align*}
(\ref{eqn:valor-medio-punto3_4}) &=\alpha+
\frac{1}{2}\left(\frac{\hbar}{2m\omega}\right)^\frac{1}{2} \left[\bra{\psi_0} + e^{i\omega t}\bra{\psi_1}\right](a+a^\dag) \left[\ket{\psi_0} + e^{-i\omega t}\ket{\psi_1}\right] =\\
&=\alpha + \frac{1}{2}\left(\frac{\hbar}{2m\omega}\right)^{\frac{1}{2}} \left[ \bra{\psi_0}a\ket{\psi_1} e^{-i\omega t} + \bra{\psi_1}a^\dag \ket{\psi_0} e^{i\omega t}\right] =\\
&=\alpha + \left(\frac{\hbar}{2m\omega}\right)^{\frac{1}{2}}\left[\frac{e^{i\omega t}+e^{-i\omega t}}{2}\right]=\alpha+\left(\frac{\hbar}{2m\omega}\right)^{\frac{1}{2}}\cos(\omega t)
\end{align*}
Come ci si potrebbe aspettare, il valor medio della misura \textit{oscilla} di moto armonico attorno ad $x=\alpha$, in analogia con il caso classico. 
\item Notiamo che l'operatore $\mathcal{P}_1$ così definito non è altro che la \textit{parità (anti)traslata}\footnote{Si noti che la posizione degli aggiunti è invertita rispetto alla traslazione \textit{diretta}. Intuitivamente, per applicare $\mathcal{P}$ a $H$, che non è \q{centrato} in $0$ ma in $\alpha$, dobbiamo prima antitraslarlo, poi calcolare $\mathcal{P}$ e poi ritraslare il risultato.} di $\alpha=-B/(2A)$: \begin{align*}
\mathcal{P}_1 \equiv \exp\left(\frac{i}{\hbar}\frac{B}{2A}P\right)\mathcal{P}\exp\left(-\frac{i}{\hbar}\frac{B}{2A}P\right)=U(\alpha)\mathcal{P}U(\alpha)^\dag
\end{align*}
Vogliamo verificare che $[\mathcal{P}_1, H]$, e che quindi il valor medio di $\mathcal{P}_1$ sia indipendente dall'evoluzione temporale del sistema. Procediamo per calcolo diretto:
\begin{align*}
[\mathcal{P}_1,H] &= [U(\alpha)\mathcal{P}U(\alpha)^\dag, H] = U(\alpha)\mathcal{P}U(\alpha)^\dag H - HU(\alpha)\mathcal{P}U(\alpha)^\dag=\\
&\underset{(a)}{=}U(\alpha) \mathcal{P}\underbrace{U(\alpha)^\dag H \textcolor{Blue}{U(\alpha)}}_{H'}\textcolor{Blue}{U^\dag(\alpha)} - \textcolor{Blue}{U(\alpha)}\underbrace{\textcolor{Blue}{U^\dag(\alpha)} HU(\alpha)}_{H'}\mathcal{P}U(\alpha)^\dag =\\
&=U(\alpha) \mathcal{P} H' U(\alpha)^\dag - U(\alpha)H' \mathcal{P}U(\alpha)^\dag = U(\alpha)[\mathcal{P},H']U(\alpha)^\dag \underset{(b)}{=} 0
\end{align*}
dove in (a) abbiamo moltiplicato opportunamente per $U(\alpha)U(\alpha)^\dag = \bb{I}$ in modo da passare a $H'$ ed evidenziare poi un commutatore, e in (b) abbiamo usato il fatto che:
\begin{align*}
[\mathcal{P},H']=0
\end{align*}
in quanto $H'$ è l'hamiltoniana di un oscillatore armonico, il cui potenziale è \textit{pari}.

\item Dato che $H$ è un ICOC, e $\mathcal{P}_1$ è compatibile con $H$ per il punto precedente, allora $\mathcal{P}_1 = f(H)$.\\
Partiamo scrivendo l'operatore parità, che possiamo poi \textit{controtraslare} per ottenere $\mathcal{P}_1 = U(\alpha)\mathcal{P}U(\alpha)^\dag$. Dall'espressione degli autostati dell'oscillatore armonico, dati dalle funzioni di Hermite:
\begin{align*}
\psi_n(x) \propto \left( x- \frac{d}{dx}\right)^{n} e^{-\alpha x^2}
\end{align*}
Notiamo che le $\psi_n(x)$ con $n$ pari sono funzioni pari, e quelle con $n$ dispari sono dispari. Possiamo allora ottenere la forma operatoriale di $\mathcal{P}$:
\begin{align*}
\mathcal{P}\ket{\psi_n} = (-1)^n \ket{\psi_n} = (-1)^N \ket{\psi_n}
\end{align*}
Basta quindi ricavare $N$ in funzione di $H$. Basta allora invertire la relazione:
\begin{align*}
H' = \hbar \omega\left(N+\frac{1}{2}\right) + V_0 \Rightarrow  N = \frac{H'-V_0}{\hbar \omega}-\frac{1}{2}
\end{align*}
E perciò otteniamo:
\begin{align*}
\mathcal{P} = (-1)^{\displaystyle\left[ \frac{H'-V_0}{\hbar \omega}-\frac{1}{2}\right]}
\end{align*}
Troviamo infine $\mathcal{P}_1$ \textit{controtraslando}:
\begin{align*}
\mathcal{P}_1 &= U(\alpha)\mathcal{P}U(\alpha)^\dag = (-1)^{\displaystyle\left[\frac{U(\alpha)H' U^\dag(\alpha) - V_0}{\hbar \omega}-\frac{1}{2} \right]} = \\
&=(-1)^{\displaystyle\left[\frac{H-V_0}{\hbar \omega}-\frac{1}{2} \right ]}
\end{align*}
\end{enumerate}
\end{document}

