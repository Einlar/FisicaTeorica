\documentclass[../../FisicaTeorica.tex]{subfiles}

\begin{document}

\section{Le traslazioni spaziali}
\label{sec:traslazioni_spaziali}
Possiamo ora facilmente discutere la rappresentazione delle \textit{traslazioni spaziali} seguendo lo stesso schema delle traslazioni temporali.\\
Partiamo considerando dall'\textbf{omogeneità dello spazio}, per cui i risultati di un esperimento non dipendono dal luogo in cui viene svolto. In \MQ ciò significa che, ogni traslazione delle coordinate $\vec{x} \to \vec{x}+\vec{a}$, con $\vec{a}\in \bb{R}^3$ non modifica le probabilità di transizione $|\braket{\phi|\psi}|^2$.\\
Consideriamo quindi il gruppo \textit{additivo} $G=(\bb{R}^3, +)$ delle \textit{traslazioni spaziali}. 
Per teorema di Wigner esiste allora una rappresentazione proiettiva $G\ni \vec{a} \mapsto \hat{U}(\vec{a}\,)$, che agisce su \textit{raggi vettori} ($\hat{U}:\mathcal{PH}\to\mathcal{PH}$) ed è compatibile con l'operazione di gruppo (addizione):
\begin{equation}
\hat{U}(\vec{a}\,)\hat{U}(\vec{b}\,)=\hat{U}(\vec{a}+\vec{b})
\label{eqn:translation-group-property}
\end{equation}
Per il teorema di Bargmann, dato che il gruppo di ricoprimento universale di $(\bb{R}^3,+)$ è se stesso (dato che è semplicemente connesso), possiamo passare da una rappresentazione proiettiva a una unitaria $\vec{a} \mapsto U(\vec{a}\,)$ sul ricoprimento.\\
Dato che gli $U(\vec{a})$ formano un gruppo di matrici, per quanto visto in (\ref{eqn:mappa-esponenziale-multivariata}), possiamo scrivere ognuno di essi come esponenziale dei generatori del gruppo, che indichiamo con $e_1, e_2, e_3$. Si ha quindi:
\begin{align}
U(\vec{a}) = U(a_1, a_2, a_3) = \exp(a_1 e_1 + a_2 e_2 + a_3e_3)
\label{eqn:traslazione-spaziale}
\end{align}
Procediamo calcolando i generatori. Spostiamoci in rappresentazione in $\{\vec{x}\}$. Detti $\ket{\vec{x}}$ gli autovettori di autovalore $x$ dell'operatore posizione $\vec{X}$. $U(\vec{a})$ \textit{agisce} su di essi per traslazione:
\begin{align*}
U(\vec{a}) \ket{\vec{x}} = \ket{\vec{x}+\vec{a}}
\end{align*}
Prendendo il coniugato di entrambi i membri si giunge a:
\begin{align}
\bra{\vec{x}} U(\vec{a})^\dag =\bra{\vec{x}+\vec{a}}
\label{eqn:bra-applicated}
\end{align}
Si ha che $U(\vec{a})^\dag$ corrisponde alla \textit{traslazione inversa} $U(\vec{a})^{-1} = U(-\vec{a})$: lo si vede della proprietà (\ref{eqn:translation-group-property}) scrivendo $\bb{I} = U(\vec{0})= U(\vec{a}-\vec{a})=U(\vec{a})U(-\vec{a})$. Ma allora sostituendo nell'equazione (\ref{eqn:bra-applicated}) e poi rinominando $-\vec{a}\to \vec{a}$:
\begin{align}
\bra{\vec{x}}U(-\vec{a})=\bra{\vec{x}+\vec{a}} \Rightarrow  \bra{\vec{x}}U(\vec{a})=\bra{\vec{x}-\vec{a}}
 \label{eqn:bra-translated}
\end{align}
Possiamo ora usare questo risultato per calcolare l'azione di $U(\vec{a})$ su $\ket{\psi}$ in rappresentazione $\{\vec{x}\}$:
\begin{align*}
(U(\vec{a}) \psi)(\vec{x}\,) \equiv \bra{\vec{x}}U(\vec{a}\,)\ket{\psi} \underset{(\ref{eqn:bra-translated})}{=} \braket{\vec{x}-\vec{a}|\psi} = \psi(\vec{x}-\vec{a})
\end{align*}
In questo modo abbiamo scritto gli stati come \textit{funzioni} (o \textit{vettori} di $L^2(\bb{R}^3,d^3x)$) che possiamo derivare facilmente. Calcoliamo perciò i generatori dall'espressione (\ref{eqn:spazio-tangente-gruppo}):
\begin{align*}
e_j \equiv \frac{\partial}{\partial a_j} (U(\vec{a}\,)\psi)(\vec{x}\,)\Big|_{\vec{a}=0} = \frac{\partial}{\partial a_j} \psi(\vec{x}-\vec{a})\Big|_{\vec{a}=0} = -\frac{\partial}{\partial x_j}\psi(\vec{x}\,)=\left(\frac{P_j}{i\hbar}\psi\right)(\vec{x}\,)_j
\end{align*}
dove $P_j = -i\hbar \partial_{x_j}$ è la $j$-esima componente del momento in rappresentazione $\{\vec{x}\}$.\\
Sostituendo allora quanto trovato in (\ref{eqn:traslazione-spaziale}), e sintetizzando la somma con un prodotto scalare, giungiamo a:
\[
U(\vec{a}\,)=\exp\left({-\frac{i}{\hbar}\vec{a}\cdot \vec{P}\,}\right)
\]

\section{Momento angolare e rotazioni}
\label{sec:rotazioni}
Una osservabile che in \MQ si discosta fortemente dal comportamento classico è il \textbf{momento angolare}.\\

In \MC, utilizzando le coordinate cartesiane in $\bb{R}^3$, le componenti del momento angolare sono (in convenzione di Einstein, con gli indici in lettere latine che vanno intesi da $1$ a $3$):
\[
l_i = (\vec{x}\times \vec{p}\,)_i= \epsilon_{ijk} x_j p_k \quad \vec{l}=\{l_1, l_2, l_3\}
\]
dove $\epsilon_{ijk}$ è il tensore di Levi-Civita così definito:
\[
\epsilon_{ijk}=\begin{cases}
1 & \text{ se } i,j,k = 1,2,3 \text{ o permutazioni pari}\\
-1 & \text{ se } i,j,k \text{ si ottiene da } 1,2,3 \text{ per permutazioni dispari}\\
0 & \text{altrimenti}
\end{cases}
\]
Il momento angolare ha spettro continuo in \MC.\\

In \MQ, nonostante $\vec{X}$ e $\vec{P}$ abbiano spettro continuo in $\bb{R}^3$, $\vec{L}$ risulterà avere spettro discreto!\\
Per semplicità non discuteremo eventuali problemi di dominio: basti sapere che tutte le operazioni che faremo sono lecite in:
\[
D=\mathcal{S}(\bb{R}^3)
\]
Volendo si potrebbe definire un dominio più ampio, ma farlo risulta complicato e oltre gli obiettivi di questo corso.\\

Calcoliamo l'\textit{algebra} di $\vec{L}$, ossia tutti i possibili commutatori tra $L_1$, $L_2$ e $L_3$. Risulta:
\[
[L_i, L_j] = i\hbar \epsilon_{ijk} L_k \quad L_i = \epsilon_{ijk}X_j P_k
\]
Tale risultato si potrebbe derivare applicando la \textit{quantizzazione canonica} di Dirac delle parentesi di Poisson $[\cdot,\cdot]=i\hbar \{\cdot,\cdot\}$, e utilizzando il risultato in \MC:
\[
\{l_i, l_j\} = \epsilon_{ijk} l_k
\] %Inserire ref a quando usato la quantizzazione delle parentesi di Poisson per la prima volta (nelle prime lezioni)

Alternativamente, si può procedere per calcolo diretto. Per esempio, calcoliamo $[L_1, L_2]$. Risulta comodo sfruttare le seguenti due proprietà del commutatore:
\begin{align*}
[A,BC]&=[A,B]C + B[A,C]\\
[AB,C] &= [A,C]B + A[B,C]
\end{align*}
che si verificano immediatamente espandendo i commutatori da definizione.\\

Allora: %[TO DO] Fare i conti
\begin{align*}
[L_1, L_2]&=[X_2 P_3 - X_3 P_2, X_3 P_1 - X_1 P_3]=\\
&=X_2[P_3, X_3] P_1 + X_1[X_3, P_3]P_2=\\
&=-i\hbar X_2 P_1 + i\hbar X_1 P_2 = i\hbar L_3
\end{align*}
(al primo conto, prova a calcolare \textit{tutti} i commutatori)\\

Definiamo ora $\vec{L}^2 = L_1^2 + L_2^2 + L_3^2$. Notiamo che $\vec{L}^2$ commuta con ogni $L_j$:
\begin{align*}
[\vec{L}^2, L_j] &= [L_i L_i, L_j]=L_i[L_i, L_j] + [L_i,L_j]L_i =\\
&=i\hbar\underbrace{ \epsilon_{ijk} }_{ik \text{ antisimmetrici}}\underbrace{(L_i L_k + L_k L_i)}_{ik \text{ simmetrici}} = 0
\end{align*}
dato che si tratta della \textit{contrazione} di un tensore $A_{ik}$ antisimmetrico con uno $S_{ik}$ simmetrico, che è nulla. Infatti, scambiando l'ordine degli indici, il tensore simmetrico non cambia, mentre quello antisimmetrico cambia di segno:
\begin{align*}
A_{ik}S_{ik} =-A_{ki}S_{ki} \Rightarrow A_{ik}S_{ik} + A_{ki}S_{ki} = 0
\end{align*}
Ora basta rinominare gli indici nel secondo termine (possiamo farlo dato che gli indici su cui si somma sono \textit{muti}), e raccogliere. Si ottiene perciò $2A_{ik} S_{ik}=0$, da cui segue il risultato utilizzato.\\

Perciò, riepilogando, mentre $L_i$ e $L_j$, per $i\neq j$ \textbf{non} sono compatibili, $\vec{L}^2$ e $L_j$ sono compatibili $\forall j$.\\

Vediamo la relazione di $\vec{L}$ con le rotazioni. Calcoliamo prima:
\begin{align*}
[L_i, X_l]&=[\epsilon_{ijk} X_j P_k, X_l] = \epsilon_{ijk} X_j[P_k, X_l] =\\
&= \epsilon_{ijk} X_j (-i\hbar \delta_{kl}) = - i\hbar \epsilon_{ijk}X_j = i\hbar \epsilon_{ilj}X_j\\
[L_i, P_l] &= i\hbar \epsilon_{ilj} P_j
\end{align*}

Analogamente a quanto trovato per le traslazioni spaziali, i cui generatori erano dati dal momento (lineare), ci aspettiamo che il momento angolare funga da generatore delle rotazioni. Procediamo, allora, \textit{a ritroso}, partendo da $e_3 = L_3/\hbar$ come generatore di un gruppo di matrici, e verifichiamo che da esso si ottengano effettivamente le rotazioni attorno all'asse $\hat{z}$.\\
Come già visto, $e_3$ è un elemento dell'algebra di Lie $\mathfrak{g}$ del gruppo di operatori unitari $U(\varphi)$ che, come vogliamo verificare, rappresenta su $\hs$ le rotazioni. Il passaggio da $e_3$ a $U(\varphi)$ è dato, come già visto, dalla mappa esponenziale:
\begin{align*}
U(\varphi) = \exp\left(-i\varphi\frac{L_3}{\hbar}\right)
\end{align*}

Come agisce $U(\varphi)$ su un'osservabile, per esempio sulla posizione $\vec{X}$? Come già visto nel caso di $H$, la formula è analoga a quella del \textit{cambio di base} vista in algebra lineare:
\begin{align*}
X_i(\varphi) =U(\varphi) X_i U(\varphi)^\dag = \exp \left( -i\varphi\frac{L_3}{\hbar}\right) X_i \exp\left(i\varphi\frac{L_3}{\hbar}\right)
\end{align*}
Procediamo per gradi, partendo dal calcolo dell'esponenziale. In generale, l'esponenziale di una matrice $A$ è:
\begin{align*}
e^A = \sum_{n=0}^{\infty}\frac{1}{n!}A^n
\end{align*}
Detta $B$ un'altra matrice, con cui $A$ potrebbe o meno commutare, proviamo ad espandere l'espressione $e^{-A}B e^A$ nella speranza di trovare un qualche \textit{schema} che velocizzi il conto:
\begin{align*}
e^A &= \bb{I} + A + \frac{1}{2!} A^2 + \dots\\
e^{-A} &= \bb{I} - A + \frac{1}{2!}A^2 + \dots\\
e^{-A}B e^A &= \left(
\bb{I} - A + \frac{1}{2!}A^2 + \dots
\right)\left(B + BA + \frac{1}{2!}BA^2 + \dots \right) =\\
&=B+BA + \frac{1}{2!}BA^2 + \dots - AB -ABA -\frac{1}{2}ABA^2 + \dots \\
&+ \frac{1}{2} A^2 B + \frac{1}{2}A^2BA + \frac{1}{4}A^2BA^2 + \dots =\\
&=B + \left(BA-AB\right) +\frac{1}{2!}\left(BA^2 - 2ABA +A^2B\right) +\dots =\\
&=B + [B,A] +\frac{1}{2!} [[B,A],A] + \dots
\end{align*}
Con un po' di formalismo in più si giunge allora alla \textit{formula di Hadamard}:
\[
e^{-A} B e^A = \sum_{n=0}^\infty \frac{1}{n!} \underbrace{[[[\dots [}_{n \text{ volte}}B,\underbrace{A],A],A],\dots,A]}_{n \text{ volte}}
\]
dove la differenza di segno è puramente convenzionale, e serve per considerare una rotazione di $\varphi$ positivo come \textit{antioraria}.\\
In questi conti stiamo ignorando ogni problema di dominio: per esempio non è detto a priori che le potenze $n$-esime di $A$ siano definite. Il tutto, in ogni caso, funziona all'interno del formalismo (ben più avanzato) degli spazi di Hilbert equipaggiati, che però non esploriamo.\\

Svolgiamo lo stesso conto per gli operatori che ci interessano, ossia poniamo nella formula di Hadamard:
\begin{align*}
A = i\frac{\varphi}{\hbar}L_3\qquad B = X_1
\end{align*}
Calcolando i vari termini, ci accorgiamo della comparsa delle espansioni di $\sin$ e $\cos$:
\begin{align*}
\exp\left(i\frac{\varphi}{\hbar}L_3 \right) X_1 \exp\left(-i\frac{\varphi}{\hbar}L_3 \right) &= X_1 - \frac{i\varphi}{\hbar}[X_1, L_3]+ \frac{1}{2}\left(\frac{-i\varphi}{\hbar}\right)^2 [[X_1, L_3], L_3] + \dots =\\
&= X_1 - i\frac{\varphi}{\hbar}(-i\hbar X_2)+\frac{1}{2}\left(\frac{i\varphi}{\hbar}\right)^2 -i\hbar \underbrace{[X_2,L_3]}_{i\hbar X_1}+\dots=\\
&=X_1 - \varphi X_2 - \frac{1}{2}\varphi^2 X_1 + \dots \\
&= X_1\hlc{Yellow}{\left(1-\frac{1}{2}\varphi^2 \dots\right)} - X_2\hlc{SkyBlue}{(\varphi + \dots)} =\\
&= X_1 \cos\varphi - X_2 \sin\varphi
\end{align*} 

Conti analoghi portano alle \textit{trasformazioni} delle altre componenti dell'operatore posizione $\vec{X}$:
\begin{align*}
\exp\left(i\frac{\varphi}{\hbar}L_3\right) X_2 \exp\left(-i\frac{\varphi}{\hbar} L_3 \right)&=X_2 \cos\varphi+X_1\sin\varphi\\
\exp\left(i\frac{\varphi}{\hbar}L_3\right) X_3 \exp\left(-i\frac{\varphi}{\hbar}L_3\right)&= X_3
\end{align*}

Mettendo tutto insieme otteniamo perciò:
\[
\exp\left(i\frac{\varphi}{\hbar}L_3\right) \vec{X} \exp\left(-i\frac{\varphi}{\hbar}L_3 \right) = R(\varphi, \hat{z}) \vec{X}
\]
con:
\[
R(\varphi, \hat{z}) = \begin{pmatrix}
\cos\varphi & -\sin\varphi & 0\\
\sin\varphi & \cos\varphi & 0\\
0 & 0 & 1
\end{pmatrix}\quad
\vec{X}=\begin{pmatrix} X_1 \\ X_2 \\ X_3 \end{pmatrix}
\]

Dove $\hat{z}$ è il versore lungo $z$, e $R(\varphi, \hat{z})$ è la matrice di rotazione (antioraria) di un angolo $\varphi$ intorno all'asse $z$.\\
Più in generale, se $\vec{n}$ è un versore di $\bb{R}^3$:
\[
U(\varphi, \vec{n})\equiv \exp\left({-i\frac{\varphi}{\hbar} \vec{L}\cdot \vec{n}}\right)
\]
è l'operatore (unitario) che descrive le rotazioni di un angolo $\varphi$ attorno al versore $\vec{n}$.\\

Ripetendo gli stessi conti per l'operatore momento $\vec{P}$ si ottiene la stessa cosa:
\[
\exp\left(i\frac{\varphi}{\hbar}L_3\right)\vec{P}\exp\left(-i\frac{\varphi}{\hbar}L_3\right)=R(\varphi, \hat{x})\vec{P}
\]

Deriviamo allora la trasformazione di una generica funzione d'onda $\psi(\vec{x})$ in rappresentazione $\{\vec{x}\}$. Procediamo alla stessa maniera utilizzata per le traslazioni spaziali, notando che $U(\varphi,\vec{n})$ \textit{agisce} \q{da destra} su un bra come una \textit{rotazione inversa}:
\[
\bra{\vec{x}} U(\varphi, \vec{n}) = \bra{R^{-1}(\varphi, \vec{n}) \vec{x}}
\]
E applicando ciò ad una $\ket{\psi}$ generica otteniamo:
\[
\bra{\vec{x}}U(\varphi, \vec{n})\ket{\psi}=\braket{R^{-1}(\varphi, \vec{n})\vec{x}|\psi}=\psi(R^{-1}(\varphi, \vec{n}) \vec{x})
\]
Confrontando con le rappresentazioni delle algebre di Lie dei gruppi riconosciamo che:
\[
\frac{L_j}{i\hbar} = \frac{\partial(U(\varphi, \vec{n})\psi)}{\partial(\varphi \,n_j)}(\vec{x})\Big|_{\varphi=0}
\] %Qui dovrebbe mancare un -i davanti?
è la rappresentazione in $\hs$ (o meglio in un dominio denso) dei generatori dell'algebra di Lie delle rotazioni. %Chiarire
\\Perché una rotazione di $2\pi$ lascia $\vec{x}$ invariante:
\[
U(2\pi, \vec{n})\psi(\vec{x})=\psi(\vec{x})
\]
ossia $U(2\pi, \vec{n})=\bb{I} = \exp\left(i\frac{2\pi}{\hbar}\vec{L}\cdot \vec{n}\right)$.
\[
1=\sigma(\bb{I}) = \sigma\left(\exp\left(i\frac{2\pi}{\hbar}\vec{L}\cdot \vec{n}\right)\right) =\ \exp\left(i\frac{2\pi}{\hbar}\sigma(\vec{L}\cdot \vec{n})\right)\Rightarrow \sigma(\vec{L}\cdot \vec{n}) \subseteq \hbar \bb{Z}
\]
\end{document}

