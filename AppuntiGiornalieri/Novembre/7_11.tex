\documentclass[../../FisicaTeorica.tex]{subfiles}

\begin{document}
\begin{comment}
\textbf{Commento sulle puntate precedenti}\\ %Inserire nella lezione del 25/10, dove c'è il [TO DO]
Quando avevamo detto:
\[
\cos\left(\frac{2\pi n x}{a}\right)\xrightarrow[\mathcal{S}']{n\to\infty}0
\]
Significa precisamente:
\[
\int dx \cos\left(\frac{2\pi n x}{a}\right) \varphi(x)\xrightarrow[n\to\infty]{}0\quad \forall \varphi \in \mathcal{S}(\bb{R})
\]
(L'idea è che per $n$ molto grande, la fluttuazione è molto frequente, e quindi \textit{in media} la funzione è nulla).\\
Ciò è vero per il \textbf{teorema di\ Riemann-Lebesgue}, per cui una $f\in L'(\bb{R})$ si ha che:
\[
\int dx f(x) e^{ikx}\xrightarrow[n\to\infty]{} 0
\]
(e quindi in particolare è vero per le funzioni nello spazio di Schwartz, dato che $\mathcal{S}(\bb{R})\subset L'(\bb{R})$)
\end{comment}
\subsection{Il potenziale a gradino}
Applichiamo\lesson{17}{7/11/2018} la procedura generale vista nella sottosezione precedente ad un esempio concreto.\\
\begin{center}
\definecolor{qqqqff}{rgb}{0.,0.,1.}
\begin{tikzpicture}[line cap=round,line join=round,>=triangle 45,x=1.0cm,y=1.0cm,scale=0.8]
\clip(5.964941401833783,5.949404160383967) rectangle (14.13349701816273,11.031898478511646);
\draw [->] (9.722818966111115,6.142882556893584) -- (9.722818966111115,10.838171364053455);
\draw [->] (5.553661757288541,6.66901415523088) -- (13.891976174933689,6.66901415523088);
\draw (8.87770388544794,9.401281931769622) node[anchor=north west] {$\overline V$};
\draw (9.76512038315154,6.714881833692771) node[anchor=north west] {$x = 0$};
\draw (13.680094647336952,6.519813594648657) node[anchor=north west] {$x$};
\draw (8.26069682200903,10.944143180932464) node[anchor=north west] {$V(x)$};
\draw [line width=1.2pt,color=qqqqff] (9.722818966111115,8.916412053444356)-- (13.680912320114903,8.916412053444356);
\draw [line width=1.2pt,color=qqqqff] (5.553661757288541,6.66901415523088)-- (9.722818966111115,6.66901415523088);
\begin{scriptsize}
\draw [fill=black] (9.722818966111115,6.66901415523088) circle (1.5pt);
\end{scriptsize}
\end{tikzpicture}
$\quad$
\definecolor{qqqqff}{rgb}{0.,0.,1.}
\begin{tikzpicture}[line cap=round,line join=round,>=triangle 45,x=1.0cm,y=1.0cm,scale=0.8]
\clip(5.964941401833783,5.949404160383967) rectangle (14.13349701816273,11.031898478511646);
\draw [->] (9.722818966111115,6.142882556893584) -- (9.722818966111115,10.838171364053455);
\draw [->] (5.553661757288541,6.66901415523088) -- (13.891976174933689,6.66901415523088);
\draw (8.97770388544794,9.201281931769622) node[anchor=north west] {$\overline V$};
\draw (9.76512038315154,6.714881833692771) node[anchor=north west] {$x = 0$};
\draw (13.680094647336952,6.519813594648657) node[anchor=north west] {$x$};
\draw (8.26069682200903,10.944143180932464) node[anchor=north west] {$V(x)$};
\draw [line width=1.2pt,color=qqqqff] (9.248452119588212,6.66901415523088)-- (10.235699217354021,8.916412053444356);
\draw [line width=1.2pt,color=qqqqff] (10.235699217354021,8.916412053444356)-- (13.599652291222702,8.916412053444356);
\draw [line width=1.2pt,color=qqqqff] (9.248452119588212,6.66901415523088)-- (5.972254343298714,6.66901415523088);
\draw [->,line width=1.2pt] (10.133318036844974,9.627813717244718) -- (9.226513295193417,9.620500775779787);
\draw (8.521750934175323,10.166590205140635) node[anchor=north west] {$\vec E$};
\begin{scriptsize}
\draw [fill=black] (9.722818966111115,6.66901415523088) circle (1.5pt);
\end{scriptsize}
\end{tikzpicture}
\end{center}
%[IMMAGINE] Potenziale a gradino e potenziale "regolarizzato" smussato (passa da 0 a $\bar{V}>0$ linearmente) [DONE]
Consideriamo il \textit{potenziale a gradino} $V(x)=\bar{V}H(x)$.\\
Fisicamente si tratta di un'\textit{idealizzazione} del potenziale di un campo elettrico frenante che agisce su una regione molto piccola, per cui la crescita (che comunque resta lineare) può essere vista come \textit{quasi istantanea}.\\

Cominciamo considerando il sistema in \textbf{Meccanica Classica}.\marginpar{Comportamento in \MC} In particolare, esaminiamo il comportamento di una particella che si trova inizialmente a $x(t=0) = x_0 < 0$ (è cioè a \textit{sinistra} del gradino) e si muove verso di esso con un momento iniziale $p(t=0)=p_0 > 0$. La sua hamiltoniana $H$ (pari all'energia $\mathcal{E}$ della particella) è quindi data da\footnote{Attenzione a non confondere $H$ dell'hamiltoniana con $H(x)$ funzione di Heaviside!}:
\[
H=\frac{p^2}{2m}+\bar{V}H(x)=\mathcal{E}
\]
%Prova a calcolare \dot{x}^2 e i due casi x<0, x\geq 0, e raccordare. oppore normalizza con una crescita lineare il potenziale e prendi il limite
Poiché $p$ deve essere un numero reale, si ha che, per certi valori dell'energia $\mathcal{E}$ (che rimane costante) la particella non può accedere a tutte le posizioni. Infatti, se risolviamo per $p$:
\[
p=\pm \sqrt{2m(\mathcal{E}-\bar{V}H(x))}\Rightarrow \begin{cases}
\mathcal{E}>\bar{V} \Rightarrow \text{ $p\in\bb{R}$ $\forall x \in \bb{R}$}\\
\mathcal{E}<\bar{V} \Rightarrow \text{ $p\in\bb{R}$ solo per $x < 0$ }
\end{cases}
\]
dato che, se $\mathcal{E}<\bar{V}$, per $x>0$ l'espressione sotto radice è negativa. Fisicamente ciò corrisponde al fatto che una particella con troppa poca energia \textit{non} può attraversare il gradino del potenziale.\\
Esaminiamo l'evoluzione temporale (classica) del sistema.\\ \begin{expl} \textbf{Derivazione dell'equazione del moto}
La particella parte da $x_0$, e si muove (inizialmente verso destra) a velocità:
\[
\dot{x}=\frac{\partial H}{\partial p}=\frac{p}{m} = \frac{\pm \sqrt{2m(\mathcal{E}-\bar{V}H(x))}}{m}=\pm \sqrt{\frac{2}{m}(\mathcal{E}-\bar{V}H(x))}
\]
Basta allora integrare per trovare l'equazione del moto. Per $x<0$, per cui $H(x)=0$, si trova:
\[
x(t)=x(t=0)\pm t\sqrt{\frac{2}{m}\mathcal{E}}
\]
Che in termini delle condizioni iniziali (e imponendo $p_0 > 0$) diviene:
\[
x_1(t)=x_0 + \frac{p_0}{m}t
\]
Il dominio di questa soluzione non è tutto $t\in \bb{R}$, perché siamo partiti supponendo $x<0$:
\[
x_1(t)<0\Rightarrow t < -x_0 \frac{m}{p_0} = \frac{|x_0| m}{p_0} \equiv t_0
\]
Quindi $x_1(t)$ vale per $t\in (-\infty, t_0)$.\\
Per $x>0$, d'altro canto, $H(x)=1$, e quindi l'equazione da integrare è:
\[
\dot{x}=\pm \sqrt{\frac{2}{m}(\mathcal{E}-\bar{V})}
\]
che, come già osservato, ha soluzione reale solo per $\mathcal{E}>\bar{V}$. Integrando otteniamo:
\[
x_2(\tau)=x(\tau =0)\pm \tau\sqrt{\frac{2}{m}(\underbrace{\mathcal{E}}_{\frac{p_0^2}{2m}}-\bar{V})}=x(\tau=0)\pm
\frac{1}{m}\sqrt{(p_0^2 - 2m\bar{V})}\tau
\]
Dove $x(\tau=0)=0$ per raccordo, e $\tau = t-t_0$ (dato che la particella che parte a $t=0$ in $x_0$ raggiunge a $t=t_0$ il gradino, dove si ha il \textit{passaggio} all'altra soluzione).\\
Notiamo poi che solo la soluzione con il $+$ ha senso, dato che l'altra ha un dominio temporale nullo (rappresenta una particella che \q{torna indietro}, ma allora $x_2(\tau=0^+)<0$, e quindi esce dall'ipotesi che abbiamo fatto prima di integrare).\\
Perciò, se $\mathcal{E}>\bar{V}$, con $x_0 < 0$, $p_0>0$, la soluzione finale si ottiene unendo $x_1(t)$ e $x_2(\tau)$:
\[
x(t)_{\mathcal{E}>\bar{V}}=\begin{cases}
x_0 + \frac{p_0}{m}t & t<t_0\\
\frac{1}{m}\sqrt{p_0^2-2m\bar{V}}(t-t_0) & t\geq t_0
\end{cases}
\]
D'altro canto, se $\mathcal{E}<\bar{V}$, la regione $x>0$ è proibita. Quando la particella vi arriva può perciò solo \q{tornare indietro}\footnote{Se non si è convinti di ciò, si può provare a considerare un potenziale \textit{continuo}, che cresce linearmente tra $-a$ e $a$ da $0$ a $\bar{V}$, quindi risolvere il moto e prendere il limite per $a\to0$. In tal caso nella regione \textit{di crescita lineare} la particella subisce un'accelerazione negativa. Se non ha abbastanza energia per superare tale regione, allora ritornerà in $x=0$ con momento $-p_0$.}, e quindi passare alla soluzione per $x<0$ che è compatibile con un momento iniziale negativo. Avremo quindi:
\[
x(t)_{\mathcal{E}<\bar{V}}=\begin{cases}
x_0 + \frac{p_0}{m}t & t<t_0\\
-\frac{p_0}{m}(t-t_0) & t\geq t_0
\end{cases}
\]
\end{expl}
Osservando che $-p_0 t_0 /m = x_0$, possiamo riscrivere in maniera più sintetica le soluzioni:
\begin{align*}
x(t) &= -\frac{p_0}{m}|t-t_0| && \mathcal{E}<\bar{V}\\
x(t)&=-\frac{p_0}{m}(t_0-t)H(t_0-t)+\frac{1}{m}\sqrt{p_0^2-2m\bar{V}}(t-t_0)H(t-t_0)\quad &&\mathcal{E}>\bar{V}
\end{align*}
Perciò per $\mathcal{E}<\bar{V}$ la particella \textit{sbatte} contro il gradino di potenziale e \textbf{immediatamente} torna indietro, mentre per $\mathcal{E}>\bar{V}$ la particella \textit{supera} \textbf{immediatamente} il gradino.\\
Vedremo che in \MQ non si verifica nulla di tutto ciò: particelle di alta energia avranno una certa probabilità di essere riflesse, e in ogni caso il gradino avrà un \textit{tempo di attraversamento} non istantaneo.

\begin{center}
\definecolor{ffqqqq}{rgb}{1.,0.,0.}
\definecolor{qqqqff}{rgb}{0.,0.,1.}
\begin{tikzpicture}[line cap=round,line join=round,>=triangle 45,x=1.0cm,y=1.0cm]
\clip(6.053292454015546,6.005760464406535) rectangle (14.00683151268907,10.954470979875314);
\draw [->] (9.722818966111115,6.142882556893584) -- (9.722818966111115,10.838171364053455);
\draw [->] (5.553661757288541,6.66901415523088) -- (13.891976174933689,6.66901415523088);
\draw (9.142938868614624,9.217395709384747) node[anchor=north west] {$\overline V$};
\draw (9.834249696769032,6.639480228257932) node[anchor=north west] {$x = 0$};
\draw (13.579290960586133,6.51843263291553) node[anchor=north west] {$x$};
\draw (8.537700891902671,10.947350533090468) node[anchor=north west] {$V(x)$};
\draw [line width=1.2pt,color=qqqqff] (9.248452119588212,6.66901415523088)-- (5.972254343298714,6.66901415523088);
\draw [shift={(7.072237283823781,7.086366067341649)},color=ffqqqq]  plot[domain=4.730537232285336:6.018540471060794,variable=\t]({1.*0.4174206508579858*cos(\t r)+0.*0.4174206508579858*sin(\t r)},{0.*0.4174206508579858*cos(\t r)+1.*0.4174206508579858*sin(\t r)});
\draw [shift={(-13.7697640958203,11.506973127548651)},color=ffqqqq]  plot[domain=6.073113152327166:6.124643432986935,variable=\t]({1.*21.722438620354932*cos(\t r)+0.*21.722438620354932*sin(\t r)},{0.*21.722438620354932*cos(\t r)+1.*21.722438620354932*sin(\t r)});
\draw [shift={(7.817744063338652,8.051048414109957)},color=ffqqqq]  plot[domain=1.6165446834164001:2.951776398605257,variable=\t]({1.*0.1400154672681026*cos(\t r)+0.*0.1400154672681026*sin(\t r)},{0.*0.1400154672681026*cos(\t r)+1.*0.1400154672681026*sin(\t r)});
\draw [shift={(7.804937576513381,8.05104841410996)},color=ffqqqq]  plot[domain=0.18981625498451515:1.5250479701734174,variable=\t]({1.*0.1400154672681011*cos(\t r)+0.*0.1400154672681011*sin(\t r)},{0.*0.1400154672681011*cos(\t r)+1.*0.1400154672681011*sin(\t r)});
\draw [shift={(29.392445735674983,11.506973127549141)},color=ffqqqq]  plot[domain=3.300134527782448:3.35166480844221,variable=\t]({1.*21.722438620357632*cos(\t r)+0.*21.722438620357632*sin(\t r)},{0.*21.722438620357632*cos(\t r)+1.*21.722438620357632*sin(\t r)});
\draw [shift={(8.55044435602825,7.086366067341648)},color=ffqqqq]  plot[domain=3.4062374897085834:4.694240728484043,variable=\t]({1.*0.4174206508579848*cos(\t r)+0.*0.4174206508579848*sin(\t r)},{0.*0.4174206508579848*cos(\t r)+1.*0.4174206508579848*sin(\t r)});
\draw [color=ffqqqq] (5.6078947776206896,6.66901415523088)-- (7.0798123231101515,6.66901415523088);
\draw [color=ffqqqq] (8.542869316741879,6.66901415523088)-- (9.722818966111115,6.66901415523088);
\draw [line width=1.2pt,color=qqqqff] (9.722818966111115,8.90915101510004)-- (13.68541315022054,8.916412053444356);
\draw [line width=1.2pt,color=qqqqff] (5.6078947776206896,6.66901415523088)-- (9.722818966111115,6.66901415523088);
\draw [dash pattern=on 1pt off 1pt] (7.811340819926015,8.471014144949809)-- (7.811340819926015,6.403831258969872);
\draw [->,line width=1.2pt] (7.442607428439651,8.805537516539923) -- (8.231739484498316,8.805537516539923);
\draw (7.540838342024164,9.467238836878676) node[anchor=north west] {$p_0$};
\draw (7.559477001669624,6.397385037573129) node[anchor=north west] {$x_0$};
\begin{scriptsize}
\draw [fill=black] (9.722818966111115,6.66901415523088) circle (1.5pt);
\end{scriptsize}
\end{tikzpicture}
\end{center}
%[IMMAGINE] Pacchetto d'onda che si muove verso il gradino ideale [DONE]
Consideriamo ora l'analogo sistema \textbf{quantistico}. \marginpar{Comportamento in \MQ}
La particella classica è sostituita da un \textbf{pacchetto d'onde} inizialmente \textit{piccato} attorno a $x_0$ e con momento \textit{piccato} attorno a $p_0$.\\
\begin{expl}
\textbf{Richiami sulla nozione di pacchetto d'onda}\\
Un \textbf{pacchetto d'onda} non è altro che \q{un'onda di dimensioni finite}, ossia una perturbazione dello spazio che ha valori $\neq 0$ solo in una \textbf{regione ristretta}, come in genere si verifica sperimentalmente, dato che non si ha a che fare con segnali infiniti.\\
%Inserire immagine
Matematicamente, un pacchetto d'onda si ottiene dalla sovrapposizione di un numero arbitrario di onde ideali (es. onde piane), ciascuna con una certa lunghezza d'onda $\lambda$ e un relativo numero d'onda (\textit{angolare}) $k = 2\pi/\lambda$. Scegliendo bene i termini della somma, si ottiene un'interferenza costruttiva in una regione limitata, e distruttiva in tutto il resto dello spazio.\\
\textbf{Nota}: per \textit{localizzare} il pacchetto è necessario incrementare le regioni di interferenza distruttiva, e quindi considerare nella somma un numero maggiore di termini, ossia più possibili scelte per $k$. Al diminuire dell'\textit{incertezza sulla posizione} del pacchetto corrisponde il crescere dell'\textit{incertezza sul suo momento}, e viceversa.\\
In generale, partendo da un pacchetto con andamento arbitrario $\psi(x,t)$, la sua rappresentazione in momenti (o $k$, o frequenze, dato che $\omega$ è funzione di $k$) si ottiene per trasformata di Fourier:
\[
\tilde{\psi}(k,t)=\mathcal{F}[\psi(x,t)](k)=\frac{1}{\sqrt{2\pi}}\int_{-\infty}^{+\infty}
\psi(x,t) e^{-ikx}\,dx
\]
E il passaggio inverso si ottiene mediante l'antitrasformata:
\[
\psi(x,t)=\mathcal{F}^{-1}[\tilde{\psi}(k,t)](x)=\frac{1}{\sqrt{2\pi}}\int_{-\infty}^{+\infty}\tilde{\psi}(k,t)e^{ikx}\,dk
\]
Perciò basta fissare $\psi(x,t)$ o $\tilde{\psi}(k,t)$ per ricavare l'altro. Per esempio potremmo partire fissando l'\textit{energia} del pacchetto, ossia il range delle $k$ delle onde che lo compongono. Fissiamo allora la funzione $\tilde{\psi}(k,t)$ che dà l'ampiezza delle singole onde-componenti a seconda del loro $k$. Se separiamo componente dell'ampiezza da quella dell'evoluzione temporale:
\[
\tilde{\psi}(k,t)=\tilde{f}(k)e^{-i\omega(k)t}
\]
giungiamo alla scrittura di un generico pacchetto come:
\begin{equation}
\psi(x,t)=\frac{1}{\sqrt{2\pi}}\int_{-\infty}^{+\infty}\tilde{f}(k) e^{i(kx-\omega(k)t)}dk
\label{eqn:generico_pacchetto}
\end{equation}
Se la funzione $\omega=\omega(k)$ è lineare, allora tutte le onde che compongono il pacchetto \textit{si muovono alla stessa velocità}, che è detta \textit{velocità di fase} ed è data da:
\[
v=\frac{\omega(k)}{k}
\]
(per cui se $\omega(k)=\bar{v}k$ si ha immediatamente $v\equiv \bar{v}$).\\
In tal caso il mezzo in cui si propaga l'onda è detto \textbf{non dispersivo} (un esempio è il vuoto per un'onda classica): dato che tutte le componenti del pacchetto si muovono in maniera \textit{sincrona}, il pacchetto \q{non si sfalda} e mantiene lo stesso \q{profilo} (envelope). In particolare la velocità del pacchetto è la stessa delle onde che lo compongono. Ecco perché la funzione $\omega = \omega(k)$ è detta \textbf{relazione di dispersione.}\\
Se invece $\omega=\omega(k)$ non è lineare, si ha \textbf{dispersione}. In tal caso non è detto che il pacchetto, ossia \q{l'inviluppo} della sovrapposizione delle onde, in genere \textbf{non} viaggia alla stessa velocità delle onde che lo compongono, ma a una certa $v_g$, detta \textbf{velocità di gruppo}, che ora deriviamo.\\
Chiaramente, ha senso parlare di \q{velocità di un pacchetto} è \q{ben localizzato}, ossia se la sovrapposizione avviene tra onde di simili $k$. Ciò si ha se scegliamo la $\tilde{f}(k)$ con un picco \textit{sharp} attorno a $k_0$. Possiamo allora \textit{linearizzare} $\omega(k)$ sviluppandola in serie di Taylor attorno a $k_0$ (che è il \textit{massimo} che ci interessa) e troncando al primo ordine:
\begin{equation}
\omega(k)\approx \underbrace{\omega(k_0)}_{\omega_0} + \underbrace{\frac{\partial \omega(k)}{\partial k}\Big|_{k=k_0}}_{\omega_0'}(k-k_0)
\label{eqn:linearizzazione}
\end{equation}
Sostituendo (\ref{eqn:linearizzazione}) in (\ref{eqn:generico_pacchetto}) otteniamo (omettendo il fattore di normalizzazione per semplicità):
\begin{align*}
\psi(x,t)_{lin}&=\int_{-\infty}^{+\infty}\tilde{f}(k)e^{i[kx-(\omega_0+\omega_0'(k-k_0)]t}\,dk=\\
&=\int_{-\infty}^{+\infty}\tilde{f}(k)\exp\left [i\left(
\textcolor{Blue}{+k_0 x}-\omega_0 t \textcolor{Blue}{- k_0 x } +k_0 \omega_0' t+ kx-k\omega_0't
\right)\right ] =\\
&=\hlc{Yellow}{e^{i(k_0x-\omega_0 t)}}\hlc{SkyBlue}{\int_{-\infty}^{+\infty} dk\,\tilde{f}(k) e^{i(k-k_0)(x-\omega_0't)}}
\end{align*}
e $\psi(x,t)_{lin}$ è un'approssimazione molto buona di $\psi(x,t)$ se $\tilde{f}(k)$ ha un picco molto stretto attorno a $k_0$. In quest'approssimazione il pacchetto è dato da una componente localizzata di \textit{segnale} (in azzurro) che costituisce l'\textit{inviluppo} (envelope) del pacchetto, ossia la sua \q{sagoma}. Tale espressione è moltiplicata per il termine in giallo, che è una normalissima onda piana con numero d'onda $k_0$, che si muove alla velocità di fase $\omega_0/k_0$. L'espressione azzurra, invece, dipende da $t$ per una $e^{i(x-\omega_0't)}$, e quindi i suoi massimi si propagano alla \textbf{velocità di gruppo} (in quanto è quella dell'inviluppo) che sarà proprio la $\omega_0'$:
\[
\omega_0' = \frac{d\omega}{dk}\Big|_{k=k_0}
\]
\textbf{Nota}: se $\omega(k)$ è decrescente (cosa che può succedere localmente in un mezzo irregolare) è possibile che la velocità di gruppo sia di segno opposto a quella di fase: si ha allora che le onde componenti si muovono in una direzione, mentre l'inviluppo \textit{risale} lungo il verso opposto!\footnote{Delle animazioni carine dei vari casi si trovano su wikipedia, a \url{http://en.wikipedia.org/wiki/Group_velocity} e \url{http://en.wikipedia.org/wiki/Wave_packet}}
\end{expl}
 Utilizzando la relazione (già vista per il fotone):
\begin{equation}
p = \hbar k \Rightarrow k = \frac{p}{\hbar}
\label{eqn:momento_fotone}
\end{equation}
possiamo usare le $k$ come variabili, che sono più comode per una trattazione ondulatoria. Da $\mathcal{E}=p^2/(2m)$ si ha poi $p=\sqrt{2m\mathcal{E}}$, e giungiamo quindi a:
\begin{equation}
k=\frac{\sqrt{2m\mathcal{E}}}{\hbar} \Rightarrow \mathcal{E}(k)=\frac{k^2 \hbar^2}{2m}
\label{eqn:kappa-en}
\end{equation}
Un'onda piana in 1D ha la forma generale di $u(x,t)=e^{i(kx-\omega t)}$. Un pacchetto d'onda è la combinazione di \textit{infinite} onde, la cui ampiezza dipende da $k$ ed è regolata da una funzione $A(k)$ che descrive la \q{forma} del pacchetto $\bar{\psi}$. Avremo quindi:
\[
\bar{\psi}(x,t)=\frac{1}{\sqrt{2\pi}}\int_{-\infty}^{+\infty} A(k)\,u(x,t)\,dk
\]
dove il coefficiente $1/\sqrt{2\pi}$ compare per convenzioni legate alle trasformate di Fourier\footnote{In effetti, la $u(x,t)$ così scritta ha la forma di una trasformata di Fourier. Per approfondimenti vedere i \textit{richiami} all'inizio di questa sezione}. 
%Ripassare bene i pacchetti d'onda e inserire maggiori ref qui
Nel caso quantistico non tutte le $u(x,t)$ sono ammesse, ma solo quelle che sono autofunzioni di $H$, e che quindi indichiamo come $\varphi_{\mathcal{E}}(x)$. Per le ampiezze usiamo invece la notazione $\tilde{f}_{k_0}(k)$, dove il pedice indica la posizione del \textit{picco}, e la $\sim$ ne evidenzia la natura di trasformata di Fourier.\\
Un'ultima importante modifica sta nel variare gli estremi di integrazione: vogliamo infatti considerare la particella che inizialmente si muove verso il gradino, e che quindi ha $p$ (e di conseguenza $k$) positivo. Basta allora partire da $0$ invece che da $-\infty$.\footnote{Come vedremo, le funzioni d'onda si evolvono in maniera \textit{dispersiva}. La presenza di $k<0$ provocherebbe quindi un veloce \q{sfaldarsi} dell'onda, che - seppur fisicamente accettabile - non corrisponde in questo caso all'analogia classica che stiamo portando avanti.}\\
Definiamo allora il pacchetto d'onda iniziale come:
%Se prendessi dei k negativi avrei un'onda che "si sfalda" dilatandosi anche in direzione opposta. 
\begin{equation}
\psi(x,t=0)=\frac{1}{\sqrt{2\pi}}\int_0^{\infty} dk\,\tilde{f}_{k_0}(k) \varphi_\mathcal{E}(x)
\label{eqn:pacchettoiniziale}
\end{equation}
La $\tilde{f}_{k_0}(k)$ è nello specifico una funzione positiva piccata su $k_0$ (ossia il valor medio $\langle k\rangle_{\tilde{f}_{k_0}}\approx k_0$) - e perciò è analoga alla particella classica di momento iniziale $p_0$. Portando avanti l'analogia classica abbiamo ora due casi: il pacchetto d'onda ha \q{nel complesso} un'energia più bassa della barriera di potenziale, oppure ne ha una più alta. Esaminiamoli separatamente.
\paragraph{Energia più bassa del gradino}
Scegliamo in questo caso $\tilde{f}_{k_0}(k)$ con un \textit{supporto}\footnote{Dove il supporto di una funzione  è il range delle coordinate nel dominio per cui ha valore non nullo.} $\Delta$ tale che:
\[
\mathcal{E}(k) =
\frac{\hbar^2 k^2}{2m}<\bar{V} \quad \forall k\in \Delta
\]
In questo modo anche le onde \q{più energetiche} che compongono il pacchetto non hanno comunque abbastanza energia (almeno classicamente) per superare il gradino (come per la particella con $\mathcal{E}<V$).\\
Per esempio potremmo prendere:
\[
\tilde{f}_{k_0}(k)=\chi_{\left[k_0-\frac{a}{2}, k_0+\frac{a}{2}\right]}(k)
\]
%[DOMANDA] Ma le \varphi_{\mathcal{E}} non dovrebbero dipendere anche da $k$ oltre che da x? Beh è ovvio, dato che sono associate ad una certa \mathcal{E}... ma allora dovrebbero dipendere anche dal tempo? No perché il sistema non cambia col tempo (forse dovrei specificarlo)

Per studiare l'evoluzione temporale di $\psi(x,t)$ dobbiamo prima trovare le $\varphi_\mathcal{E}(x)$, che saranno autovettori (eventualmente generalizzati) dell'operatore:
\[
H=-\frac{\hbar^2}{2m}\frac{d^2}{dx^2}+\bar{V}H(x)
\]
con una \textit{fase} per i coefficienti di $\varphi_\mathcal{E}(x)$ scelta opportunamente perché $\psi(x,0)$ sia piccata attorno a $x_0$.\\

Risolviamo quindi l'equazione agli autovalori:
\begin{align}
H\varphi_\mathcal{E}(x)&=\mathcal{E}\varphi_\mathcal{E}(x)\nonumber \\
\left(-\frac{\hbar^2}{2m} \frac{d^2}{dx^2}+\bar{V}H(x)\right)\varphi_{\mathcal{E}}(x)&=\mathcal{E}\varphi_\mathcal{E}(x)
\label{eqn:eqz_step}
\end{align}
Per risolverla conviene spezzare il dominio e raccordare poi le soluzioni. Per $x<0$ si ha che $H(x)=0$, e quindi (\ref{eqn:eqz_step}) diviene:
\begin{align}
\varphi_{\mathcal{E}}''(x)+\frac{2m\mathcal{E}}{\hbar^2}\varphi_\mathcal{E}(x)=0 &\Rightarrow t^2 + \frac{2m\mathcal{E}}{\hbar^2}=0\Rightarrow t_{1,2}=\pm\frac{i}{\hbar}\sqrt{2m\mathcal{E}}=\pm ik
\nonumber
\\
&\Rightarrow \varphi_{\mathcal{E}}(x)=c_+^1 e^{ikx}+c_-^1 e^{-ikx}; \quad x < 0
\label{eqn:conti_negativi}
\end{align} 
Analogamente, per $x\geq 0$ si ha che $H(x)=1$, e quindi l'equazione diviene:
\begin{align}
\varphi_{\mathcal{E}}''(x)+\frac{2m}{\hbar^2}\underbrace{(\mathcal{E}-\bar{V})}_{<0}\varphi_\mathcal{E}(x)=0 &\Rightarrow t_{1,2}=\pm \overbrace{\frac{\sqrt{2m(\bar{V}-\mathcal{E})}}{\hbar}}^{\chi}
\nonumber
\\
&\Rightarrow \varphi_\mathcal{E}(x)=c^2_+e^{\chi x}+c_-^2e^{-\chi x};\quad x\geq 0
\label{eqn:conti_positivi}
\end{align}
Riepilogando:
\begin{equation}
\varphi_\mathcal{E}(x) = 
\begin{cases}
c^1_+ e^{ikx} + c^1_- e^{-ikx} & x<0\\
\hlc{Yellow}{\bcancel{c^2_+}}e^{\chi x} + c^2_- e^{-\chi x} & x\geq0
\end{cases} \quad \chi=\frac{\sqrt{2m(\bar{V}-\mathcal{E})}}{\hbar};
\quad k=\frac{\sqrt{2m\mathcal{E}}}{\hbar}
\label{eqn:autofunzioni}
\end{equation}
Dalla teoria sappiamo che le autofunzioni $\varphi_\mathcal{E}(x)$ appartengono a $\hs$ se corrispondono a un autovalore $\mathcal{E} \in \sigma_P(H)$ dello spettro discreto, o al più a $\mathcal{S}'$ se invece $\mathcal{E}\in\sigma_C(H)$. In entrambi i casi \textbf{non} possono perciò divergere esponenzialmente. Ne segue allora che $c^2_+ = 0$, sennò la seconda soluzione divergerebbe per $x\to\infty$.\\
Notiamo poi che $H$ non ha spettro discreto in questo caso: infatti $\varphi_\mathcal{E}(x)\notin L^2(\bb{R})$, poiché l'integrale del modulo quadro diverge:
\[
\int_{-\infty}^\Lambda |\varphi_\mathcal{E}(x)|^2 \xrightarrow[\Lambda \to +\infty]{}+\infty
\]
data la presenza di termini $e^{ikx}$ che oscillano all'infinito.\\
Perciò avremo che:
\[
\sigma(H)\Big|_{\mathcal{E}<\bar{V}}=\sigma_C(H)\Big|_{0<\mathcal{E}<\bar{V}}
\]
Notiamo che le $\varphi$ dipendono da $3$ coefficienti, e imponendo le $2$ equazioni di \textit{raccordo} (per continuità e derivabilità) resta un coefficiente non determinato, e quindi la degenerazione degli autovalori è $1$.\\
Le equazioni di raccordo per $\varphi_{\mathcal{E}}$ e $\varphi'_{\mathcal{E}}$ portano a:
\begin{align}
\lim_{x\to 0^-}\varphi_\mathcal{E}(x) \overset{!}{=}\lim_{x\to 0^+}\varphi_\mathcal{E}(x) &\Rightarrow c^1_+ + c^1_- =\ c^2_- \label{eqn:cont}
\\
\lim_{x\to 0^-}\varphi'_\mathcal{E}(x) \overset{!}{=}\lim_{x\to 0^+}\varphi'_\mathcal{E}(x) &\Rightarrow 
ik(c^1_+ - c^1_-) = -\chi c^2_- \label{eqn:deriv}
\end{align}
Dividendo per $ik$ la (\ref{eqn:deriv}) e sommando membro a membro con (\ref{eqn:cont}) si può ricavare immediatamente $c^2_-$:
\begin{align*}
c_-^2 = \frac{2}{1+\frac{i\chi}{k}}c_+^1\Rightarrow c^1_- = c^2_--c^1_+ = \frac{1-i\frac{\chi}{k}}{1+i\frac{\chi}{k}}c_+^1
\end{align*}
Scriviamo le costanti in rappresentazione polare (più comoda per le trasformate di Fourier). Notando che il termine $A_+=1+i\frac{\chi}{k}$ si ripete, partiamo da quello:
\begin{align*}
A_+=1+i\frac{\chi}{k}&=\left|1+i\frac{\chi}{k}\right|e^{i\theta(k)}\\
\tan\theta(k)&=\frac{\hlc{Yellow}{\chi}}{\hlc{SkyBlue}{k}}
=
\hlc{Yellow}{\frac{\sqrt{2m(\bar{V}-\mathcal{E})}}{\hbar}} \hlc{SkyBlue}{\frac{\hbar}{\sqrt{2m\mathcal{E}}}
}=\sqrt{\frac{\bar{V}-\mathcal{E}}{\mathcal{E}}} =\sqrt{\frac{\bar{V}}{\mathcal{E}}-1}=\\
&= \sqrt{\frac{\bar{k}^2}{k^2}-1}; \quad \bar{k}=\frac{\sqrt{2m\bar{V}}}{\hbar}
\end{align*}
dove $\bar{k}$ è il numero dell'onda che ha la stessa energia $\bar{V}$ del gradino. Il complesso coniugato $A_- =\ 1-i\frac{\chi}{k}$ ha quindi lo stesso modulo e angolo $\theta(k)$ opposto.\\
Le costanti $c_-^1$ e $c_-^2$ divengono quindi:
\begin{align}
c^2_- &= \frac{2}{|1+i\frac{\chi}{k}|e^{i\theta(k)}}c^1_+ = \frac{2 c^1_+}{|1+i\frac{\chi}{k}|}e^{-i\theta(k)}
\label{eqn:c2-}
\\
c^1_- &= \frac{|1-i\frac{\chi}{k}|e^{-i\theta(k)}}{|1+i\frac{\chi}{k}|e^{i\theta(k)}}c^1_+ = c^1_+ e^{-2i\theta(k)}
\label{eqn:c1-}
\end{align}
Conviene ora \textit{collegare} le costanti alle condizioni iniziali. In particolare, possiamo scegliere $c^1_+$ (che è un \textit{parametro libero} dato dalla degenerazione degli stati) in modo che il pacchetto sia piccato attorno a $x_0$ all'istante iniziale. In particolare poniamo:
\begin{equation}
c^1_+ = e^{-ikx_0}
\label{eqn:c1+}
\end{equation}
Verifichiamo che in tal modo il pacchetto d'onda sia davvero piccato in $x_0$. Calcoliamo la (\ref{eqn:pacchettoiniziale}) per $\varphi_\mathcal{E}(x)$ in (\ref{eqn:autofunzioni}) in $x<0$. Con la scelta fatta per $c^1_+$ otteniamo:
\begin{equation}
\psi(x,t=0)=\frac{1}{\sqrt{2\pi}}\int_0^\infty dk\, \tilde{f}_{k_0}(k) e^{-ikx_0}(\hlc{Yellow}{e^{ikx}}+\hlc{SkyBlue}{e^{-2i\theta(k)}e^{-ikx}})
\label{eqn:pacchettonegativo}
\end{equation}
Concentriamoci sul termine evidenziato in giallo, che corrisponde all'onda \textit{incidente}. A meno di un fattore costante si ha:
\begin{equation}
\psi(x,0)_i\propto \int_0^{+\infty}\tilde{f}_{k_0}(k) e^{ik(x-x_0)} dk \underset{(a)}{=}\int_0^{+\infty} \tilde{f}_{0}(k')e^{i(k'+k_0)(x-x_0)}dk
\label{eqn:pacchetto_max1}
\end{equation}
dove in (a) si è effettuato il cambio di variabile $k\to k'+k_0$, e si è \textit{traslata} la funzione $\tilde{f}_{k_0}(k)$:
\[
\tilde{f}_{k_0}(k+k_0) = \tilde{f}_{0}(k')
\]
che, nella nuova variabile $k'$, è ora piccata in $0$ (nei prossimi passaggi rimuoveremo l'apice per alleggerire la notazione).\\
Notiamo ora la presenza di un termine $e^{ik_0(x-x_0)}$ che corrisponde all'onda piana \q{di fondo} del pacchetto d'onda\footnote{Possiamo vedere le costituenti del pacchetto come una componente di \textit{segnale} data da $e^{ik(x-x_0)}$ e un fondo sinusoidale comune a tutte dato da $e^{ik_0(x-x_o)}$}
%Inserire interpretazione grafica
 che non dipende da $k$ e quindi possiamo estrarre dall'integrale:
\begin{equation}
(\ref{eqn:pacchetto_max1}) = \hlc{Yellow}{e^{ik_0(x-x_0)}}\int_0^{+\infty} \tilde{f}_0(k) e^{ik(x-x_0)}dk
\label{eqn:pacchetto_max2}
\end{equation}
Il fattore evidenziato \textit{non} modifica la posizione dei massimi, e quindi ce ne disinteressiamo. Giungiamo allora a:
\begin{equation}
\psi(x,0)_i\propto\int_0^{+\infty}\tilde{f}_0(k)e^{ik(x-x_0)}dk
\label{eqn:pacchetto_max3}
\end{equation}
che ha un picco in $x_0$. Infatti, calcolandone derivata prima:
\begin{align*}
\frac{d}{dx}\int_0^{+\infty}\tilde{f}_0(k)e^{ik(x-x_0)}dk \Big|_{x=x_0}&=\int_0^{+\infty}\tilde{f}_0(k)\,ik\,e^{ik(x-x_0)}dk\Big|_{x=x_0}=\\
&=\int_0^{+\infty} \tilde{f}_0(k)\,ik\,dk \underset{(a)}{\approx} 0
\end{align*}
dove l'integrale in (a) corrisponde, a meno di un fattore $i$, al valor atteso di $\tilde{f}_0(k)$ su $[0,+\infty)$. Ma la $\tilde{f}_0(k)$ (limitata e a supporto compatto) è piccata in $k=0$, per cui gli unici valori per cui è $\neq 0$ sono per $k$ piccoli, e perciò l'integrale va a $0$.\\
Esaminando la derivata seconda otteniamo:
\[
\frac{d^2}{dx^2}\int_0^{+\infty}
\tilde{f}_0(k)e^{ik(x-x_0)}
 \Big|_{x=x_0} = \int_0^{+\infty} dk (-k^2)\tilde{f}_0(k) < 0
\]
per cui $x=x_0$ è proprio un massimo per $\psi(x,0)_i$.\\
Notiamo che tutti i ragionamenti effettuati valgono indipendentemente dal segno di $k$, e perciò si giunge alla stessa conclusione anche esaminando il termine evidenziato in azzurro nella (\ref{eqn:pacchettonegativo}).\\
Perciò l'intera $\psi(x,0)$ ha massimo in $x=x_0$, come volevamo.\\

Sostituendo allora (\ref{eqn:autofunzioni}) in (\ref{eqn:pacchettoiniziale})\ assieme alle espressioni trovate per le costanti in (\ref{eqn:c1-}) e (\ref{eqn:c1+}), giungiamo all'espressione per il pacchetto a $t=0$:
\begin{equation}
\psi(x,0)=\begin{cases}
\displaystyle
\frac{1}{\sqrt{2\pi}}\int_0^\infty dk \tilde{f}_{k_0}(k) \left[
e^{ik(x-x_0)}+e^{-ik(x+x_0)}e^{-2i\theta(k)}
\right] & x < 0\\ \displaystyle
\frac{1}{\sqrt{2\pi}}\int_0^\infty dk \tilde{f}_{k_0}(k) \frac{2}{\left|1+i\frac{\chi}{k} \right|}e^{-\chi x}e^{-ikx_0}e^{-i\theta(k)} & x\geq 0
\end{cases}
\label{eqn:pacchettoinizialerisolto}
\end{equation}
%Ma quindi fin dall'inizio c'è un'onda riflessa? E allora come faccio a sapere che inizialmente l'onda è piccata in x0? - perché la parte di onda riflessa che c'è all'inizio è "trascurabile", nel senso che è quella che corrisponde alla riflessione della componente stazionaria dell'onda iniziale (quella e^{ik_0 x} "di fondo" che genera le code.
%CFR note del prof. Feruglio per i conti
L'evoluzione di uno stato puro, come da assioma, è data da:%Inserire ref [TO DO] all'Assioma dell'evoluzione degli stati puri
\[
\varphi_\mathcal{E}(x)\xrightarrow[t]{}\varphi_\mathcal{E}(x)e^{-i\frac{\mathcal{E}}{\hbar}t}
\]
In analogia con l'onda piana classica $u(x,t)=e^{i(kx-\omega t)}$, poniamo:
\begin{equation}
\omega \equiv \frac{\mathcal{E}}{\hbar} = \frac{k^2 \hbar}{2m}
\label{eqn:omega}
\end{equation}
\textbf{Nota}: $\omega$ non è una funzione lineare di $k$, e quindi il pacchetto d'onda subisce un effetto di \textit{dispersione}: le onde che lo compongono viaggiano a \textit{velocità di fase} differenti (e che dipendono da $k$) e quindi ci aspettiamo che il pacchetto \q{non mantenga la stessa forma} nel muoversi verso il gradino.\\
L'evoluzione temporale è allora data dalla sostituzione $\varphi_\mathcal{E}\to \varphi_\mathcal{E}(x)e^{-i\omega t}$, che, in (\ref{eqn:pacchettoinizialerisolto}) porta a:
\[
\psi(x,t)=\begin{cases}\displaystyle
\frac{1}{\sqrt{2\pi}} \int_0^\infty dk \tilde{f}_{k_0}(k)
\left[
\overbrace{
e^{i[k(x-x_0)-\omega t]}}^{\psi_i \text{ incidente }}
+ \overbrace{e^{-i[k(x+x_0)+\omega t]}e^{-2i\theta(k)}}^{\psi_r \text{ riflessa }} 
\right] & x < 0
\\ \displaystyle
\frac{1}{\sqrt{2\pi}} \int_0^\infty dk \tilde{f}_{k_0}(k) \frac{2}{\left |1+i\frac{\chi}{k} \right|}e^{-\chi x}e^{-ikx_0}e^{-i[\theta(k)+\omega t]} & x\geq 0
\end{cases}
\]
Ritroviamo perciò per $x<0$ la sovrapposizione di un'onda incidente con una riflessa, e per $x\geq 0$ si ha un'onda che decresce esponenzialmente (a causa del termine $e^{-\chi x}$).\\
Poiché il picco del pacchetto d'onda rappresenta la \q{posizione più probabile} della particella, possiamo calcolare il suo andamento per confrontarlo con il caso classico.\\
Poiché $\tilde{f}_{k_0}$ è piccata su $k_0$, esaminiamo la funzione integranda per quel valore di $k$. Per $x<0$ e $\psi_i$ incidente, il massimo di ordine $0$ è dato dall'annullarsi della derivata:
\begin{align}
\frac{\partial}{\partial k}\Big|_{k_0} \psi_i\overset{!}{=}0 &\Rightarrow \cancel{i}\left [(x-x_0)-\frac{d \omega}{d k}t \right ]\cancel{e^{i[k(x-x_0)-\omega t)]}}=0\nonumber\ \\
&\Rightarrow x-x_0 -\frac{d\omega}{dk}\Big|_{k_0}t=0
\label{eqn:evoluzionemax}
\end{align}
Ricordando la definizione di $\omega$ in (\ref{eqn:omega}) si ha:
\[
\omega = \frac{k^2\hbar}{2m} \Rightarrow \frac{d\omega}{dk} = \frac{k\hbar}{m}
\]
%Approfondimento: d\omega/dk è la velocità di gruppo del pacchetto d'onda!
Sostituendo in (\ref{eqn:evoluzionemax}) si ottiene allora l'equazione del moto \textit{classica}:
\[
x=x_0+\frac{\hlc{Yellow}{\hbar k_0}}{m}t\underset{(\ref{eqn:momento_fotone})}{=}x_0 + \frac{\hlc{Yellow}{p_0}}{m}t=\frac{p_0}{m}(t-t_0)
\]
Perciò, prima dell'urto, il sistema si comporta \q{classicamente}, con il picco che arriva al gradino a $t=t_0$.\\
Se invece esaminiamo, per $x<0$, il picco per $\psi_r$ (sempre calcolando il massimo di ordine $0$ per $k_0$), dobbiamo considerare anche un altro termine:
\begin{equation}
x=-x_0-\frac{\hbar k_0 }{m}t-2\frac{d\theta}{dk}(k_0) = \frac{p_0}{m}(t_0-t) - \underbrace{2\frac{d\theta}{dk}(k_0)}_{\text{offset della riflessione}}
\label{eqn:motoclassicoriflesso}
\end{equation}
Perciò è come se l'onda riflessa non ripartisse da $x=0$, ma da un punto diverso. Calcolando la derivata:
\begin{align*}
\tan\theta(k)=\frac{\chi}{k}=\sqrt{\frac{\bar{k}^2}{k^2}-1} &\Rightarrow \theta(k)=\arctan\left(
\sqrt{\frac{\bar{k}^2}{k^2}-1}\,
 \right)\\
&\Rightarrow \frac{d\theta}{dk}(k_0)=-\frac{1}{\sqrt{\bar{k}^2-k_0^2}} <0
\end{align*}
Sostituendo in (\ref{eqn:motoclassicoriflesso}) si giunge a:
\[
x= \frac{p_0}{m}(t_0-t) + \underbrace{\frac{2}{\sqrt{\bar{k}^2-k_0^2}}}_{\bar{x} > 0}
\]
Perciò il picco dell'onda riflessa riparte da un punto $\bar{x}>0$ che si trova \q{oltre} il gradino di potenziale. Ciò significa che l'onda \q{riemerge} dal gradino con un certo ritardo, dato dividendo l'offset per la \textit{velocità}:
\[
\tau = \frac{\displaystyle -2\frac{d\theta}{dk}(k_0)}{
\displaystyle 
\frac{\hbar k_0}{m} 
}
=\frac{2}{\sqrt{\bar{k}^2-k_0^2}}\frac{m}{\hbar k_0} > 0
\]
Classicamente\marginpar{Differenze con il caso classico per $\mathcal{E}<\bar{V}$} ciò non ha senso: è come se una pallina che sbatte contro il muro si fermasse per un certo $\tau$ prima di rimbalzare!\\
Ancora più stupefacente: vi è una soluzione anche per $x>0$, dove classicamente l'energia cinetica è negativa, sebbene questo contributo sia esponenzialmente depresso ($e^{-\chi x}$). Tale $\psi(t)$ corrisponde a un'\textit{onda di probabilità evanescente} con probabilità massima a $\tau/2$.\\ %Verificare!
Perciò, graficamente, il pacchetto \textit{entra leggermente} nel gradino di potenziale, e quando \textit{non riesce più ad andare avanti} \textit{rimbalza indietro}. Questo \textit{tentativo di entrata} produce il ritardo che viene osservato.\\

\paragraph{Energia più alta del gradino} 
Cosa succede invece se l'energia del pacchetto è maggiore del potenziale $\bar{V}$ del gradino?\\
Ripetendo i conti dall'equazione agli autovalori, il risultato in (\ref{eqn:conti_negativi}) vale ancora, ma stavolta in (\ref{eqn:conti_positivi}) si ha un'espressione negativa dentro la radice, e perciò la parte per $x>0$ è differente:
\begin{equation}
\varphi_{\mathcal{E}>\bar{V}}(x)=\begin{cases}
c^1_+ e^{ikx}+c^1_- e^{-ikx} & x < 0\\
c^2_+ e^{ik_2 x} + \cancel{c^2_-}e^{-ik_2 x} & x\geq 0
\end{cases} \quad k_2=\frac{\sqrt{2m(\mathcal{E}-\bar{V})}}{\hbar}
\label{eqn:autofunzioni_energetiche}
\end{equation}

Le condizioni di raccordo sono:
\begin{align}
\varphi_\mathcal{E}(0^-)=\varphi_\mathcal{E}(0^+) &\Rightarrow c^1_+ + c^1_- = c^2_+ + c_-^2
\label{eqn:raccordo_continua}
\\
\varphi'_\mathcal{E}(0^-)=\varphi_\mathcal{E}(0^+) &\Rightarrow ik(c_+^1 - c_-^1)=ik_2(c_+^2-c_-^2)
\label{eqn:raccordo_derivata}
\end{align}
Lo spettro è nuovamente solo continuo (le soluzioni oscillano all'infinito, e quindi non sono in $\hs$):
\[
\sigma(H)\Big|_{\mathcal{E}>\bar{V}}=\sigma_C(H)\Big|_{\mathcal{E}>\bar{V}}
\]
Abbiamo $4$ coefficienti e $2$ equazioni di raccordo, per cui la degenerazione degli autovalori è $4-2=2$.\\
Notiamo che $c^2_- \neq 0$ conduce alla soluzione di una particella che, giungendo da $x=+\infty$, attraversa il gradino da destra a sinistra. Poiché ci interessa invece la dinamica di attraversamento da sinistra a destra, imponiamo $c^2_-=0$.\\
Allora dividendo (\ref{eqn:raccordo_derivata}) per $ik$ e sommando membro a membro con (\ref{eqn:raccordo_continua}), si ricava immediatamente $c^2_+$, e di conseguenza $c^1_-$:
\begin{equation}
c_+^2 = \frac{2}{1+\frac{k_2}{k}}c_+^1; \quad c_-^1=c^2_+-c^1_+=\frac{1-\frac{k_2}{k}}{1+\frac{k_2}{k}}c_+^1
\label{eqn:costanti_energetiche}
\end{equation}
Sostituendo in (\ref{eqn:autofunzioni_energetiche}) e in (\ref{eqn:pacchettoiniziale}):
\begin{align*}
\psi_\mathcal{E}(x,t) = 
\begin{cases}
\displaystyle
\frac{1}{\sqrt{2\pi}}\int_0^{+\infty} dk \tilde{f}_{k_0}(k) \left[
\underbrace{
e^{i[k(x-x_0)-\omega t]}}_{\psi_i}
+
\frac{1-\frac{k_2}{k}}{1+\frac{k_2}{k}}
\underbrace{e^{-i[k(x+x_0)+\omega t]}}_{\psi_r}
\right]
& x <0 \\
\displaystyle
\frac{1}{\sqrt{2\pi}}\int_0^{+\infty} dk\,\tilde{f}_{k_0}(k) \frac{2}{1+\frac{k_2}{k}}\underbrace{e^{i[k_2 x - kx_0-\omega t]}}_{\psi_t} & x\geq 0
\end{cases}
\end{align*}
%Valuta paragone tra coefficienti di $\psi_i$ e $\psi_r$ e $\psi_t$ e i coefficienti di Fresnel in MC
Notiamo che, nonostante la particella abbia \textit{classicamente} abbastanza energia per superare il gradino di potenziale, vi è ancora una soluzione $\psi_r$ corrispondente ad una riflessione, a cui è associata una probabilità finita.\\
Di nuovo, ciò va contro l'intuizione: nonostante il potenziale \textit{non basti a frenare} una particella troppo energetica, c'è comunque una certa probabilità che essa venga riflessa!\\

\subsubsection{L'equazione di continuità}
Cerchiamo di quantificare la componente riflessa e trasmessa dell'onda. Per farlo ci servirà un modo per esaminare l'evoluzione non della funzione d'onda, ma del suo modulo quadro, che esprime le probabilità che ci interessano.\\
Partendo dall'equazione di Schrodinger possiamo giungere alla relazione che ci interessa sul modulo quadro, derivando la cosiddetta \textbf{equazione di continuità} per la densità di probabilità. Limitiamoci al caso unidimensionale, in cui l'equazione di Schrodinger è: 
\begin{equation}
i\hbar \frac{\partial \psi}{\partial t}(x,t)=-\frac{\hbar^2}{2m}\frac{\partial^2}{\partial x^2}\psi(x,t)+V(x)\psi(x,t)
\label{eqn:schrodinger1D}
\end{equation}
e la sua (duale) complessa coniugata è data da:
\begin{equation}
-i\hbar \frac{\partial \psi^*}{\partial t}(x,t)=-\frac{\hbar^2}{2m}\frac{\partial^2}{\partial x^2}\psi^*(x,t)+V(x)\psi^*(x,t)
\label{eqn:schrodinger1Dduale}
\end{equation}
Moltiplicando (\ref{eqn:schrodinger1D}) per $\psi^*$ e (\ref{eqn:schrodinger1Dduale}) per $\psi$ e sottraendo membro a membro si giunge a:
\begin{align}
\nonumber
\span
i\hbar\underbrace{\left(
\psi^*\frac{\partial \psi}{\partial t}+\frac{\partial \psi^*}{\partial t}\psi
\right)}_{\frac{\partial}{\partial t}(\psi^*\psi)=\frac{\partial }{\partial t}|\psi(x,t)|^2} =-\frac{\hbar^2}{2m}\left [
\psi^*\left(\frac{\partial^2}{\partial x^2}+\bcancel{V}\right) \psi-\psi\left(\frac{\partial^2}{\partial x^2}+\bcancel{V}\right) \psi^*
\right ]\span\span\span\\
\nonumber
&\Rightarrow &&
\frac{\partial}{\partial t}|\psi(x,t)|^2 + \frac{\hbar}{2mi}\left [
\underbrace{\psi^*\frac{\partial^2 \psi}{\partial x^2}
+\textcolor{Blue}{\frac{\partial \psi^*}{\partial x}\frac{\partial \psi}{\partial x}}
}_{\frac{\partial}{\partial x}\left (\psi^* \frac{\partial \psi}{\partial x} \right)}-\underbrace{\frac{\partial^2 \psi^*}{\partial x^2}\psi
 \textcolor{Blue}{-\frac{\partial \psi^*}{\partial x}\frac{\partial \psi}{\partial x}}
 }_{\frac{\partial}{\partial x}\left(\psi\frac{\partial \psi^*}{\partial x}\right)}\right ] &=0\\
\nonumber
&\Rightarrow && \frac{\partial}{\partial t}|\psi(x,t)|^2 +
\frac{\partial}{\partial x}
\underbrace{\left[\frac{\hbar}{2mi}
\left(
\psi^*(x,t)\frac{\partial \psi(x,t)}{\partial x} - \frac{\partial \psi^*(x,t)}{\partial x} \psi(x,t)
 \right) \right]}_{j(x,t)}
 &= 0\\
&\Rightarrow  && \frac{\partial}{\partial t}|\psi(x,t)|^2 +\frac{\partial}{\partial x}j(x,t) = 0 \span
\label{eqn:probability_continuity}
\end{align}
L'equazione (\ref{eqn:probability_continuity}) è detta \textbf{equazione di continuità} per la probabilità\index{Equazione di continuità}. Si tratta di una relazione analoga a quella già ricavata per i fluidi o per la conservazione locale di carica dalle leggi di Maxwell. Intuitivamente, significa che la \textit{probabilità} che una particella si trovi in un dato punto $x$ al tempo $t$ (codificata dal modulo quadro della sua funzione d'onda $\psi(x,t)$) può variare solamente a seguito della variazione delle probabilità dei punti circostanti. In altre parole, se la probabilità di trovarsi a $x=0$ aumenta, quella nei punti \textbf{limitrofi} deve essere diminuita, cioè la probabilità \q{non fa salti}, ma si propaga come un \textit{fluido} che scorre e si addensa nelle regioni \q{più probabili}. La variazione di probabilità in una regione sarà quindi pari al \q{flusso} di probabilità che attraversa il suo bordo: in particolare la probabilità salirà se il flusso è entrante (negativo), e scenderà se il flusso è uscente (positivo) e rimarrà costante se il flusso è nullo. Tale idea, espressa matematicamente in forma differenziale, è in sintesi quella dell'equazione (\ref{eqn:probability_continuity}).\\


Nel caso particolare di una autofunzione di $H$:
\[
\varphi_\mathcal{E}(x,t)=\exp\left({-\frac{i}{\hbar}\mathcal{E}t}\right)\varphi_{\mathcal{E}}(x)
\Rightarrow |\varphi_\mathcal{E}(x,t)|^2=
\underbrace{\left |\exp\left({-\frac{i}{\hbar}\mathcal{E}t}\right)\right|
}_{=1}|\varphi_\mathcal{E}(x)|^2 =|\varphi_\mathcal{E}(x)|^2
\]
Perciò la densità di probabilità associata alle autofunzioni è \textit{costante} nel tempo. Sostituendo nell'equazione di continuità:
\[
\frac{\partial}{\partial t}|\varphi_\mathcal{E}(x,t)|^2=0=
\frac{d}{dx}j(x)
\]
\lesson{18}{8/11/2018}
Dove la corrente di probabilità $j(x)$ ha la forma vista nei passaggi per giungere a (\ref{eqn:probability_continuity}):
\[
j(x)=\frac{\hbar}{2mi}\left(\varphi_\mathcal{E}(x)^* \frac{d\varphi_\mathcal{E}(x)}{dx}-\frac{d\varphi_\mathcal{E}(x)^*}{dx}\varphi_\mathcal{E}(x)\right)
\]
In cui sostituiamo la $\varphi_\mathcal{E}(x)$ ricavata in (\ref{eqn:autofunzioni_energetiche}):
\[
\varphi_\mathcal{E}(x) = \begin{cases}
c^1_+ e^{ikx}+c^1_- e^{-ikx} & x < 0\\
c^2_+ e^{ik_2 x} + c^2_- e^{-i k_2 x} & x\geq 0
\end{cases} \quad k=\frac{\sqrt{2m\mathcal{E}}}{\hbar};\quad k_2=\frac{\sqrt{2m(\mathcal{E}-\bar{V})}}{\hbar}
\]
Svolgendo allora i conti per ricavare $j(x)$, per $x<0$ otteniamo:
\begin{align*}
\varphi_\mathcal{E}(x)&=c^1_+ e^{ikx} + c^1_- e^{-ikx} && \varphi_\mathcal{E}^*(x)=c^{1*}_+ e^{-ikx}+c^{1*}_-e^{ikx}\\
\varphi_\mathcal{E}'(x)&=ik(c^1_+ e^{ikx}-c^1_-e^{-ikx})&&\varphi_\mathcal{E}'^*(x)=ik(-c^{1*}_+e^{-ikx}+c^{1*}_-e^{ikx})
\end{align*}
\begin{align*}
\Rightarrow j(x<0) &= \frac{\hbar}{2mi}ik \big [
(c^{1*}_+e^{-ikx}+c^{1*}_-e^{ikx})(c^1_+ e^{ikx}-c^1_-e^{-ikx})-
\\&\quad
-(-c^{1*}_+ e^{-ikx}+c_-^{1*}e^{ikx})(c^1_+e^{ikx}+c^1_-e^{-ikx})\big ] =\\
&=\frac{\hbar k}{2m}\big[
c^{1*}_+ c^1_+ - \bcancel{c^{1*}_+c^1_-e^{-2ikx}} + \cancel{c^{1*}_-c^1_+e^{2ikx}}-c^{1*}_-c^1_-\\
&\quad-(-c^{1*}_+c^1_+-\bcancel{c^{1*}_+c^1_-e^{-2ikx}}+\cancel{c^{1*}_-c^1_+ e^{2ikx}}+c^{1*}_-c^1_-)\big] =\\
&=\frac{\hbar k}{2m}\big[2|c^1_+|^2-2|c^1_-|^2
\big] =\frac{\hbar k}{m}(\underbrace{|c^1_+|^2}_{j_i}-\underbrace{|c^1_-|^2}_{j_r})
\end{align*}
D'altro canto, per $x\geq 0$:
\begin{align*}
\varphi_\mathcal{E}(x)&=c^2_+ e^{ik_2x} && \varphi_\mathcal{E}^*(x)=c^{2*}_+e^{-ik_2x}\\
\varphi'_\mathcal{E}(x)&=ik_2 c^2_+ e^{ik_2 x} && \varphi_\mathcal{E}'^*(x)=-ik_2 c^{2*}_+ e^{-ik_2 x}
\end{align*}
\begin{align*}
\Rightarrow j(x>0)&=\frac{\hbar}{2mi}ik_2\big[(c^{2*}_+ e^{-ik_2 x})(c^2_+ e^{ik_2 x})-(-c^{2*}_+e^{-ik_2 x})(c^2_+ e^{ik_2}x)\big] =\\
&=\frac{\hbar k_2}{2m}\big[2 |c^2_+|^2\big] = \frac{\hbar k_2}{m}\underbrace{|c^2_+|^2}_{j_t}
\end{align*}
Ma dall'equazione di continuità sappiamo che:
\[
\frac{d}{dx}j(x)=0
\]
ossia $j(x) \equiv j$ è costante e non dipende dalla $x$. Ma allora deve essere:
\begin{equation}
j(x<0)=j(x>0) \Rightarrow  |c^1_+|^2-|c^1_-|^2=\frac{k_2}{k}|c^2_+|^2
\label{eqn:pareggio_correnti}
\end{equation}
Definiamo allora il \textbf{coefficiente di riflessione} $R$ come il rapporto tra l'intensità della corrente di probabilità in $x<0$ che va verso destra ($\hbar k |c^1_+|^2/m$) e quella che va verso sinistra ($\hbar k |c^1_-|^2/m$):
\[
R=\left| \frac{j_r}{j_i} \right |=\left | \frac{c^1_-}{c^1_+}\right |^2
\]
Il \textbf{coefficiente di trasmissione} $T$ è invece dato dal rapporto tra l'intensità di corrente in $x>0$ che ha superato il gradino e va verso destra ($\hbar k_2 |c^2_+|^2/m$) e quella che in $x<0$ si avvicina al gradino andando verso destra ($\hbar k |c^1_+|^2/m$):
\[
T=\left | \frac{j_t}{j_i} \right |
=\frac{k_2}{k}\left | \frac{c^2_+}{c^1_+}\right |^2
\]
L'equazione (\ref{eqn:pareggio_correnti}) collega i due termini $R$ e $T$:
\[
|c^1_+|^2 - |c^1_-|^2 = \frac{k_2}{k}
|c^2_+|^2 \Rightarrow 1=
\underbrace{\left |\frac{c^1_-}{c^1_+}\right |^2}_{R} +
\underbrace{\frac{k_2}{k}\left | \frac{c^2_+}{c^1_+}\right |^2}_{T} \Rightarrow 1=R+T
\]
Per la scelta di $c^1_+=e^{-ikx_0}$, con $|c^1_+|=1$, dalle espressioni dei coefficienti ricavate in (\ref{eqn:costanti_energetiche}) si ha:
\[
R=\left(\frac{k-k_2}{k+k_2}\right)^2;\quad T=
\frac{4k^2}{(k+k_2)^2}
\]

\end{document}

