\documentclass[../../FisicaTeorica.tex]{subfiles}

%\usepackage[usenames, dvipsnames, table]{xcolor}
\usepackage[utf8]{inputenc}
\usepackage[T1]{fontenc}
\usepackage{lmodern}
\usepackage{amsmath}
\usepackage{amsfonts}
\usepackage{comment}
\usepackage{wrapfig}
\usepackage{booktabs}
\usepackage{braket}
\usepackage{tikz}
\usepackage{gnuplottex}
\usepackage{epstopdf}
\usepackage{marginnote}
\usepackage{float}
\usetikzlibrary{tikzmark}
\usepackage{graphicx}
\usepackage{cancel}
\usepackage{bm}
\usepackage{mathtools}
\usepackage{hyperref}
\usepackage{ragged2e}
\usepackage[stable]{footmisc}
\usepackage{enumerate}
\usepackage{mathdots}
\usepackage[framemethod=tikz]{mdframed}
\PassOptionsToPackage{table}{xcolor}
\usepackage{soul}
\usepackage{enumerate}
\usepackage{mathdots}
\usepackage[framemethod=tikz]{mdframed} %Added 16/10
\usepackage[italian]{babel} %Added 16/10
\usepackage{amssymb} %Added

%%BOOKTAB
\setlength{\aboverulesep}{0pt}
\setlength{\belowrulesep}{0pt}
\setlength{\extrarowheight}{.75ex}
\setlength\parindent{0pt} %Rimuove indentazione


%%GEOMETRIA
\usepackage[a4paper]{geometry}
 \newgeometry{inner=20mm,
            outer=49mm,% = marginparsep + marginparwidth 
                       %   + 5mm (between marginpar and page border)
            top=20mm,
            bottom=25mm,
            marginparsep=6mm,
            marginparwidth=30mm}

\makeatletter
\renewcommand{\@marginparreset}{%
  \reset@font\small
  \raggedright
  \slshape
  \@setminipage
}
\makeatother
 

%%COMANDI
\newcommand{\q}[1]{``#1''}
\newcommand{\lamb}[2]{\Lambda^{#1}_{\>{#2}}}
\newcommand{\norm}[1]{\left\lVert#1\right\rVert}
\newcommand{\hs}{\mathcal{H}}
\newcommand{\minus}{\scalebox{0.75}[1.0]{$-$}}
\newcommand{\hlc}[2]{%
  \colorbox{#1!50}{$\displaystyle#2$}}
\newcommand{\bb}[1]{\mathbb{#1}}
\newcommand{\op}[1]{\operatorname{#1}}
\renewcommand{\figurename}{Fig.}

\usepackage{fancyhdr}
\pagestyle{fancy}
\fancyhead{} % clear all header fields
\renewcommand{\headrulewidth}{0pt} % no line in header area
\fancyfoot{} % clear all footer fields
\fancyfoot[R]{Francesco Manzali, 2018-19} % other info in "inner" position of footer line
\cfoot{\thepage}

%%AMBIENTI
\newtheorem{thm}{Teorema}[section]
\newtheorem{dfn}{Definizione}
\newtheorem{oss}{Osservazione}
\newtheorem{es}{Esempio}
\newtheorem{axi}{Assioma}
%%Domande di Marchetti
\newtheorem{question}{Domanda}


%%OPERATORI
\DeclareMathOperator{\sech}{sech}
\DeclareMathOperator{\csch}{csch}
\DeclareMathOperator{\arcsec}{arcsec}
\DeclareMathOperator{\arccot}{arcCot}
\DeclareMathOperator{\arccsc}{arcCsc}
\DeclareMathOperator{\arccosh}{arcCosh}
\DeclareMathOperator{\arcsinh}{arcsinh}
\DeclareMathOperator{\arctanh}{arctanh}
\DeclareMathOperator{\arcsech}{arcsech}
\DeclareMathOperator{\arccsch}{arcCsch}
\DeclareMathOperator{\arccoth}{arcCoth} 




\begin{document}
\section{L'operatore momento}
In questa sezione si analizza l'osservabile momento in meccanica quantistica. Tale analisi, oltre ad essere fondamentale in fisica, mostra come il cambiamento del dominio degli operatori in meccanica quantistica possa modificare completamente la fisica del sistema.
\subsection{Momento in $\bb{R}$}
Consideriamo l'operatore momento $P=-i\hbar \frac{d}{dx}$ \q{a dominio reale}, cioè su $\hs = L^2(\bm{\bb{R}}, dx)$. Esaminiamone nel dettaglio il dominio, espandendo tutte le condizioni che avevamo prima sintetizzato con una semplice \q{regolarità}.\\
In particolar modo, è necessario chiedere che:
\begin{itemize}
    \item Esista la derivata $\psi'$ \textit{quasi ovunque}.
    \item Poiché dovremo fare un'\textit{integrazione per parti}, ci serve che $\psi' \in L^2(\bb{R}, dx)$ e sia \q{integrabile alla Lebesgue su ogni compatto}:
    \[
    \int_a^b \psi'(x)dx = \psi(b)-\psi(a) \qquad \forall [a,b] \subset \bb{R}
    \]
\end{itemize}
\textbf{Nota}: È necessario che le condizioni siano così \q{larghe}. Per esempio, potremmo essere tentati di sintetizzarle chiedendo che $\psi \in \mathcal C^1$. Ciò, tuttavia, fa perdere l'autoaggiuntezza dell'operatore, e perciò fa sì che il suo spettro non sia più solamente reale!\\
Definiamo perciò il dominio di $P$ come:
\begin{align*}
D\left(P\right)= \big\{\psi\in L^2\left(\mathbb{R},\ dx\right)\ \ |\ \ 
&\exists\psi^\prime\ q.o.\, \int_{a}^{b}{\psi^\prime\left(x\right)dx}=\psi\left(b\right)-\psi\left(a\right)\ \forall\left[a,b\right]\subset\mathbb{R},\\
&\psi^\prime\in L^2(\mathbb{R},\ dx)\ \big\}
\end{align*}
\textbf{Nota}: per \textit{definire} un operatore è necessario indicarne in maniera esplicita il dominio. Come vedremo tra poco, infatti, lo \q{stesso operatore} su domini diversi può dar luogo a risultati assurdi, o essere associato a diverse osservabili.\\

Fissate queste condizioni, si ha subito che se $\psi$ , $\psi^\prime\in L^2(R, dx)$ allora $\psi(x)$ si annulla all'infinito:
\[
\lim_{x\rightarrow\pm\infty}{\psi\left(x\right)}=0
\]
A priori non è nemmeno ovvio che tale limite esista. Potremmo per esempio considerare una funzione che si annulla ovunque, se non in determinati intervalli (di misura non nulla), in cui \q{salta} a un certo valore positivo. Calibrando opportunamente la larghezza dei \q{salti} e distanziandoli tra loro si può far sì che l'integrale di tale $\psi$ converga a un valore finito - e quindi che $\psi \in L^2$. Tuttavia, poiché tali \q{salti} sono sempre presenti, per quanto grande sia $M$, per $x>M$ la funzione assume valori che non si avvicinano a $0$ più del valore del \q{salto}, e perciò il limite a $\infty$ non esiste.\\
Tuttavia, se $\psi$ e $\psi' \in L^2(\bb{R})$, allora lo sono anche su $\mathbb{R}_+$: $\psi$,$\psi^\prime\in L^2([0, +\infty [, dx)$. Ma allora:
\[
\int_0^\infty dx (\psi^*(x)\psi'(x)+{\psi^*}'(x)\psi(x)) = \int_0^\infty dx \frac{d}{dx}(\psi^*(x), \psi(x)) = \norm{\psi(\infty)}^2 - \norm{\psi(0)}^2
\]
e tale integrale esiste finito perché abbiamo richiesto sia possibile l'integrazione per parti.\\
Ma allora esiste $\norm{\psi(\infty)}$, e quindi anche:
\begin{equation}
\psi(\infty) = \lim_{x\to\infty} \psi(x) = 0
\label{eqn:psitozero}
\end{equation}
(Se fosse qualsiasi altro valore eccetto $0$ si avrebbe che $\psi \notin L^2$).\\

Verifichiamo allora l'autoaggiuntezza $P = P^\dag$. Essendo $D(P)$ denso in $\hs$ partiamo dall'uguaglianza degli elementi di matrice per definire $P^\dag$ e osservare quali condizioni è necessario imporre per il suo dominio:
\[
\left(\phi,P\psi\right)=(P^\dag\phi,\psi); \quad \forall \phi \in D(P^\dag);\> \forall \psi \in D(P)
\]
Calcolando il prodotto scalare:
\[
\int_{-\infty}^{+\infty}{dx\,\phi^\ast\left(x\right)\left[\hlc{SkyBlue}{-i\hbar\frac{d}{dx}\psi\left(x\right)}\right]\underset{(a)}{=}
\hlc{Yellow}{-i\hbar\phi^\ast\left(x\right)\psi\left(x\right)\big|_{-\infty}^{+\infty}}+\int{\left[\hlc{SkyBlue}{-i\hbar\frac{d}{dx}\phi\left(x\right)}\right]^\ast\psi\left(x\right)dx}}
\]
(dove in (a) si è integrato per parti).\\
Se il termine evidenziato in giallo si annullasse avremmo dimostrato che l'aggiunto $P^\dag$ ha la stessa forma di $P$, ossia che $P$ è simmetrico (le espressioni corrispondenti a $P$ e $P^\dag$ sono evidenziati in azzurro). Sappiamo che tale termine si annulla poiché, come visto in (\ref{eqn:psitozero}), $\psi(x)$ si annulla all'infinito. Tale conclusione è però valida solamente se l'integrazione per parti effettuata in (a) è sensata, ossia se anche per $\phi$ valgono le richieste che abbiamo fatto per $\psi$ a tal proposito, e cioè che $\phi'$ esista quasi ovunque, $\phi, \phi' \in L^2(\bb{R},dx)$ e $\phi'$ Lebesgue-integrabile su compatti.\\
Ma allora le condizioni che abbiamo imposto per trovare $D(P)$ sono esattamente le stesse che caratterizzano $D\left(P^\dag\right)$, e quindi:
\[
D\left(P\right)=D(P^\dag)
\]

\subsection{Momento in $\bb{R}_+$}
Paradossalmente, se cerchiamo di restringere la definizione di $P$ ai soli reali positivi, ossia ponendo $\hs=L^2\left(\mathbb{R}_+,\ dx\right)$, la costruzione di prima non porta ad alcun operatore autoaggiunto. Verifichiamolo.\\
Sia $P= -i\hbar \frac{d}{dx}$, e fissiamo:
\[
D\left(P\right)= \left\{\psi\in L^2\left(\mathbb{R}_+\right)\ |\ \exists\> \psi'
\text{ q.o. e sia definita l'integrazione per parti}, \psi^\prime\in L^2(\mathbb{R}_+)\right\}
\]
Ripetendo gli stessi passaggi di prima, si arriva a dover definire le condizioni per cui il primo termine dell'integrazione per parti (quello evidenziato in giallo nel caso precedente) si annulli:
\[
\phi^*(x)\psi(x)\big|_{0}^{+\infty} = \cancel{\phi^*(+\infty)\psi(+\infty)} - \phi^*(0)\psi(0) = 0
\]
Il problema è che dalle condizioni che abbiamo imposto finora sappiamo solo che $\psi$ si annulla all'infinito, ma nulla si sa sul suo comportamento in $0$. È quindi necessario imporre un'altra condizione - ma non c'è modo di farlo in maniera \q{simmetrica}. Se infatti risolvessimo ponendo $\psi(0) = 0$ non dovremmo fare la stessa richiesta per la $\phi$ (non sarebbe necessario), e quindi $D(P) \subset D(P^\dag)$. Viceversa, se imponessimo $\phi(0) = 0$ avremmo la situazione opposta, con $D(P) \supset D(P^\dag)$.\\
Perciò, se richiediamo che il momento $P$ nell'intervallo $\bb{R}_+$ sia simmetrico, allora non può essere autoaggiunto, e quindi \textit{non è un'osservabile}.\\
Qual è il significato fisico di ciò?\\
ipotizziamo per assurdo che $P$ su $\bb{R}_+$ sia un'osservabile. Allora avremo degli autovalori, per esempio $p$ autovalore di $P$. Facendo quindi una misura del momento troveremmo allora $p$, e sapremmo per certo che $p>0$.\\
Con un'analogia semiclassica, \q{fissare un momento} significa considerare un'onda piana\footnote{Pensandola con il principio di indeterminazione: l'onda piana ha estensione infinita - quindi non conosciamo la sua posizione con nessuna precisione - ma ha un \textit{unico} momento $p$}. Tuttavia, un'onda piana \q{confinata} a $\bb{R}_+$ vuol dire che \q{si è riflessa} sul piano per $x=0$, ed è quindi in realtà la sovrapposizione di un'onda incidente e una riflessa - che hanno momenti di segno opposto\footnote{In altre parole, facendo oscillare una fune \q{infinitamente lunga} ma collegata ad un punto fisso - oltre il quale non c'è più nulla, e che corrisponde all'\textit{estremo del dominio}, ossia $x=0$, si crea un'\textit{onda stazionaria}, che classicamente si ottiene come sovrapposizione di due onde con verso opposto di propagazione}. Perciò anche in questo caso non è possibile ottenere una misura univoca del momento - e infatti l'operatore $P$ su $\bb{R}_+$ non corrisponde a osservabili.

%%19-10
\subsection{Momento in $[0,2\pi]$}
\lesson{12}{19/10/2018}
\label{sec:momento_compatto}
È invece possibile definire l'operatore momento su $[0,2\pi]$, cioè ponendo $\hs = L^2([0,2\pi],dx)$.\\
Nella costruzione del suo dominio inizialmente consideriamo un operatore \q{test} $\tilde{P}$, che poi raffiniremo al $P$ vero e proprio. Poniamo quindi:
\[
D\left(\widetilde{P}\right)= \left\{\psi\in L^2\left(\left[0,2\pi\right],dx\right)\ \ |\ \ \psi\text{ regolari},\psi\left(0\right)=\psi\left(2\pi\right)=0\right\}
\]
Ma $\widetilde{P}$ così definito non può essere autoaggiunto, poiché $D(\widetilde{P})\subset D({\widetilde{P}}^\dag)$. Infatti, partendo come prima dalla condizione di simmetria:
\[
\left(\phi,\tilde{P}\psi\right)=(\tilde{P}^\dag\phi,\psi);\quad \phi\in D(\tilde{P}^\dag), \> \psi \in D(\tilde{P})
\]
quando si giunge all'integrazione per parti, il primo termine il primo termine (quello evidenziato in giallo all'inizio) si annulla solamente per la condizione che abbiamo imposto sulle $\psi$:
\[
\phi^\ast\left(x\right)\psi\left(x\right)\big|_0^{2\pi}=0
\]
Ma allora sulle $\phi$ non serve imporre nulla oltre alla regolarità:
\[
D\left({\widetilde{P}}^\dag\right)=\left\{\phi\in L^2\left(\left[0,2\pi\right],dx\right)\ |\ \phi\text{ regolari}\right\}
\]
da cui $D(\tilde{P})\subset D(\tilde{P}^\dag)$.\\
In effetti $\tilde{P}$ non produce risultati fisici. Risolvendo infatti l'equazione agli autovalori (per separazione delle variabili):
\begin{align*}
    &\tilde{P}\ket{\lambda} = \lambda \ket{\lambda} \Rightarrow -i\hbar\frac{d}{dx}\psi_\lambda = \lambda\psi_\lambda \Rightarrow \frac{\psi_\lambda'}{\psi_\lambda} = \frac{i\lambda}{\hbar}\\
    &\Rightarrow \ln \psi_\lambda = \frac{i\lambda}{\hbar}x + c \Rightarrow \psi_\lambda\left(x\right)=A e^{i\frac{\lambda}{\hbar}x}
\end{align*}
Quando si impone (come richiesto dalle condizioni sul dominio di $\tilde{P}$) che $\psi_\lambda\left(0\right)=0$ si ottiene solo una funzione nulla - che non ha senso fisico.\\
La ragione intuitiva per cui ciò non funziona è che, come nel caso di $\mathbb{R}_+$, una situazione semiclassica in cui il momento è perfettamente definito è quella di un'onda piana, ma un'onda \q{confinata tra due punti} è un'onda stazionaria, che è sovrapposizione di due onde con versi opposti, e perciò anche in questo caso resta indefinita la direzione di $p$, e si giunge ad una contraddizione.\\
Tuttavia, in questo caso, possiamo \q{aggiustare} la situazione. Per distinguere i conti dal caso precedente, chiameremo $P_0$ questa \q{versione} del momento.\\
Perché sia simmetrico, come abbiamo visto, vogliamo che si annulli il primo termine dell'integrazione per parti:
\[
\phi^\ast\left(2\pi\right)\psi\left(2\pi\right)-\phi^\ast\left(0\right)\psi \left(0\right)=0
\]
Perché sia così basta allora imporre:
\[
D\left(P_0\right)=\left\{\psi\in L^2\left(\left[0,\ 2\pi\right],\ dx\right)\ |\ \ \psi\text{ regolare},\ \psi\left(0\right)=\psi(2\pi)\right\}
\]
Ne segue che la condizione da imporre sulle $\phi$ è:
\[
\left(\phi^\ast\left(2\pi\right)-\phi^\ast\left(0\right)\right)\psi \left(0\right)=0\Rightarrow \phi \left(2\pi\right)=\phi \left(0\right)
\]
ossia la stessa che abbiamo appena imposto alle $\psi$. Perciò $D\left(P_0\right)=D\left(P_0^\dag\right)$ e si ha l'autoaggiuntezza desiderata.\\
Matematicamente, la condizione che abbiamo appena imposto è una condizione di \textbf{periodicità}. In effetti, l'aver scelto $2\pi$  suggerisce di mappare il sistema su una circonferenza, dove allora ha senso che la particella abbia un momento ben definito (basti pensare, semiclassicamente, ad un'onda che \q{gira} su una circonferenza - il suo verso è definito, e non vi sono riflessioni).\\
Risolvendo allora l'equazione agli autovalori si giunge sempre a:
\[
\psi\left(x\right)=A e^{i\frac{\lambda}{\hbar}x}
\]
e imponendo le condizioni richieste nel dominio, e cioè che:
\[
\psi\left(0\right)=A= \psi \left(2\pi\right)=e^{i\frac{\lambda}{\hbar}2\pi}
\]
Si ottiene che $\lambda = n\hbar$, $n\in \bb{Z}$, e quindi lo spettro è solo puntuale: $\sigma(P_0) = n\hbar = \sigma_P(P_0)$.
\[
\int_{0}^{2\pi}{\left|e^{i\frac{\lambda}{\hbar}x}\right|^2dx=2\pi<\infty}
\]
%DOMANDA: A cosa serve quest'integrale (forse per indicare che tali \psi sono in L^2 e che quindi sono "fisiche"?
Troviamo quindi un'importante differenza tra fisica classica e quantistica. Se consideriamo una particella che gira su un cerchio, in \MC possiamo avere un qualsiasi valore del momento, ma in \MQ solo certi valori \q{quantizzati}\footnote{Potremmo riconoscere in ciò la condizione trovata da de Broglie: perché ogni particella ha un comportamento ondulatorio, solo i momenti che fanno sì che \q{l'onda interferisca costruttivamente con se stessa} sono ammessi}: $p=\hbar n$, $n\in \bb{Z}$.\\

In realtà la scelta che abbiamo fatto sulla restrizione del dominio di $\psi$ per avere l'autoaggiuntezza non è l'unica che si può fare.\\
Se vogliamo che:
\begin{equation}
\phi^\ast\left(x\right)\psi \left(x\right)\big|_0^{2\pi}=0
\label{eqn:integrazioneperpartizero}
\end{equation}
Potremmo, matematicamente, imporre anche solo che $\psi \left(2\pi\right)=e^{i\gamma}\psi \left(0\right)$ (ossia che la $\psi$ sia una funzione \q{periodica} che torna nello stesso punto con uno sfasamento fisso dato da $\gamma$). Definiamo un operatore $P_\gamma$ che incorpora questa condizione nel suo dominio.\\
Se $\gamma =0$ ritroviamo il caso di prima, ma generalmente $\gamma \in [0, 2\pi [$. Qual è allora la condizione da imporre sulle $\phi$?\\
Imponendo (\ref{eqn:integrazioneperpartizero}) con la nuova condizione sulle $\psi$, otteniamo:
\[
0= \phi^\ast\left(2\pi\right)\psi\left(2\pi\right)- \phi^\ast\left(0\right)\psi \left(0\right)=\left(\phi^\ast\left(2\pi\right)e^{i\gamma}-\phi^\ast\left(0\right)\right)\psi\left(0\right)\phi\left(2\pi\right)=e^{i\gamma}\phi \left(0\right)
\]
che è la stessa condizione imposta sulle $\psi$! Abbiamo quindi l'autoaggiuntezza:
\[
D\left(P_\gamma\right)=D\left(P_\gamma^\dag\right)
\]
Ma che osservabile corrisponde a questo operatore? Partiamo dalla solita soluzione $A\exp\left(\frac{i\lambda}{\hbar}x\right)$ dell'equazione agli autovalori e imponiamo la condizione fatta nel dominio:
\[
e^{i\gamma}\psi_\lambda\left(0\right)=e^{i\gamma}A e^{i\frac{\lambda}{\hbar}\cdot 0}=e^{i\gamma}\overset{!}{=}\psi_\lambda\left(2\pi\right)=A e^{i\frac{\lambda}{\hbar}2\pi}
\]
Da cui: 
\[
e^{i\gamma}=e^{i\frac{\lambda}{\hbar}2\pi}\Rightarrow \lambda = \hbar \left(\frac{\gamma}{2\pi} + n\right), \> n\in \bb{Z}
\]
E perciò il dominio di $P_\gamma$ è di nuovo esclusivamente puntuale:
\[
\sigma\left(P_\gamma\right)= \sigma_P\left(P_\gamma\right)= \left\{\hbar\left(\frac{\gamma}{2\pi}+n\right),\ n\in\mathbb{Z}\right\}
\]
Sostituendo l'espressione per $\lambda$ nella soluzione:
\[
\psi_\lambda(x) = A \exp\left(i\frac{\lambda}{\hbar}x\right) = A \exp\left [ix\left(\frac{\gamma}{2\pi}+n\right)\right ]; \> n\in\bb{Z}
\]
Se condensiamo la parte che dipenda da $n$ in un'opportuna funzione $\tilde{\psi}_n(x)$ possiamo riscrivere:
\[
\psi_\lambda(x) \rightarrow \psi_n(x) = \tilde{\psi}_n(x) \exp\left(i\frac{\gamma}{2\pi}x\right); \quad \tilde{\psi}_n = \tilde{\psi}_n(2\pi); \quad \lambda = \hbar \left(\frac{\gamma}{2\pi} + n\right); \> n\in \bb{Z}
\]
E perciò, con questa notazione, l'equazione agli autovalori diviene:
\[
P_\gamma \psi_n = P_\gamma\left(\tilde{\psi}_n(x) e^{i\frac{\gamma}{2\pi}x}\right) = \hbar \left(\frac{\gamma}{2\pi}+n\right)\left(\tilde{\psi}_n(x) \hlc{SkyBlue}{e^{i\frac{\gamma}{2\pi}x}}\right )
\]
Ma allora portando l'esponenziale evidenziata in azzurro a sinistra possiamo evidenziare una nuova equazione agli autovalori:
\[
\left( \hlc{SkyBlue}{e^{-i\frac{\gamma}{2\pi}x}}P_\gamma e^{i\frac{\gamma}{2\pi}x}\right )\tilde{\psi}_n(x) = \hbar \left(\frac{\gamma}{2\pi}+n\right)\tilde{\psi}_n(x)
\]
E sostituendo la definizione di $P_\gamma$:
\[
\left(e^{-i\frac{\gamma}{2\pi}x}\ P_\gamma\ e^{i\frac{\gamma}{2\pi}x}\right){\widetilde{\psi}}_n\left(x\right)=
e^{-\frac{i\gamma}{2\pi}x}\left(-i\hbar\frac{d}{dx}\left(e^{i\frac{\gamma}{2\pi}x}\psi\right)\left(x\right)\right)=\frac{\gamma\hbar}{2\pi}\psi
\left(x\right)-i\hbar \frac{d}{dx}\psi(x)
\]
(Nota: la derivata in un prodotto di operatori agisce su tutto quello che sta alla sua destra).\\
Quindi, se consideriamo solo gli operatori (rimuovendo le $\psi(x)$) giungiamo all'uguaglianza:
\[
e^{-i\frac{\gamma}{2\pi}x}\left(-i\hbar\frac{d}{dx}\right)e^{i\frac{\gamma}{2\pi}x}=\frac{\gamma\hbar}{2\pi}\underbrace{-i\hbar \frac{d}{dx}}_{P_0} = \frac{\gamma \hbar}{2\pi} + P_0
\]
Perciò le $\psi_n(x)$ sono gli autovettori corrispondenti ad un operatore \q{momento + costante}. A che osservabile si riferisce?\\

Per capire il significato fisico di $P_\gamma$ ci riconduciamo ad un caso classico. Consideriamo un solenoide strettamente interno alla circonferenza in cui si muove la particella perpendicolare al piano della circonferenza con flusso magnetico $\phi$.\\
Poiché il campo magnetico è nullo su $S^1$ classicamente la particella non risente del flusso $\phi$  del campo magnetico, ma quantisticamente invece \q{sente} $\phi$  topologicamente. Infatti possiamo scrivere $\phi$ in termini di potenziale vettore $\vec{A}$, che non è nullo su $S^1$:
\[
\phi =\int_{\Sigma}{\vec{B}\cdot d\Sigma=\oint_{S^1}{\vec{A}\cdot d\vec{l}}}=2\pi A
\]
(dove si è applicato il teorema di Stokes). Il modulo del potenziale vettore è allora costante e dipende dal flusso:
\[
A=\frac{\phi}{2\pi}
\]
Classicamente la presenza di un campo magnetico nel formalismo hamiltoniano classico conduce alla sostituzione del momento $\vec{p}$ con un termine\footnote{L'Hamiltoniana di una particella di carica $q$ in un campo elettromagnetico con potenziale vettore $\vec{A}$ e potenziale scalare $\phi$ è infatti data da: $H = \frac{1}{2m}\left(\vec{p}-\frac{q}{c}\vec{A}\right )^2 +q\phi$}:
\[
\vec{p}\rightarrow \vec{p}+\frac{e}{c} \vec{A}
\]
Nel nostro caso 1D:
\[
p\rightarrow p+\frac{e}{c}A=p+\frac{e}{c}\frac{\phi}{2\pi}
\]%Rivedere questa parte
Confrontando questo caso classico con quello quantistico, se avessimo
\[
\frac{e}{c}\frac{\phi}{2\pi} = \frac{\gamma\hbar}{2\pi} \Rightarrow 
\gamma =\frac{e\phi}{\hbar c} 
\]
allora avremmo trovato il caso quantistico corrispondente a questo caso classico. In effetti ciò fu verificato sperimentalmente, anni dopo la prima descrizione matematica di questo operatore.\\
Pertanto $P_\gamma$ descrive il momento di una particella quantistica in $S^1$ in presenza di un solenoide interno a $S^1$ con flusso $\phi (\gamma =\frac{e\phi}{\hbar c})$ che altera lo spettro $\sigma \left(P_\gamma\right)=\sigma\left(P_0\right)+\hbar \gamma$.\\
Tale fenomeno prende il nome di effetto Aharonov-Bohm.\\ Come si intuisce dalla trattazione matematica, si dimostra sperimentalmente che si tratta di un fenomeno puramente topologico: non conta la distanza tra circonferenza e solenoide, ma basta che la circonferenza \textit{sia attorno} alla regione in cui è presente $\vec{B}$, ossia che ogni superficie delimitata dal circuito su cui si trova la particella intersechi una regione di campo magnetico non nullo.
\begin{figure}
    \centering
    \includegraphics[scale=0.6]{Immagini/aharonov_bohm.png}
    \caption{Illustrazione del setup dell'effetto Aharonov-Bohm}
    \label{fig:aharonov_bohm}
\end{figure}%Ricontrollare [TO DO]

\subsubsection{Fase di Berry}
Lo sfasamento che acquisisce una $\psi$ dopo un \textit{ciclo} è detto \textbf{fase di Berry}, e per $\psi$ normalizzate ($\braket{\psi|\psi} = 1$) ha la seguente definizione:
\[
e^{\oint_{\mathcal{C}}{\left\langle\psi\right|d\psi\rangle\ }}
\]
dove $\mathcal{C}$ è una curva chiusa nello spazio degli stati (o delle funzioni).\\
Con questa definizione la fase di Berry è un'osservabile. Mostriamo che il suo valore è non nullo se calcolata lungo un percorso ciclico (per esempio una circonferenza).\\
%[Done] (correggere negli appunti vecchi la i nella fase di Berry) OK
Partiamo considerando:
\[
\psi_n\left(x\right)=\frac{1}{\sqrt{2\pi}}e^{i\left(n+\frac{\gamma}{2\pi}\right)x}
\]
Le $\psi_n$ sono definite a meno di una fase \q{attorno al cerchio}, ossia a meno di una $\psi_\alpha(x)$:
\[
\{\psi_\alpha(x)\>|\>\psi_{2\pi}(x) = e^{i\gamma} \psi_0(x)\}; \quad \alpha \in [0,2\pi[
\]
Giungiamo quindi alla famiglia:
\[
\psi_{n,\alpha}=\frac{1}{\sqrt{2\pi}}e^{i\left(n+\frac{\gamma}{2\pi}\right)\left(x+\alpha\right)}; \quad \psi_{n,2\pi}(x) = e^{i\gamma}\psi_{n,0}(x); \quad \alpha \in [0,2\pi [
\]
Possiamo ora definire $\mathcal{C}$:
\[
C=\psi_{n,\alpha}\left(x\right)\>|\>  \psi_{n,2\pi}\left(x\right)=e^{i\gamma}\psi_{n,0}(x)
\]
e quindi calcolare l'integrale per la fase di Berry:
\begin{align*}
\oint_{\mathcal{C}}{\left\langle\psi_{n,\alpha}\right|d\psi_{n,\alpha}\rangle}&=\int_{0}^{2\pi}{d\alpha\ \left\langle\psi_{n,\alpha}\ \right|\frac{d}{d\alpha}\psi_{n,\alpha}\rangle\ }\\
\left\langle\psi_{n,\alpha}\ \right|\frac{d}{dx}\psi_{n,\alpha} \rangle &=\int_{0}^{2\pi}{dx\ \frac{1}{2\pi}\ e^{-i\left(\frac{\gamma}{2\pi}+n\right)\left(x+\alpha\right)}\left(i\left(\frac{\gamma}{2\pi}+n\right)\right)e^{i\left(\frac{\gamma}{2\pi}+n\right)\left(x+\alpha\right)\ }}=i\left(\frac{\gamma}{2\pi}+n\right)\\
e^{\oint_{\mathcal{C}}\left\langle\psi\middle| d\psi\right\rangle}&=\exp{\int_{0}^{2\pi}{d\alpha\ i\left(\frac{\gamma}{2\pi}+n\right)}}=\exp{i\ 2\pi\left(\frac{\gamma}{2\pi}+n\right)}=e^{i\gamma} 
\end{align*}

\section{Stati misti in \MQ}
Supponiamo di non conoscere in quale stato puro $\ket{\phi}$ il sistema si trovi, ma solo di sapere che può essere negli stati puri $\left|\phi_1\right\rangle,\dots, |\phi_n\rangle $ (che consideriamo normalizzati, $\braket{\phi_i|\phi_i}=1$) con probabilità $p_1\dots p_n$, $\sum_{i=1}^{n}p_i=1$.\\
\textbf{Nota}: $n$ è per forza numerabile, ma le $\phi$ non hanno vincoli di ortogonalità, e le $p_i$ sono probabilità classiche.\\
Dalle regole della probabilità, se $\Sigma$ denota lo stato di informazione non massimale (misto) sopra descritto e $A$ è un'osservabile, allora il valor medio di A nello stato $\Sigma$ è dato da una \q{media pesata} dei valor medi dei singoli stati puri:
\[
\langle A \rangle_\Sigma = \sum_{i=1}^n p_i\bra{\phi_i}A\ket{\phi_i}
\]
Osserviamo che se $\phi$ è un qualsiasi stato di $\hs$, possiamo scrivere il valor medio come:\marginpar{Valor medio in termini di traccia}
\[
\left\langle\phi\left|A\right|\phi\right\rangle
=\op{Tr}(\ket{\phi}\bra{\phi}A)
\]
Infatti, sia $\left\{\left|\chi_j\right\rangle,\
j=1,\ldots,\dim{\mathcal{H}}\right\}$ una base ON per $\hs$, con la condizione che $\left|\chi_1\right\rangle=\ket{\phi}$ (allineo il primo vettore della base con il vettore in esame). Allora:
\[
\op{Tr}(\ket{\phi}\bra{\phi}A) = \sum_{j=1}^{\op{dim}\hs}\bra{\chi_j}(\ket{\phi}\bra{\phi}A)\ket{\chi_j} \underset{(a)}{=} \sum_{j=1}^{\op{dim}\hs}\delta_{ji}\bra{\phi}A\ket{\chi_j} = \bra{\phi}A\ket{\phi}
\]
dove in (a) si è usato il fatto che $\left\langle\chi_j\middle|\phi\right\rangle=\left\langle\chi_j\middle|\chi_1\right\rangle=\delta_{j1}$ per l'ortonormalità (essendo $\ket{\phi}$ perpendicolare a tutti i $\ket{\chi_j}$ e parallelo al primo per costruzione).
\begin{mdframed}[hidealllines=true,backgroundcolor=green!20,innerleftmargin=3pt,innerrightmargin=3pt,leftmargin=-3pt,rightmargin=-3pt]
\textbf{Nota}: in uno spazio di Hilbert\index{Traccia} $\hs$, con $\ket{\chi_j}$ base ON, la \textbf{traccia} è definita come:
\[
\op{Tr}(\cdot) = \sum_j \bra{\chi_j} (\cdot) \ket{\chi_j}
\]
Ed è indipendente dalla scelta della base.\\
In effetti, nel caso finito dimensionale di $\mathbb{R}^n$, se si parte da una base canonica si ritrova la somma degli elementi sulla diagonale principale.\\
Dalla definizione segue anche che la traccia è \textbf{lineare}.
\end{mdframed}
Sostituendo questo risultato nella formula della media:
\[
\langle A \rangle_\Sigma = \sum_{i=1}^n p_i \bra{\phi_i}A\ket{\phi_i} = \sum_{i=1}^n p_i \op{Tr}(\ket{\phi_i}\bra{\phi_i}A) \underset{(a)}{=} \op{Tr}\left( \hlc{Yellow}{\sum_{i=1}^n p_i \ket{\phi_i}\bra{\phi_i}A}\right )
\]
dove in (a) si è usata la linearità della traccia.\\
Il termine evidenziato è detto \textbf{matrice densità}:\marginpar{Matrice di densità}\index{Matrice densità}
\[
\rho \equiv \sum_{i=1}^{n}{p_i\left|\phi_i\right\rangle\langle\phi_i|}
\]
Dalla sua definizione segue che: \marginpar{Proprietà della matrice densità}
\begin{enumerate}
    \item $\rho$ è simmetrico\marginpar{$\rho$ è simmetrico}, ossia $\displaystyle \rho = \rho^\dag$, in quanto $A$ è autoaggiunto, $p_i$ è reale e $\left(\middle|\phi\right\rangle{\left\langle\phi\middle|\right)}^\dag=\left|\phi\right\rangle\left\langle\phi\right|$
	\item $\rho$ è \textbf{positivo}\marginpar{$\rho$ è positivo} ($\rho \geq 0$), cioè produce solo \q{valor medi positivi}: $\forall \psi \left\langle\psi\right|\rho \ket{\psi}\geq 0$. Infatti:
	\begin{align*}
	    \left\langle\psi\left|\rho\right|\psi\right\rangle&=\sum_{i}p_i\left\langle\psi\middle|\phi_i\right\rangle\left\langle\phi_i\middle|\psi\right\rangle=
	    \sum_i p_i \braket{\psi|\phi_i}(\braket{\psi|\phi_i})^* =
	    \\
	    &=\sum_{i}p_i\left|\left\langle\psi\middle|\phi_i\right\rangle\right|^2\geq0; \quad \forall \ket{\psi}\in \hs
	\end{align*}
	(Dalla positività si ha poi subito che è autoaggiunto, come nel caso di un numero complesso, che è positivo solo se reale)
	\item $\displaystyle \op{Tr}\rho = 1$, infatti \marginpar{$\op{Tr} \rho = 1$} $\op{Tr}(\ket{\phi}\bra{\phi}) = \sum_j \braket{\chi_j|\phi}\braket{\phi|\chi_j} = |\braket{\phi|\phi}|^2 = 1$ (essendo $\ket{\chi_1} = \ket{\phi}$). Sfruttando allora la linearità della traccia, e il fatto che le $p_i$, esaurendo tutte le possibilità, si sommano a $1$ (cosa che avevamo chiesto fin dal principio):
	\[
	\op{Tr}\left(\sum_i p_i \ket{\phi_i}\bra{\phi_i}\right) = \sum_i p_i \op{Tr}(\ket{\phi_i}\bra{\phi_i}) = \sum_i p_i = 1
	\]
\end{enumerate}
\begin{dfn}\marginpar{Descrizione matematica di uno stato}
Uno \textbf{stato} (misto o puro) in \MQ è descritto da un operatore $\rho$ che soddisfa 1-3.
\end{dfn}
Gli \textbf{stati puri} sono descritti dalle $\rho$ che soddisfano $\rho =\rho^2$. In tal caso $\rho$  è un proiettore, ma poiché la traccia di $\rho$ deve fare $1$, deve essere unidimensionale (cioè c'è un solo $p_i=1$ e tutti gli altri sono $0$, poiché solo i numeri $0$ e $1$ sono uguali al loro quadrato\footnote{In particolare, se ci fosse un $p_i \neq 0,1$ allora non si avrebbe $\rho = \rho^2$, e se vi fossero più di un $p_i$ pari a $1$ allora $\op{Tr}\rho > 1$, e nel caso non ve ne fosse nessuno $\op{Tr}\rho = 0$, che va contro alla definizione di matrice di densità}).\\
Ma allora in tal caso la somma è ridotta a un solo elemento (tutti gli altri sono annullati dalle $p_i = 0$), e quindi $\rho =\ket{\psi}\bra{\psi}$ per un qualche $\ket{\psi}\in \hs$ normalizzato  ($\braket{\psi|\psi}=1$).\\
Ci si potrebbe allora chiedere\marginpar{$\rho$ $\leftrightarrow$ raggi vettori} se $\rho$, visto che è definito in termini di $\ket{\psi}\in \hs$ - che sono definiti a meno di una fase - presenti la stessa ambiguità. Ma se $\rho$ è la descrizione di uno stato, che è dato da un raggio vettore in $\mathcal{S}$, tale ambiguità deve sparire. E infatti, si verifica che:
\begin{align*}
    \ket{\psi}&\rightarrow e^{i\alpha}\ket{\psi}\\
\rho &\rightarrow e^{i\alpha}\ket{\psi}\bra{\psi}e^{-i\alpha}=\ket{\psi}\bra{\psi}
\end{align*}
Ma allora $\rho^2=\rho$ sono in corrispondenza biunivoca con i raggi vettori dello spazio di Hilbert.\\
Ciò è comodo sperimentalmente, perché $\rho$ costituisce un'osservabile ( e qui potenzialmente può essere misurata).\\
Per esempio, in un sistema di due fotoni, lo stato delle polarizzazioni è una matrice $2\times 2$ di numeri che si possono ricavare sperimentalmente. Perciò, per verificare se un sistema creato in laboratorio sia in uno stato puro o meno, basta determinare tale matrice $A$, calcolarne il quadrato e vedere se viene lo stesso risultato di partenza, ossia se si ha $A^2 = A$.\\

Vediamo che, come nel caso classico, uno stato misto quantistico può essere scritto come una combinazione \textit{convessa}\footnote{Una combinazione convessa è una combinazione lineare di elementi fatta con coefficienti non negativi a somma $1$. Per esempio: $\lambda_1 x_1 + \dots +\lambda_m x_m$ è convessa se $\lambda_i \geq 0$ per $i=1,\dots,m$ e $\sum_{i=1}^m \lambda_i = 1$} di stati puri, scritti come proiettori.\\
\textbf{Attenzione}: partendo dalla decomposizione di $\ket{\psi} = \ket{\psi_1}+\ket{\psi_2}$ si potrebbe essere tentati di scrivere la seguente relazione per la sovrapposizione di stati:
\[
|\psi_1\rangle \langle \psi_1|+\left|\psi_2\right\rangle\left\langle\psi_2\right|\neq \ket{\psi}\bra{\psi}
\]
Ma tale uguaglianza non è corretta. In effetti il primo membro è (a meno di una normalizzazione) la matrice densità di uno \textit{stato misto}, mentre il secondo è ovviamente uno \textit{stato puro} (essendo costituito da \textit{un solo} proiettore).\\
Matematicamente la faccenda risulta evidente prendendo $\ket{\psi_1}$ e $\ket{\psi_2}$ ON. Allora le espressioni del membro a sinistra sono per forza matrici diagonali, ma quella a destra generalmente non lo è.\\
Concretamente, siano per esempio (trascurando, di nuovo, le normalizzazioni):
\begin{align*}
\ket{\psi_1}&=\begin{pmatrix}
1\\
0
\end{pmatrix};\>\ket{\psi_2} = \begin{pmatrix}
0\\
1
\end{pmatrix};\>\ket{\psi}=\ket{\psi_1}+\ket{\psi_2}=\begin{pmatrix}
1\\
1
\end{pmatrix}\\
\ket{\psi_1}\bra{\psi_1} + \ket{\psi_2}\bra{\psi_2} &= 
\begin{pmatrix}
1\\
0
\end{pmatrix}
\begin{pmatrix}
1 & 0
\end{pmatrix}
+
\begin{pmatrix}
0\\
1
\end{pmatrix}
\begin{pmatrix}
0 & 1
\end{pmatrix} =
\begin{pmatrix}
1 & 0\\
0 & 0
\end{pmatrix}
+ \begin{pmatrix}
0 & 0\\
0 & 1
\end{pmatrix}
= \begin{pmatrix}
1 & 0\\
0 & 1
\end{pmatrix}\\
\ket{\psi}\bra{\psi} &=
\begin{pmatrix}
1\\
1
\end{pmatrix}
\begin{pmatrix}
1 & 1
\end{pmatrix}
=
\begin{pmatrix}
1 & 1\\
1 & 1
\end{pmatrix}
\end{align*}
Si nota quindi che nell'espressione del membro a sinistra mancano tutti i termini fuori dalla diagonale.\\

Nel caso \textbf{classico} possiamo scrivere uno stato come combinazione convessa (non numerabile) di stati puri:
\[
\rho \left(q,p\right)=\int dq_0\,dp_0 \rho \left(q_0,p_0\right) \underbrace{\delta \left(q-q_0\right)\delta \left(p-p_0\right)}_{\text{Stati puri}}
\]
È quindi ben definita (dalla $\rho(q,p)$) la \textit{famiglia} degli stati puri che definiscono lo stato misto.\\
In \MQ la situazione è diversa. Dall'autoaggiuntezza di $\rho$  ($\rho = \rho^\dag$) si ha che esistono autovettori $\ket{\lambda_n}$ di $\rho$ che formano una base ON. Possiamo allora
decomporre $\rho$ spettralmente:
\[
\rho =\sum_{n}{\lambda_n|\lambda_n\rangle\langle\lambda_n|};\quad \op{Tr}\left(\rho\right)=1= \sum_{n}\lambda_n
\]
Abbiamo allora ottenuto una descrizione completamente equivalente a quella che abbiamo finora usato:
\[
\rho = \sum_{i}{p_i|\phi_i\rangle\langle\phi_i|}
\]
in cui, tra l'altro, le $\ket{\phi_i}$ non devono neppure essere ON.\\
Perciò si hanno famiglie \textit{diverse} di stati puri che danno la stessa informazione dello stato misto!\\
In \MC posso comunque sapere gli stati puri da cui parto - non so la loro combinazione esatta), ma in \MQ in uno stato misto non è chiara neanche la famiglia di stati puri sottostante (è un livello decisamente più profondo di ignoranza).
\end{document}