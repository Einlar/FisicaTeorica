\documentclass[../../FisicaTeorica.tex]{subfiles}

\begin{document}
Nel caso più generale, tenendo conto anche della degenerazione degli autovalori, la formula (\ref{eqn:evoluzionetemporale_nodeg}) diviene:
\begin{align}
    \psi(t)&=U(t)\psi = e^{-i\frac{t}{\hbar}H}\psi=\int e^{-i\frac{t}{\hbar}\lambda}dP^H(\lambda)\psi \qquad \forall \psi \in \hs\nonumber\\
    \ket{\psi(t)} &= \sum_{\mathcal{E}_n\in \sigma_P(H)} e^{-i\frac{t}{\hbar}\mathcal{E}_n}\sum_{r=1}^{d(\mathcal{E}_n)}\ket{\mathcal{E}_{n},r}\braket{\mathcal{E}_{n},r|\psi}
+\int_{\sigma_C(H)}d\mathcal{E} e^{-i\frac{\mathcal{E}}{\hbar}t}\sum_{r=1}^{d(\lambda)}\ket{\mathcal{E},r}\braket{\mathcal{E},r|\psi}
\label{eqn:evoluzione_temporale_totale}
\end{align}
Se inoltre $\psi$ appartiene al dominio dell'operatore energia ($\psi \in D(H)$), allora $\psi(t)$ si può ottenere anche risolvendo l'equazione di Schrödinger:
\[
i\hbar \frac{\partial \psi(t)}{\partial t}=H\psi(t)
\]
Analogamente al caso degli stati puri, \marginpar{Evoluzione degli stati misti} uno \textbf{stato misto} $\rho=\sum_i c_i\ket{\phi_i}\bra{\phi_i}$ evolve come:
\begin{align}
\rho(t)=\sum_i c_i\ket{\phi_i(t)}\bra{\phi_i(t)}=\sum_i c_i U(t)\ket{\phi_i}\bra{\phi_i}U^\dag(t)=U(t)\rho U^\dag(t)
\label{eqn:rho_mista}
\end{align}
Se tutte le $\ket{\phi_i}\in D(H)$ allora vale anche un'equazione analoga a quella di Schrödinger. Partiamo scrivendo la relazione per le $\ket{\phi_i}$ e quella duale per le $\bra{\phi_i}$ (per cui basta calcolare un complesso coniugato):
\begin{align*}
i\hbar \frac{\partial}{\partial t}\ket{\phi_i(t)} &= H\ket{\phi_i(t)}\\
-i\hbar \frac{\partial}{\partial t}\bra{\phi_i(t)}&=\bra{\phi_i(t)}H
\end{align*}
Perciò per $\sum_i c_i \ket{\phi_i}\bra{\phi_i}$ si ottiene l'\textbf{equazione di von Neumann}:\\
\begin{equation}
i\hbar \frac{\partial}{\partial t}\rho(t) = \sum_i c_i\left(H\ket{\phi_i(t)}\bra{\phi_i(t)}-\ket{\phi_i(t)}\bra{\phi_i(t)}H \right) = [H,\rho(t)]
\end{equation}

\subsection{Visuali di Schrödinger e Heisenberg}
Abbiamo finora discusso l'evoluzione temporale degli stati. Agli effetti potremmo "condensare" la descrizione completa dell'evoluzione di un sistema fisico nella variazione dei suoi stati, e quindi assumere che le osservabili restino sempre le stesse, ossia che gli operatori che le descrivono non dipendano dal tempo.\\
Tale convenzione fa parte della cosiddetta \textbf{visuale di\ Schrödinger}\marginpar{Visuale di Schrödinger} dell'evoluzione\footnote{In inglese "Schrödinger picture"}:
\begin{align}
    \psi &\to \psi(t) = U(t)\psi
    \label{eqn:visuale_schrodinger}
    \\
    A &\to A\nonumber
\end{align}
Del resto, sperimentalmente abbiamo accesso esclusivamente ai valor medi, che evolvono come:\marginpar{Evoluzione di un'osservabile: $A^H(t)$}
\begin{equation}
\langle A\rangle_{\psi(t)}=\bra{\psi(t)}A\ket{\psi(t)}=\bra{\psi}U^\dag(t)AU(t)\ket{\psi} = \bra{\psi}A^H(t)\ket{\psi} = \langle A^H(t)\rangle_\psi
\label{eqn:evoluzione_valor_medi}
\end{equation}
ove $A^H(t) = U^\dag(t)AU(t)$.\\

L'equazione (\ref{eqn:evoluzione_valor_medi}) suggerisce un approccio diverso: focalizzarci sull'evoluzione delle osservabili ($A^H(t)$) e tenere fissi gli stati.\\
Così facendo si ottiene la \textbf{visuale di Heisenberg}\marginpar{Visuale di Heisenberg}\index{Visuali dell'evoluzione} dell'evoluzione\footnote{Chiaramente entrambi gli approcci danno luogo agli stessi risultati sperimentali - che in una teoria coerente devono rimanere fissi indipendentemente dalla descrizione.}:
\begin{align}
    \psi &\to \psi
    \label{eqn:visuale_heisenberg}
    \\
    A &\to A^H(t)=U^\dag (t)A\,U(t)
    \nonumber
\end{align}
Confrontando (\ref{eqn:rho_mista}) con (\ref{eqn:visuale_heisenberg}) si può notare una differenza nell'ordine degli aggiunti. Ciò non sorprende: avevamo notato un qualcosa di simile anche in \MC, quando scrivendo l'evoluzione degli stati in (\ref{eqn:evoluzione_stati_MC}) si otteneva un segno $-$, che non c'era nel caso dell'evoluzione delle sole osservabili.\\

Notiamo inoltre che la formula $A^H(t) = U^\dag(t) A U(t)$ risulta \textit{diversa} rispetto a quella di una normale simmetria fisica vista in (\ref{eqn:op_simmetrizzato}). Il punto chiave è che, nelle visuali di Schr\"odinger o Heisenberg, solo le osservabili o solo gli stati si evolvono, e mai entrambi contemporaneamente, mentre nel caso di una simmetria fisica (es. una rotazione) sia operatori che stati vengono modificati.\\

Se il commutatore $[A^H(t), H]$ esiste, allora otteniamo l'\textbf{equazione di Heisenberg}:
\begin{equation}
i\hbar \frac{\partial}{\partial t}A^H(t) = i\hbar \frac{d}{dt}(U^\dag(t)AU(t)) = i\hbar \frac{dU^\dag(t)}{dt}AU(t)+i\hbar U^\dag(t)A\frac{dU(t)}{dt}
\label{eqn:equazione_heisenberg}
\end{equation}
Sostituendo le espressioni per $U(t)$ e $U^\dag(t)$:
\[
U^\dag(t)=\exp\left(i\frac{t}{\hbar}H\right); \quad U(t)=\exp\left(-\frac{i}{\hbar}tH\right)
\]
otteniamo:
\[
(\ref{eqn:equazione_heisenberg}) = -HU^\dag(t)AU(t)+U^\dag(t)AU(t)H=[A^H(t),H]
\]
E allora:
\[
\frac{dA^H(t)}{dt}=\frac{[A^H(t),H]}{i\hbar}
\]
Confrontando l'espressione appena trovata con quella classica:
\[
\frac{df}{dt}=\{f,H\}
\]
si potrebbe dire (come fece Dirac) che il commutatore $[\cdot,\cdot]/(i\hbar)$ non è altro che la versione \q{quantizzata} delle parentesi di Poisson classiche:
\[
\{\cdot,\cdot\} \to \frac{[\cdot,\cdot]}{i\hbar}
\]
Perciò, per un sistema isolato, l'evoluzione temporale in \MQ procede in un modo molto "classico", e soprattutto è completamente \textbf{deterministica}.\\
Tuttavia, per ricavare informazioni sarà necessario fare una misura, e in quel momento il sistema non potrà più essere considerato isolato, e in particolare esibirà un comportamento quantistico (probabilistico).\\
Facciamo ora qualche esempio per familiarizzarci con il formalismo, e discuteremo poi la teoria della misura (cosa succede ad un sistema quantistico a seguito di una misurazione).\\
\subsection{Buca infinitamente profonda in 1D}
\label{sec:buca_infinita_1D}
Consideriamo una buca di potenziale infinitamente profonda tra $[-\frac{a}{2},\frac{a}{2}]$, in una dimensione\footnote{Fisicamente tale sistema descrive \q{pareti impenetrabili}, sia classicamente che quantisticamente}.\\
%Inserire figura della buca [IMMAGINE] %[TO DO]
L'energia della particella è data da:
\[
H=\frac{p^2}{2m}+V(x); \quad V(x)=\begin{cases}
0 & |x|<\frac{a}{2}\\
V_0\to +\infty & |x|\geq \frac{a}{2}
\end{cases}
\]
Consideriamo gli stati con energia fissata $\mathcal{E}$, e confrontiamo le previsioni di \MC e \MQ per il comportamento della particella.

\subsubsection{In Meccanica Classica}
Classicamente non vi è alcuna limitazione sui valori di energia che può assumere la particella, e perciò 
$\sigma(H) = \bb{R}_+$.\\
Supponiamo di sapere solo che l'energia è $\mathcal{E}$ (ossia, siamo sicuri che facendo una misura di energia si troverà con certezza $\mathcal{E}$). Nello spazio delle fasi, ciò corrisponde a considerare tutti i punti che hanno la $x$ tra\footnote{Se così non fosse, $H\to\infty$ per come abbiamo definito il potenziale, e perciò certamente non varrebbe $\mathcal{E}$} $-a/2$ e $+a/2$, e $p$ tale che l'energia cinetica $p^2/2m$ sia esattamente pari a $\mathcal{E}$. Nessuno di tali punti è preferibile ad un altro, e quindi lo stato è dato da una distribuzione uniforme:
\begin{equation}
\rho_\mathcal{E}(x,p)=A\,\chi_{\left[-\frac{a}{2},\frac{a}{2}\right]}(x)\delta\left(\frac{p^2}{2m}-\mathcal{E}\right)
\label{eqn:stato_classico}
\end{equation}
dove $\chi$ è la funzione caratteristica, e $A$ è la costante di normalizzazione, che troviamo imponendo che l'integrale di $\rho_\mathcal{E}(x,p)$ sullo spazio delle fasi sia esattamente $1$:
\begin{equation}
\int_\bb{R} \chi_{\left[-\frac{a}{2},\frac{a}{2}\right]}(x)\,dx\, \int_\bb{R} dp\,A\,\delta\left(\frac{p^2}{2m}-\mathcal{E}\right) =
\int_{-\frac{a}{2}}^{\frac{a}{2}} dx\,\int_{\bb{R}} dp\, A\,\delta\left(\frac{p^2}{2m}-\mathcal{E}\right) \overset{!}{=} 1
\label{eqn:normalizzazione_buca_classica}
\end{equation}
Ricordiamo che vale la seguente identità\footnote{Pag. 31 di \cite{spazi_hilbert}} per la composizione della delta di Dirac con una funzione differenziabile $g(x)$:
\[
\delta(g(x))=\sum_{i}\frac{\delta(x-a_i)}{|g'(a_i)|}
\]
dove $a_i$ sono gli zeri (semplici) di $g(x)$.\\
Applicandola al nostro caso:
\begin{align}\nonumber
\delta\left(\frac{p^2}{2m}-\mathcal{E}\right) &=\frac{1}{|p|/m}\Big|_{p=\pm \sqrt{2m\mathcal{E}}} (\delta(p-\sqrt{2m\mathcal{E}})+\delta(p+\sqrt{2m\mathcal{E}}))
= \\
&=\hlc{Yellow}{\frac{m}{\sqrt{2m\mathcal{E}}}(\delta(p-\sqrt{2m\mathcal{E}})+\delta(p+\sqrt{2m\mathcal{E}}))}
\label{eqn:composizione-delta-MC}
\end{align}
Sostituendo nell'integrale in (\ref{eqn:normalizzazione_buca_classica}) otteniamo:
\begin{align*}
(\ref{eqn:normalizzazione_buca_classica}) &= a\underbrace{\int_\bb{R} A
\hlc{Yellow}{
\sqrt{\frac{m}{2\mathcal{E}}}\left(
\delta(p-\sqrt{2m\mathcal{E}})+\delta(p+\sqrt{2m\mathcal{E}})
\right)\,dp 
}
}_{=2} = 2a\sqrt{\frac{m}{2\mathcal{E}}}A =\\
&= a\sqrt{\frac{\mathcal{E}}{2m}}A\overset{!}{=}1\Rightarrow A=\frac{1}{a}\sqrt{\frac{\mathcal{E}}{2m}}
\end{align*}
Perciò lo stato (\ref{eqn:stato_classico}) è:
\begin{align}
\rho_\mathcal{E}(x,p)=\frac{1}{a} \sqrt{\frac{\mathcal{E}}{2m}}\chi_{\left[-\frac{a}{2},\frac{a}{2}\right]}(x)\delta\left(\frac{p^2}{2m}-\mathcal{E}\right)
\label{eqn:rhoEMC}
\end{align}
Notiamo che \textbf{non si tratta di uno stato puro} (ossia non è possibile scriverlo come una combinazione \textit{finita} di $\delta$ di Dirac). In effetti vi sono $2$ valori possibili per il momento, e $\infty$ per la posizione.\\

Possiamo naturalmente calcolare la probabilità che la particella si trovi in un certo range.\\
Per esempio, la densità di probabilità che il valore di posizione $\lambda$ sia all'interno di $\left[-\frac{a}{2},\frac{a}{2}\right]$ è data da:
\begin{align*}
w^X_{\rho_\mathcal{E}}(\lambda) &= 
\hlc{Yellow}{\frac{d}{d\lambda}}\int_{\bb{R}}dp \int_{-\frac{a}{2}}^{\frac{a}{2}} dx\,\rho_\mathcal{E}(x,p)\hlc{Yellow}{ H(\lambda-x)}
= \int_{\bb{R}}dp \int_{-\frac{a}{2}}^{\frac{a}{2}}dx\, \rho_\mathcal{E}(x,p)\hlc{Yellow}{\delta(\lambda-x)}=\\
&\underset{(a)}{=}\int_{-\frac{a}{2}}^{\frac{a}{2}} dx\cancel{ \chi_{\left[-\frac{a}{2},+\frac{a}{2}\right]}(x)}\,\frac{\delta(\lambda-x)}{a} = \frac{1}{a}
\end{align*}
Dove in $(a)$ si è svolto l'integrale in $dp$:
\begin{align*}
\int_{\bb{R}}\rho_\mathcal{E}(x,p)\,dp&=\int_{\bb{R}}\frac{1}{a}dp\,\sqrt{\frac{\mathcal{E}}{2m}}\chi_{\left[-\frac{a}{2},\frac{a}{2}\right]}(x)\sqrt{\frac{m}{2\mathcal{E}}}\underbrace{(\delta(p-\sqrt{2m\mathcal{E}})+\delta(p+\sqrt{2m\mathcal{E}}))}_{2} =\\
&= \frac{1}{a}\chi_{\left[-\frac{a}{2}, \frac{a}{2}\right]}(x)
\end{align*}
Il risultato è proprio quello che ci si aspetta da una distribuzione uniforme.\\
Per quanto riguarda i momenti, abbiamo solo due valori possibili (uno positivo e uno negativo). Per esempio, potremmo calcolare la probabilità che una misura del sistema nello stato  $\rho_{\mathcal{E}}$ trovi esattamente $\sqrt{2m\mathcal{E}}$:
\begin{align*}
\mathcal{P}^p_{\rho_\mathcal{E}}(\{\sqrt{2m\mathcal{E}}\})&=\int_\bb{R} \int_\bb{R} \chi_{\{\sqrt{2m\mathcal{E}}\}}(p) \hlc{Yellow}{\rho_\mathcal{E}(x,p)}\,dx\,dp = \\
& \underset{(\ref{eqn:rhoEMC})}{=}
\hlc{Yellow}{\frac{1}{a}\sqrt{\frac{\mathcal{E}}{2m}}}
\underbrace{\int_\bb{R} 
\hlc{Yellow}{\chi_{\left[-\frac{a}{2},\frac{a}{2}\right]}(x)}\,dx}_{a}
\int_\bb{R} \chi_{\{\sqrt{2m\mathcal{E}}\}}(p)\, \hlc{Yellow}{\delta\left(\frac{p^2}{2m}-\mathcal{E}\right)}\,dp = \\
&\underset{(\ref{eqn:composizione-delta-MC})}{=} \cancel{a} \frac{1}{\cancel{a}}\int_\bb{R} dp\,\sqrt{\frac{\mathcal{E}}{2m}} \sqrt{\frac{m}{2\mathcal{E}}}[
\delta(p-\sqrt{2m\mathcal{E}})+\delta(p+\sqrt{2m\mathcal{E}})]\chi_{\{2m\mathcal{E}\}}(p) =\\
&= \sqrt{\frac{\mathcal{E}}{2m}}\sqrt{\frac{m}{2\mathcal{E}}} = \frac{1}{2}
\end{align*}


\subsubsection{In Meccanica Quantistica}
Si dimostra che, analogamente al caso classico, per $V_0\to +\infty$ la particella può avere solo posizioni in $[-\frac{a}{2},\frac{a}{2}]$)\footnote{A priori in \MQ non è ovvio, e ci ritorneremo più avanti}. Perciò $\hs=L^2\left(\left[-\frac{a}{2},\frac{a}{2}\right] dx\right)$.\\
Per prima cosa definiamo bene la $H$ quantistica, che deve essere un operatore autoaggiunto:
\[
H=-\frac{\hbar^2}{2m}\frac{d^2}{dx^2}
\]
Occorre perciò che il dominio $D(H)$ sia tale che $H = H^\dag$.\\
\textbf{Nota}: Specificare il dominio dell'hamiltoniana è \textbf{fondamentale} per definire tale operatore. Un'errata definizione porta a risultati assurdi!\\

Prendiamo $\psi \in D(H^\dag)$, $\varphi \in D(H)$ e verifichiamo:
\begin{equation}
(\psi, H\varphi) = (H^\dag \psi,\varphi)
\label{eqn:autoaggiuntezza_H}
\end{equation}
Espandendo i prodotti scalari\marginpar{La funzione d'onda si annulla ai margini di una buca infinita} e \q{spostando} la derivata seconda sulla $\psi$ con un'integrazione per parti (in (a) e in (b)):
\begin{align*}
-\frac{\hbar^2}{2m}\int_{-\frac{a}{2}}^{\frac{a}{2}} \psi^*(x) \varphi''(x)dx &\underset{(a)}{=} -\frac{\hbar^2}{2m}\psi^*\varphi' \big|_{-\frac{a}{2}}^{\frac{a}{2}} + \frac{\hbar^2}{2m}\int_{-\frac{a}{2}}^{\frac{a}{2}}{\psi^*}'(x)\varphi'(x)dx\\
&\underset{(b)}{=} \hlc{SkyBlue}{-\frac{\hbar^2}{2m}\psi^*\varphi'\big|_{-\frac{a}{2}}^{\frac{a}{2}}+\frac{\hbar^2}{2m}{\psi^*}'\varphi\big|_{-\frac{a}{2}}^{\frac{a}{2}}} - \underbrace{\frac{\hbar^2}{2m}\int_{-\frac{a}{2}}^{\frac{a}{2}}{\psi^*}''(x)\varphi(x)dx}_{=(H^\dag \psi,\varphi)}
\end{align*}
Perciò l'uguaglianza (\ref{eqn:autoaggiuntezza_H}) è verificata se i due termini evidenziati si annullano. Vogliamo allora imporre condizioni su $\varphi$\ in modo che si verifichi proprio questo, e che soprattutto da ciò automaticamente segua che le stesse condizioni debbano valere anche su $\psi$:
\begin{equation}
\psi^*\varphi'-{\psi^*}'\varphi\big|_{-\frac{a}{2}}^{\frac{a}{2}}=0\Rightarrow \text{ Stesse condizioni su $\psi$ }
\label{eqn:annullamento_autoaggiuntezza}
\end{equation}
Proviamo imponendo:
\[
\varphi\left(\frac{a}{2}\right)=0=\varphi\left(-\frac{a}{2}\right)
\]
Si ottiene da (\ref{eqn:annullamento_autoaggiuntezza}):
\[
\psi^*\varphi'\big|_{-\frac{a}{2}}^{\frac{a}{2}} \Rightarrow \psi\left(\frac{a}{2}\right)=0=\psi\left(-\frac{a}{2}\right)
\]
essendo $\varphi'\left(\frac{a}{2}\right), \varphi'\left(-\frac{a}{2}\right)$ arbitrarie.\\
In questo modo l'operatore $H$ è autoaggiunto. Possiamo allora scrivere il suo dominio:
\[
D(H)= \left \{\varphi \in \hs\>|\> \varphi \text{ regolare (vale l'integrazione per parti per $\varphi$ e $\varphi'$) }, \varphi\left(\pm \frac{a}{2}\right) =  0\right \}
\]
Cerchiamo lo spettro discreto $\sigma_P(H)$ risolvendo l'equazione agli autovalori in $D(H)$:
\[
-\frac{\hbar^2}{2m}\frac{d^2}{dx^2}\varphi_{\mathcal{E}}(x) = \mathcal{E}\varphi_{\mathcal{E}}(x)
\]
che ha la forma dell'equazione differenziale di un oscillatore armonico di pulsazione $k$:
\begin{equation}
\frac{d^2}{dx^2}\varphi_\mathcal{E}(x)+\underbrace{\frac{2m\mathcal{E}}{\hbar^2}}_{k^2}\varphi_\mathcal{E}(x)=0 \Rightarrow k = \frac{\sqrt{2m\mathcal{E}}}{\hbar}
\label{eqn:oscillatore_armonico}
\end{equation}
E perciò l'integrale generale è dato da:
\[
\varphi_\mathcal{E}=c_+(x) e^{ikx}+c_- e^{-ikx}
\]
Imponendo che la soluzione si annulli agli estremi della buca (come richiesto nel dominio di $H$):
\begin{equation}
\varphi_\mathcal{E}\left(\pm\frac{a}{2}\right)=0\Rightarrow
\begin{cases}
c_+ e^{ik\frac{a}{2}}+c_- e^{-ik\frac{a}{2}}=0\\
c_+ e^{-ik\frac{a}{2}} + c_- e^{ik\frac{a}{2}} = 0
\end{cases}
\Rightarrow 
\begin{pmatrix}
e^{ik\frac{a}{2}} & e^{-ik\frac{a}{2}}\\
e^{-ik\frac{a}{2}} & e^{ik\frac{a}{2}}
\end{pmatrix}
\begin{pmatrix}
c_+\\
c_-
\end{pmatrix} = 0
\label{eqn:condizioni_contorno_buca}
\end{equation}
Se la matrice delle esponenziali è invertibile, allora si ottiene immediatamente la soluzione nulla ($c_+=0=c_-$), che non definisce alcuno stato fisico. L'unica possibilità è che allora tale matrice sia singolare, e ciò si verifica se:
\begin{equation}
\op{det}\begin{pmatrix}
e^{ik\frac{a}{2}} & e^{-ik\frac{a}{2}}\\
e^{-ik\frac{a}{2}} & e^{ik\frac{a}{2}}
\end{pmatrix}
 = 0 \Rightarrow e^{ika} = e^{-ika}\Leftrightarrow \sin(ka)=0
\label{eqn:matrice_singolare}
\end{equation}
Otteniamo perciò un insieme discreto di soluzioni, le cui pulsazioni soddisfano: $k_n a = n\pi$ $n\in \bb{Z}$. Sostituendo in (\ref{eqn:oscillatore_armonico}) otteniamo che gli autovalori dell'energia, ossia i possibili risultati che si possono ottenere da una misura dell'energia della particella nella buca, sono:\marginpar{Autovalori di $H$ per la buca infinita}
\begin{equation}
    \mathcal{E}_n = \frac{\hbar^2}{2m}\left(\frac{n\pi}{a}\right)^2
    \label{eqn:autovalori_buca}
\end{equation}
Cerchiamo ora gli autovettori $\varphi_{\mathcal{E}_n}$ (o autofunzioni) ad essi associati\footnote{Ometteremo nei prossimi passaggi la $n$ a pedice per economia di notazione}. Una \q{furbizia} per ottenerli più velocemente è riscrivere la condizione per la singolarità (\ref{eqn:matrice_singolare}) come:
\begin{align*}
\sin(k_n a) = 2\sin\left (k_n\frac{a}{2}\right )\cos\left (k_n \frac{a}{2}\right)=0 \\ \Rightarrow 
\underbrace{\sin\left(\frac{k_n a}{2}\right) = 0}_{(a)}\>\lor\>
\underbrace{\cos\left(\frac{k_n a}{2}\right) = 0}_{(b)}
\end{align*}
Esaminiamo separatamente i casi (a) e (b).
\begin{enumerate}[label=\alph*)]
\item $\displaystyle \sin k_n \frac{a}{2} = 0\Rightarrow k_n \frac{a}{2}=n\pi \Rightarrow k_n = \frac{2n\pi}{a}$.\\
Riscrivendo poi il $\sin$ come:
\begin{align*}
\sin\left(k_n\frac{a}{2}\right)=0&\Rightarrow \frac{1}{2i}\left[
\exp\left(ik_n\frac{a}{2}\right)-\exp\left(-ik_n\frac{a}{2}\right)
\right] = 0\\
&\Rightarrow \exp\left({ik_n\frac{a}{2}}\right)=\exp\left({-ik_n\frac{a}{2}}\right)
\end{align*}
E quindi possiamo semplificare l'espressione (\ref{eqn:condizioni_contorno_buca}), da cui ricaviamo:
\begin{align*}
c_+ + c_- = 0 &\Rightarrow c_- = -c_+\\
\varphi_\mathcal{E}^{2n}(x)&=A\sin \left(\frac{2n\pi}{a}x\right)
\end{align*}
Notiamo che $n\neq 0$ (se così non fosse si otterrebbe la soluzione nulla, che non è fisica), e inoltre $n$ e $-n$ corrispondono allo stesso stato:
\[
\varphi_\mathcal{E}^{2n}(x)=-\varphi_{\mathcal{E}}^{-2n}(x)
\]
Normalizzando:
\begin{align*}
1 &\overset{!}{=} \int_{-\frac{a}{2}}^{\frac{a}{2}} |\varphi_\mathcal{E}^{2n}(x)|^2 dx = \int_{-\frac{a}{2}}^{\frac{a}{2}} |A|^2 \sin^2 \frac{2n\pi x}{a}dx = |A|^2 \frac{a}{2} \\
&\Rightarrow \varphi_\mathcal{E}^{2n}(x)=\sqrt{\frac{2}{a}}\sin \left(\frac{2n\pi}{a}x\right)
\end{align*}
essendo la media integrale di $\sin^2$ pari a $1/2$.
\item Analogamente, per la condizione (b) avremo:
\[
\cos\frac{k_n a}{2}=0 \Rightarrow k_n = (2n+1)\frac{\pi}{a}
\]
e riscrivendo il $\cos$ tramite esponenziali si ottiene:
\begin{align*}
\cos\left(\frac{k_n a}{2}\right ) &= \frac{1}{2}\left [
\exp\left(i k_n \frac{a}{2}\right)+\exp\left(-ik_n\frac{a}{2}\right)
\right ] = 0\\
&\Rightarrow \exp\left({ik_n \frac{a}{2}}\right)=-\exp\left({-ik_n\frac{a}{2}}\right)
\end{align*}
Possiamo allora semplificare la (\ref{eqn:condizioni_contorno_buca}):
\begin{align*}
c_+ \exp\left({-ik_n\frac{a}{2}}\right)+c_- \exp\left({ik_n \frac{a}{2}}\right)=0 &\Rightarrow c_+ = +c_-
\end{align*}
La normalizzazione è identica al caso precedente, e alla fine si ottiene:
\begin{align*}
\varphi_{\mathcal{E}}^{2n+1}(x)&=\sqrt{\frac{2}{a}}\cos\left(\frac{(2n+1)\pi}{a} x\right)
\end{align*}
\end{enumerate}
Alla fine, le soluzioni dell'equazione agli autovalori in $D(H)$ sono quindi:\marginpar{Autofunzioni di $H$ per la buca infinita}
\[
\varphi_n(x) = \begin{cases}
\sqrt{\frac{2}{a}}\cos\left(\frac{n\pi x}{a}\right) & n \text{ dispari } >0\\
\sqrt{\frac{2}{a}}\sin\left(\frac{n\pi x}{a}\right) &
n \text{ pari } > 0
\end{cases}
\]
Notiamo che tali autofunzioni sono pari per $n$ dispari, e dispari per $n$ pari (con $n>0$).\\
Abbiamo quindi dimostrato che esiste una (sola) soluzione $\forall n \in \bb{N}$, per cui la degenerazione di $\sigma_P(H)=1$.\\
Inoltre, poiché $\{\varphi_n\}_{n\in\bb{N}}$ costituisce una base ortonormale per $L^2([-\frac{a}{2},\frac{a}{2}],dx)$, lo spettro è esclusivamente puntuale, e quindi non c'è $\sigma_C(H)$.\\
\subsubsection{Differenze tra \MC e \MQ}
\begin{itemize}
\item In \MC l'energia assume valori \textbf{continui} $\sigma(H_{clas.})=\bb{R}_+$, mentre in \MQ lo spettro è \textbf{discreto}:
\[
\sigma(H_q)=\{\mathcal{E}_n =\ \frac{\hbar^2 k_n^2}{2m} = \hbar^2 \frac{n^2 \pi^2}{2m a^2}, \> n\in \bb{N}\}
\]
In questo senso i sistemi quantistici con stati legati sono \q{più indeformabili}: per modificarli è necessario dare un certo contributo minimo di energia, mentre nel caso classico basta qualsiasi contributo energetico, per quanto piccolo.
\item Mentre in \MC la conoscenza di $\mathcal{E}$ \textbf{non} è massimale (\textbf{stato misto}), in \MQ lo è (\textbf{stato puro}).
\item La particella quantistica \q{ha dei punti in cui preferisce trovarsi}, e dei punti in cui proprio non si può trovare. Ciò è molto controintuitivo se la pensiamo come particella (\q{biglia}) classica, ma facile da comprendere in un'ottica ondulatoria: un'onda stazionaria presenta infatti dei nodi fissi in cui l'ampiezza dell'oscillazione è nulla.
\end{itemize}
Il momento in questo dominio compatto non è definito (come abbiamo visto nel primo esempio di sezione \ref{sec:momento_compatto}). Possiamo però calcolare la densità di probabilità della posizione:
\[
w_{\varphi_n}^X(\lambda) = \int dx\, \delta(\lambda-x) |\varphi_n(x)|^2 = |\varphi_n(\lambda)|^2 = \frac{2}{a}\begin{cases}
\cos^2 \left(\frac{n\pi \lambda}{a}\right) & n \text{ dispari }\\
\sin^2\left(\frac{n\pi \lambda}{a}\right) & n \text{ pari}
\end{cases}
\]
\begin{figure}[H]
    \centering
    \begin{gnuplot}[terminal=epslatex, terminaloptions=color dashed,terminaloptions={size 15cm,18cm}]
set multiplot layout 2,1 title "Buca infinita di potenziale in 1D"
psi1(x) = sqrt(2)*cos(pi*x)
psi2(x) = sqrt(2)*sin(2*pi*x)
psi3(x) = sqrt(3)*cos(3*pi*x)
set xrange[-0.6:0.6]
set samples 10000
set lmargin 3
set bmargin 0
set rmargin 3
set tmargin 2


set yrange[-1.9:1.9]
set ytics -1.5,.5,1.5
set style rect fc lt -1 fs solid 0.15 noborder
set obj rect from -0.6, graph 0 to -0.5, graph 1
set obj rect from 0.5, graph 0 to 0.6, graph 1

set style line 1 lt 1 lw 6 lc rgb "#006d2c"
set style line 2 lt 1 lw 4 lc rgb "#2ca25f"
set style line 3 lt 1 lw 2 lc rgb "#99d8c9"
set style line 12 lt 2 dt 2 lw 1 lc rgb "#dddddd"
set grid ytics, xtics ls 12

set xtics ("" 0)
set key title "$\\varphi_n(x)$"
plot (x<0.5 && x > -0.5) ? psi1(x) : 0 ls 1 title "$n=1$", (x<0.5 && x > -0.5) ? psi2(x) : 0 ls 2 title "$n=2$", (x<0.5 && x > -0.5) ? psi3(x) : 0 ls 3 title "$n=3$

f1(x) = 2*cos(pi*x)**2
f2(x) = 2*sin(2*pi*x)**2
f3(x) = 2*cos(3*pi*x)**2
set tmargin 0
set bmargin 2
set xtics ("$-\\displaystyle\\frac{a}{2}$" -0.5, "$0$" 0, "$+\\displaystyle\\frac{a}{2}$" 0.5) offset 0,-0.4

set yrange[0:2.1]
set ytics 0,.4,2
set key title "$w_{\\varphi_n}^X(\\lambda)$"

plot (x<0.5 && x > -0.5) ? f1(x) : 0 ls 1 title "$n=1$", (x<0.5 && x > -0.5) ? f2(x) : 0 ls 2 title "$n=2$", (x<0.5 && x > -0.5) ? f3(x) : 0 ls 3 title "$n=3$"
\end{gnuplot}
    \caption{Grafico delle autofunzioni $\varphi_n(x)$ per la buca infinita e delle relative densità di probabilità $w_{\varphi_n}^X(\lambda)$}
    \label{fig:plot_bucainfinita}
\end{figure}

\textbf{Nota}: se consideriamo un potenziale \q{smussato}, che non tende di scatto a $\infty$ ma \q{in maniera continua} con una salita ripida, classicamente ci aspettiamo che la particella \q{rallenti} su tali salite, e quindi ci trascorra più tempo - ci aspettiamo di trovarla più facilmente in quelle parti. In realtà, quantisticamente, quelle sono le posizioni in cui la probabilità per lo stato di energia minima è minore).\\

%Inserire disegno [IMMAGINE] Grafico del potenziale smussato e della densità di probabilità della posizione, che è massima al centro della buca
La probabilità $P^X_{\varphi_n}(\lambda)$ non è poi uniforme come nel caso classico, ma tende ad esso\marginpar{Principio di corrispondenza per la buca infinita} per $n\to\infty$, infatti, dalle formule di bisezione per $\sin$ e $\cos$:
\begin{align*}
\left\{\cos^2 \left(\frac{n\pi x}{a}\right),\>
\sin^2 \left(\frac{n\pi x}{a}\right)\right\} &= \frac{1}{2}\left(1\pm \cos \frac{2n\pi x}{a} \right)\xrightarrow[n\to\infty]{\mathcal{S}'} +\frac{1}{2}\\
P_{\varphi_n}^X(\lambda) &\xrightarrow[n\to\infty]{} \frac{2}{a}\frac{1}{2} = \frac{1}{a}
\end{align*} %[TO DO] Specificare esattamente cosa succede a fronte dell'approfondimento del 7/11
Dove si fa uso della convergenza debole\footnote{CFR pag. 24 di \cite{spazi_hilbert}} in $\mathcal{S}'$.\\

Proviamo \marginpar{Esempio (buca infinita)}ora ad esaminare l'evoluzione temporale del sistema partendo dallo stato iniziale $\psi(x,t_0)$ così definito:
\begin{equation}
\psi(x,t_0) = \cos\frac{\pi}{a} x+2\sin\frac{2\pi}{a}x
\label{eqn:stato_iniziale}
\end{equation}
Per prima cosa dobbiamo riscrivere la $\ket{\psi(t_0)}$ nella base degli autostati di $H$, e quindi applicare l'evoluzione temporale:
\begin{align}
\ket{\psi(t_0)} &= \sum_n \ket{\varphi_n}\braket{\varphi_n|\psi(t_0)}\nonumber \\
\label{eqn:evoluzione_autovalori}
\ket{\psi(t)} &= \sum_n \underbrace{\exp\left(-\frac{i}{\hbar}\mathcal{E}_n(t-t_0)\right)}_{U(t)}\ket{\varphi_n}\braket{\varphi_n|\psi(t_0)}
\end{align}

Riepilogando, il procedimento è questo:
\begin{enumerate}
\item Normalizzare $\norm{\psi(t_0)}^2 = 1$\\
\item Risolvere l'equazione agli autovalori per $H$, imponendo le condizioni al contorno date dal dominio di $H$, in modo da trovare le autofunzioni $\varphi_{\mathcal{E}_n}$ associate agli autovalori $\mathcal{E}_n$.
\item Scrivere $\ket{\psi(t_0)}$ e l'operatore di evoluzione $U(t)$ in termini di autovalori e autovettori di $H$, e poi applicare la formula (\ref{eqn:evoluzione_autovalori}).
\end{enumerate}

In questo caso abbiamo già risolto l'equazione agli autovalori, trovando le $\varphi_n$ associate agli autovalori $\mathcal{E}_n$. Scrivendo $\ket{\psi(t_0)}$ in tale base e normalizzando:
\begin{align*}
\ket{\psi(t_0)} &= A\sqrt{\frac{a}{2}}(\ket{\varphi_1}+2\ket{\varphi_2})\\
\braket{\psi(t_0)|\psi(t_0)} &= |A|^2 \frac{a}{2}(1+4)\Rightarrow A=\sqrt{\frac{2}{5a}}
\end{align*}
dove si è usato il fatto che le $\ket{\varphi_i}$ sono ortonormali.\\
Applicando allora l'evoluzione temporale:
\begin{align}
\ket{\psi(t_0)} &= \frac{1}{\sqrt{5}}(\ket{\varphi_1}+2\ket{\varphi_2})\nonumber\\
\label{eqn:evoluzione_buca_t}
\braket{x|\psi(t)} &= \frac{1}{\sqrt{5}}
\left (
\exp\left(-\frac{i}{\hbar}\mathcal{E}_1(t-t_0)\right) \underbrace{\braket{x|\varphi_1(t)}}_{\varphi_1(x)} + 2\exp\left(-\frac{i}{\hbar}\mathcal{E}_2(t-t_0)\right)
\underbrace{\braket{x|\varphi_2}}_{\varphi_2(x)} \right)\\
\mathcal{E}_i &= \frac{\hbar^2}{2m}\left(\frac{n\pi}{a}\right)^2
\end{align}
%\bra{\mathcal{E}_n}H\ket{\mathcal{E}_n}=H_{nm}
\end{document}

