\documentclass[../../FisicaTeorica.tex]{subfiles}

%\usepackage[usenames, dvipsnames, table]{xcolor}
\usepackage[utf8]{inputenc}
\usepackage[T1]{fontenc}
\usepackage{lmodern}
\usepackage{amsmath}
\usepackage{amsfonts}
\usepackage{comment}
\usepackage{wrapfig}
\usepackage{booktabs}
\usepackage{braket}
\usepackage{tikz}
\usepackage{gnuplottex}
\usepackage{epstopdf}
\usepackage{marginnote}
\usepackage{float}
\usetikzlibrary{tikzmark}
\usepackage{graphicx}
\usepackage{cancel}
\usepackage{bm}
\usepackage{mathtools}
\usepackage{hyperref}
\usepackage{ragged2e}
\usepackage[stable]{footmisc}
\usepackage{enumerate}
\usepackage{mathdots}
\usepackage[framemethod=tikz]{mdframed}
\PassOptionsToPackage{table}{xcolor}
\usepackage{soul}
\usepackage{enumerate}
\usepackage{mathdots}
\usepackage[framemethod=tikz]{mdframed} %Added 16/10
\usepackage[italian]{babel} %Added 16/10
\usepackage{amssymb} %Added

%%BOOKTAB
\setlength{\aboverulesep}{0pt}
\setlength{\belowrulesep}{0pt}
\setlength{\extrarowheight}{.75ex}
\setlength\parindent{0pt} %Rimuove indentazione


%%GEOMETRIA
\usepackage[a4paper]{geometry}
 \newgeometry{inner=20mm,
            outer=49mm,% = marginparsep + marginparwidth 
                       %   + 5mm (between marginpar and page border)
            top=20mm,
            bottom=25mm,
            marginparsep=6mm,
            marginparwidth=30mm}

\makeatletter
\renewcommand{\@marginparreset}{%
  \reset@font\small
  \raggedright
  \slshape
  \@setminipage
}
\makeatother
 

%%COMANDI
\newcommand{\q}[1]{``#1''}
\newcommand{\lamb}[2]{\Lambda^{#1}_{\>{#2}}}
\newcommand{\norm}[1]{\left\lVert#1\right\rVert}
\newcommand{\hs}{\mathcal{H}}
\newcommand{\minus}{\scalebox{0.75}[1.0]{$-$}}
\newcommand{\hlc}[2]{%
  \colorbox{#1!50}{$\displaystyle#2$}}
\newcommand{\bb}[1]{\mathbb{#1}}
\newcommand{\op}[1]{\operatorname{#1}}
\renewcommand{\figurename}{Fig.}

\usepackage{fancyhdr}
\pagestyle{fancy}
\fancyhead{} % clear all header fields
\renewcommand{\headrulewidth}{0pt} % no line in header area
\fancyfoot{} % clear all footer fields
\fancyfoot[R]{Francesco Manzali, 2018-19} % other info in "inner" position of footer line
\cfoot{\thepage}

%%AMBIENTI
\newtheorem{thm}{Teorema}[section]
\newtheorem{dfn}{Definizione}
\newtheorem{oss}{Osservazione}
\newtheorem{es}{Esempio}
\newtheorem{axi}{Assioma}
%%Domande di Marchetti
\newtheorem{question}{Domanda}


%%OPERATORI
\DeclareMathOperator{\sech}{sech}
\DeclareMathOperator{\csch}{csch}
\DeclareMathOperator{\arcsec}{arcsec}
\DeclareMathOperator{\arccot}{arcCot}
\DeclareMathOperator{\arccsc}{arcCsc}
\DeclareMathOperator{\arccosh}{arcCosh}
\DeclareMathOperator{\arcsinh}{arcsinh}
\DeclareMathOperator{\arctanh}{arctanh}
\DeclareMathOperator{\arcsech}{arcsech}
\DeclareMathOperator{\arccsch}{arcCsch}
\DeclareMathOperator{\arccoth}{arcCoth} 




\begin{document}
\section{Lezione 4:\\ \large{Il Formalismo di Dirac}}
\vspace{-1em}
\begin{center}
    \small{(17-18/10/2018)}
\end{center}
Osserviamo ora più nel dettaglio la notazione di Dirac, finora occasionalmente utilizzata senza definizioni precise.\\
In effetti, Dirac elaborò il suo formalismo ciò senza utilizzare spazi di Hilbert - che del resto ancora non esistevano. Solo tramite la teoria delle distribuzioni e una matematica decisamente sofisticata è possibile però dare pienamente senso alla sua notazione, che assume così i caratteri e la consistenza propri della matematica.\\
L'idea alla base della notazione di Dirac è quella di partire dal formalismo per lo spettro discreto e applicarlo, in maniera naturale, anche quello continuo. Vediamo come.\\
\subsection{Formalismo per lo spettro discreto}
Sia $A$ un operatore autoaggiunto, il cui spettro è solo dato dagli autovalori \q{di algebra lineare}, e quindi è puramente puntuale: $\sigma \left(A\right)= \sigma_P\left(A\right)= \left\{\lambda_n\right\}$.\\
In notazione di Dirac, l'equazione agli autovalori diviene:
\[
A \left|\lambda_n\right\rangle=\lambda_n|\lambda_n \rangle 
\]
Per semplicità ci limiteremo per ora ai casi in cui i $\lambda_n$ non presentano degenerazione, ossia per cui ad ogni autovalore corrisponde un solo autovettore\footnote{Nel caso finito dimensionale questa è la situazione delle matrici diagonalizzabili}.\\
Sappiamo che i $\ket{\lambda_n}$  costituiscono una base ON, quindi possiamo scrivere un qualsiasi altro ket o bra come somme delle loro proiezioni su di essi:
\begin{equation}
\left|\psi\right\rangle=\sum_{n}{|\lambda_n\rangle \langle\lambda_n|\psi\rangle }; \quad 
\langle \phi |=\sum_{n}{\langle\phi|\lambda_n\rangle \langle\lambda_n|}
\label{eqn:bra-ket}
\end{equation}
Ove $\langle \lambda_n|$ sono i funzionali in $\mathcal{H}^\ast$ associati a $|\lambda_n \rangle$  per Riesz.\\
La relazione di Parseval diviene:
\begin{equation}
\left\langle\phi\middle|\psi\right\rangle=\sum_{n}{\langle\phi|\lambda_n\rangle \langle\lambda_n|\psi\rangle }
\label{eqn:parseval-relazione}
\end{equation}
Si ha che (\ref{eqn:bra-ket}) e (\ref{eqn:parseval-relazione}) sono compatibili solamente se:
\[
\left\langle\lambda_n\middle|\lambda_m\right\rangle=\delta_{mn}
\]
Tale risultato è sintetizzato nella \textbf{completezza di Dirac}:
\[
\sum_{n}{|\lambda_n\rangle \langle\lambda_n|}=\bb{I}
\]
In questo modo è allora definita anche la rappresentazione spettrale di $A$:
\[
A\sum_{n}{\left|\lambda_n\right\rangle\left\langle\lambda_n\right|=\sum_{n}{\lambda_n\left|\lambda_n\right\rangle\left\langle\lambda_n\right|=A}}
\]
\subsection{Formalismo per lo spettro continuo}
Se invece $\lambda \in \sigma_C\left(A\right)$, che senso ha l'equazione agli autovalori $A\left|\lambda\right\rangle=\lambda |\lambda  \rangle$?\\
Infatti, come abbiamo visto, se $\lambda \in \sigma_C(A)$, $|\lambda\rangle$  non può essere un autovettore in $\hs$!\\
Non è però escluso che possa esserlo in uno spazio \q{più grande} di $\hs$.\\
Prendiamo per esempio l'operatore posizione $A=X$, che agisce su $\ket{\lambda}$ come $X\ket{\lambda} = x \ket{\lambda}$. Allora l'equazione agli autovalori diviene (nella rappresentazione in posizioni):
\[
X\ket{\lambda} = \lambda\ket{\lambda} \Rightarrow x\ket{\lambda} = \lambda\ket{\lambda} \Rightarrow (x-\lambda)\ket{\lambda} = 0
\]
Un \textit{oggetto} che soddisfa questa condizione (e che quindi può fungere da \q{autovettore} generalizzato) è la delta di Dirac. Indichiamola con $\delta(x-\lambda) = \braket{x|\lambda}$. Sostituendola al posto dell'autovettore:
\[
(x-\lambda )\langle x | \lambda \rangle = 0 \Rightarrow (x-\lambda)\delta(x-\lambda) = 0
\]
%Sistemare qui [TO DO]
Manteniamo allora, formalmente, la struttura dell'equazione agli autovalori come nel caso discreto:
\[
A\ket{\lambda} = \lambda \ket{\lambda}
\]
anche se effettivamente tali autovalori $\lambda$ non esistono in $\hs$.\\
Ripetendo allora gli stessi ragionamenti di prima, consideriamo un operatore $A$ il cui spettro $\sigma(A)$ sia puramente continuo:
\[
\lambda \in \sigma_C\left(A\right)=\sigma (A)
\]
(di nuovo, limitiamoci al caso di autovalori senza degenerazione).\\
Allora bra e ket sono dati da:
\begin{equation}
\left|\psi\right\rangle= \int_{\sigma_C(A)}{d\lambda\ \left|\lambda\right\rangle\langle\lambda|\psi\rangle };\quad
\left\langle\phi\right|= \int_{\sigma_{C\left(A\right)}}{d\lambda\ \left\langle\phi\middle|\lambda\right\rangle\langle\lambda|}
\label{eqn:bra-ket-continui}
\end{equation}
E la relazione di Parseval diviene:
\begin{equation}
    A\left|\lambda\right\rangle=\lambda 
\left\langle\phi\middle|\psi\right\rangle= \int_{\sigma_{C\left(A\right)}}{d\lambda\ \langle\phi|\lambda\rangle \langle\lambda|\psi\rangle }
    \label{eqn:parseval-continui}
\end{equation}
%Inserire equazione agli autovalori
E (\ref{eqn:bra-ket-continui}) è compatibile con (\ref{eqn:parseval-continui}) solamente se vale:
\[
\left\langle\lambda\middle|\lambda'\right\rangle= \delta \left(\lambda-\lambda'\right)
\]
Che si sintetizza nella \textit{completezza}:
\[
\int_{\sigma_{C\left(A\right)}}{d\lambda\left|\lambda\right\rangle\left\langle\lambda\right|}=\bb{I}
\]
La rappresentazione spettrale di un operatore $A$ è dunque:
\[
A\int d\lambda  \left|\lambda\right\rangle\left\langle\lambda\right|=\int d\lambda\,\lambda \left|\lambda\right\rangle\left\langle\lambda\right|=A 
\]
\subsection{Formalismo per uno spettro generale}
Il formalismo di Dirac si generalizza subito allo spettro più generale (sia discreto che continuo) $\sigma \left(A\right)= \sigma_P(A)\cup \sigma_C\left(A\right)$ e con degenerazioni.\\
Preso un autovalore $\lambda_n\in \sigma_P(A)$ o un autovalore generalizzato ($\lambda \in \sigma_C(A)$) diremo che esso ha \textbf{degenerazione} $d(\lambda_n)$ o $d\left(\lambda\right)$ se le equazioni agli autovalori corrispondenti (nel senso di Dirac) hanno $d\left(\lambda_n\right)$ o $d\left(\lambda\right)$ soluzioni indipendenti (in $\hs$) che denotiamo rispettivamente con $\ket{\lambda_{n}, r}$ (per $r=1,\dots, d(\lambda_n$) e $\ket{\lambda, r}$ (per $r = 1,\dots, d(\lambda)$, con $d(\lambda) \in \bb{N}$ e potenzialmente $\infty$).\\ 
Allora la completezza si scrive:
\[
\sum_{\lambda_n\in\sigma_P\left(A\right)}\sum_{r=1}^{d\left(\lambda_n\right)}{\left|\lambda_n,r\right\rangle\left\langle\lambda_n,r\right|+\int_{\sigma_C\left(A\right)}\sum_{r=1}^{d\left(\lambda\right)}\left|\lambda,r\right\rangle\left\langle\lambda,r\right|=\mathbb{I}_\mathcal{H}}
\]
E tramite essa possiamo scrivere ogni funzione:
\[
f\left(A\right)= \sum_{\lambda_n\in\sigma_D\left(A\right)} f\left(\lambda_n\right)\sum_{r=1}^{d\left(\lambda_n\right)}\left|\lambda_n,r\right\rangle\left\langle\lambda_n,r\right|+\int_{\sigma_C\left(A\right)}{d\lambda f\left(\lambda\right)}\sum_{r=1}^{d\left(\lambda\right)}{\left|\lambda,r\right\rangle\langle\lambda,r|}
\]
Notiamo che anche in casi \q{semplici} è necessario la completezza nel caso più generale.\\
Per esempio, sia $H$ l'Hamiltoniana dell'atomo di idrogeno 
\[
\sigma \left(H\right)=\left\{-\frac{c}{n^2},n\in\mathbb{N}\right\}\cup \left\{x>0,x\in\mathbb{R}\right\}
\]
Infatti, negli stati in cui l'elettrone è legato l'energia è quantizzata, con degenerazione $n^2$, ma quando è libero possiamo scegliere una qualsiasi energia in un range continuo, e un qualsiasi numero di stati hanno una determinata energia (degenerazione infinita). Abbiamo perciò sia uno spettro discreto (con degenerazione) che uno continuo (con degenerazione addirittura $\infty$).

\subsection{Spazio degli autovettori dello spettro continuo}
Come visto prima, se $\lambda \in \sigma_C(A)$ gli autovettori $\ket{\lambda}$, $\ket{\lambda,r}$ non si trovano in $\hs$, ma in uno spazio più grande (sono autovettori \q{generalizzati}). Definiamolo.\\
Sia $\braket{x|\lambda} \equiv F_\lambda(x)$. Applicandovi l'operatore posizione $X$ otteniamo:
\[
XF_\lambda(x) = \lambda F_\lambda(x); \quad F_\lambda(x) = \delta(x-\lambda) \notin L^2(\bb{R})
\]
dove $\delta(x-\lambda)$ è una distribuzione, e perciò appartiene a $\mathcal{S}'(\bb{R})$. Analogamente con il momento, sia $\braket{x|\mu}\equiv F_\mu(x)$:
\[
P F_\mu(x) = \mu F_\mu(x); \quad F_\mu(x) = \exp\left ( i\frac{\mu}{\hbar}x\right ) \notin L^2(\bb{R}),\> \in \mathcal{S}'(\bb{R})
\] %Perché l'esponenziale? SIgnfiicato di F_\mu?
Il fatto che le soluzioni dell'equazione agli autovalori non siano in $L^2(\bb{R})$ significa che esse non corrispondono a \q{stati fisici}, fisicamente possibili. In particolare, ciò vuol dire che non è possibile localizzare una particella perfettamente in un punto, né assegnarle un singolo momento definito.\\
$\mathcal{S}'\left(\mathbb{R}\right)$ è più precisamente lo spazio dei funzionali lineari continui su $\hs(\bb{R})$, che è lo spazio delle funzioni $\mathcal{C}^\infty$ che descrescono all'infinito (assieme a tutte le loro derivate) più rapidamente di $x^{-n} \forall n$.\\
Possiamo generalizzare a $n$-dimensioni:
\begin{itemize}
    \item $\mathcal{S}\left(\mathbb{R}^n\right)=$ funzioni $\mathcal{C}^\infty$ su $\mathbb{R}^n$ che decrescono più velocemente della norma $\left|\left|x\right|\right|^{-m} \forall m$
    \item $\mathcal{S}'(\bb{R}^n)$ = spazio dei funzionali lineari continui su $\mathcal{S}(\bb{R}^n)$
\end{itemize}
\begin{oss}
Notiamo che anche \q{funzioni regolarissime} non appartengono a $\mathcal{S}'(\bb{R})$. Infatti, le distribuzioni regolari (descritte da funzioni) in $\mathcal{S}'(R)$ non possono crescere più rapidamente  di un polinomio in $x$, per $x\rightarrow \infty$, e perciò $e^x\notin \mathcal{S}'\left(\mathbb{R}\right)$.
\end{oss}
Sappiamo poi (da metodi) che vale la seguente catena di inclusioni:
\begin{equation}
S\left(\mathbb{R}\right)\subset L^2\left(\mathbb{R}\right)\subset \mathcal{S}'\left(\mathbb{R}\right)
\label{eqn:triplettametodi}
\end{equation}
L'idea è perciò quella di risolvere l'equazione agli autovalori nello \q{spazio più grande} $\mathcal{S}'(R)$. Continuando gli esempi su posizione $X$ e momento $P$ visti sopra, potremo trovare soluzioni $\forall \lambda \in \bb{R}$ e perciò avremo, come desiderato, che:
\[
\sigma(X) = \bb{R} = \sigma_C(X); \quad \sigma(P) = \bb{R} = \sigma_C(X)
\]
\lesson{11}{18/10/2018}
La differenza tra \q{autovettori} generalizzati che stanno in $\mathcal{S}'$ e autovettori \q{di algebra lineare} che stanno in $\hs$ è alla base della differenza tra \textit{spettro discreto} e \textit{spettro continuo}.\\
Concretamente, se la soluzione all'equazione agli autovalori è $\in \mathcal{S}'\left(\mathbb{R}\right)$, ma $\notin \hs$ si ottiene lo \textbf{spettro continuo}, se invece è $\in \hs$ si ha lo \textbf{spettro discreto}.\\

L'idea (di Gel'fand) per trovare lo\marginpar{Spettro di un operatore $A$ autoaggiunto per Gel'fand} spettro di un operatore autoaggiunto $A$ nel formalismo di Dirac sta quindi nel trovare uno spazio $\phi_A \subset \hs$ tale che:
\begin{equation}
\phi_A \subset \hs \approx \hs' \subset \phi_A'
\label{tripletta-gelfand}
\end{equation}
Tale catena di inclusioni, che generalizza quella vista in (\ref{eqn:triplettametodi}), è detta \textbf{tripletta di Gelfand}, e $\phi_A$ è detto \textbf{spazio di Hilbert equipaggiato} (o \textit{rigged}).\\
Perché allora il formalismo di Dirac sia ben definito, è necessario fare le seguenti richieste nella definizione di $\phi_A$:
\begin{enumerate}
    \item $\displaystyle \phi_A\subseteq D\left(A\right)$
	\item $\phi_A$ è denso in $\hs$ nella topologia di $\hs$, $\bar{\mathcal{S}}\left(\mathbb{R}\right)=L^2(R)$
	\item $A$ è continuo in $\phi_A$ nella topologia di $\phi_A$
	\item $\phi_A$ è \q{nucleare} cioè i funzionali lineari continui sullo spazio delle coppie $\phi_A\times \phi_A$ (con $f,g\in \mathcal{S}(\bb{R})\times \mathcal{S}\left(\mathbb{R}\right))$ sono lineari e continui anche nello spazio delle combinazioni lineari (eventualmente infinite) dei prodotti come $f\left(x,y\right)\in S\left(\mathbb{R}^2\right)$.\\
	
	%%[TO DO] Sistemare questa parte!
	(lo spazio dei funzionali lineari continui su $\mathcal{S}\left(\mathbb{R}\right)\times \mathcal{S}\left(\mathbb{R}\right)$ sono lineari continui su $\mathcal{S}(\mathbb{R}^2)$)\\
	Cerchiamo di \q{giustificare} la richiesta di nuclearità. 
	Se $\sigma \left(A\right)=\sigma_P(A)$ (senza degenerazione):
	\[
	\sum_{n}{|\lambda_n\rangle \langle\lambda_n|}=\mathbb{I}_\mathcal{H}
	\left\langle\phi\left|A\right|\psi\right\rangle=\left\langle\phi\middle|\mathbb{I}A\mathbb{I}\right\rangle= 
	= \sum_{n,m}{\left\langle\phi\middle|\lambda_n\right\rangle\left\langle\lambda_n\left|A_{nm}\right|\lambda_m\right\rangle\left\langle\lambda_m\middle|\psi\right\rangle=\ \sum_{n,m}{A_{nm}\phi_n^\ast\psi_{nm}}} 
	\left\langle\lambda_m\middle|\psi\right\rangle= \psi_m
	\left\langle\lambda_n\ \middle|\phi\right\rangle=\phi_n
	\left\langle\phi\left|A\right|\psi\right\rangle
	\] è lineare sia in $\phi_n^\ast$ che in $\psi_n$, ma è lineare anche in $\phi_n^\ast\psi_n$ (e in tutte le loro combinazioni lineari).\\
	Se si vuole generalizzare allo spettro continuo $\sigma \left(A\right)= \sigma_C\left(A\right)$ senza degenerazione (per semplicità)
	\[
	\int_{\sigma_C\left(A\right)}{d\lambda\ \left|\lambda\right\rangle\langle\lambda|}=\mathbb{I}_\mathcal{H}
	\left\langle\phi\left|A\right|\psi\right\rangle=\left\langle\phi\left|\mathbb{I}A\mathbb{I}\right|\psi\right\rangle=\int d\lambda  \lambda'\left\langle\phi\middle|\lambda'\right\rangle\left\langle\lambda\left|a\right|\lambda'\right\rangle\left\langle\lambda'\middle|\psi\right\rangle=\int d\lambda  d\lambda'\left\langle\lambda\left|A\right|\lambda'\right\rangle\left\langle\lambda|\phi\right\rangle^\ast\left\langle\lambda\middle|\psi\right\rangle
	\]
	Ma ciò è falso in $L^2$, infatti con $A=I$\\
	Siano $f,g\in L^2$
	\[
	\left(f,\mathbb{I}g\right)=\int k\left(x,y\right)f^\ast\left(x\right)g\left(y\right)dx dy
	=\int f^\ast\left(x\right) g\left(x\right)dx
	\] lineare e continuo in $f^\ast$ e in $g$
	Non esiste alcun funzionale lineare continuo
	$ k\left(x,y\right) su L^2\left(\mathbb{R}\right)\otimes L^2\left(\mathbb{R}\right) su L^2(\mathbb{R}^2)$ che per Riesz dovrebbe appartenere a $L^2\left(\mathbb{R}^2\right)$, ma $k\left(x,y\right)=\delta \left(x-y\right)\notin L^2\left(\mathbb{R}^2\right)$
	Ma $\in \mathcal{S}'\left(\mathbb{R}^2\right)$
\end{enumerate}
%Copiare da onenote!
\end{document}