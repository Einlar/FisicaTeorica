\documentclass[../../FisicaTeorica.tex]{subfiles}

%%%PACKAGES
\usepackage[usenames, dvipsnames, table]{xcolor}
\usepackage[utf8]{inputenc}
\usepackage[T1]{fontenc}
\usepackage{lmodern}
\usepackage{amsmath}
\usepackage{amsthm}
\usepackage{amsfonts}
\usepackage{comment}
\usepackage{wrapfig}
\usepackage{booktabs}
\usepackage{braket}
\usepackage{pgf,tikz}
\usepackage{mathrsfs}
\usetikzlibrary{arrows}
\usepackage{subfigure}
\usepackage{xspace}
\usepackage{gnuplottex}
\usepackage{epstopdf}
\usepackage{marginnote}
\usepackage{float}
\usetikzlibrary{tikzmark}
\usepackage{graphicx}
\usepackage{cancel}
\usepackage{bm}
\usepackage{mathtools}
\usepackage{hyperref}
\usepackage{ragged2e}
\usepackage[stable]{footmisc}
\usepackage{enumerate}
\usepackage{mathdots}
\usepackage[framemethod=tikz]{mdframed}
\PassOptionsToPackage{table}{xcolor}
\usepackage{soul}
\usepackage{enumerate}
\usepackage{mathdots}
\usepackage[framemethod=tikz]{mdframed} %Added 16/10
\usepackage[italian]{babel} %Added 16/10
\usepackage{amssymb} %Added
\usepackage{enumitem}
\usepackage{array}


%%BOOKTAB
\setlength{\aboverulesep}{0pt}
\setlength{\belowrulesep}{0pt}
\setlength{\extrarowheight}{.75ex}
\setlength\parindent{0pt} %Rimuove indentazione


%%GEOMETRIA
\usepackage[a4paper]{geometry}
 \newgeometry{inner=20mm,
            outer=49mm,% = marginparsep + marginparwidth 
                       %   + 5mm (between marginpar and page border)
            top=20mm,
            bottom=25mm,
            marginparsep=6mm,
            marginparwidth=30mm}
\makeatletter
\renewcommand{\@marginparreset}{%
  \reset@font\small
  \raggedright
  \slshape
  \@setminipage
}
\makeatother
 

%%COMANDI
\newcommand{\q}[1]{``#1''}
\newcommand{\lamb}[2]{\Lambda^{#1}_{\>{#2}}}
\newcommand{\norm}[1]{\left\lVert#1\right\rVert}
\newcommand{\hs}{\mathcal{H}}
\newcommand{\minus}{\scalebox{0.75}[1.0]{$-$}}
\newcommand{\hlc}[2]{%
  \colorbox{#1!50}{$\displaystyle#2$}}
\newcommand{\bb}[1]{\mathbb{#1}}
\newcommand{\op}[1]{\operatorname{#1}}
\renewcommand{\figurename}{Fig.}
\newcommand{\dom}[1]{D#1}
\newcommand{\avg}[1]{\left\langle{#1}\right\rangle}
\newcommand{\NN}{\mathbb N}
\newcommand{\RR}{\mathbb R}
\newcommand{\CC}{\mathbb C}
\newcommand{\mS}{\mathcal S}
\newcommand{\de}{d}
\newcommand{\abs}[1]{\left|#1\right|}

\newcommand{\lesson}[2]{\marginpar{(Lezione #1 del #2)}}
\DeclareRobustCommand{\MQ}{{\small\textsc{MQ}}\xspace}
\DeclareRobustCommand{\MC}{{\small\textsc{MC}}\xspace}
%Prima era \small\textsc{MQ}\xspace

%%TESTATINE
\usepackage{fancyhdr}
\pagestyle{fancy}
\fancyhead{} % clear all header fields
\renewcommand{\headrulewidth}{0pt} % no line in header area
\fancyfoot{} % clear all footer fields
%\fancyfoot[R]{A.A. 2018/19} % other info in "inner" position of footer line
\cfoot{\thepage}


%%AMBIENTI
\theoremstyle{plain}
\newtheorem{thm}{Teorema}[section]
\newtheorem{lem}{Lemma}[section]
\newtheorem{prop}{Proposizione}[section]
\newtheorem{axi}{Assioma}
\newtheorem{pst}{Postulato}

\theoremstyle{definition}
\newtheorem{dfn}{Definizione}

\theoremstyle{remark}
\newtheorem{oss}{Osservazione}
\newtheorem{es}{Esempio}
\newtheorem{ex}{Esercizio}

%Spiegazioni/verifiche
\newenvironment{expl}{\begin{mdframed}[hidealllines=true,backgroundcolor=green!20,innerleftmargin=3pt,innerrightmargin=3pt,leftmargin=-3pt,rightmargin=-3pt]}{\end{mdframed}} %Box di colore verde

\newenvironment{appr}{\begin{mdframed}[hidealllines=true,backgroundcolor=blue!10,innerleftmargin=3pt,innerrightmargin=3pt,leftmargin=-3pt,rightmargin=-3pt]}{\end{mdframed}} %Approfondimenti matematici (box di colore blu)

%%Domande di Marchetti
\newtheorem{question}{Domanda}


%%OPERATORI
\DeclareMathOperator{\sech}{sech}
\DeclareMathOperator{\csch}{csch}
\DeclareMathOperator{\arcsec}{arcsec}
\DeclareMathOperator{\arccot}{arcCot}
\DeclareMathOperator{\arccsc}{arcCsc}
\DeclareMathOperator{\arccosh}{arcCosh}
\DeclareMathOperator{\arcsinh}{arcsinh}
\DeclareMathOperator{\arctanh}{arctanh}
\DeclareMathOperator{\arcsech}{arcsech}
\DeclareMathOperator{\arccsch}{arcCsch}
\DeclareMathOperator{\arccoth}{arcCoth} 




\begin{document}
\subsection{Spettro matematico}
\begin{comment}
Abbiamo definito lo spettro di un'osservabile $O$ descritta dall'operatore $A$ come:
\begin{equation}
\sigma \left(A\right)= \left\{a\in\mathbb{R}\ |\inf_{\psi\in D\left(A\right)}{\left(\Delta A\right)_{a,\psi}=0}\right\}
\label{eqn:spettroA}
\end{equation}
\end{comment}
Come già accennato in precedenza, esiste un modo più generale per definire lo spettro di un operatore. Lo \textbf{spettro matematico} di un operatore $A$ (anche non autoaggiunto,\marginpar{Nuova definizione di spettro $\sigma(A)$ di un operatore $A$} per esempio nel caso di un operatore unitario) è dato da:
\[
\sigma \left(A\right)_{\text{mat}}= \left\{\mathbb{C}\setminus \{ z\in\mathbb{C}\ |\ \left|\left|\left(A-z\mathbb{I}\right)^{-1}\right|\right|<\infty\right\}\}
=\left\{z\in\mathbb{C}\ |\
\left(A-z\mathbb{I}\right)^{-1}\notin\mathcal{B}(\mathcal{H})\right\}
\]
In altre parole, lo spettro di $A$ è dato da tutti i numeri complessi $z$ per cui $(A-z\bb{I})^{-1}$ non è un operatore limitato.\\
In effetti, nel caso matriciale, se $a$ è un autovalore, allora $(A-a\bb{I})$ è singolare (è proprio la matrice di cui facciamo il $\op{ker}$ per calcolare gli autovettori associati ad $a$), e quindi l'inversa $\left(A-a\mathbb{I}\right)^{-1}\notin \mathcal{B}(\hs)$ proprio non esiste (e di certo non è un operatore limitato!).\\
Verifichiamo che tale nuova definizione è equivalente a quella precedente (\ref{eqn:spettroA}) nel caso di operatori autoaggiunti ($A=A^\dag$).\\
Innanzitutto, se lo spettro di $A$ è reale (come vogliamo che sia), allora ogni numero non reale $a\notin \bb{R}$ fa sì che $(A-a\bb{I})^{-1}$ sia un operatore limitato:
\[ \sigma \left(A\right)_{\text{mat}}\subseteq \bb{R} \Rightarrow \left(A-a\mathbb{I}\right)^{-1}\in \mathcal{B}\left(\mathcal{H}\right), a\notin \bb{R} 
\]
Consideriamo ora gli $a\in \bb{R}$ che generano un $(A-a\bb{I})^{-1}$ limitato. Allora, per la limitatezza, vale:
\[
\norm{\left(A-a\mathbb{I}\right)^{-1}}<C
\]
Dimostriamo che tali $a$ non soddisfano la definizione in (\ref{eqn:spettroA}):
\begin{align*}
\inf_{\psi\in D\left(A\right)}{\left(\Delta A\right)_{a,\psi}^2}
&=\inf_{\psi\in D\left(A\right)}{\frac{\left|\left|\left(A-a\bb{I}\right)\psi\right|\right|^2}{\left|\left|\psi\right|\right|^2}}
\underset{(a)}{=}\inf_\phi{\frac{\left|\left|\phi\right|\right|^2}{\left|\left|\left(A-a\bb{I}\right)^{-1}\phi\right|\right|^2}}=\\
&\underset{(b)}{=}\frac{1}{\sup_\phi{\frac{\left|\left|\left(A-a\right)^{-1}\phi\right|\right|^2}{\left|\left|\phi\right|\right|^2}}}\underset{(c)}{=}\frac{1}{\left|\left|\left(A-a\bb{I}\right)^{-1}\right|\right|^2}>\frac{1}{C^2}
\end{align*}
In (a) si è definito $\phi = (A-a\bb{I})\psi$ per $\psi \in D\left(A\right)$, da cui $\left(A-a\mathbb{I}\right)^{-1}\phi =\psi$. In (b) portiamo tutto al denominatore, e l'estremo inferiore di una frazione con numeratore fissato si ottiene \q{massimizzando} il denominatore. Ma allora applicando Riesz (norma di un funzionale) in (c) si ha che l'$\inf$ di partenza è maggiore di un numero \textit{piccolo} ma certamente maggiore di $0$. Perciò gli $a \notin \sigma(A)_{\text{mat}}$, ossia quelli per cui $(A-a\bb{I})^{-1}\in\mathcal{B}(\hs)$, non sono autovalori per la (\ref{eqn:spettroA}). Perciò:
\[
\sigma(A)_{\text{mat}}^C \subseteq \sigma(A)^C \Leftrightarrow \sigma(A) \subseteq \sigma(A)_{\text{mat}}
\]
(dove $B^C$ è l'insieme complementare di $B$).\\
Ci serve ora dimostrare l'inclusione inversa, ossia che tutti gli autovalori \q{matematici} siano anche autovalori per la (\ref{eqn:spettroA}).\\
Se $a\in\sigma(A)_{\text{mat}}$, cioè se $\left(A-a\right)^{-1}\notin B\left(\mathcal{H}\right)$,  abbiamo 3 casi:
\begin{enumerate}
    \item 
    $\displaystyle \exists  \psi_a\>|\> \left(A-a  \bb{I}\right)\psi_a=0\Rightarrow \left(\Delta A\right)_{a,\psi_a}=0$\\
    In altre parole, come nel caso matriciale, esiste un autovettore $\psi_a$ di autovalore $a$ dell'operatore $A$, per cui $(A-a\bb{I})^{-1}$ \textbf{non è definito}.\\
    Inoltre si ha che il codominio (o range) di $(A-a\bb{I})$  non è denso in $\hs$, poiché da esso manca il sottospazio generato da $\psi_a$ (che essendo un autovettore è costituito da vettori del tipo $a\psi_a$, che vengono rimossi nella differenza).\\
    Questo caso corrisponde allo \textbf{spettro discreto} precedentemente definito.
	\item $\left(A-a\right)^{-1}$ esiste in $D$ denso, è un \textbf{operatore illimitato}:
	\[
	\sup_{\phi\in D\left(A\right)}{\frac{\left|\left|\left(A-a\right)^{-1}\phi\right|\right|}{\left|\left|\phi\right|\right|}}=\infty 
	\]
	Ciò significa che possiamo trovare una sequenza di vettori $\phi_n$ di norma unitaria che fanno sì che $(A-a)^{-1}$ diverga. Matematicamente: 
	\[ 
	\exists \left\{\phi_n\right\}\subset D{(\left(A-a\right)}^{-1}), \left|\left|\phi_n\right|\right|=1 \text{ t.c. }\left|\left|\left(A-a\right)^{-1}\phi_n\right|\right|\xrightarrow[n\to\infty]{} \infty
	\]
	Ma allora esiste una sequenza di \q{autovettori approssimati} $\psi_n$ per cui $\norm{(A-a)\psi_n}\xrightarrow[n\to\infty]{} 0$. Infatti, definendo opportunamente:
	\[
	\psi_n=\frac{\left(A-a\right)^{-1}\phi_n}{\norm{\left(A-a\right)^{-1}\phi_n}}
	\]
	Si ha che:
	\[
	\norm{\left(A-a\right)\psi_n}=\frac{\norm{\left(A-a\right)\left(A-a\right)^{-1}\phi_n}}{\norm{\left(A-a\right)^{-1}\phi_n}}=\frac{\norm{\phi_n}}{\norm{\left(A-a\right)^{-1}\phi_n}}\xrightarrow[n\to\infty]{} 0
	\]
	Ma questi $\psi_n$ soddisfano la definizione di autovalori che abbiamo dato in (\ref{eqn:spettroA}):
	\[
	\inf_{\psi\in D\left(A\right)}{\left(\Delta A\right)_{a,\psi}^2}\leq \lim_{n\rightarrow\infty}{\left|\left|\left(A-a\right)\psi_n\right|\right|^2=0}
	\]
	(si è usata nuovamente la tecnica di calcolare l'$\inf$ maggiorandolo con l'$\inf$ di un sottoinsieme di tutte le $\psi$ che contiene solo le $\psi_n$ appena costruite).\\
	Questo caso corrisponde allora allo \textbf{spettro continuo}.
	\item Per esclusione, c'è anche il caso in cui non esiste $\psi_a$ autovettore, e quindi $(A-a\bb{I})^{-1}$ è definito, ma non con dominio denso (perciò non è possibile estenderlo univocamente a tutto $\hs$, e quindi ci saranno dei vettori di $\hs$ per cui non è definito).\\
	Questo caso costituisce lo \textbf{spettro residuo}, che non ha alcuna interpretazione fisica.\\
	Fortunatamente vale il seguente teorema:
	\begin{thm}
	Se $A$ è autoaggiunto o unitario allora non ha spettro residuo.
	\end{thm}
\end{enumerate}
Avendo allora mostrato che tra le componenti dello spettro matematico $\sigma(A)_{\text{mat}}$ vi sono tutti gli autovalori che abbiamo già incontrato nell'esaminare lo spettro $\sigma(A)$, abbiamo dimostrato che $\sigma(A)_{\text{mat}} \subseteq \sigma(A)$, e ciò, unito all'inclusione inversa già discussa, fa sì che le due definizioni siano equivalenti.\\
Notiamo infine che basta spostarsi di poco da un operatore autoaggiunto per perdere completamente diverse proprietà fisiche di fondamentale importanza. Per esempio, quando avevamo definito l'operatore momento $P_0$ (che non è autoaggiunto, in quanto la definizione giusta sarebbe quella di $P$), ponendo $\hs = L^2([0,1],dx)$:
\begin{align*}
P_0&=-i\hbar \frac{d}{dx}\\
D\left(P_0\right)&=\left\{\psi\in\mathcal{H}\ |\ \ \psi\text{ regolare e } \psi\left(0\right)=\ \psi\left(1\right)=0\right\}
\end{align*}
\[
-i\hbar \frac{d}{dx}\psi(x) = \lambda \psi(x); \quad \psi(x) = \lambda e^{i\frac{\lambda}{\hbar}x}
\]
Imponendo $\psi \left(0\right)=0$ si ha che non esiste alcun $\lambda$ autovalore.\\ Ciò significa che lo spettro di questo operatore (che non può essere l'insieme vuoto, come si può dimostrare in generale) è unicamente \textbf{residuo}, e quindi non ha interpretazione fisica.
%Domanda: perché non continuo?








% Sezione (interna al capitolo della formulazione assiomatica) sul formalismo di Dirac

\section{Il Formalismo di Dirac}
Osserviamo ora più nel dettaglio la notazione di Dirac, finora occasionalmente utilizzata senza definizioni precise.
Dirac elaborò il suo formalismo senza l'utilizzo degli spazi di Hilbert, dato che questi ancora non erano stati ideati. Solo tramite la teoria delle distribuzioni e una matematica decisamente sofisticata è stato possibile dare un senso logico ed una giustificazione rigorosa al suo formalismo, che assume così i caratteri e la consistenza propri della matematica. L'idea alla base della notazione di Dirac è quella di partire dal formalismo per lo spettro discreto e applicarlo, in maniera naturale, anche quello continuo. Questo paragrafo espone il formalismo con dei cenni alla complessa matematica che permette a tutto questo di \q{reggersi} in modo consistente.\\
\subsection{Notazione e formalismo per lo spettro discreto}
Sia $A$ un operatore autoaggiunto, il cui spettro è solo dato dagli autovalori \q{di algebra lineare}, e quindi è puramente puntuale: $\sigma \left(A\right)= \sigma_P\left(A\right)= \left\{\lambda_n\right\}$.\\
Per rappresentare i funzionali e gli stati in notazione di Dirac si utilizzano i bra e i ket già presentati negli scorsi paragrafi.
Come già discusso in precedenza, questa notazione viene introdotta in generale per denotare i vettori \q{astratti}, cioè permette di rappresentare gli elementi di $\hs$ senza la necessità di scegliere una \emph{base}, dunque una rappresentazione \q{concreta} dei vettori.
Per semplicità ci limiteremo per ora ai casi in cui i $\lambda_n$ non presentano degenerazione, ossia per cui ad ogni autovalore corrisponde un solo autovettore\footnote{Nel caso finito dimensionale questa è la situazione delle matrici diagonalizzabili.}.
In notazione di Dirac l'equazione agli autovalori diviene:
\[
A \left|\lambda_n\right\rangle=\lambda_n|\lambda_n \rangle 
\]
Siccome i $\ket{\lambda_n}$ costituiscono una base ON, possiamo scrivere un qualsiasi altro ket $\ket{\psi}$ o bra $\bra{\phi}$ come somme delle loro proiezioni su di essi:
\begin{equation}
\left|\psi\right\rangle=\sum_{n}{|\lambda_n\rangle \langle\lambda_n|\psi\rangle }; \quad 
\langle \phi |=\sum_{n}{\langle\phi|\lambda_n\rangle \langle\lambda_n|}
\label{eqn:bra-ket}
\end{equation}
dove $\langle \lambda_n|$ sono i funzionali in $\mathcal{H}^\ast$ associati a $|\lambda_n \rangle$  per Riesz. In notazione di Dirac la relazione di Parseval, che permette di scrivere il prodotto scalare come una serie, ha la seguente forma:
\begin{equation}
\left\langle\phi\middle|\psi\right\rangle=\sum_{n}{\langle\phi|\lambda_n\rangle \langle\lambda_n|\psi\rangle }
\label{eqn:parseval-relazione}
\end{equation}
Si ha che (\ref{eqn:bra-ket}) e (\ref{eqn:parseval-relazione}) sono compatibili solamente se $\left\langle\lambda_n\middle|\lambda_m\right\rangle=\delta_{mn}$, e questo fatto è sintetizzato nella \textbf{relazione di completezza di Dirac}:
\[
\sum_{n}{|\lambda_n\rangle \langle\lambda_n|}=\bb{I}
\]
Tale relazione esprime il fatto che \emph{ogni} vettore di $\hs$ può essere rappresentato mediante i vettori della base $\ket{\lambda_n}$.
Dato che $\ket{\lambda_n} \bra{\lambda_n}$ è il proiettore nell'autospazio di $\lambda_n$, mediante la relazione di completezza è definita anche la rappresentazione spettrale di $A$:
\[
A = A \mathbb I = A\sum_{n}{\left|\lambda_n\right\rangle\left\langle\lambda_n\right|} = \sum_{n}{A \left|\lambda_n\right\rangle\left\langle\lambda_n\right| = \sum_{n}{\lambda_n\left|\lambda_n\right\rangle\left\langle\lambda_n\right|}}
\]

\subsection{Estensione del formalismo allo spettro continuo}
Nei precedenti paragrafi abbiamo visto che  l'equazione agli autovalori $A\left|\lambda\right\rangle=\lambda |\lambda  \rangle$ non fornisce le soluzioni per lo spettro continuo.
Infatti se $\lambda \in \sigma_C(A)$, abbiamo dimostrato che $|\lambda\rangle$ non può essere un autovettore in $\hs$.
L'intuizione di Dirac fu che $\ket{\lambda}$ potesse comunque risolvere l'equazione agli autovalori, dunque essere un autovettore anche se in $\sigma_C(A)$, ma solo in uno spazio diverso da $\hs$, sicuramente \q{più grande}, ovvero uno spazio che estendesse $\hs$.\\
Per comprendere di cosa si tratta consideriamo ad esempio l'operatore posizione $A=X$, che agisce su $\ket{\lambda}$ come $X\ket{\lambda} = x \ket{\lambda}$. Allora l'equazione agli autovalori diviene (nella rappresentazione in posizioni):
\[
X\ket{\lambda} = \lambda\ket{\lambda} \quad \Rightarrow \quad x\ket{\lambda} = \lambda\ket{\lambda} \quad \Rightarrow \quad (x-\lambda)\ket{\lambda} = 0
\]
Solo la funzione nulla (che non è in $\hs$) soddisfa questa equazione. Tuttavia un altro \q{\textit{oggetto}} che soddisfa questa condizione (e che quindi può fungere da \q{autovettore} generalizzato) è la delta di Dirac. Per definire rigorosamente la $\delta$ il formalismo matematico elementare ovviamente non basta. Tuttavia Dirac suppose che la $\delta$ fosse un \q{nuovo} elemento dello spazio degli stati, e che potesse comunque essere manipolata come  $\delta(x-\lambda) = \braket{x|\lambda}$. Sostituendola al posto dell'autovettore la $\delta$ può essere considerata una soluzione dell'equazione agli autovalori:
\[
(x-\lambda )\langle x | \lambda \rangle = 0 \quad \Rightarrow \quad (x-\lambda)\delta(x-\lambda) = 0
\]
%Sistemare qui [TO DO]
infatti una nota proprietà della $\delta$ (da metodi) è che $(x - \lambda) \delta(x - \lambda) = 0$ anche per $x = \lambda$ dove la $\delta$ vale (idealmente) $\infty$. Un altro modo per scrivere tale proprietà è
\[
x \delta(x - \lambda) = \lambda \delta(x - \lambda)
\]
la quale esprime il fatto che $\delta(x - \lambda)$ sia proprio una soluzione dell'equazione agli autovalori dell'operatore posizione.
La soluzione non è identicamente nulla: la proprietà più importante della $\delta$ è il fatto che
\[
\int_{\mathbb R} \delta(x - \lambda) \de x = 1
\]
Dunque la $\delta$, a meno del fatto di non essere una funzione \q{ordinaria}, ha le stesse proprietà di una funzione di stato \q{ammessa} in $\hs$! Questo fatto ha diversi vantaggi. Innanzitutto si mantiene, almeno formalmente, la struttura dell'equazione agli autovalori $A\ket{\lambda} = \lambda \ket{\lambda}$ identica a quella del caso dello spettro puramente discreto. Ma in questo modo ad ogni $\lambda \in \mathbb R$ corrisponde l'autostato \q{generalizzato} $\delta(x - \lambda)$, cosicché $\lambda$ può assumere qualsiasi valore reale, e lo spettro $\sigma(X)$ è tutto $\mathbb R$, esattamente quello che ci aspettiamo dallo spettro dell'operatore posizione!
In realtà gli autostati di tali autovalori $\lambda$ non esistono in $\hs$, per il fatto che le $\delta$ non sono funzioni rigorosamente definite nello spazio degli stati, almeno nel formalismo matematico utilizzato finora.
Trattare la $\delta$ come uno stato \q{possibile} porta in ogni caso a conclusioni corrette sullo spettro di $X$, e questo, almeno a prima vista, può sembrare una coincidenza. In realtà questa formulazione è perfettamente equivalente a quella di von Neumann, ma dimostrare ciò è tutt'altro che banale.

Vediamo ora le relazioni per gli stati nel caso dello spettro continuo. Consideriamo un operatore $A$ il cui spettro $\sigma(A)$ sia puramente continuo, dunque $\lambda \in \sigma_C\left(A\right)=\sigma (A)$
limitiandoci ancora al caso di autovalori senza degenerazione. Se gli autovalori \q{generalizzati} di $\sigma_C(A)$ sono, come per il caso dello spettro dell'operatore posizione, una quantità non numerabile in $\mathbb R$, come si può scrivere uno stato $\ket\psi$ come somma di un infinito non numerabile di autostati \q{generalizzati} di base? La notazione di Dirac lo permette, semplicemente integrando tutte le \q{proiezioni} di $\ket\psi$ su tutti gli autostati generalizzati dello spettro continuo di $A$:
\begin{equation}
\left|\psi\right\rangle= \int_{\sigma_C(A)}{d\lambda\ \left|\lambda\right\rangle\langle\lambda|\psi\rangle } \qquad \quad
\left\langle\phi\right|= \int_{\sigma_{C\left(A\right)}}{d\lambda\ \left\langle\phi\middle|\lambda\right\rangle\langle\lambda|}
\label{eqn:bra-ket-continui}
\end{equation}
Si noti l'analogia con il caso dello spettro discreto: l'unica differenza è che l'integrale sostituisce la sommatoria, non essendo possibile includere tutti gli elementi di $\sigma_C(A)$ mediante una somma di una quantità numerabile di termini. 
In questo modo l'estensione allo spettro continuo è molto naturale, non richiedendo di definire oggetti complessi come le famiglie spettrali. Tutto questo rispecchia il minimalismo di Dirac, il quale antepose la semplicità a tutto il resto nella formulazione della \MQ.
Sempre in analogia con lo spettro discreto, la relazione di Parseval per lo spettro continuo diviene:
\begin{equation}
    A\left|\lambda\right\rangle=\lambda 
\left\langle\phi\middle|\psi\right\rangle= \int_{\sigma_{C}\left(A\right)}{d\lambda\ \langle\phi|\lambda\rangle \langle\lambda|\psi\rangle }
    \label{eqn:parseval-continui}
\end{equation}
%Inserire equazione agli autovalori
E la (\ref{eqn:bra-ket-continui}) è compatibile con la (\ref{eqn:parseval-continui}) solamente se vale $\left\langle\lambda\middle|\lambda'\right\rangle= \delta \left(\lambda-\lambda'\right)$, la quale si sintetizza nella \textit{relazione di completezza}, valida anche per il caso dello spettro continuo:
\[
\int_{\sigma_{C}\left(A\right)}{d\lambda\left|\lambda\right\rangle\left\langle\lambda\right|}=\bb{I}
\]
Infine l'ultima analogia con lo spettro discreto consiste nella \emph{rappresentazione spettrale} di un operatore $A$ con spettro continuo:
\[
A\int d\lambda  \left|\lambda\right\rangle\left\langle\lambda\right|=\int d\lambda\,\lambda \left|\lambda\right\rangle\left\langle\lambda\right|=A 
\]

\subsection{Formalismo per uno spettro generale}
I casi visti fin ora sono quelli con spettro puramente discreto e puramente continuo, in entrambi i casi senza degenerazioni. Fortunatamente il formalismo di Dirac si generalizza facilmente ad uno spettro qualsiasi (sia discreto che continuo) $\sigma \left(A\right)= \sigma_P(A)\cup \sigma_C\left(A\right)$ e con eventuali degenerazioni.\\
\begin{dfn}[degenerazione]
Dato un autovalore $\lambda_n\in \sigma_P(A)$ o un autovalore generalizzato ($\lambda \in \sigma_C(A)$) diremo che ha \textbf{degenerazione} $d(\lambda_n)$ o $d\left(\lambda\right)$ se le equazioni agli autovalori corrispondenti (nel senso di Dirac) hanno $d\left(\lambda_n\right)$ o $d\left(\lambda\right)$ soluzioni indipendenti (in $\hs$) che denotiamo rispettivamente con $\ket{\lambda_{n}, r}$ (per $r=1,\dots, d(\lambda_n$) e $\ket{\lambda, r}$ per $r = 1,\dots, d(\lambda)$ (con $d(\lambda) \in \bb{N}$ ed eventualmente $\infty$).
\end{dfn}
Nel caso in cui lo spettro abbia una componente discreta e una continua allora la notazione di Dirac permette semplicemente di sommare le due componenti nelle formule del paragrafo precedente.
Ad esempio la relazione di completezza nel caso di uno spettro qualsiasi diviene:
\[
\sum_{\lambda_n\in\sigma_P\left(A\right)}{\left|\lambda_n\right\rangle\left\langle\lambda_n\right|+\int_{\sigma_C\left(A\right)} d\lambda\left|\lambda\right\rangle\left\langle\lambda\right|=\mathbb{I}_\mathcal{H}}
\]
Se inoltre sono presenti degenerazioni, per ciascuna di esse si dovranno ovviamente sommare tutti gli stati corrispondenti soggetti a degenerazione:
\[
\sum_{\lambda_n\in\sigma_P\left(A\right)}\sum_{r=1}^{d\left(\lambda_n\right)}{\left|\lambda_n,r\right\rangle\left\langle\lambda_n,r\right|+\int_{\sigma_C\left(A\right)}d\lambda\sum_{r=1}^{d\left(\lambda\right)}\left|\lambda,r\right\rangle\left\langle\lambda,r\right|=\mathbb{I}_\mathcal{H}}
\]
Tramite la relazione di completezza possiamo scrivere funzioni per osservabili nel caso più generale:
\[
f\left(A\right)= \sum_{\lambda_n\in\sigma_D\left(A\right)} f\left(\lambda_n\right)\sum_{r=1}^{d\left(\lambda_n\right)}\left|\lambda_n,r\right\rangle\left\langle\lambda_n,r\right|+\int_{\sigma_C\left(A\right)}{d\lambda f\left(\lambda\right)}\sum_{r=1}^{d\left(\lambda\right)}{\left|\lambda,r\right\rangle\langle\lambda,r|}
\]
Notiamo che per lo spettro generale anche nei casi più \q{semplici} è necessaria la completezza.\\

Ad esempio sia $H$ l'hamiltoniana dell'atomo di idrogeno. Negli stati in cui l'elettrone è legato l'energia è quantizzata, con degenerazione $n^2$, ma quando è libero possiamo scegliere una qualsiasi energia in un range continuo, e vi è un qualsiasi numero di stati che hanno una determinata energia (la degenerazione è infinita):
\[
\sigma \left(H\right)=\left\{-\frac{c}{n^2},n\in\mathbb{N}\right\}\cup \left\{x>0,x\in\mathbb{R}\right\}
\]
Abbiamo perciò una componente discreta dello spettro (con degenerazione finita) e una continua (con degenerazione addirittura $\infty$).


\subsection{Spazio degli autovettori dello spettro continuo}
Come visto prima, se $\lambda \in \sigma_C(A)$ gli autovettori $\ket{\lambda}$, $\ket{\lambda,r}$ non si trovano in $\hs$, ma in uno spazio più grande (sono autovettori \q{generalizzati}). In questo paragrafo si cerca di definire questo spazio più ampio, accennando alla matematica che ha permesso di giustificare il formalismo di Dirac.\\
Se si definisce  $F_\lambda(x) \equiv \braket{x|\lambda}$ e lo si applica all'operatore posizione $X$ abbiamo visto che la soluzione dell'equazione agli autovalori risulta:
\[
XF_\lambda(x) = \lambda F_\lambda(x) \qquad \quad F_\lambda(x) = \delta(x-\lambda) \notin L^2(\bb{R})
\]
Tuttavia la $\delta$ non è una funzione reale a valori reali, pertanto non può essere una funzione di $L^2(\mathbb R)$. Tuttavia (da metodi) sappiamo che  $\delta(x-\lambda)$ appartiene allo spazio delle distribuzioni $\mathcal{S}'(\bb{R})$, il quale contiene $L^2(\mathbb R)$.

Proviamo la stessa cosa con l'operatore momento. Se si definisce $F_\mu(x) \equiv \braket{x|\mu}$ la soluzione dell'equazione agli autovalori risulta:
\[
P F_\mu(x) = \mu F_\mu(x) \qquad \quad F_\mu(x) = e^{\frac{i}{\hbar}\mu x} \notin L^2(\bb{R}),\> \in \mathcal{S}'(\bb{R})
\] %Perché l'esponenziale? SIgnfiicato di F_\mu?
infatti
\[
P F_{\mu}(x) = - i \hbar \frac{\partial}{\partial x} e^{\frac{i}{\hbar} \mu x} = - i \hbar \frac{i}{\hbar} \mu \, e^{\frac{i}{\hbar} \mu x} = \mu F_{\mu}(x)
\]
La situazione è analoga, anche se opposta a quella dell'operatore posizione: in questo caso l'autovettore generalizzato identifica un preciso valore del momento (una sinusoide \q{monocromatica}). Tuttavia si noti questa soluzione \emph{non} è in $L^2(\mathbb R)$, infatti tale funzione non è quadrato sommabile. Non a caso la trasformata di Fourier di $e^{\frac{i}{\hbar}\mu x}$ è proprio la $\delta$ di Dirac. La trasformata di Fourier per funzioni di questo tipo è però definita solo nello spazio delle distribuzioni $\mathcal S'$ e non in $L^2$.

Viste in quest'ottica le distribuzioni sono estremamente comode, ma la domanda sorge spontanea: possono essere considerate come \q{stati fisici}, cioè stati corrispondenti a situazioni reali? Se questo fosse possibile, allora tutta la formulazione assiomatica della \MQ non sarebbe coerente con il principio di indeterminazione. Il fatto che le soluzioni dell'equazione agli autovalori non siano in $L^2(\bb{R})$ significa che esse non corrispondono a \q{stati fisici}, fisicamente possibili. In particolare, ciò vuol dire che non è comunque possibile localizzare una particella perfettamente in un punto, né assegnarle un singolo momento definito.
Questo artefatto matematico permette solo di estendere lo spazio degli autostati, ma non \q{aggiunge} funzioni di $L^2$ bensì istituisce nuove soluzioni \q{non fisiche} dell'equazione agli autovalori.

La teoria delle distribuzioni permette di formalizzare in modo rigoroso questi concetti. Lo spazio $\mathcal{S}'\left(\mathbb{R}\right)$ è definito come l'insieme dei funzionali lineari continui su $\hs(\bb{R})$, ovvero lo spazio delle funzioni $\mathcal{C}^\infty$ che descrescono all'infinito (assieme a tutte le loro derivate) più rapidamente di $x^{-n}$ per ogni $n \in \mathbb N$. Questa definizione è facilmente generalizzabile in dimensione $n$:
\begin{itemize}
    \item $\mathcal{S}\left(\mathbb{R}^n\right)$ è lo spazio delle funzioni $\mathcal{C}^\infty$ su $\mathbb{R}^n$ che decrescono più velocemente della norma $\left|\left|x\right|\right|^{-m}$ per ogni $m \in \mathbb N$;
    \item $\mathcal{S}'(\bb{R}^n)$ è lo spazio dei funzionali lineari continui su $\mathcal{S}(\bb{R}^n)$.
\end{itemize}
\begin{oss}
Lo spazio delle distribuzioni $\mathcal S'$ è più esteso di $L^2$ e sicuramente ammette funzioni particolari come la $\delta$. Tuttavia $\mathcal S'$ ha molte limitazioni: notiamo che esistono \q{funzioni regolarissime} che non appartengono a $\mathcal{S}'(\mathbb {R})$. Infatti, le distribuzioni regolari (descritte da funzioni) in $\mathcal{S}'(\mathbb R)$ non possono crescere più rapidamente  di un polinomio in $x$, per $x\rightarrow \infty$, quindi ad esempio $e^x\notin \mathcal{S}'\left(\mathbb{R}\right)$.
\end{oss}
Definendo i nuovi spazi $\mathcal S$ e $\mathcal S'$ si dimostra (da metodi) che vale la seguente catena di inclusioni:
\begin{equation}
\mathcal S\left(\mathbb{R}\right)\subset L^2\left(\mathbb{R}\right) \approx [L^2\left(\mathbb{R}\right)]' \subset \mathcal{S}'\left(\mathbb{R}\right)
\label{eqn:triplettametodi}
\end{equation}
Dunque l'idea, come già anticipato, è quella di risolvere l'equazione agli autovalori proprio nello \q{spazio più grande} $\mathcal{S}'(\mathbb R)$. Questo spazio è quello per cui è possibile trovare soluzioni per tutti i $\lambda \in \bb{R}$ per gli operatori $X$ e $P$, e dunque trovare lo spettro voluto per entrambi gli operatori:
\[
\sigma(X) = \sigma_C(X) = \bb{R} \qquad \quad \sigma(P) = \sigma_C(P) = \bb{R}
\]
\lesson{11}{18/10/2018}
Abbiamo visto che, almeno nel caso di posizione e momento, gli autovalori generalizzati sono quelli dello spettro continuo. Questo fatto vale in realtà in generale per quelaiasi osservabile: la differenza tra gli autovettori generalizzati di $\mathcal{S}'$ e gli autovettori \q{algebrici} di $\hs$ è esattamente ciò che distingue lo \textit{spettro discreto} dallo \textit{spettro continuo}.
Concretamente, se la soluzione dell'equazione agli autovalori è in $\mathcal{S}'\left(\mathbb{R}\right)$, ma non è in $\hs$ l'autovalore corrispondente appartiene allo \textbf{spettro continuo}. Se invece l'autovettore è in $\hs$ lo spettro corrispondente appartiene allo \textbf{spettro discreto}.\\

Queste idee ovviamente non appartengono più alle intuizioni di Dirac, ma fanno parte della successiva teoria distribuzionale. Uno dei massimi esponenti di questa teoria fu Gel'fand della scuola russa. Come trovare lo\marginpar{Spettro di un operatore $A$ autoaggiunto per Gel'fand} spettro di un operatore autoaggiunto $A$ nel formalismo di Dirac? L'idea di Gel'fand fu quella di trovare uno spazio $\Phi_A$ più \q{piccolo} di $\hs$, ma tale che il suo duale $\Phi'_A$ avesse potuto estendere lo spazio degli stati $\hs$, in pratica:
\begin{equation}
\Phi_A \subset \hs \approx \hs' \subset \Phi_A'
\label{tripletta-gelfand}
\end{equation}
Tale catena di inclusioni, che generalizza quella vista in (\ref{eqn:triplettametodi}), è detta \textbf{tripletta di Gel'fand}, e $\Phi_A'$ è detto \textbf{spazio di Hilbert equipaggiato} o \textbf{allargato} (\textit{rigged Hilbert space}). Gel'fand comprese che creando uno spazio con una topologia più forte, dunque uno spazio $\Phi_A$ contenuto in $\hs$, la topologia del suo duale $\Phi_A'$ sarebbe stata più debole, e dunque si sarebbe creata un'estensione $\hs'$. Un esempio di spazio di Hilbert equipaggiato è lo spazio delle ditribuzioni $\mathcal S'$.

La teoria che a partire da queste definizioni fornisce una spiegazione completa del formalismo di Dirac è molto sofisticata. Infatti è necessario fare le seguenti ulteriori richieste nella definizione di $\Phi_A$:
\begin{enumerate}
    \item $\Phi_A$ dev'essere \emph{contenuto nel dominio} di $A$, dunque $\displaystyle \Phi_A\subseteq D\left(A\right)$. Ad esempio nel caso delle distribuzioni e dell'operatore posizione, lo spazio delle funzioni di prova $\mathcal S\left(\mathbb{R}\right)$ è contenuto in $D(X)$;
	\item Nella topologia di $\hs$, $\Phi_A$ dev'essere \emph{denso} in $\hs$, ovvero $\overline{\Phi_A} = \hs$. Nel caso particolare delle distribuzioni si dimostra che $\overline{\mathcal{S}\left(\mathbb{R}\right)} = L^2(\mathbb R)$;
	\item $A$ dev'essere \emph{continuo} in $\Phi_A$ nella topologia forte (quella di $\Phi_A$), e questo è vero per $X$ nella topologia di $\mathcal S\left(\mathbb{R}\right)$;
	\item $\Phi_A$ dev'essere \q{\emph{nucleare}} cioè i funzionali lineari continui sullo spazio\footnote{Il simbolo $\otimes$ indica il prodotto tensoriale tra due spazi.} delle coppie $\Phi_A\otimes \Phi_A$ sono lineari e continui anche nello spazio delle combinazioni lineari (eventualmente infinite) degli elementi dello spazio dei prodotti. Nel caso delle distribuzioni una coppia è ad esempio $f,g\in \mathcal{S}(\bb{R})\otimes \mathcal{S}\left(\mathbb{R}\right)$ mentre i prodotti sono $f\left(x,y\right)\in \mathcal S\left(\mathbb{R}^2\right)$. Dunque la nuclearità richiede che i funzionali lineari continui su $\mathcal{S}\left(\mathbb{R}\right)\otimes \mathcal{S}\left(\mathbb{R}\right)$ siano lineari e continui su $\mathcal S(\mathbb R \times \mathbb R) = \mathcal{S}(\mathbb{R}^2)$.
	
	\subsubsection{Giustificazione della richiesta di nuclearità}
	Perché è necessario richiedere l'ultima richiesta, quella di nuclearità? 
	Consideriamo un'osservabile $A$ con spettro puramente discreto $\sigma \left(A\right)=\sigma_P(A)$ (e senza degenerazione). La relazione di completezza è
	\[
	\sum_{n}{|\lambda_n\rangle \langle\lambda_n|}=\mathbb{I}_\mathcal{H}
	\]
	pertanto si ha
	\[
	\left\langle\phi\left|A\right|\psi\right\rangle = \bra\phi\mathbb{I}A\mathbb{I}\ket\psi
	= \sum_{n,m}{\left\langle\phi\middle|\lambda_n\right\rangle\left\langle\lambda_n\left|A\right|\lambda_m\right\rangle\left\langle\lambda_m\middle|\psi\right\rangle}
	= \ \sum_{n,m}{A_{nm}\phi_n^\ast\psi_{m}}
	\]
	$\bra\phi A \ket\psi$ è ovviamente lineare separatamente sia in $\phi_n^\ast$ che in $\psi_m$. Ma da questo risultato si osserva che è lineare anche nei prodotti $\phi_n^\ast\psi_m$, e quindi in tutte le loro combinazioni lineari.
	
	Il formalismo di Dirac deve permettere di generalizzare anche questo in modo naturale allo spettro continuo $\sigma \left(A\right)= \sigma_C\left(A\right)$. Poniamo che sia senza degenerazione (sempre per semplicità di notazione).
	La relazione di completezza nel caso dello spettro continuo è:
	\[
	\int_{\sigma_C\left(A\right)}{d\lambda\ \left|\lambda\right\rangle\langle\lambda|}=\mathbb{I}_\mathcal{H}
	\]
	dunque analogamente al caso dello spettro discreto si ottiene:
	\begin{align}
	\left\langle\phi\left|A\right|\psi\right\rangle=\left\langle\phi\left|\mathbb{I}A\mathbb{I}\right|\psi\right\rangle&=\int d\lambda \  d\lambda'\left\langle\phi\middle|\lambda'\right\rangle\left\langle\lambda\left|A\right|\lambda'\right\rangle\left\langle\lambda'\middle|\psi\right\rangle \notag \\
	& =\int d\lambda \ d\lambda' \ \left\langle\lambda\left|A\right|\lambda'\right\rangle\left\langle\lambda|\phi\right\rangle^\ast\left\langle\lambda'\middle|\psi\right\rangle \notag
	\end{align}
	Ma ciò non è possibile in $L^2$. Infatti si consideri ad esempio $A=\mathbb I$, e siano $f,g\in L^2$. Allora se $x = \lambda$ e $y = \lambda'$ deve verificarsi
	\[
	\left(f,\mathbb{I}g\right)=\int dx \, dy \, k\left(x,y\right)f^\ast\left(x\right)g\left(y\right)
	=\int f^\ast\left(x\right) g\left(x\right)dx
	\] ovvero il funzionale dev'essere lineare e continuo in $f^\ast$ e in $g$ separatamente e nei loro prodotti. Tuttavia non esiste alcuna funzione $k\left(x,y\right)$ che renda il funzionale lineare e continuo sia in $L^2\left(\mathbb{R}\right)\otimes L^2\left(\mathbb{R}\right)$ che in $L^2(\mathbb{R}^2)$.
	% che per Riesz dovrebbe appartenere a $L^2\left(\mathbb{R}^2\right)$ [non mi è chiara questa parte]
	Ma una soluzione possibile è prendere $k\left(x,y\right)=\delta \left(x-y\right)$ che come sappiamo non è in $L^2\left(\mathbb{R}^2\right)$
	ma in $\mathcal{S}'\left(\mathbb{R}^2\right)$. Per questo, se lo spazio equipaggiato $\Phi'_A$ è generico, è necessario richiedere come ipotesi che la condizione di nuclearità sia verificata.
\end{enumerate}





\end{document}






