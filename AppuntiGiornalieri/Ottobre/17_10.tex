%\documentclass[12pt]{article}
\begin{comment}
%%PACKAGES
\usepackage[usenames, dvipsnames, table]{xcolor}
\usepackage[utf8]{inputenc}
\usepackage[T1]{fontenc}
\usepackage{lmodern}
\usepackage{amsmath}
\usepackage{amsthm}
\usepackage{amsfonts}
\usepackage{comment}
\usepackage{wrapfig}
\usepackage{booktabs}
\usepackage{braket}
\usepackage{pgf,tikz}
\usepackage{mathrsfs}
\usetikzlibrary{arrows}
\usepackage{subfigure}
\usepackage{xspace}
\usepackage{gnuplottex}
\usepackage{epstopdf}
\usepackage{marginnote}
\usepackage{float}
\usetikzlibrary{tikzmark}
\usepackage{graphicx}
\usepackage{cancel}
\usepackage{bm}
\usepackage{mathtools}
\usepackage{hyperref}
\usepackage{ragged2e}
\usepackage[stable]{footmisc}
\usepackage{enumerate}
\usepackage{mathdots}
\usepackage[framemethod=tikz]{mdframed}
\PassOptionsToPackage{table}{xcolor}
\usepackage{soul}
\usepackage{enumerate}
\usepackage{mathdots}
\usepackage[framemethod=tikz]{mdframed} %Added 16/10
\usepackage[italian]{babel} %Added 16/10
\usepackage{amssymb} %Added
\usepackage{enumitem}
\usepackage{array}


%%BOOKTAB
\setlength{\aboverulesep}{0pt}
\setlength{\belowrulesep}{0pt}
\setlength{\extrarowheight}{.75ex}
\setlength\parindent{0pt} %Rimuove indentazione


%%GEOMETRIA
\usepackage[a4paper]{geometry}
 \newgeometry{inner=20mm,
            outer=49mm,% = marginparsep + marginparwidth 
                       %   + 5mm (between marginpar and page border)
            top=20mm,
            bottom=25mm,
            marginparsep=6mm,
            marginparwidth=30mm}
\makeatletter
\renewcommand{\@marginparreset}{%
  \reset@font\small
  \raggedright
  \slshape
  \@setminipage
}
\makeatother
 

%%COMANDI
\newcommand{\q}[1]{``#1''}
\newcommand{\lamb}[2]{\Lambda^{#1}_{\>{#2}}}
\newcommand{\norm}[1]{\left\lVert#1\right\rVert}
\newcommand{\hs}{\mathcal{H}}
\newcommand{\minus}{\scalebox{0.75}[1.0]{$-$}}
\newcommand{\hlc}[2]{%
  \colorbox{#1!50}{$\displaystyle#2$}}
\newcommand{\bb}[1]{\mathbb{#1}}
\newcommand{\op}[1]{\operatorname{#1}}
\renewcommand{\figurename}{Fig.}
\newcommand{\dom}[1]{D#1}
\newcommand{\avg}[1]{\left\langle{#1}\right\rangle}
\newcommand{\NN}{\mathbb N}
\newcommand{\RR}{\mathbb R}
\newcommand{\CC}{\mathbb C}
\newcommand{\mS}{\mathcal S}
\newcommand{\de}{d}
\newcommand{\abs}[1]{\left|#1\right|}

\newcommand{\lesson}[2]{\marginpar{(Lezione #1 del #2)}}
\DeclareRobustCommand{\MQ}{{\small\textsc{MQ}}\xspace}
\DeclareRobustCommand{\MC}{{\small\textsc{MC}}\xspace}
%Prima era \small\textsc{MQ}\xspace

%%TESTATINE
\usepackage{fancyhdr}
\pagestyle{fancy}
\fancyhead{} % clear all header fields
\renewcommand{\headrulewidth}{0pt} % no line in header area
\fancyfoot{} % clear all footer fields
%\fancyfoot[R]{A.A. 2018/19} % other info in "inner" position of footer line
\cfoot{\thepage}


%%AMBIENTI
\theoremstyle{plain}
\newtheorem{thm}{Teorema}[section]
\newtheorem{lem}{Lemma}[section]
\newtheorem{prop}{Proposizione}[section]
\newtheorem{axi}{Assioma}
\newtheorem{pst}{Postulato}

\theoremstyle{definition}
\newtheorem{dfn}{Definizione}

\theoremstyle{remark}
\newtheorem{oss}{Osservazione}
\newtheorem{es}{Esempio}
\newtheorem{ex}{Esercizio}

%Spiegazioni/verifiche
\newenvironment{expl}{\begin{mdframed}[hidealllines=true,backgroundcolor=green!20,innerleftmargin=3pt,innerrightmargin=3pt,leftmargin=-3pt,rightmargin=-3pt]}{\end{mdframed}} %Box di colore verde

\newenvironment{appr}{\begin{mdframed}[hidealllines=true,backgroundcolor=blue!10,innerleftmargin=3pt,innerrightmargin=3pt,leftmargin=-3pt,rightmargin=-3pt]}{\end{mdframed}} %Approfondimenti matematici (box di colore blu)

%%Domande di Marchetti
\newtheorem{question}{Domanda}


%%OPERATORI
\DeclareMathOperator{\sech}{sech}
\DeclareMathOperator{\csch}{csch}
\DeclareMathOperator{\arcsec}{arcsec}
\DeclareMathOperator{\arccot}{arcCot}
\DeclareMathOperator{\arccsc}{arcCsc}
\DeclareMathOperator{\arccosh}{arcCosh}
\DeclareMathOperator{\arcsinh}{arcsinh}
\DeclareMathOperator{\arctanh}{arctanh}
\DeclareMathOperator{\arcsech}{arcsech}
\DeclareMathOperator{\arccsch}{arcCsch}
\DeclareMathOperator{\arccoth}{arcCoth} 




\begin{document}
\section{Lezione 3:\\ \large{Formulazione assiomatica Hilbertiana (o Standard)}}
\vspace{-1em}
\begin{center}
    \small{(17/10/2018)}
\end{center}
\end{comment}

\subsection{Spettro matematico}
\begin{comment}
Abbiamo definito lo spettro di un'osservabile $O$ descritta dall'operatore $A$ come:
\begin{equation}
\sigma \left(A\right)= \left\{a\in\mathbb{R}\ |\inf_{\psi\in D\left(A\right)}{\left(\Delta A\right)_{a,\psi}=0}\right\}
\label{eqn:spettroA}
\end{equation}
\end{comment}
Come già accennato in precedenza, esiste un modo più generale per definire lo spettro di un operatore. Lo \textbf{spettro matematico} di un operatore $A$ (anche non autoaggiunto,\marginpar{Nuova definizione di spettro $\sigma(A)$ di un operatore $A$} per esempio nel caso di un operatore unitario) è dato da:
\[
\sigma \left(A\right)_{\text{mat}}= \left\{\mathbb{C}\setminus \{ z\in\mathbb{C}\ |\ \left|\left|\left(A-z\mathbb{I}\right)^{-1}\right|\right|<\infty\right\}\}
=\left\{z\in\mathbb{C}\ |\
\left(A-z\mathbb{I}\right)^{-1}\notin\mathcal{B}(\mathcal{H})\right\}
\]
In altre parole, lo spettro di $A$ è dato da tutti i numeri complessi $z$ per cui $(A-z\bb{I})^{-1}$ non è un operatore limitato.\\
In effetti, nel caso matriciale, se $a$ è un autovalore, allora $(A-a\bb{I})$ è singolare (è proprio la matrice di cui facciamo il $\op{ker}$ per calcolare gli autovettori associati ad $a$), e quindi l'inversa $\left(A-a\mathbb{I}\right)^{-1}\notin \mathcal{B}(\hs)$ proprio non esiste (e di certo non è un operatore limitato!).\\
Verifichiamo che tale nuova definizione è equivalente a quella precedente (\ref{eqn:spettroA}) nel caso di operatori autoaggiunti ($A=A^\dag$).\\
Innanzitutto, se lo spettro di $A$ è reale (come vogliamo che sia), allora ogni numero non reale $a\notin \bb{R}$ fa sì che $(A-a\bb{I})^{-1}$ sia un operatore limitato:
\[ \sigma \left(A\right)_{\text{mat}}\subseteq \bb{R} \Rightarrow \left(A-a\mathbb{I}\right)^{-1}\in \mathcal{B}\left(\mathcal{H}\right), a\notin \bb{R} 
\]
Consideriamo ora gli $a\in \bb{R}$ che generano un $(A-a\bb{I})^{-1}$ limitato. Allora, per la limitatezza, vale:
\[
\norm{\left(A-a\mathbb{I}\right)^{-1}}<C
\]
Dimostriamo che tali $a$ non soddisfano la definizione in (\ref{eqn:spettroA}):
\begin{align*}
\inf_{\psi\in D\left(A\right)}{D\left(A\right)_{a,\psi}^2}
&=\inf_{\psi\in D\left(A\right)}{\frac{\left|\left|\left(A-a\bb{I}\right)\psi\right|\right|^2}{\left|\left|\psi\right|\right|^2}}
\underset{(a)}{=}\inf_\phi{\frac{\left|\left|\phi\right|\right|^2}{\left|\left|\left(A-a\bb{I}\right)^{-1}\phi\right|\right|^2}}=\\
&\underset{(b)}{=}\frac{1}{\sup_\phi{\frac{\left|\left|\left(A-a\right)^{-1}\phi\right|\right|^2}{\left|\left|\phi\right|\right|^2}}}\underset{(c)}{=}\frac{1}{\left|\left|\left(A-a\bb{I}\right)^{-1}\right|\right|^2}>\frac{1}{C^2}
\end{align*}
In (a) si è definito $\phi = (A-a\bb{I})\psi$ per $\psi \in D\left(A\right)$, da cui $\left(A-a\mathbb{I}\right)^{-1}\phi =\psi$. In (b) portiamo tutto al denominatore, e l'estremo inferiore di una frazione con numeratore fissato si ottiene "massimizzando" il denominatore. Ma allora applicando Riesz (norma di un funzionale) in (c) si ha che l'$\inf$ di partenza è maggiore di un numero \textit{piccolo} ma certamente maggiore di $0$. Perciò gli $a \notin \sigma(A)_{\text{mat}}$, ossia quelli per cui $(A-a\bb{I})^{-1}\in\mathcal{B}(\hs)$, non sono autovalori per la (\ref{eqn:spettroA}). Perciò:
\[
\sigma(A)_{\text{mat}}^C \subseteq \sigma(A)^C \Leftrightarrow \sigma(A) \subseteq \sigma(A)_{\text{mat}}
\]
(dove $B^C$ è l'insieme complementare di $B$).\\
Ci serve ora dimostrare l'inclusione inversa, ossia che tutti gli autovalori "matematici" siano anche autovalori per la (\ref{eqn:spettroA}).\\
Se $a\in\sigma(A)_{\text{mat}}$, cioè se $\left(A-a\right)^{-1}\notin B\left(\mathcal{H}\right)$,  abbiamo 3 casi:
\begin{enumerate}
    \item 
    $\displaystyle \exists  \psi_a\>|\> \left(A-a  \bb{I}\right)\psi_a=0\Rightarrow \left(\Delta A\right)_{a,\psi_a}=0$\\
    In altre parole, come nel caso matriciale, esiste un autovettore $\psi_a$ di autovalore $a$ dell'operatore $A$, per cui $(A-a\bb{I})^{-1}$ \textbf{non è definito}.\\
    Inoltre si ha che il codominio (o range) di $(A-a\bb{I})$  non è denso in $\hs$, poiché da esso manca il sottospazio generato da $\psi_a$ (che essendo un autovettore è costituito da vettori del tipo $a\psi_a$, che vengono rimossi nella differenza).\\
    Questo caso corrisponde allo \textbf{spettro discreto} precedentemente definito.
	\item $\left(A-a\right)^{-1}$ esiste in $D$ denso, è un \textbf{operatore illimitato}:
	\[
	\sup_{\phi\in D\left(A\right)}{\frac{\left|\left|\left(A-a\right)^{-1}\phi\right|\right|}{\left|\left|\phi\right|\right|}}=\infty 
	\]
	Ciò significa che possiamo trovare una sequenza di vettori $\phi_n$ di norma unitaria che fanno sì che $(A-a)^{-1}$ diverga. Matematicamente: 
	\[ 
	\exists \left\{\phi_n\right\}\subset D{(\left(A-a\right)}^{-1}), \left|\left|\phi_n\right|\right|=1 \text{ t.c. }\left|\left|\left(A-a\right)^{-1}\phi_n\right|\right|\xrightarrow[n\to\infty]{} \infty
	\]
	Ma allora esiste una sequenza di "autovettori approssimati" $\psi_n$ per cui $\norm{(A-a)\psi_n}\xrightarrow[n\to\infty]{} 0$. Infatti, definendo opportunamente:
	\[
	\psi_n=\frac{\left(A-a\right)^{-1}\phi_n}{\norm{\left(A-a\right)^{-1}\phi_n}}
	\]
	Si ha che:
	\[
	\norm{\left(A-a\right)\psi_n}=\frac{\norm{\left(A-a\right)\left(A-a\right)^{-1}\phi_n}}{\norm{\left(A-a\right)^{-1}\phi_n}}=\frac{\norm{\phi_n}}{\norm{\left(A-a\right)^{-1}\phi_n}}\xrightarrow[n\to\infty]{} 0
	\]
	Ma questi $\psi_n$ soddisfano la definizione di autovalori che abbiamo dato in (\ref{eqn:spettroA}):
	\[
	\inf_{\psi\in D\left(A\right)}{\left(\Delta A\right)_{a,\psi}^2}\leq \lim_{n\rightarrow\infty}{\left|\left|\left(A-a\right)\psi_n\right|\right|^2=0}
	\]
	(si è usata nuovamente la tecnica di calcolare l'$\inf$ maggiorandolo con l'$\inf$ di un sottoinsieme di tutte le $\psi$ che contiene solo le $\psi_n$ appena costruite).\\
	Questo caso corrisponde allora allo \textbf{spettro continuo}.
	\item Per esclusione, c'è anche il caso in cui non esiste $\psi_a$ autovettore, e quindi $(A-a\bb{I})^{-1}$ è definito, ma non con dominio denso (perciò non è possibile estenderlo univocamente a tutto $\hs$, e quindi ci saranno dei vettori di $\hs$ per cui non è definito).\\
	Questo caso costituisce lo \textbf{spettro residuo}, che non ha alcuna interpretazione fisica.\\
	Fortunatamente vale il seguente teorema:
	\begin{thm}
	Se $A$ è autoaggiunto o unitario allora non ha spettro residuo.
	\end{thm}
\end{enumerate}
Avendo allora mostrato che tra le componenti dello spettro matematico $\sigma(A)_{\text{mat}}$ vi sono tutti gli autovalori che abbiamo già incontrato nell'esaminare lo spettro $\sigma(A)$, abbiamo dimostrato che $\sigma(A)_{\text{mat}} \subseteq \sigma(A)$, e ciò, unito all'inclusione inversa già discussa, fa sì che le due definizioni siano equivalenti.\\
Notiamo infine che basta spostarsi di poco da un operatore autoaggiunto per perdere completamente diverse proprietà fisiche di fondamentale importanza. Per esempio, quando avevamo definito l'operatore momento $P_0$ (che non è autoaggiunto, in quanto la definizione giusta sarebbe quella di $P$), ponendo $\hs = L^2([0,1],dx)$:
\begin{align*}
P_0&=-i\hbar \frac{d}{dx}\\
D\left(P_0\right)&=\left\{\psi\in\mathcal{H}\ |\ \ \psi\text{ regolare e } \psi\left(0\right)=\ \psi\left(1\right)=0\right\}
\end{align*}
\[
-i\hbar \frac{d}{dx}\psi(x) = \lambda \psi(x); \quad \psi(x) = \lambda e^{i\frac{\lambda}{\hbar}x}
\]
Imponendo $\psi \left(0\right)=0$ si ha che non esiste alcun $\lambda$ autovalore.\\ Ciò significa che lo spettro di questo operatore (che non può essere l'insieme vuoto, come si può dimostrare in generale) è unicamente \textbf{residuo}, e quindi non ha interpretazione fisica.
%Domanda: perché non continuo?
%\end{document}