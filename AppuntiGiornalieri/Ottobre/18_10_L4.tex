\documentclass[../../FisicaTeorica.tex]{subfiles}

%\usepackage[usenames, dvipsnames, table]{xcolor}
\usepackage[utf8]{inputenc}
\usepackage[T1]{fontenc}
\usepackage{lmodern}
\usepackage{amsmath}
\usepackage{amsfonts}
\usepackage{comment}
\usepackage{wrapfig}
\usepackage{booktabs}
\usepackage{braket}
\usepackage{tikz}
\usepackage{gnuplottex}
\usepackage{epstopdf}
\usepackage{marginnote}
\usepackage{float}
\usetikzlibrary{tikzmark}
\usepackage{graphicx}
\usepackage{cancel}
\usepackage{bm}
\usepackage{mathtools}
\usepackage{hyperref}
\usepackage{ragged2e}
\usepackage[stable]{footmisc}
\usepackage{enumerate}
\usepackage{mathdots}
\usepackage[framemethod=tikz]{mdframed}
\PassOptionsToPackage{table}{xcolor}
\usepackage{soul}
\usepackage{enumerate}
\usepackage{mathdots}
\usepackage[framemethod=tikz]{mdframed} %Added 16/10
\usepackage[italian]{babel} %Added 16/10
\usepackage{amssymb} %Added

%%BOOKTAB
\setlength{\aboverulesep}{0pt}
\setlength{\belowrulesep}{0pt}
\setlength{\extrarowheight}{.75ex}
\setlength\parindent{0pt} %Rimuove indentazione


%%GEOMETRIA
\usepackage[a4paper]{geometry}
 \newgeometry{inner=20mm,
            outer=49mm,% = marginparsep + marginparwidth 
                       %   + 5mm (between marginpar and page border)
            top=20mm,
            bottom=25mm,
            marginparsep=6mm,
            marginparwidth=30mm}

\makeatletter
\renewcommand{\@marginparreset}{%
  \reset@font\small
  \raggedright
  \slshape
  \@setminipage
}
\makeatother
 

%%COMANDI
\newcommand{\q}[1]{``#1''}
\newcommand{\lamb}[2]{\Lambda^{#1}_{\>{#2}}}
\newcommand{\norm}[1]{\left\lVert#1\right\rVert}
\newcommand{\hs}{\mathcal{H}}
\newcommand{\minus}{\scalebox{0.75}[1.0]{$-$}}
\newcommand{\hlc}[2]{%
  \colorbox{#1!50}{$\displaystyle#2$}}
\newcommand{\bb}[1]{\mathbb{#1}}
\newcommand{\op}[1]{\operatorname{#1}}
\renewcommand{\figurename}{Fig.}

\usepackage{fancyhdr}
\pagestyle{fancy}
\fancyhead{} % clear all header fields
\renewcommand{\headrulewidth}{0pt} % no line in header area
\fancyfoot{} % clear all footer fields
\fancyfoot[R]{Francesco Manzali, 2018-19} % other info in "inner" position of footer line
\cfoot{\thepage}

%%AMBIENTI
\newtheorem{thm}{Teorema}[section]
\newtheorem{dfn}{Definizione}
\newtheorem{oss}{Osservazione}
\newtheorem{es}{Esempio}
\newtheorem{axi}{Assioma}
%%Domande di Marchetti
\newtheorem{question}{Domanda}


%%OPERATORI
\DeclareMathOperator{\sech}{sech}
\DeclareMathOperator{\csch}{csch}
\DeclareMathOperator{\arcsec}{arcsec}
\DeclareMathOperator{\arccot}{arcCot}
\DeclareMathOperator{\arccsc}{arcCsc}
\DeclareMathOperator{\arccosh}{arcCosh}
\DeclareMathOperator{\arcsinh}{arcsinh}
\DeclareMathOperator{\arctanh}{arctanh}
\DeclareMathOperator{\arcsech}{arcsech}
\DeclareMathOperator{\arccsch}{arcCsch}
\DeclareMathOperator{\arccoth}{arcCoth} 




\begin{document}
\section{Lezione 4:\\ \large{Il Formalismo di Dirac}}
\vspace{-1em}

% Questa parte non era formattata. Ho messo la mia parte, ma non ho ovviamente eliminato la tua che si trova in un commento successivo.
\lesson{11}{18/10/2018}
Spettro continuo $\lambda \in \sigma_c(A)$, $A \ket{\lambda} = \lambda \ket{\lambda}$ equazione agli autovalori \q{generalizzati}.
Ma in che spazio sta $\ket\lambda$?
\[
X F_\lambda(x) = \lambda F_\lambda(x) \qquad F_\lambda(x) = \delta(x - \lambda) \in \mS'(\RR)
\]
\[
\mathcal S(\RR) \subset L^2(\RR) \subset \mS'(\RR)
\]
A generalizzati $\Phi_A \subset \hs \subset \Phi'_A$ proprietà di $\Phi_A$

1) $\Phi_A \subseteq \dom(A)$

2) $\Phi_A = \hs$

3) $A$ è continuo in $\Phi_A$ nella topologia di $\Phi_A$

4) $\Phi_A$ è \q{nucleare}

$F_\lambda \in \Phi'_A$, $\lambda \in \sigma(A)$
equazione agli autovalori $A F_\lambda = \lambda F_\lambda$ che si deve intendere (come nel caso di $\mS'$)
\[
A F_\lambda (\varphi) \equiv F_\lambda (A \varphi) = \lambda F_\lambda(\varphi) \qquad \forall \varphi \in \Phi_A
\]
Pertanto
\[
\delta(\lambda - x) = \lambda \delta(\lambda - x)
\]
è interpretata come
\[
\int d x \, x \delta(\lambda - x) \phi(x) = \int \de x \lambda \delta(\lambda - x) \varphi(x) \qquad \forall \varphi \in \mS(\RR)
\]
$A F_\lambda (\varphi)$ è continuo e $A \varphi$ con $A$ continuo.

Si dimostra che

1) $\forall A$ autoaggiunto $\exists$ almeno un $\Phi_A$

2) si può prendere $\Phi_A = \mS(\RR^n)$ se $A$ è un polinomio di $\vec x$ e $\vec p$ + $V(\vec x)$

Con $V \in {\mathcal C}^\infty(\RR^n)$ con derivate limitate. Se $V$ ha discontinuità in un insieme di misura nulla $\mathcal N$
\[
\Phi_A = \set{\varphi \in \mS(\RR^n) \,\, | \,\, \varphi(x) = 0 \,\, x \in \mathcal N}
\]

Conveniamo di

1) denotare con $\bra\lambda$ gli autofunzionali in $\Phi'_A$, cioè se $\hs$ è lo spazio astratto, allora l'equazione agli autovalori si scrive
\[
\bra\lambda \, A = \lambda \bra\lambda \qquad \qquad \lambda \in \sigma(A)
\]
(invece che scriverla in ket la scriviamo in bra)

2) denotare con $\ket\varphi$ un ket in $\Phi_A$ e con
\[
\braket{\lambda}{\varphi} \equiv F_\lambda (\varphi)
\]
(con $F-\lambda$ opportunamente normalizzato), cioè l'applicazione del bra $\bra\lambda$ al ket $\ket\varphi$.
Se $\ket\lambda \in \hs$ ($\lambda \in \sigma_p(A)$) allora $\braket{\lambda}{\varphi}$ è semplicemente il prodotto scalare per il quale vale
\[
\braket{\varphi}{\lambda} = \braket{\lambda}{\varphi}^* \qquad \qquad (\star)
\]
Si noti che non ha nessun significato $\braket{\varphi}{\lambda}$ (non abbiamo definito un prodotto scalare).

3) Estendere come definizione $(\star)$ anche se $\bra\lambda \in \Phi'_A$ ma $\bra\lambda \notin \hs$.

4) Estendere la notazione $\braket{x}{\psi} = \psi(x) \in L^2(\RR, d x)$ (per denotare $\ket\psi \in \hs$ in rappresentazione $x$) denotando con $\langle{x}|{x}\rangle$ l'autofunzionale $\bra{\lambda}$ in rappresentazione $x$, cioè $F_\lambda(x) \equiv \braket{\lambda}{x}$ e nuovamente se 
\[
\braket{x}{\lambda} \equiv \braket{\lambda}{x}^*
\]

Allora si dimostra che $\forall A$ autoaggiunto esistono:
\[
\set{\bra{\lambda_{n,r},r} \in \hs' \subset \Phi'_A, \lambda_n \in \sigma_p(A), r = 1,\dots,d(\lambda_n)}
\]
e
\[
\set{\bra{\lambda,r} \in \Phi'_A, \bra{\lambda,r} \notin \hs, \lambda \in \sigma_c(A), r = 1, \dots, d(\lambda)}
\]
tali che per ogni $\ket\psi \in \Phi_A$, $\ket\chi \in \Phi_A$
\[
\ket\varphi = \sum_{\lambda_n \in \sigma_p(A)} \sum_{r = 1}^{d(\lambda_n)} \braket{\lambda_n r}{\varphi} + \int_{\sigma_c(A)} d\lambda \sum_{r = 1}^{d(\lambda)} {\ket{\lambda,r} \braket{\lambda,r}{\varphi}}
\]
\[
\braket{\psi}{\varphi} = \sum_{\lambda_n \in \sigma_p(A)} \sum_{r = 1}^{d(\lambda_n)} \braket{\psi}{\lambda_n, r} \braket{\lambda_n, r}{\varphi} + \int_{\sigma_c(A)} d\lambda \sum_{r = 1}^{d(\lambda)} \braket{\psi}{\lambda,r} \braket{\lambda,r}{\varphi}
\]
Dove $\braket{\psi}{\lambda,r} \braket{\lambda,r}{\psi}^*$

Ma siccome $\Phi_A$ è denso in $\hs$ queste formule per densità si estendono a ogni $\ket{\varphi_0}$, $\ket\phi \in \hs$

Posso calcolare
\[
\bra\psi A \ket\varphi = \sum_{\lambda_n \in \sigma_p(A)} \sum_{r = 1}^{d(\lambda_n)} \lambda_n \braket{\varphi}{\lambda_n, r} \braket{\lambda_n, r}{\varphi} + \int_{\sigma_c(A)} \de\lambda \sum_{r = 1}^{d(\lambda)} \braket{\varphi}{\lambda,r} \braket{\lambda,r}{\varphi}
\]
per $\ket\psi = \ket\varphi$ si ha che $\avg{A}_\varphi = \int \lambda \de (\varphi, P^A(\lambda) \varphi)$

\[
\de P^A(\lambda) \bigg|_{\sigma_p(A)} = \sum_{\lambda_n \in \sigma_p(A)} \delta(\lambda - \lambda_n) \sum_{r = 1}^{d(\lambda_n)} \ket{\lambda_n, r} \bra{\lambda_n, r} \de \lambda
\]
\[
\de P^A(\lambda) \bigg|_{\sigma_c(A)} = \sum_{r = 1}^{d(\lambda)} \ket{\lambda, r} \bra{\lambda, r} \de\lambda
\]
\[
{\mathbb I}_\hs = \sum_{\lambda_n \in \sigma_p(A)} \sigma_{r = 1}^{d(\lambda_n)} \ket{\lambda_n, r} \bra{\lambda_n, r} + \int_{\sigma_c(A)} d\lambda \sum_{r = 1}^{d(\lambda)} \ket{\lambda,r} \bra{\lambda,r}
\]

Vantaggi e svantaggi del formalismo di Dirac

Vantaggi 

1) Semplicità ed eleganza, in particolare la completezza è molto potente.

2) È basato anche per lo spettro continuo $\sigma_c(A)$ su un'equazione agli autovalori.

C'è un modo semplice per vedere se un punto è dello spettro discreto o continuo: per $\lambda \in \sigma_p(A)$ è in $\dom(A)$, $\lambda \notin \sigma_p(A)$ non è in $\hs$.

Scartiamo le soluzioni che divergono esponenzialmente (figura accanto) perché non stanno in $\mS'$. Quanto larga posso prendere la soluzione? La rappresentazione di Dirac funziona solo in $\mS'$.
%devo includere l'immagine dove si toglie l'esponenziale

Esempio. Se prendiamo l'Hamiltoniana $H = \frac{p^2}{2m} + V(x)$, $\Phi_H \subset \mS'(\RR)$, e $\Phi'_H \subset \mS(\RR)$
Non posso dunque prendere le soluzioni che divergono esponenzialmente perché stanno in $\Phi'_H$.

Svantaggi

1) $\phi_A$ non è unico (non c'è un $\phi_A$ \qm{canonico}). La scelta del $\phi_A$ ha una certa arbitrarietà. Questo non piace a certi matematici.

2) La scrittura $\int \de\lambda \ket\lambda\bra\lambda$ è matematicamente imprecisa, e la sua formalizzazione rigorosa richiede una matematica sofisticata (per esempio la \qm{nuclearità}).

3) Se $\bra\lambda \notin \hs' \simeq \hs$ non rappresenta uno stato fisico.
È analogo al caso di de Broglie:
\[
p = \frac{h}{\lambda} \qquad \nu = \frac{\mathcal E}{h}
\]
\[
\lambda \nu = \frac{\mE}{p} = \frac{m c^2 \gamma}{m \nu \gamma} = \frac{c^2}{\nu} > c
\]
Un onda piana non appartiene ad $L^2$, quindi $e^{\frac{i}{\hbar}p_0 x} \notin L^2$, ma se si prende
\[
\int f_{p_0}(p) e^{\frac{i}{\hbar} p} x \de p
\]
può stare in $L^2$ purchè $f_{p_0} \in L^2(\RR, \de p)$, e la velocità effettiva della particella è quella di gruppo:
\[
\deriv{\omega}{k} = \deriv{\mE}{p} = \nu
\]

4) Quando si scrive la completezza (per esempio) per $X$, con Dirac si scrive come
\[
\int \de x \ket x \bra x = \id
\]
assomiglia ad una base \qm{continua} nel caso di una base $\set{\lambda_n}$ ortonormale
\[
\sum_n \ket{\lambda_n} \bra{\lambda_n} = \id
\]
Ma 
\[
\bra x \in \Phi'_X (\bra x \notin \hs)
\]
non è base in $\hs$, e comunque non viola la separabilità di $\hs$!





%% Questi sono i tuoi appunti:
\begin{comment}
Formalismo di Dirac
Spettro continuo \lambda\in \sigma_C\left(A\right), A\left|\lambda\right\rangle=\lambda\ket{\lambda} equazione agli autovalori \q{generalizzata}.
In che spazio sta \ket{\lambda}?
XF_\lambda\left(x\right)=\lambdaF_\lambda\left(x\right)
F_\lambda\left(x\right)=\delta \left(x-\lambda\right)\in \mathcal{S}^\prime\left(\mathbb{R}\right)
S\left(\mathbb{R}\right)\subset L^2\left(\mathbb{R}\right)\subset \mathcal{S}^\prime\left(\mathbb{R}\right)

Gerl'fand generalizza a \phi_A\subset H\subset \phi_A^\prime
Proprietà di \phi_A:
	1) \phi \left(A\right)\subseteq D\left(A\right)
	2) {\bar{\phi}}_A=H
	3) A è continuo in \phi_A nella topologia di \phi_A 
	4) \phi_A è nucleare

F_\lambda\in \phi_A′
\lambda\in \sigma \left(A\right)

Equazione agli autovalori: AF_\lambda=\lambdaF_\lambda che si deve intendere (come nel caso di S′)
AF_\lambda\left(\varphi\right)\equiv F_\lambda\left(A\varphi\right)=\lambdaF_\lambda\left(\varphi\right) \forall \phi \in \phi_A
 
Perciò: X\delta \left(\lambda-x\right)=\lambda\delta \left(\lambda-x\right) si intende come:
\int dx x \delta \left(\lambda-x\right)\phi \left(x\right)=\int dx \lambda\delta \left(\lambda-x\right)\phi \left(x\right) \forall \phi \in S(R)

Si dimostra che:
	1) \forall A autoaggiunto \exists  almeno un \phi_A
	2) Si può prendere \phi_A=S(\mathbb{R}^n) se A= polinomio di \vec{x} o \vec{p} + V(\vec{x})
	Con V\in C^\infty\left(\mathbb{R}^n\right) con derivate limitate. Se V ha discontinuità in un insieme di misura nulla sulla N, \phi_A=\left\{\varphi\in\mathcal{S}\left(\mathbb{R}^n\right)|\varphi\left(x\right)=0\ \ \ x\in N\right\}

(Con questa lezione vogliamo mostrare che esiste una maniera consistente di trattare il formalismo di Dirac)
Conveniamo di:
	1) Denotare con \bra{\lambda} gli autofunzionali in \phi_A^\prime, cioè se H è lo spazio astratto, allora l'equazione agli autovalori si scrive \bra{\lambda}A=\lambda\bra{\lambda}, \lambda\in \sigma \left(A\right)
	(invece di scriverla in ket, la scriviamo in bra)
	2) Denotare con |\phi ⟩ un ket in \phi_A e con \left\langle\lambda\middle|\varphi\right\rangle\equiv F_\lambda\left(\varphi\right) (con F_\lambda opportunamente normalizzato), cioè l'applicazione del bra \bra{\lambda} al ket \left|\varphi\right\rangle.
	Se \left|\lambda\right\rangle\in H (\lambda\in \sigma_P(A)) allora \bra{\lambda}\phi ⟩ è semplicemente il prodotto scalare, per il quale vale la seguente uguaglianza:
	\left\langle\varphi\middle|\lambda\right\rangle={\left\langle\lambda\middle|\varphi\right\rangle}^\ast (*)
	3) Estendere (*) come definizione anche se \left\langle\lambda\right|\in \phi_A^\prime, ma \bra{\lambda}\notin H (altrimenti non avrebbe senso, dato che il prodotto scalare è su H, non su \phi_A. Cioè non posso prendere \phi \in \phi_A^\prime come bra)
	4) Estendere la notazione \left\langle x\middle|\psi\right\rangle=\psi \left(x\right)\in L^2\left(\mathbb{R},\ dx\right) (per denotare \left|\psi\right\rangle\in H in rappresentazione x), denotando con \bra{\lambda}x⟩ l'autofunzionale \bra{\lambda} in rappresentazione x, cioè F_{\lambda\left(x\right)}\equiv \bra{\lambda}x⟩ e nuovamente se 
	\left\langle x\middle|\lambda\right\rangle\equiv {\left\langle\lambda\middle| x\right\rangle}^\ast
Allora si dimostra che \forall A autoaggiunto esiste una famiglia \left\{\left\langle\lambda_n,r\right|\in\mathcal{H}^\prime\subset\phi_A^\prime,\ \lambda_n\in\sigma_P\left(A\right),\ r=1,\ \ldots,\ d(\lambda_n)\right\} e \left\{\langle\lambda_n,\ r|\in\phi_A^\prime,\ \left\langle\lambda,\ r\right|\notin\mathcal{H},\ \lambda\in\sigma_C\left(A\right),\ r=1,\ \ldots,\ d\left(\lambda\right)\right\} tali che \forall \left|\varphi\right\rangle\in \phi_A, \left|\psi\right\rangle\in \phi_A
\left|\varphi\right\rangle=\sum_{\lambda_n\in\sigma_P\left(A\right)}\sum_{r=1}^{d(\lambda_n)}{|\lambda_n,\ r\ket\langle\lambda_n,\ r|\varphi\ket}+\int_{\sigma_C\left(A\right)}{d\lambda\sum_{r=1}^{d\left(\lambda\right)}{|\lambda,\ r\ket\langle\lambda,\ r|\varphi\ket}}
(controllare)

\left\langle\psi\middle|\varphi\right\rangle=\sum_{\lambda_n\in\sigma_P(A)}\sum_{r=1}^{d(\lambda_n)}{\langle\psi|\lambda_n,\ r\ket\langle\lambda_{n,r}|\varphi\ket}+\int_{\sigma_{C\left(A\right)}}{d\lambda\sum_{r=1}^{d\left(\lambda\right)}{\langle\psi|\lambda,\ r\ket\langle\lambda,\ r|\varphi\ket}}
Con \left\langle\psi\middle|\lambda,r\right\rangle={\left\langle\lambda,r\middle|\psi\right\rangle}^\ast
Ma siccome \phi_A è denso in H queste formule per densità si estendono a ogni \left|\varphi\right\rangle, \left|\psi\right\rangle\in H

Posso calcolare:
\left\langle\psi\left|A\right|\varphi\right\rangle=\sum_{\lambda_n\in\sigma_P\left(A\right)}\sum_{r=1}^{d(\lambda_n)}{\lambda_n\langle\psi|\lambda_{n,r}\ket\langle\lambda_{n,r}|\varphi\ket}+ \int_{\sigma_C\left(A\right)}{d\lambda\sum_{r=1}^{d\left(\lambda\right)}{\lambda\langle\psi|\lambda,r\ket\langle\lambda,r|\varphi\ket}}
Per \left|\psi\right\rangle=|\phi ⟩ si ha che:


dP^A\left(\lambda\right)|_{\sigma_P\left(A\right)}
= \sum_{\lambda_n\in\sigma_P\left(A\right)}{\delta\left(\lambda-\lambda_n\right)\sum_{r=1}^{d(\lambda_n)}{|\lambda_n,r\ket\langle\lambda_n,\ r|d\lambda}}
dP^A\left(\lambda\right)|_{\sigma_C\left(A\right)}=
= \sum_{r=1}^{d(\lambda_n)}\left|\lambda,r\ \right\rangle\left\langle\lambda,r\right|d\lambda

Formalmente, la completezza di Dirac:
\mathbb{I}_\mathcal{H}=\sum_{\lambda_n\in\sigma_P(A)}{\sum_{r=1}^{d(\lambda_n)}{|\lambda_n,r\ \ket\langle\lambda_n,\ r|}+\int_{\sigma_C\left(A\right)}{d\lambda\sum_{r=1}^{d\left(\lambda\right)}{|\lambda,r\ket\langle\lambda,r|}}}
(se applicata ad una qualsiasi formula dà il risultato corretto)

Vantaggi e svantaggi del formalismo di Dirac
Vantaggi:
	1) Semplicità ed eleganza, in particolare la completezza è potentissima
	2) È basato anche per \sigma_C(A) su una equazione agli autovalori (per \lambda\in \sigma_P(A) la soluzione è in D(A), mentre se \lambda\in \sigma_C(A) non è neanche in H!)
	3) La notazione di Dirac non ammette soluzioni che divergono a infinito (non stanno in \mathcal{S}^\prime\left(\mathbb{R}\right)). Per esempio, H=\frac{p^2}{2m}+V(x), \phi_H\subseteq S(R), \phi_H^\prime\subseteq S\left(\mathbb{R}\right)
	Non posso prendere le soluzioni che divergono esponenzialmente perché non stanno in \phi_H^\prime (un'onda piana invece non è in L^2, ma in S\left(\mathbb{R}\right) sì, e quindi funziona)

Svantaggi:
	1) \phi_A non è unico, non è \q{canonico} (mentre una famiglia spettrale è unica). 
	2) La scrittura \int d\lambda \ket{\lambda}\bra{\lambda} è matematicamente imprecisa e la sua formalizzazione rigorosa richiede una matematica sofisticata (es. \q{nuclearità})
	3) Se \left\langle\lambda\right|\notin \mathcal{H}^\prime≈H non rappresenta uno stato fisico (es. nel caso di un'onda piana e^{\frac{i}{\hbar}px}\notin L^2. Ma lì p=h/\lambda, e \nu =\frac{E}{h}, \lambda\nu =\frac{E}{p}=\frac{mc^2}{mv\gamma}=\frac{c^2}{v}>c. De Broglie considerò allora un pacchetto d'onde, con \int f_{p_0}\left(p\right)e^{\frac{i}{\hbar}px}dp\in L^2, purché f_{p_0}\in L^2(R, dp), dove f_{p_0} è una funzione \q{piccata} su un punto p_0. Ma allora in questo caso \frac{d\omega}{dk}=\frac{d\mathcal{E}}{dp}=v. Tuttavia, fisicamente si preferisce lavorare con l'onda piana - per semplicità - e nel caso vengano risultati infiniti si considerano le ipotesi di pacchetto d'onda), ma Dirac lo tratta analogamente agli altri stati (nel 99\% di volte produrrà conti giusti, ma ci sono circostanti in cui si giunge ad assurdità)
	4) Quando scrivete la completezza per X:
	\int dx \left|x\right\rangle\left\langle x\right|=I
	Assomiglia ad una \q{versione continua} del caso di una base \left\{|\lambda_n\ket\right\} ON, per cui H sembra non separabile, ma sappiamo che lo è:
	\sum_{n}{\left|\lambda_n\right\rangle\left\langle\lambda_n\right|=\mathbb{I}}
	
	Tuttavia tale relazione è comunque giusta (il formalismo di Dirac funziona, ma \q{va interpretato} nei suoi risultati)
\end{comment}

\end{document}