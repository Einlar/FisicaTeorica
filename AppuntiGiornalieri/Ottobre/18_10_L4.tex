\documentclass[12pt]{article}

\usepackage[usenames, dvipsnames, table]{xcolor}
\usepackage[utf8]{inputenc}
\usepackage[T1]{fontenc}
\usepackage{lmodern}
\usepackage{amsmath}
\usepackage{amsfonts}
\usepackage{comment}
\usepackage{wrapfig}
\usepackage{booktabs}
\usepackage{braket}
\usepackage{tikz}
\usepackage{gnuplottex}
\usepackage{epstopdf}
\usepackage{marginnote}
\usepackage{float}
\usetikzlibrary{tikzmark}
\usepackage{graphicx}
\usepackage{cancel}
\usepackage{bm}
\usepackage{mathtools}
\usepackage{hyperref}
\usepackage{ragged2e}
\usepackage[stable]{footmisc}
\usepackage{enumerate}
\usepackage{mathdots}
\usepackage[framemethod=tikz]{mdframed}
\PassOptionsToPackage{table}{xcolor}
\usepackage{soul}
\usepackage{enumerate}
\usepackage{mathdots}
\usepackage[framemethod=tikz]{mdframed} %Added 16/10
\usepackage[italian]{babel} %Added 16/10
\usepackage{amssymb} %Added

%%BOOKTAB
\setlength{\aboverulesep}{0pt}
\setlength{\belowrulesep}{0pt}
\setlength{\extrarowheight}{.75ex}
\setlength\parindent{0pt} %Rimuove indentazione


%%GEOMETRIA
\usepackage[a4paper]{geometry}
 \newgeometry{inner=20mm,
            outer=49mm,% = marginparsep + marginparwidth 
                       %   + 5mm (between marginpar and page border)
            top=20mm,
            bottom=25mm,
            marginparsep=6mm,
            marginparwidth=30mm}

\makeatletter
\renewcommand{\@marginparreset}{%
  \reset@font\small
  \raggedright
  \slshape
  \@setminipage
}
\makeatother
 

%%COMANDI
\newcommand{\q}[1]{``#1''}
\newcommand{\lamb}[2]{\Lambda^{#1}_{\>{#2}}}
\newcommand{\norm}[1]{\left\lVert#1\right\rVert}
\newcommand{\hs}{\mathcal{H}}
\newcommand{\minus}{\scalebox{0.75}[1.0]{$-$}}
\newcommand{\hlc}[2]{%
  \colorbox{#1!50}{$\displaystyle#2$}}
\newcommand{\bb}[1]{\mathbb{#1}}
\newcommand{\op}[1]{\operatorname{#1}}
\renewcommand{\figurename}{Fig.}

\usepackage{fancyhdr}
\pagestyle{fancy}
\fancyhead{} % clear all header fields
\renewcommand{\headrulewidth}{0pt} % no line in header area
\fancyfoot{} % clear all footer fields
\fancyfoot[R]{Francesco Manzali, 2018-19} % other info in "inner" position of footer line
\cfoot{\thepage}

%%AMBIENTI
\newtheorem{thm}{Teorema}[section]
\newtheorem{dfn}{Definizione}
\newtheorem{oss}{Osservazione}
\newtheorem{es}{Esempio}
\newtheorem{axi}{Assioma}
%%Domande di Marchetti
\newtheorem{question}{Domanda}


%%OPERATORI
\DeclareMathOperator{\sech}{sech}
\DeclareMathOperator{\csch}{csch}
\DeclareMathOperator{\arcsec}{arcsec}
\DeclareMathOperator{\arccot}{arcCot}
\DeclareMathOperator{\arccsc}{arcCsc}
\DeclareMathOperator{\arccosh}{arcCosh}
\DeclareMathOperator{\arcsinh}{arcsinh}
\DeclareMathOperator{\arctanh}{arctanh}
\DeclareMathOperator{\arcsech}{arcsech}
\DeclareMathOperator{\arccsch}{arcCsch}
\DeclareMathOperator{\arccoth}{arcCoth} 




\begin{document}
\section{Lezione 4:\\ \large{Il Formalismo di Dirac}}
\vspace{-1em}
\begin{center}
    \small{(18/10/2018)}
\end{center}

Formalismo di Dirac
Spettro continuo \lambda\in \sigma_C\left(A\right), A\left|\lambda\right\rangle=\lambda\ket{\lambda} equazione agli autovalori "generalizzata".
In che spazio sta \ket{\lambda}?
XF_\lambda\left(x\right)=\lambdaF_\lambda\left(x\right)
F_\lambda\left(x\right)=\delta \left(x-\lambda\right)\in \mathcal{S}^\prime\left(\mathbb{R}\right)
S\left(\mathbb{R}\right)\subset L^2\left(\mathbb{R}\right)\subset \mathcal{S}^\prime\left(\mathbb{R}\right)

Gerl'fand generalizza a \phi_A\subset H\subset \phi_A^\prime
Proprietà di \phi_A:
	1) \phi \left(A\right)\subseteq D\left(A\right)
	2) {\bar{\phi}}_A=H
	3) A è continuo in \phi_A nella topologia di \phi_A 
	4) \phi_A è nucleare

F_\lambda\in \phi_A′
\lambda\in \sigma \left(A\right)

Equazione agli autovalori: AF_\lambda=\lambdaF_\lambda che si deve intendere (come nel caso di S′)
AF_\lambda\left(\varphi\right)\equiv F_\lambda\left(A\varphi\right)=\lambdaF_\lambda\left(\varphi\right) \forall \phi \in \phi_A
 
Perciò: X\delta \left(\lambda-x\right)=\lambda\delta \left(\lambda-x\right) si intende come:
\int dx x \delta \left(\lambda-x\right)\phi \left(x\right)=\int dx \lambda\delta \left(\lambda-x\right)\phi \left(x\right) \forall \phi \in S(R)

Si dimostra che:
	1) \forall A autoaggiunto \exists  almeno un \phi_A
	2) Si può prendere \phi_A=S(\mathbb{R}^n) se A= polinomio di \vec{x} o \vec{p} + V(\vec{x})
	Con V\in C^\infty\left(\mathbb{R}^n\right) con derivate limitate. Se V ha discontinuità in un insieme di misura nulla sulla N, \phi_A=\left\{\varphi\in\mathcal{S}\left(\mathbb{R}^n\right)|\varphi\left(x\right)=0\ \ \ x\in N\right\}

(Con questa lezione vogliamo mostrare che esiste una maniera consistente di trattare il formalismo di Dirac)
Conveniamo di:
	1) Denotare con \bra{\lambda} gli autofunzionali in \phi_A^\prime, cioè se H è lo spazio astratto, allora l'equazione agli autovalori si scrive \bra{\lambda}A=\lambda\bra{\lambda}, \lambda\in \sigma \left(A\right)
	(invece di scriverla in ket, la scriviamo in bra)
	2) Denotare con |\phi ⟩ un ket in \phi_A e con \left\langle\lambda\middle|\varphi\right\rangle\equiv F_\lambda\left(\varphi\right) (con F_\lambda opportunamente normalizzato), cioè l'applicazione del bra \bra{\lambda} al ket \left|\varphi\right\rangle.
	Se \left|\lambda\right\rangle\in H (\lambda\in \sigma_P(A)) allora \bra{\lambda}\phi ⟩ è semplicemente il prodotto scalare, per il quale vale la seguente uguaglianza:
	\left\langle\varphi\middle|\lambda\right\rangle={\left\langle\lambda\middle|\varphi\right\rangle}^\ast (*)
	3) Estendere (*) come definizione anche se \left\langle\lambda\right|\in \phi_A^\prime, ma \bra{\lambda}\notin H (altrimenti non avrebbe senso, dato che il prodotto scalare è su H, non su \phi_A. Cioè non posso prendere \phi \in \phi_A^\prime come bra)
	4) Estendere la notazione \left\langle x\middle|\psi\right\rangle=\psi \left(x\right)\in L^2\left(\mathbb{R},\ dx\right) (per denotare \left|\psi\right\rangle\in H in rappresentazione x), denotando con \bra{\lambda}x⟩ l'autofunzionale \bra{\lambda} in rappresentazione x, cioè F_{\lambda\left(x\right)}\equiv \bra{\lambda}x⟩ e nuovamente se 
	\left\langle x\middle|\lambda\right\rangle\equiv {\left\langle\lambda\middle| x\right\rangle}^\ast
Allora si dimostra che \forall A autoaggiunto esiste una famiglia \left\{\left\langle\lambda_n,r\right|\in\mathcal{H}^\prime\subset\phi_A^\prime,\ \lambda_n\in\sigma_P\left(A\right),\ r=1,\ \ldots,\ d(\lambda_n)\right\} e \left\{\langle\lambda_n,\ r|\in\phi_A^\prime,\ \left\langle\lambda,\ r\right|\notin\mathcal{H},\ \lambda\in\sigma_C\left(A\right),\ r=1,\ \ldots,\ d\left(\lambda\right)\right\} tali che \forall \left|\varphi\right\rangle\in \phi_A, \left|\psi\right\rangle\in \phi_A
\left|\varphi\right\rangle=\sum_{\lambda_n\in\sigma_P\left(A\right)}\sum_{r=1}^{d(\lambda_n)}{|\lambda_n,\ r\ket\langle\lambda_n,\ r|\varphi\ket}+\int_{\sigma_C\left(A\right)}{d\lambda\sum_{r=1}^{d\left(\lambda\right)}{|\lambda,\ r\ket\langle\lambda,\ r|\varphi\ket}}
(controllare)

\left\langle\psi\middle|\varphi\right\rangle=\sum_{\lambda_n\in\sigma_P(A)}\sum_{r=1}^{d(\lambda_n)}{\langle\psi|\lambda_n,\ r\ket\langle\lambda_{n,r}|\varphi\ket}+\int_{\sigma_{C\left(A\right)}}{d\lambda\sum_{r=1}^{d\left(\lambda\right)}{\langle\psi|\lambda,\ r\ket\langle\lambda,\ r|\varphi\ket}}
Con \left\langle\psi\middle|\lambda,r\right\rangle={\left\langle\lambda,r\middle|\psi\right\rangle}^\ast
Ma siccome \phi_A è denso in H queste formule per densità si estendono a ogni \left|\varphi\right\rangle, \left|\psi\right\rangle\in H

Posso calcolare:
\left\langle\psi\left|A\right|\varphi\right\rangle=\sum_{\lambda_n\in\sigma_P\left(A\right)}\sum_{r=1}^{d(\lambda_n)}{\lambda_n\langle\psi|\lambda_{n,r}\ket\langle\lambda_{n,r}|\varphi\ket}+ \int_{\sigma_C\left(A\right)}{d\lambda\sum_{r=1}^{d\left(\lambda\right)}{\lambda\langle\psi|\lambda,r\ket\langle\lambda,r|\varphi\ket}}
Per \left|\psi\right\rangle=|\phi ⟩ si ha che:


dP^A\left(\lambda\right)|_{\sigma_P\left(A\right)}
= \sum_{\lambda_n\in\sigma_P\left(A\right)}{\delta\left(\lambda-\lambda_n\right)\sum_{r=1}^{d(\lambda_n)}{|\lambda_n,r\ket\langle\lambda_n,\ r|d\lambda}}
dP^A\left(\lambda\right)|_{\sigma_C\left(A\right)}=
= \sum_{r=1}^{d(\lambda_n)}\left|\lambda,r\ \right\rangle\left\langle\lambda,r\right|d\lambda

Formalmente, la completezza di Dirac:
\mathbb{I}_\mathcal{H}=\sum_{\lambda_n\in\sigma_P(A)}{\sum_{r=1}^{d(\lambda_n)}{|\lambda_n,r\ \ket\langle\lambda_n,\ r|}+\int_{\sigma_C\left(A\right)}{d\lambda\sum_{r=1}^{d\left(\lambda\right)}{|\lambda,r\ket\langle\lambda,r|}}}
(se applicata ad una qualsiasi formula dà il risultato corretto)

Vantaggi e svantaggi del formalismo di Dirac
Vantaggi:
	1) Semplicità ed eleganza, in particolare la completezza è potentissima
	2) È basato anche per \sigma_C(A) su una equazione agli autovalori (per \lambda\in \sigma_P(A) la soluzione è in D(A), mentre se \lambda\in \sigma_C(A) non è neanche in H!)
	3) La notazione di Dirac non ammette soluzioni che divergono a infinito (non stanno in \mathcal{S}^\prime\left(\mathbb{R}\right)). Per esempio, H=\frac{p^2}{2m}+V(x), \phi_H\subseteq S(R), \phi_H^\prime\subseteq S\left(\mathbb{R}\right)
	Non posso prendere le soluzioni che divergono esponenzialmente perché non stanno in \phi_H^\prime (un'onda piana invece non è in L^2, ma in S\left(\mathbb{R}\right) sì, e quindi funziona)

Svantaggi:
	1) \phi_A non è unico, non è "canonico" (mentre una famiglia spettrale è unica). 
	2) La scrittura \int d\lambda \ket{\lambda}\bra{\lambda} è matematicamente imprecisa e la sua formalizzazione rigorosa richiede una matematica sofisticata (es. "nuclearità")
	3) Se \left\langle\lambda\right|\notin \mathcal{H}^\prime≈H non rappresenta uno stato fisico (es. nel caso di un'onda piana e^{\frac{i}{\hbar}px}\notin L^2. Ma lì p=h/\lambda, e \nu =\frac{E}{h}, \lambda\nu =\frac{E}{p}=\frac{mc^2}{mv\gamma}=\frac{c^2}{v}>c. De Broglie considerò allora un pacchetto d'onde, con \int f_{p_0}\left(p\right)e^{\frac{i}{\hbar}px}dp\in L^2, purché f_{p_0}\in L^2(R, dp), dove f_{p_0} è una funzione "piccata" su un punto p_0. Ma allora in questo caso \frac{d\omega}{dk}=\frac{d\mathcal{E}}{dp}=v. Tuttavia, fisicamente si preferisce lavorare con l'onda piana - per semplicità - e nel caso vengano risultati infiniti si considerano le ipotesi di pacchetto d'onda), ma Dirac lo tratta analogamente agli altri stati (nel 99% di volte produrrà conti giusti, ma ci sono circostanti in cui si giunge ad assurdità)
	4) Quando scrivete la completezza per X:
	\int dx \left|x\right\rangle\left\langle x\right|=I
	Assomiglia ad una "versione continua" del caso di una base \left\{|\lambda_n\ket\right\} ON, per cui H sembra non separabile, ma sappiamo che lo è:
	\sum_{n}{\left|\lambda_n\right\rangle\left\langle\lambda_n\right|=\mathbb{I}}
	
	Tuttavia tale relazione è comunque giusta (il formalismo di Dirac funziona, ma "va interpretato" nei suoi risultati)


\end{document}