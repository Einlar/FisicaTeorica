\documentclass[../../FisicaTeorica.tex]{subfiles}

\begin{document}

\section{Lezione X:\\ \large{Particelle indistinbuibili - parte 2}}
\vspace{-1em}
\begin{center}
    \small{(12/12/2018)}
\end{center}

Come visto nella precedente lezione, in \MQ particelle identiche sono per forza \textit{indistinguibili}. Non potendo determinare l'identità di ogni particella, non ha senso che le previsioni teoriche dipendano da essa, e perciò i valor medi non possono differire per scambi di particelle identiche.\\
Consideriamo i gruppi $S_N$ delle permutazioni di $N$ oggetti. Gli elementi $\sigma \in S_N$ sono \textit{generiche permutazioni}, che possono essere scritte come prodotto di scambi $\sigma_i$, con $i=1, \dots, N-1$, che rappresentano gli scambi di $i$-esimo e $i+1$-esimo componente.\\
Tali $\sigma_i$ soddisfano le seguenti proprietà:
\begin{itemize}
\item $\sigma_i^2 = 1$
\item $\sigma_i \sigma_j = \sigma_j \sigma_i$ $|i-j| \geq 2$
\item $\sigma_i \sigma_{i+1} \sigma_i = \sigma_{i+1}\sigma_i \sigma_{i+1}$
\end{itemize}
Dato un $\sigma \in \hs$, con $\{e_j\}_{j\in J}$ base per $\hs$, esso agisce \textit{permutando}:
\begin{align*}
U(\sigma): e_{j_1} \otimes \dots \otimes e_{j_N} \to e_{\sigma(j_1)} \otimes \dots \otimes e_{\sigma(j_N)}
\end{align*}

Matematicamente, $\forall \ket{\psi}\in \hs^{\otimes \bb{N}}$ e $\forall$ osservabile $A$ in $\hs^{\otimes \bb{N}}$, si ha:
\begin{align}
\bra{\psi} U^\dag(\sigma)AU(\sigma)\ket{\psi} =\ \bra{\psi}A\ket{\psi}\quad \forall \sigma \in S_N \Rightarrow U^\dag(\sigma)AU(\sigma)=A
\label{eqn:condizione-simmetria-scambi}
\end{align}
Perciò non tutti gli operatori autoaggiunti in\ $\hs^{\otimes \bb{N}}$ sono osservabili, ma solo quelli per cui vale (\ref{eqn:condizione-simmetria-scambi}). Poiché ogni operatore in $\hs^{\otimes \bb{N}}$ può scriversi come combinazione lineare (eventualmente infinita) di operatori $A_1 \otimes A_2 \otimes \dots \otimes A_N$, ove gli $A_i$, detti operatori di particella singola, agiscono sulla $i$-esima coppia di $\hs$ in $\hs^{\otimes \bb{N}}$, La (\ref{eqn:condizione-simmetria-scambi}) ci dice che \textbf{sono osservabili solo operatori autoaggiunti} in $\hs^{\otimes \bb{N}}$ \textbf{simmetrici} per \textit{ogni} scambio degli operatori di particella singola.\\

Inoltre $U(\sigma)\ket{\psi}$, $\forall \sigma \in S_N$, $\ket{\psi} \in \hs^{\otimes \bb{N}}$ deve descrivere lo stesso stato di $\ket{\psi}$, ma poiché\ per ogni scambio $\sigma_i$, $\sigma_i^2 = 1$, deve essere:
\begin{align}
U(\sigma_i)^2 = U(\bb{I}) = \bb{I}
\label{eqn:condizione-quadrato}
\end{align}
perché avvenga:
\begin{align*}
U(\sigma_i) \ket{\psi} = c(\sigma_i)^2 \ket{\psi}
\end{align*}
E (\ref{eqn:condizione-quadrato}) impone che $c(\sigma_i)^2=1$.\\

Quindi abbiamo due possibilità. $\forall \sigma_i$ deve succedere:
\begin{itemize}
\item $c(\sigma_i) = 1$, ossia la funzione d'onda o il vettore di stato è \textit{simmetrica per scambi}. Le particelle che soddisfano questa condizone sono detti \textbf{bosoni}.
\item $c(\sigma_i)=-1$, dove invece funzione d'onda/vettore di stato sono \textit{antisimmetrici per scambi}. Le particelle che soddisfano quiesta proprietà sono i \textbf{fermioni}.
\end{itemize}

Tutto ciò porta a un fenomeno nuovo: i \textbf{settori di superselezione}\marginpar{Settori di superselezione}\index{Settori di superselezione}.\\
Poiché le osservabili sono \textit{simmetriche per scambi}, i sottospazi di $\hs^{\otimes \bb{N}}$ sede delle rappresentazioni simmetriche (bosoni) o antisimmetriche (fermioni) sono lasciati invariati sia dall'evoluzione temporale che è l'esponenziale di un'osservabile, $H$, che è simmetrica, sia dalle misure di qualsiasi osservabile, che sono descritte da proiettori, che essendo osservabili sono simmetrici.\\
\textit{Tali sottospazi sono \q{mondi separati}, ossia un vettore - uno stato - che sta in uno di essi, per esempio lo spazio simmetrico, è destinato a rimanervi per sempre - non c'è modo di passare da uno all'altro. Questo è il concetto di superselezione}

Quindi gli stati simmetrici e antisimmetrici non si mescolano mai qualsiasi operazione si faccia. Descrivono qindi \q{mondi distinti} che tecnicamente si chiamano \textbf{settori di superselezione}.

\textit{Un altro esempio, meno ovvio, di settore di superselezione è dato dalla carica elettrica. Non è possibile sovrapporre due stati di carica diversa - e qualsiasi operazione si faccia non è possibile cambiarla. Operatori che non rispettano tali condizioni non costituiscono osservabili.}

Se $\ket{\psi_i}$, $i=1, \dots, N$ sono ket di particella singola, ad esempio:
\begin{align*}
&\frac{1}{\sqrt{N!}}\sum_{\sigma \in S_N} \prod_i \ket{\psi_{\sigma(i)}} && \text{ket per $N$ bosoni}\\
&\frac{1}{\sqrt{N!}}\sum_{\sigma \in S_N} (-1)^{|\sigma|}\prod_i \ket{\psi_{\sigma(i)}} && \text{ket per $N$-fermioni}
\end{align*}
Dove con $|\sigma|$ indichiamo il numero di scambi effettuati per formare la permutazione $\sigma$ (che a priori non è ben definito, ma è ben definita la sua \textit{parità}, che è l'unica cosa che ci interessa nella formula di sopra).\\
Notiamo che tali stati per il sistema composto \textbf{non} sono il semplice prodotto di ket di particelle singole $\ket{\psi_i}$, ma combinazioni lineari di tali prodotti, cioè sono \q{entangled}. Questa è la conseguenza \textit{fisica} della \textit{perdita dell'individualità}: il fatto che non si possa discutere di particelle singole (poiché non è possibile identificarle), fa sì che l'unico modo di descrivere il sistema sia in una \q{visione globale} che non può essere spezzata nelle singole parti.\\

Una conseguenza immediata dell'antisimmetria per i fermioni è il seguente principio:
\begin{thm}
\marginpar{Principio di esclusione di Pauli}\index{Principio di esclusione di Pauli}
Non è possibile per un sistema di $N$ fermioni identici avere due particelle con lo stesso insieme di autovalori di un ICOC.
\end{thm}
Ad esempio, consideriamo come lo spazio di Hilbert $\hs$ di particella singola per un atomo di idrogeno. Un ICOC è $\{H, \vec{L^2}, L_z, S_z\}$. Indichiamo i rispettivi autovalori di tali osservabili con $\{n, l, m, s_z\}$.\\
Per il principio di esclusione, è impossibile mettere due elettroni con gli stessi $n, l, m, s_z$ nell'atomo di idrogeno - cioè al massimo vi sono due elettroni per \textit{orbitale} (dove un orbitale è definito da i numeri quantici $n,l,m$).\\
Lo stesso ICOC vale anche per un qualsiasi sistema coulombiano - e in particolare per atomi più complessi.\\
In un atomo con molti elettroni siamo costretti a mettere elettroni in orbitali a energia crescente per rispettare il principio di Pauli, e da ciò deriva tutta la \textit{chimica}.\\
In modo analogo sappiamo di un ICOC $\{\vec{X}, S_z\}$, e quindi Pauli ci dice in modo immaginifico che due elettroni non possono avere neppure \q{potenzialmente} stessa posizione e spin. Infatti, se ipotizziamo che $\vec{x}_i, s_{z_i} = \vec{x_j}, s_{z_j}$ (gli elettroni $i$-esimo e $j$-esimo hanno la stessa posizione e spin). Se partiamo da una funzione d'onda:
$\psi(\vec{x}_1, s_{z_1}, \dots, \vec{x}_N, s_{z_N}$
e scambiamo i due elettroni $i$ e $j$, sappiamo che il segno deve cambiare per l'antisimmetria (essendo gli elettroni fermioni):
\begin{align*}
\psi(\vec{x}_1, s_{z_1}, \dots, \hlc{Yellow}{\vec{x}_i, s_{z_i}}, \dots \hlc{SkyBlue}{\vec{x}_j, s_{z_j}}, \dots, \vec{x}_N, s_{z_N})=-
\psi(\vec{x}_1, s_{z_1}, \dots, \hlc{SkyBlue}{\vec{x}_j, s_{z_j}}, \dots \hlc{Yellow}{\vec{x}_i, s_{z_i}}, \dots, \vec{x}_N, s_{z_N})
\end{align*}
Ma se i due termini scambiati sono uguali (per ipotesi), allora le due $\psi$ devono essere uguali (scambiare gli stessi termini \textit{non} modifica la $\psi$). Ma allora l'unica possibilità è che tale $\psi=0$.\\


Poiché la funzione d'onda descrive le probabilità di misura, anche prima che essa venga effettuata, si ha che \textit{anche prima di una misura} gli elettroni non possono ocupare gli stessi stati di posizione-spin.\\

Qualitativamente questa è l'\textit{origine dell'incompenetrabilità} della materia.\\
Perché nonostante l'atomo sia quasi interamente vuoto, e la probabilità di interazione tra gli elettroni sia bassa, la sola \textit{potenzialità} di occupazione degli stessi stati è impedita dal principio di esclusione.\\

Il principio di esclusione è utile per comprendere anche la struttura delle \textbf{stelle di neutroni}. Se la forza di gravitazionale è sufficientemente intensa, gli orbitali sono \q{schiacciati} uno contro l'altro, ma non possono compenentrarsi per principio di esclusione. Avviene così un altro processo: un protone e un elettrone si \textit{fondono} a formare un neutrone.
\begin{align*}
p^+ + e^- \to n + \nu_e
\end{align*}
E i neutroni possono \q{compattarsi} molto di più, dato che la distanza tra ciascuno di essi è dell'ordine di $10^{-15}m$. Ciò permetterebbe all'intera massa del Sole, che normalmente ha un raggio $\sim 700\cdot 10^3$km, di essere concentrata in un raggio di $10$km.\\
Un $cm^3$ di \q{neutronio} ha una massa dell'ordine $\sim 2\cdot 10^6$kg.\\

C'è un modo lievemente diverso di vedere la statistica (comportamento per scambi) che è rilevante in $d<3$ (trascuriamo, per ora, lo spin).\\
Possiamo realizzare concretamente uno scambio di posizioni $x_i$, $x_j$ di particelle identiche con un cammino continuo che porti $x_i$, $x_j$ in $x_j$, $x_i$.\\
%Inserire disegnetto

Se introduciamo un parametro $t$ per descrivere il cammino:

Se $d=3$, possiamo far passare una particella \q{sotto} all'altra, e i seguenti scambi sono completamente equivalenti:
%Disegnetto

dato che possiamo passare da uno all'altro \textit{ruotando} di $\pi$ \textit{nella terza dimensione}.\\

Se però $d=2$, se le particelle non possono occupare la stessa posizione, come nel caso dei fermioni, non c'è modo di mappare con continuità uno scambio \textit{orario} in uno \textit{antiorario}, senza intersecare i cammini. Perciò le due possibilità non sono più equivalenti:
%Disegnetto

Gli scambi $\sigma_i$ in 2D sono perciò \textit{orientati}, $\sigma_i^2 \neq 1$.
%Disegnetto
(Volendo si potrebbe continuare ad intrecciare)\\

Il gruppo generato da prodotti di scambi \textit{con orientazione} è detto \textbf{gruppo delle trecce} $B_N$ per $N$ oggetti\footnote{La $B$ sta per \textit{braid}}, e per i suoi elementi valgono le seguenti proprietà:
\begin{itemize}
\item $\sigma_i \sigma_j = \sigma_j \sigma_i$, $|i-j|>2$
\item $\sigma_i \sigma_{i+1}\sigma_i = \sigma_{i+1}\sigma_i\sigma_{i+1}$
\item Con $\sigma_i$ indichiamo gli scambi orari, e $\sigma_i^{-1}\neq \sigma_i$ indica uno scambio antiorario
\item $U(\sigma_i)^2 \neq 1$, e valgono:
\begin{align*}
U(\sigma_i) \ket{\psi} &= e^{i\theta} \ket{\psi}\\
U(\sigma_i^{-1}) \ket{\psi} &= e^{-i\theta}\ket{\psi}
\end{align*}
\end{itemize}
Tale idea, concepita dal punto di vista completamente teorico negli anni '70 da Leinaas-Myrheim.\\
Nel 1986, invece, fu osservata una conseguenza di ciò nell'effetto Hall quantistico frazionario. Per certe particelle, il $\theta$ di sopra non è né $0$ (come per i bosoni) né $\pi$ (come per i fermioni), ma assume un valore \textit{generico}. Tali particelle sono perciò indicate come \textit{anyoni}.\\

Sperimentalmente, consideriamo un interfaccia tra semiconduttori (che costituisce un sistema bidimensionale), su cui mettiamo un campo magnetico $\vec{B}$ perpendicolare ad essa. Le particelle seguono le orbite di ciclotrone, per cui abbiamo i livelli di Landau per le energie, dati da $\hbar \omega \left(n+\frac{1}{2}\right)$, dove $\omega$ è la pulsazione di ciclotrone. Gli elettroni si dispongono inizialmente nel livello energetico inferiore, e man mano riempiono quelli più alti.\\
In realtà, però, i livelli di Landau non sono ben definiti a causa delle \textit{impurezze} dell'interfaccia. Tali impurezze causano delle \q{buche di potenziale localizzate}, e può capitare che degli elettroni rimangano intrappolati in esse \textit{tra due livelli di Landau}.\\
Tale fenomeno costituisce l'effetto Hall intero (si verifica per $T\sim 10K$), e fu scoperto nel 1983.\\
Nel 1986, si provò a ripetere l'esperimento abbassando ulteriormente la temperatura, a $T \sim 1K$ e si scoprì un fenomeno leggermente diverso.\\
Detta $n$ la densità di elettroni, per l'effetto Hall intero si trovava:
\begin{align*}
\frac{nhc}{eB} = \frac{1}{3}
\end{align*}
In questo nuovo sistema si trovò che gli elettroni avevano uno stato di energia minimo uniforme, e i livelli eccitati erano costituiti da eccitazioni che obbedivano alla statistica di trecce, con $\theta=2\pi/3$.\\
Se $x,y$ sono le coordinate del piano, definiamo $z=x+iy$. Allora le funzioni d'onda di $N$ di queste eccitazioni sono della forma:
\begin{align*}
\psi(z_1, \dots, z_N) = \prod_{i<j} (z_i - z_j)^{\frac{1}{3}}\prod_{i=1}^N e^{-c|z_i|^2}
\end{align*}
Scambiando in sensio antiorario due particelle, normalmente avremmo $(z_i - z_j) \to e^{i\pi} (z_i -z_j)$, ma ora abbiamo:
\begin{align*}
(z_i-z_j)^{\frac{1}{3}} \to (e^{i\pi}(z_i-z_j))^{\frac{1}{3}} = e^{i\frac{\pi}{3}}(z_i - z_j)^{\frac{1}{3}}
\end{align*}
Analogamente, uno scambio orario sarebbe $(z_i-z_j) \to e^{-i\pi (z_i-z_j)}$, ma qui diviene:
\begin{align*}
(z_i-z_j)^{\frac{1}{3}} \to (e^{-i\pi (z_i - z_j)})^{\frac{1}{3}}= e^{-i\frac{\pi}{3}}(z_i- z_j)^{\frac{1}{3}}
\end{align*}
(forse non ci va l'esponente nella prima parentesi).\\

Tali fenomeni possono essere sfruttati per la costruzione di comouter quantistici, dato che si preannunciano particolarmente stabili per la costruzione di porte logiche.


\end{document}

