\documentclass[../../FisicaTeorica.tex]{subfiles}

\begin{document}

\section{Stati legati per l'atomo di idrogeno}
Occupiamoci ora di studiare il caso di un potenziale $U(r)$ coulombiano, analizzando il caso specifico dell'\textit{atomo di idrogeno}, e seguendo da vicino la trattazione di tale sistema data da Dirac.\\

L'atomo di idrogeno consiste in un protone di carica $+e$ e un elettrone di carica $-e$ che si \textit{attraggono} tramite l'interazione elettromagnetica, il cui potenziale è:
\begin{align*}
U(r) = -\frac{e^2}{r}
\end{align*}
Possiamo perciò subito scrivere l'equazione radiale:
\begin{align}
\frac{d^2\chi_{\epsilon l}(r)}{dr^2}(r) +\left[\frac{2m}{\hbar^2}\left(\mathcal{E}+\frac{e^2}{r}\right)-\frac{l(l+1)}{r^2}\right]\chi_{\epsilon l}(r) = 0
\label{eqn:autovalori_idrogeno_rad}
\end{align}
%Inserire plot di U(r)

Siamo interessati agli \textbf{stati legati}, ossia a quelli per cui $\mathcal{E}<0$. Sappiamo che:
\begin{align*}
rh_{\epsilon l}(r) = \chi_{\epsilon l}(r)
\end{align*}
e $h_{\epsilon l}(r) \in L^2(\bb{R}_+, r^2 dr)$ oppure $\chi_{\epsilon l}(r) \in L^2(\bb{R}_+, dr)$.\\
Dall'analisi della precedente lezione, sappiamo anche le condizioni al contorno e l'andamento per $r\to 0$ di $\chi_{\epsilon l}$:
\begin{align*}
\chi_{\epsilon l}(0) = 0 \qquad \chi_{\epsilon l}(r) \underset{r \to 0}{\sim} r^{l+1}
\end{align*}
Vogliamo allora trovare esplicitamente le soluzioni radiali all'equazione (\ref{eqn:autovalori_idrogeno_rad}). Prima di farlo, riscriviamola in maniera più semplice. In maniera analoga a quanto fatto nel caso di $U(r) \equiv 0$, definiamo delle opportune costanti $k$ e $a$:
\begin{align*}
\frac{d^2 \chi_{\epsilon l}(r)}{dr^2} + \left[-k^2 + \frac{2}{ra} - \frac{l(l+1)}{r^2}\right] \chi_{\epsilon l}(r)=0 \qquad k \equiv \sqrt{-\frac{2m}{\hbar^2}\mathcal{E}}; \quad a \equiv \frac{\hbar^2}{me^2}
\end{align*}
dove $a \approx 0.52 \AA$ è detto \textit{raggio di Bohr}.\\
Un'ulteriore semplificazione è data da un cambio di variabili che permette di \q{estrarre} una costante $k^2$ e rimuoverla dividendo per $4k^2$:
\begin{align}\nonumber
x \equiv 2kr &\Rightarrow  r=\frac{x}{2k}\\ \nonumber
4k^2\frac{d^2 \chi_{\epsilon l}(x)}{dx^2} &+\left[-k^2 + \frac{4k}{xa} - \frac{l(l+1)}{x^2}4k^2 \right] \chi_{\epsilon l}(x) =0\\
\underset{/(4k^2)}{\Rightarrow} \frac{d^2 \chi_{\epsilon l}(x)}{dx^2}&+\left[-\frac{1}{4} + \frac{\nu}{x} - \frac{l(l+1)}{x^2} \right] \chi_{\epsilon l}(x) = 0 \qquad \nu\equiv \frac{1}{ka}
\label{eqn:equazione-radiale-idrogeno-semplificata}
\end{align}
Tale cambio di variabili è motivato dal fatto che la $x$ così definita è adimensionale. Infatti:
\begin{align*}
[k] = \sqrt{\frac{\si{\kilogram}}{\si{\joule}^2 \si{\second}^2}\si{\joule}} = \sqrt{\frac{\si{\kilogram}}{\si{\newton\meter \second^2}}}=\sqrt{\frac{1}{\si{\meter^2}}}=\frac{1}{m}; \quad [r]=\si{\meter} \Rightarrow [kr]=\mathcal{N}
\end{align*}
In generale\footnote{CFR Lecture 24 di \url{bit.ly/2GLsM66}, mentre il processo è detto \q{Nondimensionalization}}, quando si ha a che fare con lo studio di comportamenti asintotici (come vedremo nei prossimi paragrafi) è comodo lavorare in una scala \q{fondamentale} con quantità adimensionali.\\ %inserire citazione nella sitografia
Per risolvere la (\ref{eqn:equazione-radiale-idrogeno-semplificata}), partiamo analizzando i punti di possibile singolarità dell'equazione differenziale, che si possono avere per $x\to \infty$, oppure, data la presenza di denominatori, per $x\to 0$. La strategia è quella di risolvere \q{approssimazioni} dell'equazione in questi limiti in cui consideriamo solo i termini dominanti, e poi \q{cercare di raccordare} il tutto per ottenere una soluzione generale\footnote{Il termine tecnico è \textbf{ansatz}, ossia un \textit{educated guess} sulla soluzione di un'equazione differenziale difficile.}.\\

Per $x\to \infty$ l'unico termine dominante è $-1/4$, e l'equazione approssimata è quella di un \textit{repulsore armonico}:
\begin{align*}
\frac{d^2}{dx^2}\chi_{\epsilon l}(x) - \underbrace{\frac{1}{4}}_{\omega^2}\chi_{\epsilon l}(x) = 0 \Rightarrow  \chi_{\epsilon l}(x) \underset{x \to \infty}{\sim} e^{\pm\omega x} = e^{\pm\frac{x}{2}} = e^{\pm kr}
\end{align*}
Di queste due, $e^{+kr}$ di sicuro non sta in $L^2$ (diverge esponenzialmente), e perciò solo $e^{-kr}$ è accettabile:
\begin{align*}
\chi_{\epsilon l}(x) \underset{x \to 0}{\sim} e^{-\frac{x}{2}}
\end{align*}

Per $x\to 0$, invece, il problema si riconduce a quello già visto nella lezione precedente, dato che $U(r)$ ed $\mathcal{E}$ sono trascurabili rispetto all'\textit{energia centrifuga}:
\begin{align*}
\frac{d^2 \chi_{\epsilon l}}{dx^2} - \frac{l(l+1)}{x^2}\chi_{\epsilon l} (x) = 0 \Rightarrow  \chi_{\epsilon l}(x) \underset{x\to 0}{\sim} x^{l+1}
\end{align*}

Dalle informazioni appena ricavate, ipotizziamo che una soluzione generale sia data dal prodotto tra le soluzioni \q{locali} appena trovate e una certa funzione $v_l(x)$ che non alteri il comportamento di $\chi_{\epsilon l}(x)$ ai limiti $x\to 0$ e $x\to \infty$:
\begin{align*}
\chi_{\epsilon l}(x) = \underbrace{x^{l+1} }_{f}\underbrace{e^{-\frac{x}{2}}}_{g} \underbrace{v_l(x)}_{h}
\end{align*}
Inserendo nell'equazione (\ref{eqn:autovalori_idrogeno_rad}), e calcolando la derivata seconda iterando più volte Leibniz:
\begin{align*}
(fgh)'' &=(f'gh + fg'h + fgh')' \\
&= (f'' g h + f'g'h + f'gh')  +\\
&\quad\>(f'g'h + fg''h+fg'h') +\\
&\quad\>(f'gh' + fg'h' + fgh'') =\\
&= \hlc{Yellow}{(f''g + 2f'g' + fg'')h} + \hlc{SkyBlue}{2(f'g + 2fg')h' }+ \hlc{ForestGreen}{fgh}''\\
(\ref{eqn:autovalori_idrogeno_rad}) =
&\hlc{Yellow}{\left[\cancel{l(l+1)x^{l-1} }+ 2(l+1)x^l\left(-\frac{1}{2}\right)
+ \bcancel{x^{l+1}\left(-\frac{1}{2}\right)^2} \right] e^{-\frac{x}{2}}v_l(x)} +\\
& +\hlc{SkyBlue}{ 2\left((l+1)x^l +x^{l+1}\left(-\frac{1}{2}\right) \right ) e^{-\frac{x}{2}} v_l'(x)} + \hlc{ForestGreen}{x^{l+1} e^{-\frac{x}{2}} v_l''(x)}+\\
&+ \left[-\bcancel{\frac{1}{4}} + \frac{\nu}{x} -\cancel{\frac{l(l+1)}{x^2}} \right] e^{-\frac{x}{2}} x^{l+1}v_l(x) = 0 
\end{align*}
Dividendo per $x^l$ e $e^{-x/2}$ giungiamo a:
\begin{align}
xv_l'' + (2l+2-x)v_l' - \left((l+1)-\nu\right)v_l = 0
\label{eqn:diff_vl}
\end{align}
che dobbiamo risolvere per $v_l$. Possiamo farlo sviluppando $v_l(x)$ in serie:
\begin{align}
v_l(x) = \sum_{p=0}^{\infty} a_p x^p
\label{eqn:vl_series}
\end{align}
con $a_0 = 1$, dato che non vogliamo turbare l'andamento a $x\to 0$, e perciò deve essere $v_l(0) \neq 0$, dato che altrimenti non avremmo $\chi_{\epsilon l}(x) \underset{x\to 0}{\sim}x^{l+1}$.\\
Inserendo (\ref{eqn:vl_series}) in (\ref{eqn:diff_vl}) otteniamo:
\begin{align*}
\sum_{p=2}^\infty p(p-1) a_p x^{p-1} + \sum_{p=1}^\infty p[2(l+1)x^{p-1}-x^p]a_p -\sum_{p=0}^\infty ((l+1)-\nu) a_p x^p = 0
\end{align*}
Conviene riportare tutto sotto la stessa sommatoria, sostituendo \hbox{$p-1 \equiv p'\Rightarrow p=p'+1$}:
\begin{align*}
\hlc{Yellow}{\sum_{p'={0}}^\infty} (p'+1)p' a_{p'+1} x^{p'} + \sum_{p'=0}^\infty a_{p'+1} (p'+1) (2(l+1))x^{p'} - \hlc{Yellow}{\sum_{p=0}^\infty} a_p p x^p - \sum_{p=0}^\infty a_p(l+1-v)x^p=0
\end{align*}
\textbf{Nota}: nella prima sommatoria evidenziata $p'$ parte da $0$, e non da $1$ come si avrebbe dalla situazione. Possiamo far ciò notando che essa conteneva già un termine $p'$, e quindi partire da $p'=0$ non fa altro che aggiungere in testa alla successione un termine nullo, che non modifica la serie. Un discorso analogo (stavolta per $p$ e non per $p'$) vale per la seconda sommatoria evidenziata.\\

Dato che $p'$ e $p$ sono indici muti, possiamo anche usare lo stesso indice per tutti e due: così facendo compariranno termini $a_{p+1}$ oltre a quelli $a_p$. Potremo allora trovare una \textit{relazione di ricorrenza} scrivendo $a_{p+1}$ in funzione del termine precedente $a_p$, e usare tale relazione per \textit{definire} la serie. Riarrangiando, otteniamo allora:
\begin{align*}
\Rightarrow  \sum_{p=0}^\infty  (p+1)(p + 2(l+1)) a_{p+1}   x^{p} = \sum_{p=0}^\infty [p+ l+1-\nu]a_p x^p
\end{align*}
Tale uguaglianza deve essere vera $\forall x$, e perciò deve valere \textit{potenza per potenza} (ossia togliendo le sommatorie):
\begin{align}
(p+1)(p+2l+2)a_{p+1} = (p+l+1-\nu)a_p
\label{eqn:relazione_ricorrenza}
\end{align}
\q{Diminuendo} $p$ di $1$ otteniamo una relazione leggermente più semplice:
\begin{align}
p(p+2l+1)a_{p} = (p+l-\nu)a_{p-1}
\label{eqn:ricorsione-semplificata}
\end{align}
Verifichiamone la convergenza, per esempio tramite il criterio del rapporto:
\begin{align*}
\frac{a_p}{a_{p-1}} = \frac{p+l-\nu}{p(p+2l+1)} \underset{p\to \infty}{\sim} \frac{1}{p} \Rightarrow a_p \underset{p \to \infty}{\sim}\frac{1}{p!}
\end{align*}
La serie definitivamente converge, ma il suo andamento asintotico è esponenziale:
\begin{align*}
v_l(x) = \sum_{p=0}^\infty a_p x^p \underset{x\to\infty}{\sim} \sum_{p=0}^\infty \frac{x^p}{p!} = e^x
\end{align*}
E sostituendo nell'espressione per la $\chi_{\epsilon l}(x)$:
\begin{align*}
\chi_{\epsilon l}(x) \sim x^{l+1} e^{-\frac{x}{2}} e^x \notin L^2
\end{align*}
e questo è un gran problema, perché sappiamo (per il fatto che la materia esiste) che gli autostati legati dell'atomo di idrogeno sono \q{fisici}, e quindi devono stare in $L^2$.\\

Perché ciò non succeda, la serie di $v_l(x)$ deve troncarsi, ossia se per un certo $p = n' \in \bb{N}$ opportuno si ha $a_{n'}\neq 0$, ma $a_{n' +1}=0$, da cui, per ricorrenza, tutti i termini successivi sono nulli. Imponendo $a_{n'+1}=0$ si ha che il coefficiente di $a_{n'}\neq 0$ in (\ref{eqn:relazione_ricorrenza}) deve annullarsi, ma ciò è possibile solo se $\nu$ assume valori interi:
\begin{align*}
n' + l +1 -\nu=0 \Rightarrow \nu = \{n' + l +1, n' \in \bb{N}\} = \{n\in \bb{N}, n\geq l+1\}
\end{align*}
Ricordando l'espressione per $\nu$:
\begin{align*}
\nu = \frac{1}{ka}=\frac{me^2}{\hbar^2} \sqrt{\frac{\hbar^2}{-2m\mathcal{E}}}
\end{align*}
si ha che la condizione su $\nu$ si traduce in una condizione sull'energia. Indichiamo con $\mathcal{E}_n$ il valore dell'energia a cui corrisponde $\nu = n$:
\begin{align}
\mathcal{E}_n = -\frac{\hbar^2}{2m}\frac{1}{(\nu a)^2} = -\frac{\hbar^2}{2m}\left(\frac{me^2}{\hbar^2}\right)^2 \frac{1}{n^2} = -\frac{me^4}{2\hbar^2}\frac{1}{n^2} \approx -\frac{\SI{13.6}{\electronvolt}}{n^2}
\label{eqn:bohr_energy}
\end{align}
Ritroviamo allora le \textbf{energie di Bohr}, che erano state ricavate inizialmente imponendo artificialmente che il momento angolare fosse multiplo intero di una certa costante. Le $\mathcal{E}_n$ spiegano anche le frequenze di Rydberg per l'atomo di idrogeno: i fotoni emessi hanno infatti un'energia pari a $\mathcal{E}_m-\mathcal{E}_n$, con $m<n$, dato che sono generati quando un elettrone passa da uno stato \textit{eccitato} a uno a più bassa energia. Ecco allora perché, come avevamo osservato a inizio del corso, si scopre che tali frequenze sono sempre ottenibili come differenza di due numeri.\\

Come sono fatte le autofunzioni associate alle $\mathcal{E}_n$?\\
Da (\ref{eqn:ricorsione-semplificata}) ricaviamo:
\begin{align*}
\frac{a_{p}}{a_{p-1}}=\frac{p+l-\nu}{p(p+2l+1)}
\end{align*}
Fissando $a_0 = 1$, avremo allora che il coefficiente $p$-esimo è dato da:
\begin{align*}
a_p = \frac{a_{p}}{a_{p-1}} \frac{a_{p-1}}{a_{p-2}}\dots \frac{a_1}{a_0}1
\end{align*}
Sappiamo che $a_p \neq 0$ per $p$ che vanno da $0$ a $n' = \nu -l-1 = n - l -1$. L'espressione per $v_l(x)$ è quindi data espandendo la serie:
\begin{align*}
v_l(x) = \sum_{p=0}^{n'} a_p x^p
\end{align*}
Sostituendola nell'espressione per la soluzione radiale $\chi_{nl}(x)$ otteniamo:
\begin{align*}
\chi_{nl}(x) =\ x^{l+1} e^{-x/2} \left[
\sum_{p=0}^{n' = n - l -1} \frac{\hlc{Yellow}{p+l-n}}{\hlc{SkyBlue}{p}\hlc{ForestGreen}{(p+2l+1)}}
\frac{\hlc{Yellow}{(p+l-n-1)}}{\hlc{SkyBlue}{(p-1)}\hlc{ForestGreen}{(p-1+2l+1)}}\dots\frac{\hlc{Yellow}{(1+l-n)}}{\hlc{SkyBlue}{1}\hlc{ForestGreen}{(1+2l+1)}} x^p
\right]
\end{align*}
Possiamo ricondurci ad una forma più familiare scrivendo:
\begin{align*}
\hlc{Yellow}{(-1)^p \frac{(n-l-1)!}{(n-l-1-p)!}} &=
(p+l-n)(p+l-n-1)\dots(1+l-n)\\
\hlc{SkyBlue}{\frac{1}{p!}} &= \frac{1}{p}\frac{1}{p-1}\dots \frac{1}{1}\\
\hlc{ForestGreen}{\frac{(2l+1)!}{(p+2l+1)!}} &=\frac{1}{(p+2l+1)}\frac{1}{(p-1+2l+1)}\dots\frac{1}{(1+2l+1)}
\end{align*}
Dove il fattore alternante $(-1)^p$ presente nel termine evidenziato in giallo è una scelta convenzionale di fase, che permette di riscrivere le autofunzioni $\chi_{nl}(x)$:
\begin{align}
\chi_{nl} = \sum_{p=0}^{n-l-1} (-1)^p \frac{(n-l-1)!}{(n-l-1-p)!}\frac{(2l+1)!}{(p+2l+1)!}\frac{1}{p!} x^p
\label{eqn:chi_sum}
\end{align}
utilizzando i cosiddetti \textbf{polinomi di Laguerre generalizzati}, che hanno la forma\footnote{\url{mathworld.wolfram.com/AssociatedLaguerrePolynomial.html}}:
\begin{align}
L_j^k (x) =\ \sum_{p=0}^j (-1)^p \frac{(j+k)!}{(j-p)!(k+p)!p!}x^p
\label{eqn:laguerre}
\end{align}
Notiamo che il numeratore di (\ref{eqn:chi_sum}) non dipende da $p$, ed è fissato una volta scelti $n$ e $l$. Portandolo fuori dalla sommatoria possiamo considerarlo come un fattore moltiplicativo costante, che sarà fissato dalla normalizzazione. Così facendo, confrontando (\ref{eqn:chi_sum}) con (\ref{eqn:laguerre}), ponendo $j=n-l-1$ e $k = 2l+1$ nel polinomio di Laguerre otteniamo (a meno del fattore moltiplicativo $A$ al numeratore) $\chi_{nl}$:
\begin{align*}
\chi_{nl}(x) = A x^{l+1} e^{-\frac{x}{2}} L^{2l+1}_{n-l-1}(x)
\end{align*} 
Sappiamo dalla precedente sezione che $H, \vec{L}, L_3$ formano un ICOC, e perciò esistono delle autofunzioni comuni per tali operatori, che hanno la forma:
\begin{align*}
\psi_{nlm}(r,\theta,\varphi) = h_{nlm}(r) Y^m_l(\theta,\varphi)
\end{align*}
dove $h=\chi/r$.\\
Ripercorrendo a ritroso i vari cambi di variabile effettuati si ha che:
\begin{align*}
x = 2kr \qquad k = \sqrt{-\frac{2m}{\hbar^2}\mathcal{E}_n} \qquad \mathcal{E}_n = -\frac{me^4}{2\hbar^2}\frac{1}{n^2}\qquad a =\frac{\hbar^2}{me^2}
\end{align*}
da cui otteniamo:
\begin{align*}
x = \frac{2r}{na}
\end{align*}
Possiamo allora costruire l'espressione per $\psi(r,\theta,\varphi)$:
\begin{align*}
\psi(r,\theta,\varphi) &= B\frac{1}{r}\left(\frac{2r}{na}\right)^{l+1}\exp\left(-\frac{r}{na}\right)L^{2l+1}_{n-l-1}\left(\frac{2r}{na}\right)Y^m_l(\theta,\varphi)=\\
&\underset{(a)}{=}C\left(\frac{2r}{na}\right)^l\exp\left(-\frac{r}{na}\right)L^{2l+1}_{n-l-1}\left(\frac{2r}{na}\right)Y^m_l(\theta,\varphi)
\end{align*}
dove in (a) abbiamo semplificato l'$r$ al denominatore inglobando il fattore rimanente nella costante $C$, il cui valore è fissato dalla normalizzazione\footnote{CFR pag. 154 \cite{griffiths}.}:
\begin{align}
\psi_{nlm} (r,\theta,\varphi) = 
\sqrt{\left(\frac{2}{na}\right)^3 \frac{(n-l-1)!}{2n(n+l)!}}\left( \frac{2r}{na}\right)^l\exp\left(-\frac{r}{na}\right) L^{2l+1}_{n-l-1}\left(\frac{2r}{na}\right) Y^m_l(\theta,\varphi)
\label{eqn:autostati_idrogeno_normalizzati}
\end{align}

In particolare per $n=l+1$ (il massimo momento angolare $l$ per $n$ fissato), poiché il relativo polinomio di Laguerre è costante, $L_0^{2l+1}(x)=q$, si ha che la componente radiale di $\psi_{nlm}$ è proporzionale a:
\begin{align*}
h_{nl}(r)\propto
\left(\frac{2r}{na}\right)^{\bm{l}} \exp\left({-\frac{r}{na}}\right) = \left(\frac{2r}{na}\right)^{\bm{n-1}} \exp\left(-\frac{r}{na}\right)
\end{align*}

Quindi la probabilità di trovare la particella tra $r$ e $r+dr$ è proporzionale al modulo quadro (con un $r^2$ dovuto alla misura in coordinate sferiche):
\begin{align*}
r^2|h_{nl}(r)|^2 \propto
r^2 \left(\frac{2r}{na}\right)^{2(n-1)}\exp\left(-\frac{2r}{na}\right) =\left(\frac{na}{2}\right)^2 \left(\frac{2r}{na}\right)^{2n} \exp\left(-\frac{2r}{na}\right)
\end{align*}
Cerchiamone il massimo. Con la sostituzione $y=2r/(na)$, imponendo l'annullamento della derivata prima giungiamo a:
\begin{align*}
\frac{d}{dy}\left( y^{2n} e^{-y} \right) = \left(
2n y^{2n-1} - y^{2n}
\right) e^{-y} \overset{!}{=} 0 \Rightarrow  y=2n \Rightarrow r_n = \frac{na}{2}y = n^2 a
\end{align*}
Abbiamo allora ritrovato in $r_n$ il \textbf{raggio di Bohr} dell'orbita $n$-esima.\\

\textbf{Riepilogando}: risolvendo l'equazione di Schr\"odinger stazionaria per gli stati legati dell'atomo di idrogeno ritroviamo le energie di Bohr, e nei casi in cui $l$ è massimo, anche che la probabilità della posizione dell'elettrone si addensa attorno ai corrispondenti raggi di Bohr.\\
La dipendenza angolare dovuta alle $Y^m_l(\theta,\varphi)$ porta infine ad ottenere i \q{disegnini stravaganti} dei vari orbitali.

\begin{comment}
\subsection{Esercizio}
Si consideri una particella quantistica priva di spin di massa $m$ e carica $e$ in un piano in presenza di un campo magneitco uniforme e costante $\vec{B}$ perpendicolare al piano. Si denoti con $\vec{A}$ il potenziale vettore osservato a $\vec{B}$ e si ponga $c=1$.
\begin{enumerate}
\item Si scriva l'Hamiltoniana quantistica del sistema
\item Si determini in funzione di $\vec{X}=\{X,Y\}$ e $\vec{P}= \{P_x, P_y\}$ la velocità $\vec{V}=\{V_x, V_y\}$ e si calcoli il commutatore di $V_x$ con $V_y$ in rappresentazione $\{\vec{x}\}$.
\item Si determini lo spettro $\sigma(H)$ di $H$.
\item Definite le coordinate $X_0=\frac{m}{eB}V_y + X$ e $Y_0 = -\frac{m}{eB}V_x + Y$ (analogo quantistico del centro dell'orbita di ciclotrone classica), si determini l'algebra dei commutatori di $\vec{X}_0$ e $\vec{V}$
\item Si determini un ICOC per il sistema e, utilizzando le $\vec{X}_0$, la degenerazione di $\sigma(H)$.
\item Supponendo ora che la particella sia confinata (pareti impenetrabili) in un quadrato di lato $L$ e scegliendo
$A_x = -B_y$ e $A_y=0$ per il potenziale vettore, si stimi la degenerazione di $\sigma(H)$ in questo caso.
\end{enumerate}
\end{comment}

\end{document}

