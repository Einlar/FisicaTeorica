%%PACKAGES
\usepackage[usenames, dvipsnames, table]{xcolor}
\usepackage[utf8]{inputenc}
\usepackage[T1]{fontenc}
\usepackage{lmodern}
\usepackage{amsmath}
\usepackage{amsthm}
\usepackage{amsfonts}
\usepackage{comment}
\usepackage{wrapfig}
\usepackage{booktabs}
\usepackage{braket}
\usepackage{pgf,tikz}
\usepackage{mathrsfs}
\usetikzlibrary{arrows}
\usepackage{subfigure}
\usepackage{xspace}
\usepackage{gnuplottex}
\usepackage{epstopdf}
\usepackage{marginnote}
\usepackage{float}
\usetikzlibrary{tikzmark}
\usepackage{graphicx}
\usepackage{cancel}
\usepackage{bm}
\usepackage{mathtools}
\usepackage{hyperref}
\usepackage{ragged2e}
\usepackage[stable]{footmisc}
\usepackage{enumerate}
\usepackage{mathdots}
\usepackage[framemethod=tikz]{mdframed}
\PassOptionsToPackage{table}{xcolor}
\usepackage{soul}
\usepackage{enumerate}
\usepackage{mathdots}
\usepackage[framemethod=tikz]{mdframed} %Added 16/10
\usepackage[italian]{babel} %Added 16/10
\usepackage{amssymb} %Added
\usepackage{enumitem}
\usepackage{array}


%%BOOKTAB
\setlength{\aboverulesep}{0pt}
\setlength{\belowrulesep}{0pt}
\setlength{\extrarowheight}{.75ex}
\setlength\parindent{0pt} %Rimuove indentazione


%%GEOMETRIA
\usepackage[a4paper]{geometry}
 \newgeometry{inner=20mm,
            outer=49mm,% = marginparsep + marginparwidth 
                       %   + 5mm (between marginpar and page border)
            top=20mm,
            bottom=25mm,
            marginparsep=6mm,
            marginparwidth=30mm}
\makeatletter
\renewcommand{\@marginparreset}{%
  \reset@font\small
  \raggedright
  \slshape
  \@setminipage
}
\makeatother
 

%%COMANDI
\newcommand{\q}[1]{``#1''}
\newcommand{\lamb}[2]{\Lambda^{#1}_{\>{#2}}}
\newcommand{\norm}[1]{\left\lVert#1\right\rVert}
\newcommand{\hs}{\mathcal{H}}
\newcommand{\minus}{\scalebox{0.75}[1.0]{$-$}}
\newcommand{\hlc}[2]{%
  \colorbox{#1!50}{$\displaystyle#2$}}
\newcommand{\bb}[1]{\mathbb{#1}}
\newcommand{\op}[1]{\operatorname{#1}}
\renewcommand{\figurename}{Fig.}
\newcommand{\dom}[1]{D#1}
\newcommand{\avg}[1]{\left\langle{#1}\right\rangle}
\newcommand{\NN}{\mathbb N}
\newcommand{\RR}{\mathbb R}
\newcommand{\CC}{\mathbb C}
\newcommand{\mS}{\mathcal S}
\newcommand{\de}{d}
\newcommand{\abs}[1]{\left|#1\right|}

\newcommand{\lesson}[2]{\marginpar{(Lezione #1 del #2)}}
\DeclareRobustCommand{\MQ}{{\small\textsc{MQ}}\xspace}
\DeclareRobustCommand{\MC}{{\small\textsc{MC}}\xspace}
%Prima era \small\textsc{MQ}\xspace

%%TESTATINE
\usepackage{fancyhdr}
\pagestyle{fancy}
\fancyhead{} % clear all header fields
\renewcommand{\headrulewidth}{0pt} % no line in header area
\fancyfoot{} % clear all footer fields
%\fancyfoot[R]{A.A. 2018/19} % other info in "inner" position of footer line
\cfoot{\thepage}


%%AMBIENTI
\theoremstyle{plain}
\newtheorem{thm}{Teorema}[section]
\newtheorem{lem}{Lemma}[section]
\newtheorem{prop}{Proposizione}[section]
\newtheorem{axi}{Assioma}
\newtheorem{pst}{Postulato}

\theoremstyle{definition}
\newtheorem{dfn}{Definizione}

\theoremstyle{remark}
\newtheorem{oss}{Osservazione}
\newtheorem{es}{Esempio}
\newtheorem{ex}{Esercizio}

%Spiegazioni/verifiche
\newenvironment{expl}{\begin{mdframed}[hidealllines=true,backgroundcolor=green!20,innerleftmargin=3pt,innerrightmargin=3pt,leftmargin=-3pt,rightmargin=-3pt]}{\end{mdframed}} %Box di colore verde

\newenvironment{appr}{\begin{mdframed}[hidealllines=true,backgroundcolor=blue!10,innerleftmargin=3pt,innerrightmargin=3pt,leftmargin=-3pt,rightmargin=-3pt]}{\end{mdframed}} %Approfondimenti matematici (box di colore blu)

%%Domande di Marchetti
\newtheorem{question}{Domanda}


%%OPERATORI
\DeclareMathOperator{\sech}{sech}
\DeclareMathOperator{\csch}{csch}
\DeclareMathOperator{\arcsec}{arcsec}
\DeclareMathOperator{\arccot}{arcCot}
\DeclareMathOperator{\arccsc}{arcCsc}
\DeclareMathOperator{\arccosh}{arcCosh}
\DeclareMathOperator{\arcsinh}{arcsinh}
\DeclareMathOperator{\arctanh}{arctanh}
\DeclareMathOperator{\arcsech}{arcsech}
\DeclareMathOperator{\arccsch}{arcCsch}
\DeclareMathOperator{\arccoth}{arcCoth} 


