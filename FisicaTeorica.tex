\documentclass[12pt]{report}
\usepackage{subfiles} %Struttura modulare
\usepackage{makeidx} %Indice analitico
\usepackage[ddmmyyyy,hhmmss]{datetime} %Data di compilazione
\makeindex
\usepackage[usenames, dvipsnames, table]{xcolor}
\usepackage[utf8]{inputenc}
\usepackage[T1]{fontenc}
\usepackage{lmodern}
\usepackage{amsmath}
\usepackage{amsfonts}
\usepackage{comment}
\usepackage{wrapfig}
\usepackage{booktabs}
\usepackage{braket}
\usepackage{tikz}
\usepackage{gnuplottex}
\usepackage{epstopdf}
\usepackage{marginnote}
\usepackage{float}
\usetikzlibrary{tikzmark}
\usepackage{graphicx}
\usepackage{cancel}
\usepackage{bm}
\usepackage{mathtools}
\usepackage{hyperref}
\usepackage{ragged2e}
\usepackage[stable]{footmisc}
\usepackage{enumerate}
\usepackage{mathdots}
\usepackage[framemethod=tikz]{mdframed}
\PassOptionsToPackage{table}{xcolor}
\usepackage{soul}
\usepackage{enumerate}
\usepackage{mathdots}
\usepackage[framemethod=tikz]{mdframed} %Added 16/10
\usepackage[italian]{babel} %Added 16/10
\usepackage{amssymb} %Added

%%BOOKTAB
\setlength{\aboverulesep}{0pt}
\setlength{\belowrulesep}{0pt}
\setlength{\extrarowheight}{.75ex}
\setlength\parindent{0pt} %Rimuove indentazione


%%GEOMETRIA
\usepackage[a4paper]{geometry}
 \newgeometry{inner=20mm,
            outer=49mm,% = marginparsep + marginparwidth 
                       %   + 5mm (between marginpar and page border)
            top=20mm,
            bottom=25mm,
            marginparsep=6mm,
            marginparwidth=30mm}

\makeatletter
\renewcommand{\@marginparreset}{%
  \reset@font\small
  \raggedright
  \slshape
  \@setminipage
}
\makeatother
 

%%COMANDI
\newcommand{\q}[1]{``#1''}
\newcommand{\lamb}[2]{\Lambda^{#1}_{\>{#2}}}
\newcommand{\norm}[1]{\left\lVert#1\right\rVert}
\newcommand{\hs}{\mathcal{H}}
\newcommand{\minus}{\scalebox{0.75}[1.0]{$-$}}
\newcommand{\hlc}[2]{%
  \colorbox{#1!50}{$\displaystyle#2$}}
\newcommand{\bb}[1]{\mathbb{#1}}
\newcommand{\op}[1]{\operatorname{#1}}
\renewcommand{\figurename}{Fig.}

\usepackage{fancyhdr}
\pagestyle{fancy}
\fancyhead{} % clear all header fields
\renewcommand{\headrulewidth}{0pt} % no line in header area
\fancyfoot{} % clear all footer fields
\fancyfoot[R]{Francesco Manzali, 2018-19} % other info in "inner" position of footer line
\cfoot{\thepage}

%%AMBIENTI
\newtheorem{thm}{Teorema}[section]
\newtheorem{dfn}{Definizione}
\newtheorem{oss}{Osservazione}
\newtheorem{es}{Esempio}
\newtheorem{axi}{Assioma}
%%Domande di Marchetti
\newtheorem{question}{Domanda}


%%OPERATORI
\DeclareMathOperator{\sech}{sech}
\DeclareMathOperator{\csch}{csch}
\DeclareMathOperator{\arcsec}{arcsec}
\DeclareMathOperator{\arccot}{arcCot}
\DeclareMathOperator{\arccsc}{arcCsc}
\DeclareMathOperator{\arccosh}{arcCosh}
\DeclareMathOperator{\arcsinh}{arcsinh}
\DeclareMathOperator{\arctanh}{arctanh}
\DeclareMathOperator{\arcsech}{arcsech}
\DeclareMathOperator{\arccsch}{arcCsch}
\DeclareMathOperator{\arccoth}{arcCoth} 




\begin{document}
\newgeometry{total={170mm,257mm}, left=20mm, top=20mm}
%Impostazioni booktab (Rimuove quello spazio odioso tra righe)
\setlength{\aboverulesep}{0pt}
\setlength{\belowrulesep}{0pt}
\setlength{\extrarowheight}{.75ex}
\begin{center}
		\line (1,0){350} \\
		\textsc{\normalsize Trascrizione degli appunti delle lezioni di}\\
		[0.25in]
		\huge{\bfseries Istituzioni di Fisica teorica}\\
		[2mm]
		\textsc{\normalsize Tenute dal Prof. \textit{Pieralberto Marchetti}}
		\vspace{-0.5em}\\
		\textsc{\normalsize Presso l'università di Padova}\\
		\vspace{-1em}
		\line (1,0){350} \\
		[0.2cm]
		\textsc{\normalsize Anno accademico 2018-2019}\\ 
        {\scriptsize Compilato il \today}
	\end{center}

\newgeometry{inner=20mm,
            outer=49mm,% = marginparsep + marginparwidth 
                       %   + 5mm (between marginpar and page border)
            top=20mm,
            bottom=25mm,
            marginparsep=6mm,
            marginparwidth=30mm}

\makeatletter
\renewcommand{\@marginparreset}{%
  \reset@font\small
  \raggedright
  \slshape
  \@setminipage
}
\makeatother

\tableofcontents 
\clearpage
\chapter*{Introduzione}
Buonsalve!\\
In questo documento ho cercato di riordinare gli appunti del corso di Istituzioni di Fisica Teorica tenuto dal professor Pieralberto Marchetti  presso il Dipartimento di Fisica dell'Università di Padova nel corso del primo semestre del 2018-19.\\
Potrebbero esserci errori di formattazione, parentesi saltate, o peggio, coefficienti/esponenti/segni errati in giro (ma non dovrebbero essere tanti). Se ne sgamate qualcuno, fatemi sapere. Ditemi anche (se avete tempo e non vi scoccia) se ci sono passaggi non chiari.\\
Prima di iniziare, ultimo disclaimer (che dovrebbe essere scontato dato che non ho una laurea): questi appunti non sono da intendere come sostituzione delle lezioni, o di altre dispense già presenti.\\
Buon viaggio! :)

\begin{flushright}
\textit{Francesco Manzali}, 23/10/2018
\end{flushright}
\section*{Aggiornamenti}
\begin{table}[hb]
    \centering
    \begin{tabular}{|cccc|}\toprule
        Data & Aggiunte & Errata corrige & Commenti\\\midrule
        \textbf{23/10/2018} & Prima pubblicazione & & \\\bottomrule
    \end{tabular}
    \caption{Cronologia di modifiche/aggiornamenti agli appunti}
    \label{updates}
\end{table}

\clearpage

\subfile{Lezione1.tex}
\subfile{Lezione2.tex}
\subfile{Lezione3.tex}

\clearpage
\begin{thebibliography}{9}
\bibitem{spazi_hilbert} 
Luca Martucci, Roberto Volpato. 
\textit{Appunti di Istituzioni di Metodi Matematici per la Fisica}. 
Unipd, 10 maggio 2018.
 
\bibitem{mq} 
Kenichi Konishi, Giampiero Paffuti
\textit{Meccanica quantistica: nuova introduzione}.
Pisa university press, ed. 2012
\end{thebibliography}

\printindex

\end{document}