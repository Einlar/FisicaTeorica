\documentclass[12pt]{article}

%%PACKAGES
\usepackage[usenames, dvipsnames, table]{xcolor}
\usepackage[utf8]{inputenc}
\usepackage[T1]{fontenc}
\usepackage{lmodern}
\usepackage{amsmath}
\usepackage{amsthm}
\usepackage{amsfonts}
\usepackage{comment}
\usepackage{wrapfig}
\usepackage{booktabs}
\usepackage{braket}
\usepackage{pgf,tikz}
\usepackage{mathrsfs}
\usetikzlibrary{arrows}
\usepackage{subfigure}
\usepackage{xspace}
\usepackage{gnuplottex}
\usepackage{epstopdf}
\usepackage{marginnote}
\usepackage{float}
\usetikzlibrary{tikzmark}
\usepackage{graphicx}
\usepackage{cancel}
\usepackage{bm}
\usepackage{mathtools}
\usepackage{hyperref}
\usepackage{ragged2e}
\usepackage[stable]{footmisc}
\usepackage{enumerate}
\usepackage{mathdots}
\usepackage[framemethod=tikz]{mdframed}
\PassOptionsToPackage{table}{xcolor}
\usepackage{soul}
\usepackage{enumerate}
\usepackage{mathdots}
\usepackage[framemethod=tikz]{mdframed} %Added 16/10
\usepackage[italian]{babel} %Added 16/10
\usepackage{amssymb} %Added
\usepackage{enumitem}
\usepackage{array}


%%BOOKTAB
\setlength{\aboverulesep}{0pt}
\setlength{\belowrulesep}{0pt}
\setlength{\extrarowheight}{.75ex}
\setlength\parindent{0pt} %Rimuove indentazione


%%GEOMETRIA
\usepackage[a4paper]{geometry}
 \newgeometry{inner=20mm,
            outer=49mm,% = marginparsep + marginparwidth 
                       %   + 5mm (between marginpar and page border)
            top=20mm,
            bottom=25mm,
            marginparsep=6mm,
            marginparwidth=30mm}
\makeatletter
\renewcommand{\@marginparreset}{%
  \reset@font\small
  \raggedright
  \slshape
  \@setminipage
}
\makeatother
 

%%COMANDI
\newcommand{\q}[1]{``#1''}
\newcommand{\lamb}[2]{\Lambda^{#1}_{\>{#2}}}
\newcommand{\norm}[1]{\left\lVert#1\right\rVert}
\newcommand{\hs}{\mathcal{H}}
\newcommand{\minus}{\scalebox{0.75}[1.0]{$-$}}
\newcommand{\hlc}[2]{%
  \colorbox{#1!50}{$\displaystyle#2$}}
\newcommand{\bb}[1]{\mathbb{#1}}
\newcommand{\op}[1]{\operatorname{#1}}
\renewcommand{\figurename}{Fig.}
\newcommand{\dom}[1]{D#1}
\newcommand{\avg}[1]{\left\langle{#1}\right\rangle}
\newcommand{\NN}{\mathbb N}
\newcommand{\RR}{\mathbb R}
\newcommand{\CC}{\mathbb C}
\newcommand{\mS}{\mathcal S}
\newcommand{\de}{d}
\newcommand{\abs}[1]{\left|#1\right|}

\newcommand{\lesson}[2]{\marginpar{(Lezione #1 del #2)}}
\DeclareRobustCommand{\MQ}{{\small\textsc{MQ}}\xspace}
\DeclareRobustCommand{\MC}{{\small\textsc{MC}}\xspace}
%Prima era \small\textsc{MQ}\xspace

%%TESTATINE
\usepackage{fancyhdr}
\pagestyle{fancy}
\fancyhead{} % clear all header fields
\renewcommand{\headrulewidth}{0pt} % no line in header area
\fancyfoot{} % clear all footer fields
%\fancyfoot[R]{A.A. 2018/19} % other info in "inner" position of footer line
\cfoot{\thepage}


%%AMBIENTI
\theoremstyle{plain}
\newtheorem{thm}{Teorema}[section]
\newtheorem{lem}{Lemma}[section]
\newtheorem{prop}{Proposizione}[section]
\newtheorem{axi}{Assioma}
\newtheorem{pst}{Postulato}

\theoremstyle{definition}
\newtheorem{dfn}{Definizione}

\theoremstyle{remark}
\newtheorem{oss}{Osservazione}
\newtheorem{es}{Esempio}
\newtheorem{ex}{Esercizio}

%Spiegazioni/verifiche
\newenvironment{expl}{\begin{mdframed}[hidealllines=true,backgroundcolor=green!20,innerleftmargin=3pt,innerrightmargin=3pt,leftmargin=-3pt,rightmargin=-3pt]}{\end{mdframed}} %Box di colore verde

\newenvironment{appr}{\begin{mdframed}[hidealllines=true,backgroundcolor=blue!10,innerleftmargin=3pt,innerrightmargin=3pt,leftmargin=-3pt,rightmargin=-3pt]}{\end{mdframed}} %Approfondimenti matematici (box di colore blu)

%%Domande di Marchetti
\newtheorem{question}{Domanda}


%%OPERATORI
\DeclareMathOperator{\sech}{sech}
\DeclareMathOperator{\csch}{csch}
\DeclareMathOperator{\arcsec}{arcsec}
\DeclareMathOperator{\arccot}{arcCot}
\DeclareMathOperator{\arccsc}{arcCsc}
\DeclareMathOperator{\arccosh}{arcCosh}
\DeclareMathOperator{\arcsinh}{arcsinh}
\DeclareMathOperator{\arctanh}{arctanh}
\DeclareMathOperator{\arcsech}{arcsech}
\DeclareMathOperator{\arccsch}{arcCsch}
\DeclareMathOperator{\arccoth}{arcCoth} 




\begin{document}

\section{Lezione 5}
\vspace{-1em}
\begin{center}
    \small{(24/10/2018)}
\end{center}
\section{Riassuntino}
\begin{equation}
S(E,V,N) = Nk_B \ln\left[V\left(\frac{4\pi m}{3h^2}\right)^\frac{3}{2}\left(\frac{E}{N}\right)^\frac{3}{2}\right]+\frac{3}{2}k_B N
\label{eqn:entropianonestensiva}
\end{equation}

\begin{equation}
S=k\ln\left[\Gamma(E)\right] = k\ln[\Sigma(E)]
\end{equation}
Ma tale formula non soddisfa l'omogeneità $S(\lambda N, \lambda V, \lambda E)=\lambda S(N,V,E)$.\\

\section{24/10}
Si può notare ciò nel \textbf{paradosso di Gibbs}. Consideriamo un sistema a "due compartimenti", con $N_1,V_1$ in $1$ e $N_2,V_2$ in $2$, e $N_1+N_2=N$, $V_1+V_2=V$, pieni dello stesso gas perfetto. La parete tra i due impedisce lo scambio di particelle.\\
All'equilibrio, vogliamo che:
\[
\frac{N_1}{V_1} = \frac{N_2}{V_2}; \quad \frac{E_1}{N_1}=\frac{E_2}{N_2}
\]
(essendo $E_i = \frac{3}{2}N_i k_B T$).\\
Se ora rimuoviamo la paratia, ci aspettiamo che "non succeda nulla": effettivamente il sistema non "cambia" a livello fisico. In particolare, ci aspettiamo che anche l'entropia del sistema rimanga la stessa. Tuttavia, se calcoliamo:
\begin{align*}
S_1 &= k N_1 \ln\left[V_1 \left(\frac{4\pi m}{3h^2}\right)^\frac{3}{2}\left(\frac{E_1}{N_1}\right)^\frac{3}{2}\right]+\frac{3}{2}kN_1\\
S_2 &= k N_2 \ln\left[V_2 \left(\frac{4\pi m}{3h^2}\right)^\frac{3}{2}\left(\frac{E_2}{N_2}\right)^\frac{3}{2}\right]+\frac{3}{2}kN_2
\end{align*}
Tuttavia, l'entropia totale del sistema:
\[
S=kN\ln\left[(V_1+V_2)\left(\frac{4\pi m}{3h^2}\right)^\frac{3}{2}\left(\frac{E_1+E_2}{N_1+N_2}\right)^\frac{3}{2}\right]
\]
differisce da $S_1 + S_2$ per un termine:
\[
\Delta S = kN_1\ln\left(\frac{V_1+V_2}{V_1}\right)+kN_2\ln\left(\frac{V_1+V_2}{V_2}\right)
\]
Tale fattore andrebbe bene se i gas contenuti in $1$ e in $2$ fossero diversi (e allora il mescolamento sarebbe effettivamente irreversibile), ma di certo non in questo caso, in cui sono effettivamente uguali.\\
Del resto, tale calcolo porterebbe a risultati assurdi, per cui l'entropia di un sistema dipenderebbe dal "numero di scompartimenti" che sono stati uniti nella preparazione del sistema!\\
Il problema si risolve rendendo estensiva la (\ref{eqn:entropianonestensiva}), ossia apportando la modifica che "rende le particelle dello stesso gas indistinguibili":
\[
S(E,V,N) = Nk_B \ln\left[\frac{V}{\bm{N!}}\left(\frac{4\pi m}{3h^2}\right)^\frac{3}{2}\left(\frac{E}{N}\right)^\frac{3}{2}\right]+\frac{3}{2}k_B N
\]

\subsection{Esercizio}
Abbiamo visto che il valore di una grandezza macroscopica è dato "mediando" sullo spazio delle fasi:
\[
\langle b\rangle = \frac{\displaystyle \int_{E\leq H\leq E+\Delta}dq\,dp\,b(q,p)}{\displaystyle \int_{E\leq H\leq E+\Delta}dq\,dp}
\]

Consideriamo un sistema in due "scompartimenti" %Inserisci disegno
 che però non sono divisi da una parete (possono scambiarsi particelle), il primo con $V_1$, $N_1$, e il secondo con $V_2$, $N_2$.\\
Consideriamo noti il numero totale di particelle $N=N_1+N_2$, il volume $V=V_1+V_2$ e l'energia $E$ del sistema. Possiamo calcolare la probabilità che $N_1$ particelle si trovino in $1$? Possiamo cioè calcolare le probabilità delle varie possibili ripartizioni di particelle tra i due volumi?
\[
\Sigma(E)=\frac{1}{N! h^{3N}} (V_1+V_2)^N \Omega_{3N}(E)
\]
Qual è il volume associato a $N_1$ particelle a sinistra e $N-N_1$ a destra?\\
Lavoriamo inizialmente considerando le particelle distinguibili, e poi compensiamo l'ambiguità dividendo per un opportuno fattore.\\
Il volume degli stati in cui esattamente $N_1$ particelle sono in $1$ è dato da:
\[
\Sigma_{N_1}(E) =
\frac{1}{N!\,h^{3N}} {{N}\choose{N_1}} V_1^{N_1}V_2^{N-N_1} \Omega_{3N}(E) =\ \frac{1}{N!\,h^{3N}} \frac{N!}{N_1! (N-N_1)!}V_1^{N_1} V_2^{N-N_1}\Omega_{3N}(E)
\]
(stiamo considerando tutte le combinazioni possibili per scegliere le $N_1$ particelle che stanno in $1$ a partire da un gruppo di $N$ particelle totali (coefficiente binomiale. Ad ognuna di queste combinazioni è associato un volume $V_1^{N_1} V_2^{N_2}$ spaziato dalle $q$, e un $\Omega_{3N}(E)$ spaziato dalle $p$. Dopodiché si adimensionalizza con $h^{3N}$ e si divide per le permutazioni $N!$ per l'indistinbuibilità delle particelle).\\
Dividendo per il volume di \textit{tutti} gli stati, pari a $(V_1+V_2)^N \Omega_{3N}(E)$, otteniamo la probabilità:
\[
P(N_1)=\frac{N!}{N_1!(N-N_1)!}\frac{V_1^{N_1}V_2^{N-N_1}}{(V_1+V_2)^N}
\]
Passiamo al logaritmo per cercare l'$N_1$ che massimizza la probabilità:
\[
\ln P(N_1)=N\ln N-N-N_1\ln N_1 + N_1 -(N-N_1)\ln(N-N_1)+N-N_1 + N_1\ln V+(N-N_1)\ln V_2 -N\ln V
\]
Imponendo che la derivata si annulli (per il massimo):
\[
\frac{\partial}{\partial N_1}\ln P(N_1) = 0
\]
Otteniamo la condizione:
\[
\frac{\bar{N}_1}{V_1} = \frac{N-\bar{N}_1}{V_2}
\]
(fisicamente: le particelle si distribuiscono omogeneamente in entrambe le sottosezioni)\\
e calcolando la derivata seconda in $\bar{N}_1$ ci accorgiamo che è effettivamente un massimo:
\[
\frac{\partial^2 \ln P(N_1)}{\partial N_1^2}\big|_{\bar{N}_1} = -\frac{1}{\bar{N}_1}-\frac{1}{N-\bar{N}_1} < 0
\]

\[
\rho_m(q,p) = \begin{cases}
\frac{1}{\displaystyle \int_{E\leq H \leq E+\Delta} dq\,dp} & E\leq H(q,p) \leq E+\Delta\quad \Omega_{3N}(E)\\
0 & \text{altrimenti}
\end{cases}
\]

\section{Ensemble canonico}
Consideriamo un sistema in contatto con un serbatoio termico a temperatura $T$, con cui può scambiare energia. Sperimentalmente, ciò è più semplice da realizzare dell'ensemble microcanonico, per cui è necessario un completo isolamento termico. Qui, invece che fissare l'energia, stiamo fissando la \textit{temperatura}.\\
Introduciamo un potenziale termico noto come energia libera di Helmholtz:
\[
A=E-TS
\]
Consideriamo una transizione da uno stato $A$ a uno stato $B$. Sappiamo che:
\[
\int_A^B \frac{dQ}{T}\leq S_B-S_A
\]
(con l'uguaglianza che vale solo se la trasformazione è reversibile).\\
Se consideriamo $T$ costante, l'espressione diviene:
\[
\frac{Q}{T}\leq S_B-S_A
\]
Applicando allora il primo principio della termodinamica:
\[
Q = \Delta E + W= E_B-E_A + W \leq (S_B-S_A)T
\]
Otteniamo:
\[
\Delta E+W\leq T\Delta S \Rightarrow W \leq \Delta -(\Delta E-T\Delta S) = -\Delta A
\]
Ma se il sistema non fa lavoro sull'ambiente, $W=0$, allora:
\[
-\Delta A \geq 0\Rightarrow \Delta A\leq 0
\]
Da qui nasce l'appellativo di \textit{potenziale termodinamico}. Se la trasformazione è reversibile, $-\Delta A$ dà esattamente il lavoro, e a seguito di trasformazioni "libere" tende a diminuire (proprio come una particella carica positivamente tende ad andare verso i punti in cui il potenziale è minore).\\
Consideriamo $A$ funzione di $N,V,T$ (mentre si aveva $S(E,N,V)$). Il passaggio da $S$ ad $A$ è detto \textit{trasformazione di Legendre.}\\
Partendo dalla relazione:
\[
\frac{\partial S}{\partial E}(E,N,V)=\frac{1}{T} \Rightarrow E=E(N,V,T)
\]
e definiamo allora $A(N,V,T)$ sostituendo:
\[
A(N,V,T)=E(N,V,T)-TS(E(N,V,T),N,V)
\]
Derivando rispetto alla temperatura:
\[
\frac{\partial A}{\partial T}=\hlc{Yellow}{\frac{\partial E}{\partial T}(T,V,N)} - S(E(T,V,N), V,N)-\hlc{Yellow}{T}\underbrace{\hlc{Yellow}{\frac{\partial S}{\partial E}(E,V,N)}}_{1/T}\frac{\partial E}{\partial T}(T,V,N)
\]
dove i termini evidenziati si elidono. Si ottiene quindi:
\[
-\frac{\partial A}{\partial T} = S
\]

Derivando invece $A$ rispetto a $V$:
\[
\frac{\partial A}{\partial V}(T,V,N) = \bcancel{\frac{\partial E}{\partial V}(T,V,N)}-\bcancel{T\frac{\partial S}{\partial E}\frac{\partial E}{\partial V}}-T\frac{\partial S}{\partial V} = -T\frac{\partial S}{\partial V} = -p
\]
Infatti:
\[
dS= \frac{dE}{T}+\frac{p dV}{T} \Rightarrow \frac{\partial S}{\partial V}=\frac{p}{T}
\]

Ritorniamo a considerare un sistema $1$ a contatto con un serbatoio termico $2$. Globalmente, $1+2$ è un sistema isolato, a cui perciò possiamo applicare l'ensemble microcanonico. Agli effetti, possiamo vedere tale sistema come quello "diviso in due scompartimenti", solo che qui il II scompartimento è enormemente più grande.\\
Avevamo trovato:
\begin{equation}
\Gamma(E)=\sum_{i=0}^{E/\Delta}\Gamma_1(E_i)\Gamma_2(E-E_i); \quad \Delta \ll 1; \>E_i=i\Delta,\>i=0,1,\dots,\frac{E}{\Delta}
\label{eqn:gammaE_12}
\end{equation}
Il sistema $2$ in questa situazione è però fortemente dominante: vogliamo studiare le conseguenze di ciò. In particolare, qual è la probabilità $P_H$ di avere il sottosistema $1$ in uno stato microscopico con $E_i\leq H_1(q_1, p_1)\leq E_i+\Delta$? Notiamo che questa situazione è data da uno dei singoli termini della somma in (\ref{eqn:gammaE_12}). Dividendo allora per tutte le possibilità troviamo la probabilità cercata:
\[
P_H =\ \frac{\displaystyle \Gamma_1(E_i)\Gamma_2(E-E_i)}{\displaystyle \sum_{j=0}^{E/\Delta}\Gamma_1(E_j)\Gamma_2(E-E_j)
}
=
\frac{\displaystyle
\frac{\Gamma_1(E_i)\Gamma_2(E-E_i)}{\Gamma_2(E)}
}{
\displaystyle
\frac{\displaystyle \sum_{j=0}^{E/\Delta} \Gamma_1(E_j)\Gamma_2(E-E_j)}{\Gamma_2(E)}
}
\]
Dalla definizione di entropia nell'ensemble microcanonico:
\begin{align*}
\frac{\Gamma_2(E-E_i)}{\Gamma_2(E)}&=\exp\left(
\frac{S_2}{k}(E-E_i)-\frac{S_2}{k}(E)
\right) =\\
&\underset{(a)}{=} 
\exp\left(-\frac{1}{k}\frac{\partial S_2}{\partial E}\big|_E E_i -\frac{1}{2k}\frac{\partial^2 S_2(E)}{\partial E^2}(E_i)^2+\dots \right) =\\
&= \exp\left(-\frac{1}{k_B T}E_i - \frac{1}{2k_B}\frac{\partial^2 S_2(E)}{\partial E^2}(E_i)^2+\dots\right)
\end{align*}
dove in (a) si è fatta un'espansione in serie di\ Taylor.\\
Ma la derivata seconda:
\[
\frac{\partial^2 S_2(E)}{\partial E^2}=\frac{\partial}{\partial E}\frac{1}{T}=-\frac{1}{T^2}\frac{\partial T}{\partial E}; \quad d\frac{1}{T}=-\frac{1}{T}dT
\]
Essendo il serbatoio termico molto grande, $\frac{\partial T}{\partial E}$ deve essere $0$ (non posso cambiarne la temperatura, per quanta energia ci trasferisca). E in effetti:
\[
\frac{\partial E}{\partial T}=C_{V,2}
\]
(capacità termica del sistema $2$ a voluime costante).\\
(Notiamo che così facendo spariscono completamente le caratteristiche materiali del sistema $2$, e rimangono solo alcune proprietà "universali").\\
\[
P_H = \frac{\displaystyle \Gamma_1(E_1) \exp\left(-\frac{E_i}{k_B T}\right )}{\displaystyle \sum_{j=0}^{E/\Delta} \Gamma_1(E_j) \exp\left(-\frac{E_j}{k_B T}\right)}
\]

\end{document}

