\documentclass[12pt]{article}
\usepackage[usenames, dvipsnames, table]{xcolor}
\usepackage[utf8]{inputenc}
\usepackage[T1]{fontenc}
\usepackage{lmodern}
\usepackage{amsmath}
\usepackage{amsfonts}
\usepackage{comment}
\usepackage{wrapfig}
\usepackage{booktabs}
\usepackage{braket}
\usepackage{tikz}
\usepackage{gnuplottex}
\usepackage{epstopdf}
\usepackage{marginnote}
\usepackage{float}
\usetikzlibrary{tikzmark}
\usepackage{graphicx}
\usepackage{cancel}
\usepackage{bm}
\usepackage{mathtools}
\usepackage{hyperref}
\usepackage{ragged2e}
\usepackage[stable]{footmisc}
\DeclareMathOperator{\sech}{sech}
\DeclareMathOperator{\csch}{csch}
\DeclareMathOperator{\arcsec}{arcsec}
\DeclareMathOperator{\arccot}{arcCot}
\DeclareMathOperator{\arccsc}{arcCsc}
\DeclareMathOperator{\arccosh}{arcCosh}
\DeclareMathOperator{\arcsinh}{arcsinh}
\DeclareMathOperator{\arctanh}{arctanh}
\DeclareMathOperator{\arcsech}{arcsech}
\DeclareMathOperator{\arccsch}{arcCsch}
\DeclareMathOperator{\arccoth}{arcCoth} 
\usepackage{enumerate}
\usepackage{mathdots}

\renewcommand{\figurename}{Fig.}

\PassOptionsToPackage{table}{xcolor}

\usepackage{soul}

\newcommand{\hlc}[2]{%
  \colorbox{#1!50}{$\displaystyle#2$}}


\usepackage[a4paper,
            total={170mm,257mm},
 left=20mm,
 top=20mm]{geometry}

\newcommand{\q}[1]{``#1''}
\newcommand{\lamb}[2]{\Lambda^{#1}_{\>{#2}}}
\newcommand{\norm}[1]{\left\lVert#1\right\rVert}

\newcommand{\bb}[1]{\mathbb{#1}}
\newcommand{\op}[1]{\operatorname{#1}}
\newcommand{\sumn}[1]{\sum_{i=1}^{#1}}

\usepackage{fancyhdr}
\pagestyle{fancy}
\fancyhead{} % clear all header fields
\renewcommand{\headrulewidth}{0pt} % no line in header area
\fancyfoot{} % clear all footer fields
\fancyfoot[R]{Francesco Manzali, 2018-19} % other info in "inner" position of footer line

\newtheorem{thm}{Teorema}[section]
\newtheorem{dfn}{Definizione}
\newtheorem{oss}{Osservazione}

\begin{document}
%Impostazioni booktab (Rimuove quello spazio odioso tra righe)
\setlength{\aboverulesep}{0pt}
\setlength{\belowrulesep}{0pt}
\setlength{\extrarowheight}{.75ex}
\setlength\parindent{0pt} %Rimuove indentazione
\newgeometry{inner=20mm,
            outer=49mm,% = marginparsep + marginparwidth 
                       %   + 5mm (between marginpar and page border)
            top=20mm,
            bottom=25mm,
            marginparsep=6mm,
            marginparwidth=30mm}

\makeatletter
\renewcommand{\@marginparreset}{%
  \reset@font\small
  \raggedright
  \slshape
  \@setminipage
}
\makeatother

\begin{comment} %INTESTAZIONE PRIMA PAGINA
\begin{center}
		\line (1,0){350} \\
		\textsc{\normalsize Trascrizione degli appunti delle lezioni di}\\
		[0.25in]
		\huge{\bfseries Istituzioni di Fisica teorica}\\
		[2mm]
		\textsc{\normalsize Prof. Marchetti}\\
		\line (1,0){350} \\
		[0.5cm]
		\textsc{\normalsize Anno accademico 2018-2019}\\ 

	\end{center}
\end{comment}

\newgeometry{inner=20mm,
            outer=49mm,% = marginparsep + marginparwidth 
                       %   + 5mm (between marginpar and page border)
            top=20mm,
            bottom=25mm,
            marginparsep=6mm,
            marginparwidth=30mm}

\makeatletter
\renewcommand{\@marginparreset}{%
  \reset@font\small
  \raggedright
  \slshape
  \@setminipage
}
\makeatother



\section{Lezione 3}
\vspace{-1em}
\begin{center}
    \small{(10/10/2018)}
\end{center}
%Editare \mu al posto di \gamma
\textbf{Riepilogo}\\ %%Pag. 75-85
Consideriamo lo spazio delle fasi $\mu$, i cui punti sono dati da \((q,p)\), e suddividiamolo in celle di dimensioni piccole (volume $\omega_i$) rispetto alla scala del sistema, ma comunque grande rispetto alla scala delle particelle che contengono.\\
Contiamo quindi il numero $n_i$ di particelle in ogni cella. Uno stato microscopico è definito dalla collezione degli $n_i$ di tutte le celle: $\{n\}=\{n_i, \dots, n_M\}$. Se assegnamo un'unica energia $\mathcal{E}_i$ ad ogni celletta (che possiamo fare se le cellette sono ragionevolmente piccole), allora l'energia del sistema è data da:
\[
E = \sum_{i=1}^M n_i\mathcal{E}_i
\]
Tuttavia, poiché in fisica classica le particelle sono distinguibili (grazie alle loro traiettorie), possiamo "scambiare" particelle tra celle diverse, mantenendo gli stessi $n_i$ trovando così lo stesso \textit{stato macroscopico}, ma con una diversa \textit{configurazione microscopica}.\\
Se consideriamo tutti i punti sullo spazio delle fasi (ciascuno dei quali identifica una configurazione microscopica) che corrispondono allo stesso stato macroscopico, si ha che tale insieme avrà un certo volume $W(\{n\})$. Consideriamo questo volume come una misura dell'\textit{importanza} dello stato $\{n\}$. Se consideriamo il volume associato ad una sola configurazione microscopica corrispondente allo stato $\{n\}$ otteniamo:
\[
\int_{\omega_1} d^3 q_1 d^3p_1 \int_{\omega_1} d^3\_2 d^3p_2 \cdots \int_{\omega_1}d^3 q_{n_1}d^3p_{n_1} \int_{\omega_2}d^3 q_{n_1+1}d^3 p_{n_1+1} \int_{\omega_2}\cdots
\]
Con ognuno degli integrali che dà luogo ad un "volume" $\omega_i$. Tenendo conto di ciò, e del fattore combinatorio che "conta" tutti i possibili stati ottenuti scambiando particelle tra cellette, si giunge a:
\[
W\{n\} =\ 1/(h^{3N})\ \frac{N!}{n_1! n_2! \dots n_M!} \omega_1^{n_1} \omega_2^{n_2} \dots \omega_M^{n_m}
\]
Il fattore $h$ serve a rendere $W$ adimensionale, in modo da poter calcolare il logaritmo e massimizzare più agevolmente.\\
Teniamo poi conto dei vincoli dell'ensemble micronanonico (l'energia del sistema è costante, così come il suo numero di particelle):
\[
\sum_{i=1}^M n_i = N; \quad \sum_{i=1}^M n_i \mathcal{E}_i =\ E
\]
Con il metodo dei moltiplicatori di Lagrange:
\[
\ln W\{n\} -\alpha \left(\sum_{i=1}^M n_i - N \right) -\beta\left(\sum_{i=1}^M n_i \mathcal{E}_i - E\right)
\]
In approssimazione di Stirling: $\ln n! \approx n\ln n - n$ per $n\to \infty$, calcoliamo:
\[
N\ln N-N -\sum_{i=1}^M n_i \ln n_i + \sum_{i=1}^M n_i + \sum_{i=1}^M n_i \ln \omega_i -\alpha \left(\sum_{i=1}^M n_i - N\right ) -\beta\left(\sum_{i=1}^M n_i \mathcal{E}_i - E\right)
\]
Derivando rispetto a $n_k$:
\[
-\ln n_k +\ln \omega_k - \alpha - \beta\mathcal{E}_k =\ 0 \quad \forall k
\]
Esponenziando:
\[
\frac{n_k}{\omega_k} =\ e^{-\alpha -\beta\mathcal{E}_k} \Rightarrow n_k =\ \omega_k e^{-\alpha -\beta \mathcal{E}_k}
\]
Questo $n_k =\ \bar{n}_k$ è effettivamente un massimo. Esaminando il rapporto incrementale:
\[
\ln W\{\bar{n}+\Delta n\}-\ln W\{\bar{n}\} 0 -\frac{1}{2} \sum_{i=1}^M \frac{1}{\bar{n}_k} \Delta n_k^2 + \dots
\]
Notiamo ora che:
\begin{equation}
\bar{n}_k =\ \omega_k e^{-\alpha -\beta \mathcal{E}_k}
\label{eqn:Maxwell}
\end{equation}
è esattamente la distribuzione di Maxwell.\\
Maxwell la ricavò considerando un gas perfetto $N$, $V$, $m$, $T$, volendo trovare il numero di particelle contenute in un elemento $d^3 p\, d^3 q$ (piccolo rispetto alla scala del sistema, grande rispetto a quella microscopica), ossia $\int f_M(\vec{q}, \vec{p} d^3 p\, d^3 q =\ dN$. Trovò che:
\[
f_M(\vec{q}, \vec{p}) =\ \begin{cases}
0 &\ \vec{q}\notin V\\
\frac{N}{V} {(2\pi mk_B\ T)^{-3/2}} \exp{\left (-\frac{\vec{p}^2}{2mk_b T}\right )} & \vec{q} \in V
\end{cases}
\]

Notò anche che:
\[
E\ =\ \int_V d^3 q \int d^3 p f_M(\vec{q}, \vec{p}) \frac{\vec{p}^2}{2m} =\ \frac{3}{2} N k_B\ T
\]
Maxwell derivò tutto ciò partendo da considerazioni di simmetria. Tuttavia non risulta chiaro perché tale distribuzione debba ritrovarsi all'equilibrio di un gas perfetto. Ciò fu chiarito da Boltzmann.\\

Eravamo arrivati a:
\[
\bar{n}_k =\ \omega_k \exp(-\alpha -\beta\mathcal{E}_k)
\]
Come possiamo trovare $\alpha$ e $\beta$? Esaminiamo i vincoli:
\begin{align*}
N &=\ \sum_{i=1}^M \bar{n_i} =\ \sum_{i=1}^M \omega_i e^{-\alpha -\beta \mathcal{E}_i}\\
E &=\ \sum_{i=1}^M \underbrace{\omega_i}_{\Delta^3 q_i \Delta^3 p_j} e^{-\alpha -\beta\mathcal{E_i}} \overbrace{\mathcal{E}_i}^{\vec{p}_i^2/(2m)} = \int_V d^3 q \int_{-\infty}^{+\infty} d^3 p\, \exp \left (-\alpha -\beta \frac{\vec{p}^2}{2m} \right ) \frac{\vec{p}^2}{2m}
\end{align*}
Dove l'integrale si intende con $d^3 q $ e $d^3 p$ "macroscopicamente infinitesimi", ossia non puntiformi (contengono comunque molte particelle!).\\
\begin{align*}
N\ &=\ \underbrace{\int_V d^3 q}_{=V} \int d^3 p\, e^{-\alpha} \exp\left(-\frac{\beta}{2}\frac{\vec{p}^2}{m}\right )\\
E\ &= \int_V d^3q \int d^3 p\, e^{-\alpha} \exp\left(-\frac{\beta}{2}\frac{\vec{p}^2}{m}\right) \frac{\vec{p}^2}{2m}
\end{align*}
Scompongo il secondo integrale in coordinate, ottenendo il prodotto di tre integrali gaussiani:
\begin{align*}
&\int d^3 p\, e^{-\alpha} \exp\left(-\frac{\beta}{2}\frac{\vec{p}^2}{m}\right ) =\\
&=\ \int_{-\infty}^{\infty} dp_x \, \exp\left(-\frac{\beta}{2}\frac{p_x^2}{m}\right ) \int_{-\infty}^{\infty} dp_y \exp\left(-\frac{\beta}{2}\frac{p_y^2}{m}\right )
\int_{-\infty}^{\infty} dp_z \, \exp\left(-\frac{\beta}{2}\frac{p_z^2}{m}\right )
 =\ \left (\ \frac{2\pi m}{\beta}\right)^{3/2}
\end{align*}

Per il secondo integrale, basta notare che:
\[
E =\ \int d^3 p\, \exp \left (-\frac{\beta}{2}\frac{\vec{p}^2}{m}\right ) \frac{\vec{p}^2}{2m} = -\frac{\partial}{\partial \beta} \int d^3 p\, \exp\left (-\frac{\beta}{2} \frac{\vec{p}^2}{m}\right ) = \frac{3}{2}\frac{1}{\beta} \left(\frac{2\pi m}{\beta} \right )^{3/2}
\]

Alla fine giungiamo quindi a:
\begin{align*}
N &=\ Ve^{-\alpha}\left (\ \frac{2\pi m}{\beta} \right)^{3/2}\\
E\ &= V e^{-\alpha}\left (\ \frac{2\pi m}{\beta}^{3/2}\right ) \frac{3}{2} \frac{1}{\beta} =\ \frac{3}{2}N \frac{1}{\beta}
\end{align*}
\[
\ln W\ \{\bar{n}+\Delta n\} - \ln W\{\bar{n}\} \approx -\frac{1}{2} \sumn{M} \bar{n}_i \left ( \frac{\Delta n}{\bar{n}_i}\right)^2
\]
possiamo vedere questa come una \textbf{media pesata} sulle celle di $\left ( \frac{\Delta n}{\bar{n}_i}\right)^2$, dove i pesi sono dati da $n_i$. Possiamo quindi scriverla come:
\[
\ln W\ \sim -\frac{1}{2}N \langle \left ( \frac{\Delta n}{\bar{n}_i}\right)^2 \rangle 
\]
Se $N\sim 10^{23}$, mettendo deviazioni dell'ordine di $10^{-4}$, il risultato sarà comunque un numero molto grande. Perciò:
\[
\frac{W\{\bar{n}+\Delta n\}}{W \{\bar{n}\}} \approx e^{-10^{15}}
\]
Perciò anche spostandosi "di poco" dallo stato di massimo volume, gli altri stati hanno un volume \textit{decisamente} più basso.\\
\[
\rho_{\text{micro}}(\vec{q}, \vec{p}; q,p) =\ \sum_{i=1}^N \delta^3 (\vec{q}-\vec{q}_i) \delta^3(\vec{p}-\vec{p}_i)
\]
\[
\langle \rho_{\text{micro}}(\vec{q}, \vec{p}; q,p) \rangle =\ \frac{\int_{E\leq H \leq E+\Delta} dq\,dp\, \rho_{\text{micro}}(\vec{q},\vec{p}; q,p)}{\int_{E\leq H \leq E+\Delta} dq\,dp} = f_{MB}(\vec{q},\vec{p})
\]
poiché praticamente tutti gli stati microscopici corrispondo allo stato di volume massimo, che è descritto dalla distribuzione di Maxwell-Boltzmann.\\

\section{Il secondo postulato}
Nella Termodinamica macroscopica di equilibrio \textit{non tutte le quantità} hanno un corrispettivo microscopico che, mediato, le restituisce. Vi sono grandezze "proprie della termodinamica", la principale delle quali è l'\textbf{entropia} (in un sistema isolato).\\
Non basta quindi la formula data dal primo postulato per "comprendere" tutta la termodinamica. Boltzmann pone l'entropia $S$, come funzione di $E$, $N$ e $V$ pari a:
\[
S(E,N,V) =k \ln \left (\Gamma(E,V,N) \right)
\]
dove
\[
\Gamma(E,N,V) =\frac{1}{h^{3N}}\int_{E\leq H(q,p)\leq E+\Delta} dq\,dp
\]
(CFR il denominatore di prima, che è un "volume adimensionalizzato della shell", che possiamo fare corrispondere ad una nozione di \textit{numero di stati} nella fisica classica).\\ %Inserire ref
%Inserisci disegno (cerchio diviso in due parti)
Consideriamo un sistema grande ripartito in due sottosistemi ragionevolmente grandi $1$ e $2$. Il sistema complessivo ha un'Hamiltoniana $H(q,p)$ che dipende da tutte le particelle di $1$ e $2$. Chiameremo $(q_1,p_1)$ le coordinate delle particelle che stanno in $1$ e $(q_2, p_2)$ le coordinate di quelle in $2$. Consideriamo poi che le particelle non possano mescolarsi (c'è una "barriera" tra $1$ e $2$), ma solo energia.\\
Avremo quindi $N_1+N_2=N$, con tutti e tre i numeri costanti.\\
Assumiamo poi che $H(q,p) =\ H_1(q_1,p_1) + H_2(q_2,p_2)$. Le interazioni, che sono certamente presenti, interessano però solo le particelle vicine all'interfaccia, e quindi un'eventuale termine $H_{12}(q_1, p_1, q_2, p_2)$ è proporzionale alla sua area. Se consideriamo $V \to \infty$, l'area cresce quadraticamente, mentre il volume ha una crescita cubica: possiamo quindi considerare solo le $H_1$ e $H_2$ (che dipendono dal\ "bulk", dal volume dei due sottosistemi) e trascurare le interazioni (che sono di ordine inferiore).\\
Abbiamo allora che:
\[
S(E,N,V)\ =\ S_1(E_1, V_1, N_1) + S_2(E_2, V_2, N_2)
\]
con $E_1 + E_2 =\ E$ (il sistema è isolato, quindi $E$ si conserva).\\
Supponiamo ora che $H_1(q_1, p_1) \geq 0$, $H_2(q_2, p_2) \geq 0$ (fissiamo a $0$ l'energia minima dei due sottosistemi).\\
$
\Gamma(E,V,N)$ è il "volume" su $(q,p)$ della shell $E\leq H(q,p) \leq E+2\Delta$.

%Disegno: spazio delle fasi (q_1, p_1), shell tra E_1\leq H_1 \leq E_1+\Delta % di volume \Gamma_1  
%e in (q_2, p_2) shell tra E-E_1 \leq H_2 \leq E_E_1 +\Delta
Consideriamo i due spazi delle fasi di $1$ e $2$.\\
Se individuo una shell $E_1\leq H_1 \leq E_1+\Delta$ di volume $\Gamma_1$ nello spazio $(q_1, p_1)$, ne devo selezionare un'altra $E-E_1 \leq H_2 \leq E-E_1 +\Delta$ con volume $\Gamma_2(E-E_1)$ su $(q_2, p_2)$.\\
Sommando membro a membro:
\[
E \leq H_1(q_1, p_1) + H_2(q_2, p_2) \leq E+2\Delta
\]
Moltiplicando le due $\Gamma_1$ e $\Gamma_2$ ottengo allora un contributo alla $\Gamma$ totale. Un altro contributo sarà dato esaminando un'altra shell su $(q_1, p_1)$, a cui corrisponde un'altra $(q_2, p_2)$. $\Gamma$ sarà quindi data "spazzando" tutto lo spazio delle fasi.\\
Sia quindi $E_i =\ i\Delta$, con\ $i =\ 0,1,2,\dots, \frac{E}{\Delta}$. Allora:
\[
\Gamma(E,N,V) =\sum_{i=0}^{E/\Delta} \Gamma_1(E_i, N_i,V_i) \Gamma_2(E-E_i, N_2, V_2)
\]
(stiamo considerando "tutte" le possibili ripartizioni dell'energia).\\
Di nuovo, se il sistema comprende un sufficiente numero di particelle, su questa somma domina \textit{un unico termine di massimo} $\bar{E}_1$ tale che:
\[
\Gamma(E,N,V) \approx \bar{E}_1
\]
Proviamo a giustificarlo. La somma comprende un numero grande ma finito di termini. Possiamo scrivere:
\[
\Gamma_1(\bar{E}_1)\Gamma_2(E-\bar{E}_1) \leq \Gamma(E) \leq \left (\frac{E}{\Delta}+1\right ) \Gamma_1(\bar{E}_1)\Gamma_2(E-\bar{E}_1)
\]
(il termine maggiore moltiplicato per il numero di termini chiaramente maggiora la sommatoria. Allo stesso tempo, visto che tutti i termini sono positivi per ipotesi, la $\Gamma$ è maggiore del suo termine maggiore).\\
Passando ai logaritmi:
\[
\underbrace{k\ln \Gamma_1(\bar{E}_1)}_{S_1} + \underbrace{k\ln(\Gamma_2(E-\bar{E}_1))}_{S_2} \leq k\ln\Gamma(E) \leq k\ln\Gamma_1(\bar{E}_1) + k\ln\Gamma_2(E-\bar{E}_1) + k\ln \left (\frac{E}{\Delta} +1 \right )
\]
Se $S$ è estensiva, ossia se $S_1 \propto N_1$ e $S_2 \propto N_2$, allora il termine finale, pari al logaritmo del numero di termini, è decisamente inferiore rispetto a termini che scalano linearmente con il numero di termini, e quindi possiamo trascurarlo:
\[
S(E,V,N) =\ S_1(\bar{E}_1,V_1,N_1) + S_2(E-\bar{E}_1, V_2, N_2)
\]
perciò l'entropia del sistema è \textit{dominata} da quella di un unico stato di \textit{massimo}.\\
Chiaramente questo è solo un argomento di plausibilità (non una dimostrazione, del resto stiamo prendendo il risultato come postulato), l'effettiva dimostrazione sarà data dal confronto con gli esperimenti.

%pag. 130-134
\end{document}
