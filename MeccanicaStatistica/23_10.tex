\documentclass[12pt]{article}

\usepackage[usenames, dvipsnames, table]{xcolor}
\usepackage[utf8]{inputenc}
\usepackage[T1]{fontenc}
\usepackage{lmodern}
\usepackage{amsmath}
\usepackage{amsfonts}
\usepackage{comment}
\usepackage{wrapfig}
\usepackage{booktabs}
\usepackage{braket}
\usepackage{tikz}
\usepackage{gnuplottex}
\usepackage{epstopdf}
\usepackage{marginnote}
\usepackage{float}
\usetikzlibrary{tikzmark}
\usepackage{graphicx}
\usepackage{cancel}
\usepackage{bm}
\usepackage{mathtools}
\usepackage{hyperref}
\usepackage{ragged2e}
\usepackage[stable]{footmisc}
\usepackage{enumerate}
\usepackage{mathdots}
\usepackage[framemethod=tikz]{mdframed}
\PassOptionsToPackage{table}{xcolor}
\usepackage{soul}
\usepackage{enumerate}
\usepackage{mathdots}
\usepackage[framemethod=tikz]{mdframed} %Added 16/10
\usepackage[italian]{babel} %Added 16/10
\usepackage{amssymb} %Added

%%BOOKTAB
\setlength{\aboverulesep}{0pt}
\setlength{\belowrulesep}{0pt}
\setlength{\extrarowheight}{.75ex}
\setlength\parindent{0pt} %Rimuove indentazione


%%GEOMETRIA
\usepackage[a4paper]{geometry}
 \newgeometry{inner=20mm,
            outer=49mm,% = marginparsep + marginparwidth 
                       %   + 5mm (between marginpar and page border)
            top=20mm,
            bottom=25mm,
            marginparsep=6mm,
            marginparwidth=30mm}

\makeatletter
\renewcommand{\@marginparreset}{%
  \reset@font\small
  \raggedright
  \slshape
  \@setminipage
}
\makeatother
 

%%COMANDI
\newcommand{\q}[1]{``#1''}
\newcommand{\lamb}[2]{\Lambda^{#1}_{\>{#2}}}
\newcommand{\norm}[1]{\left\lVert#1\right\rVert}
\newcommand{\hs}{\mathcal{H}}
\newcommand{\minus}{\scalebox{0.75}[1.0]{$-$}}
\newcommand{\hlc}[2]{%
  \colorbox{#1!50}{$\displaystyle#2$}}
\newcommand{\bb}[1]{\mathbb{#1}}
\newcommand{\op}[1]{\operatorname{#1}}
\renewcommand{\figurename}{Fig.}

\usepackage{fancyhdr}
\pagestyle{fancy}
\fancyhead{} % clear all header fields
\renewcommand{\headrulewidth}{0pt} % no line in header area
\fancyfoot{} % clear all footer fields
\fancyfoot[R]{Francesco Manzali, 2018-19} % other info in "inner" position of footer line
\cfoot{\thepage}

%%AMBIENTI
\newtheorem{thm}{Teorema}[section]
\newtheorem{dfn}{Definizione}
\newtheorem{oss}{Osservazione}
\newtheorem{es}{Esempio}
\newtheorem{axi}{Assioma}
%%Domande di Marchetti
\newtheorem{question}{Domanda}


%%OPERATORI
\DeclareMathOperator{\sech}{sech}
\DeclareMathOperator{\csch}{csch}
\DeclareMathOperator{\arcsec}{arcsec}
\DeclareMathOperator{\arccot}{arcCot}
\DeclareMathOperator{\arccsc}{arcCsc}
\DeclareMathOperator{\arccosh}{arcCosh}
\DeclareMathOperator{\arcsinh}{arcsinh}
\DeclareMathOperator{\arctanh}{arctanh}
\DeclareMathOperator{\arcsech}{arcsech}
\DeclareMathOperator{\arccsch}{arcCsch}
\DeclareMathOperator{\arccoth}{arcCoth} 




\begin{document}

%Impostazioni booktab (Rimuove quello spazio odioso tra righe)
\setlength{\aboverulesep}{0pt}
\setlength{\belowrulesep}{0pt}
\setlength{\extrarowheight}{.75ex}
\setlength\parindent{0pt} %Rimuove indentazione
\newgeometry{inner=20mm,
            outer=49mm,% = marginparsep + marginparwidth 
                       %   + 5mm (between marginpar and page border)
            top=20mm,
            bottom=25mm,
            marginparsep=6mm,
            marginparwidth=30mm}

\makeatletter
\renewcommand{\@marginparreset}{%
  \reset@font\small
  \raggedright
  \slshape
  \@setminipage
}
\makeatother

\begin{comment} %INTESTAZIONE PRIMA PAGINA
\begin{center}
                \line (1,0){350} \\
                \textsc{\normalsize Trascrizione degli appunti delle lezioni di}\\
                [0.25in]
                \huge{\bfseries Istituzioni di Fisica teorica}\\
                [2mm]
                \textsc{\normalsize Prof. Marchetti}\\
                \line (1,0){350} \\
                [0.5cm]
                \textsc{\normalsize Anno accademico 2018-2019}\\ 

        \end{center}
\end{comment}

\newgeometry{inner=20mm,
            outer=49mm,% = marginparsep + marginparwidth 
                       %   + 5mm (between marginpar and page border)
            top=20mm,
            bottom=25mm,
            marginparsep=6mm,
            marginparwidth=30mm}

\makeatletter
\renewcommand{\@marginparreset}{%
  \reset@font\small
  \raggedright
  \slshape
  \@setminipage
}
\makeatother



\section{Lezione 4:\\ \large{Il Secondo Principio e la termodinamica dei gas perfetti}}
\vspace{-1em}
\begin{center}
    \small{(23/10/2018)}
\end{center}
\subsection{Riassuntino}
Stavamo enunciando il II postulato dell'ensemble microcanonico. Per un sistema isolato con energia $E$, volume $V$ e $N$ poarticelle definiamo l'\textbf{entropia} come:
\[
S(E,V,N) =\ k\ln\underbrace{\left[\frac{1}{h^{3N}}\int_{E\leq H(q,p)\leq E+\Delta}dq\,dp\right]}_{\Gamma(E)}
\]
(la costante $h$ serve per "adimensionalizzare", e infatti ha le dimensioni di un'azione alla $3N$).\\
Il calcolo dell'integrale el caso di un sistema interagente non è per nulla banale.\\
Abbiamo cercato di dare argomenti di plausibilità per questo postulato (che del resto non si può dimostrare, essendo un postulato).\\
In particolare siamo partiti mostrando che, con questa definizione, l'entropia è estensiva. Considerando un sistema diviso in $2$ sezioni non comunicanti ad energia $E$ abbiamo trovato:
\[
\Gamma(E) =\ \sum_{i=0}^{E/\Delta} \Gamma_1(E_i)\Gamma_2(E-E_i); \quad E_i =\ i\Delta
\]
In questa somma di termini (positivi) ce n'è uno massimo, presso l'energia $\bar{E}$. La somma totale sarà certamente maggiore di quest'ultimo, ma minore della somma di $E/\Delta$ volte il termine massimo (è come se calcolassi la somma sostituendo ad ogni termine il massimo di tutti):
\[
\Gamma_1(\bar{E}_1)\Gamma_2(E-\bar{E}_1)\leq \Gamma(E)\leq \left(\frac{E}{\Delta}+1\right )\Gamma_1(\bar{E}_1)\Gamma_2(E-\bar{E}_1)
\]
Calcolando i $\log$ e moltiplicando per $k$ ambo i membri:
\[
\underbrace{k\ln\Gamma_1(\bar{E}_1)}_{S_1(\bar{E}_1)}+\underbrace{k\ln\Gamma_2(E-\bar{E}_1)}_{S_2(E-\bar{E}_1)}\leq k\ln\Gamma(E)\leq \underbrace{k\ln\Gamma_1(\bar{E}_1)}_{S_1(\bar{E}_1)}+\underbrace{k\ln\Gamma_2(E-\bar{E}_1)}_{S_2(E-\bar{E}_1)} + \hlc{Yellow}{k\ln\left(\frac{E}{\Delta}+1\right)}
\]
Ma $S_1(\bar{E}_1)$ e $S_2(E-\bar{E}_1)$ crescono linearmente con $N$ (se l'entropia è estensiva), mentre il termine evidenziato cresce come $\ln N$, ed è quindi trascurabile. Ponendo $\bar{E}_2 = E-\bar{E}_1$ si ottiene:
\[
S(E) = S_1(\bar{E}_1)+S_2(\bar{E}_2)
\]
\textbf{Nota}: nonostante vi siano un sacco di modi per "spartire" l'energia tra i due sistemi, è solo uno che prevale - statisticamente - su tutti gli altri, ed è questo che si vede anche empiricamente.\\

\subsection{23/10}
Ma che energia hanno queste energie $\bar{E}_1$ e $\bar{E}_2$ a cui si \textit{assesta} il sistema? Abbiamo visto che corrispondono al massimo di $\Gamma$, perciò:

\begin{align}
&\frac{\partial}{\partial E_i}\left [k\ln(\Gamma_1(E_i)\Gamma_2(E-E_i))\right]\big|_{E_i = \bar{E}_1} = 0\nonumber \\
&\Rightarrow \frac{\partial}{\partial E_i}\left[S_1(E_i)+S_2(E-E_i)\right]\big|_{\bar{E}_i} = 0\nonumber \\
&\Rightarrow \frac{\partial S_1(E_1)}{\partial E_1}\big|_{\bar{E}_1} =\ \frac{\partial S_2}{\partial E_2}(E_2)\big|_{E_2=E-\bar{E}_1}
\label{eqn:equilibrio-entropie}
\end{align}
All'equilibrio, le derivate delle entropie dei due sottosistemi rispetto alle loro energie sono uguali.\\
In termodinamica abbiamo:
\[
dS = \frac{dQ}{T} \underset{(a)}{=} \frac{1}{T}(dE + p\,dV) =\frac{dE}{T}+\frac{P}{T}dV\Rightarrow \frac{dS}{dE} = \frac{1}{T}
\]
in (a) abbiamo applicato il I principio della termodinamica. Ma allora applicando quanto appena trovato a (\ref{eqn:equilibrio-entropie}) si ottiene:
\[
\frac{1}{T_1}=\frac{1}{T_2}=\frac{1}{T}
\]
Abbiamo cioè ottenuto che due sistemi all'equilibrio hanno la stessa temperatura.\\
La \textit{temperatura} è quindi individuata con quel parametro che è uguale in tutte le sezioni di un sistema termodinamico all'equilibrio.\\
In termodinamica abbiamo già osservato che se un sistema isolato subisce una trasformazione spontanea, allora la sua entropia può solo aumentare, e mai diminuire.\\
Consideriamo per esempio un sistema in pareti adiabatiche. Possiamo pensare di aprire una nuova sezione del contenitore e far espandere liberamente il sistema ad un volume più grande. In questo processo la $S$ deve aumentare. Verifichiamolo.\\
Ma per come è definita l'entropia:
\[
S =\ k\ln\Gamma(E); \quad \Gamma(E)=\frac{1}{h^{3N}}\int_{E\leq H \leq E+\Delta}dq\,dp
\]
Ampliare il volume del sistema, fa sì che si aumenti il dominio delle coordinate dell'integrale in $\Gamma(E)$. Essendo allora l'integrale di una quantità positiva ($1>0$), ne segue immediatamente che $\Gamma(E)$ cresce, e con esso l'entropia (come ci aspettiamo).

\subsection{Gas perfetti}
Consideriamo un sistema di particelle con hamiltoniana $H$:
\[
H(q,p) =\ \sum_{i=1}^N \frac{\vec{p}_i^{\,2}}{2m}+\sum_{i=1}^N U(\vec{q}_i); \quad U(\vec{q}) = \begin{cases}
+\infty &\text{se } \vec{q}\notin V\\
0 &\text{se } \vec{q}\in V
\end{cases}
\]
Definiamo:
\[
\Sigma(E)\equiv \frac{1}{h^{3N}}\int \int_{0\leq H(q,p)\leq E} dq\,dp
\]
Ma allora:
\[
\Gamma(E) = \Sigma(E+\Delta)-\Sigma(E)
\]
Nel dettaglio:
\[
\Sigma(E) \equiv \frac{1}{h^{3N}} \int_{0\leq H(q,p)\leq E} d^3q_1\int d^3q_2\cdots \int d^3q_N\int d^3 p_1\cdots \int d^3p_N
\]
Tutte le integrazioni sulle $q$ sono limitate (dalla condizione sulla $H$) all'interno del volume $V$ (se una particella uscisse, il potenziale avrebbe un contributo $+\infty$ all'integrale, portando decisamente $H>E+\Delta$.\\
Per quanto riguarda i momenti avremo, inoltre, che l'energia cinetica di tutte le particelle non può superare $H$:
\[
\sum_{i=1}^N \vec{p}_i^{\,2}\leq 2mE
\]
Perciò:
\[
\Sigma(E) \equiv \frac{1}{h^{3N}} \int_V d^3q_1\int_V d^3q_2\cdots \int_V d^3q_N\underbrace{\int d^3 p_1\cdots \int d^3p_N}_{\sum_{i=1}^N \vec{p}_i^{\,2}\leq 2mE}
\]
Tale condizione fa sì che l'integrazione sia all'interno di una ipersfera a $3N$ dimensioni, raggio $R=\sqrt{2mE}$. Il volume di una tale ipersfera è (in quanto volume), un monomio in $R^n$:
\[
\Omega_n(R) = C_n R^n \Rightarrow \int_0^{+\infty} \frac{d}{dR}\Omega_n(R) = \int_0^{+\infty} n\, C_n\,R^{n-1}dR
\]
Utilizzando quest'espressione possiamo integrare facilmente una funzione $f(R)$ che dipende solo da $R$ (e non dai vari angoli in coordinate sferiche). Avremo infatti:
\[
\int_{S^n(\bb{R})} f(R) = \int_{0}^{+\infty} n\,C_n\,R^{n-1}f(R)\,dR
\]
Vediamo come ricavare il coefficiente $C^n$.\\
Partiamo esaminando il caso di un integrale Gaussiano:
\[
\int_{-\infty}^{+\infty} dx\,e^{-x^2} =\pi^{\frac{1}{2}}
\]
Estendendo a $n$-dimensioni:
\[
(*)=\int_{-\infty}^{+\infty} dx_1 \int_{-\infty}^{+\infty} dx_2 \cdots \int_{-\infty}^{+\infty} dx_n\, \underbrace{e^{-(x_1^2+x_2^2+\dots+x_n^2)}}_{e^{-R^2}}=\pi^{\frac{n}{2}}
\]
Ma allora, riscrivendo in coordinate polari:
\[
\pi^{\frac{n}{2}} = nC_n\int_0^{+\infty} dR\,R^{n-1}e^{-R^2}\equiv (*)
\]
Quest'ultimo integrale può essere scritto in termini di funzione $\Gamma(n)$ di Eulero:
\[
\Gamma(n) =\int_0^{+\infty} dt\,t^{n-1}e^{-t}; \quad \Gamma(n+1)=n!
\]
con la sostituzione $R^2=t \Rightarrow R=t^{1/2}$, $dr = t^{-1/2}dt/2$:
\[
\int_0^{+\infty} \frac{1}{2}t^{-1/2} t^{\frac{1}{2}(n-1)}e^{-t}dt = \frac{1}{2}\int_0^{+\infty}t^{\frac{1}{2}-1}e^{-t}dt \underset{m=n/2}{=} \frac{1}{2}\int_0^{+\infty}t^{m-1}e^{-t}dt=\Gamma(m)
\]
Utilizzando il risultato in (*):
\[
\pi^{\frac{n}{2}} =\ nC_n \frac{1}{2}\int_0^{+\infty}dt\,t^{\frac{n}{2}-1}e^{-t}
\]
Ma per le proprietà della $\Gamma$:
\[
\frac{n}{2}\Gamma\left(\frac{n}{2}\right) =\Gamma\left(\frac{n}{2}+1\right)
\]
da cui:
\[
C_n = \frac{\pi^{\frac{n}{2}}}{\Gamma\left(\frac{n}{2}+1\right)}
\]
E sostituendo nell'espressione per la $\Sigma(E)$:
\[
\Sigma(E) = \frac{1}{h^{3N}}V^N \Omega_{3N}(\sqrt{2mE}) = \frac{1}{h^{3N}}V^N \frac{\pi^{\frac{3N}{2}}}{\Gamma\left(\frac{3N}{2}+1\right)}(2mE)^{\frac{3N}{2}}
\]
\begin{align*}
\Sigma(E+\Delta)-\Sigma(E)&=\frac{V^N}{h^{3N}}\frac{\pi^{\frac{3}{2}N}}{\Gamma\left(\frac{3N}{2}+1\right)}(2m)^{\frac{3N}{2}}
\left[(E+\Delta)^{\frac{3N}{2}}-E^{\frac{3N}{2}}\right] =\\
&= \frac{V^N}{h^{3N}}\frac{\pi^{\frac{3}{2}N}}{\Gamma\left(\frac{3N}{2}+1\right)}(2m)^{\frac{3N}{2}} \left [\left(1+\frac{\Delta}{E}\right)^{\frac{3N}{2}}-1\right] \equiv \Gamma(E) 
\end{align*}
Se ogni particella ha un'energia media $u$, allora aumentando il numero di particelle l'energia del sistema cresce linearmente $E=Nu$, e così anche il volume. Portando $N\to\infty$
\begin{align*}
\Gamma(E) = \frac{V^N}{h^{3N}}\frac{\pi^{\frac{3}{2}N}}{\Gamma\left(\frac{3N}{2}+1\right)}(2m)^{\frac{3N}{2}} \underbrace{\left [\left(1+\frac{3}{2}\frac{\Delta}{\frac{3}{2}E}\right)^{\frac{3N}{2}}-1\right]}_{\xrightarrow[N\to\infty]{}\>\exp\left(\frac{3}{2}\frac{\Delta}{u}\right)-1}
\end{align*}
Prendendo il $\ln$ e moltiplicando per $k$:
\[
k\log\Gamma(E)=k\log\Sigma(E)+\hlc{Yellow}{k\log\left[\exp\left(\frac{3}{2}\frac{\Delta}{u}\right)-1\right]
}\]
Ma quest'ultimo termine evidenziato è fisso e finito, mentre il resto scala con il sistema. Perciò, per sistemi molto grandi, tale termine è trascurabile, e vale:
\[
S=k\log\Sigma(E)=k\ln\left\{
\frac{1}{h^{3N}}V^N \frac{\pi^{\frac{3N}{2}}}{\Gamma\left(\frac{3N}{2}+1\right)}(2mE)^{\frac{3N}{2}}
 \right \}
\]
(Il volume di una ipersfera ad altissima dimensionalità è praticamente dominato dalla sua superficie).\\
Per $N$ alti possiamo usare l'approssimazione di Stirling:
\[
\ln\Gamma(n+1)=n\ln n-n \Rightarrow \ln\Gamma\left(\frac{3N}{2}+1\right)\approx \frac{3N}{2}\ln \frac{3N}{2}-\frac{3N}{2} = N\ln\left(\left(\frac{3N}{2}\right)^{\frac{3}{2}}\right)-\frac{3}{2}N
\]
e sostituendo nell'espressione dell'entropia:
\begin{align*}
S(E)&=kN\log\left(
\frac{V}{h^3}\frac{\pi^{\frac{3}{2}}}{\left(\frac{3N}{2}\right)^{\frac{3}{2}}}
(2mE)^{\frac{3}{2}}
\right)+k\frac{3}{2}N =\\
&= kN\log\left(V\left(\frac{4}{3}\frac{\pi m}{h^2}\right)^{\frac{3}{2}}\left(\frac{E}{N}\right)^{\frac{3}{2}}\right)+\frac{3}{2}kN
\end{align*}
%Qual è il problema in questa formula?
Derivando rispetto ad $E$:
\[
\frac{\partial S}{\partial E}(E,N,V) = \frac{1}{T} = kN\frac{3}{2}\frac{1}{E}
\]
E infatti nella termodinamica classica per un gas perfetto avevamo $E=\frac{3}{2}Nk_B T$.\\
Invece, facendo:
\[
\frac{\partial S}{\partial V}(E,N,V) =\frac{P}{T} = kN\frac{1}{V}=\frac{P}{T}
\]
e perciò:
\[
PV=NkT
\]
Ossia la termodinamica che deriviamo da questo modello non interagente e dai due postulati della meccanica statistica è consistente con quella che avevamo già determinato classicamente.\\
Ritornando all'espressione per $S(E)$ che abbiamo attenuto, noi vorremmo, per l'estensività:
\[
S(\lambda E, \lambda V, \lambda N)=\lambda S(E,V,N)
\]
Ma calcolandola nel risultato a cui siamo giunti:
\[
S(\lambda E, \lambda V, \lambda N)= \bm{\lambda}kN\log\left(\hlc{Yellow}{\bm{\lambda}}V\left(\frac{4}{3}\frac{\pi m}{h^2}\right)^{\frac{3}{2}}\left(\frac{\cancel{\bm{\lambda}} E}{\cancel{\bm{\lambda}}N}\right)^{\frac{3}{2}}\right)+\frac{3}{2}kN\bm{\lambda}
\]
La presenza del $\lambda$ evidenziato fa sì che tale proprietà non sia verificata.\\
Secondo Boltzmann ciò salta fuori dal fatto che stiamo considerando come stati distinti quelli costituiti scambiando particelle che sono \textit{indistinguibili} tra loro. Ma ciò è chiaro solo dal punto di vista quantistico (in MC, come abbiamo visto, a priori è tutto distinguibile).\\
Tenendo conto di ciò, nel conteggio degli stati bisogna dividere per il numero delle "permutazioni indistinguibili" di $n$ particelle, ossia per $n!$
\[
\Gamma(E)=\frac{1}{N!}\frac{1}{h^{3N}}\int_{E\leq H \leq E+\Delta}dq\,dp
\]
Da:
\[
\ln \frac{1}{N!} = -N\ln N+N
\]
giungiamo, ripetendo i conti, ad una nuova espressione per $S(E)$:
\[
S(E,V,N)=kN\log\left(\frac{V}{N}\left(\frac{4}{3}\frac{\pi m}{h^2}\right)^{\frac{3}{2}}\left(\frac{E}{N}\right)^{\frac{3}{2}}\right ) + kN\frac{5}{2}
\]
che stavolta soddisfa la proprietà di omogeneità voluta.
\end{document}

